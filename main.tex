\documentclass[a4paper]{report}

% Number of "levels" the label can reference. Default is 2.
\setcounter{secnumdepth}{3}


\usepackage[english, science, titlepage]{ku-frontpage}

\usepackage[latin1]{inputenc}
\usepackage{palatino}
% \usepackage[usenames]{color}

\usepackage{hyperref}

\usepackage{chngpage}
\usepackage{graphicx}

\usepackage{booktabs}
\usepackage{multirow}
\usepackage[nounderscore]{syntax}
\setlength{\grammarparsep}{0.25cm}
\setlength{\grammarindent}{3.5cm}

\usepackage[]{algorithm2e}
\usepackage{varwidth}

\usepackage{subcaption}

\usepackage{setspace}


\usepackage{todonotes}


\usepackage{listings}
\usepackage[outputdir=.tmp]{minted}
% \usepackage[cache=false]{minted}
\setminted{frame=lines,linenos=true,framesep=2mm,fontsize=\small}


% \usepackage[outputdir=.tmp,chapter,newfloat=true,cache=true]{minted}
% \setminted{fontsize=\scriptsize, frame=single, framesep=2mm, autogobble}
% \usemintedstyle{default}
\usepackage{times}

\usepackage{amsmath}
\usepackage{amssymb}
\normalfont
%\usepackage[T1]{fontenc}
\renewcommand{\ttdefault}{cmtt}
\usepackage{xcolor}
\usepackage{booktabs,siunitx}
\usepackage[font=small]{caption}

\usepackage{parcolumns}

\usepackage{pdfpages}


% NOTE: The margins become wider just by including the geometry package
\usepackage{geometry}
\geometry{
%  % a4paper,
%  % total={170mm,257mm},
%  % left=20mm,
 top=30mm,
 bottom=40mm
 }

% \usepackage{url}
\usepackage{pgfplots}
\usepackage{tikz}


% \usepackage[%
%   backend=biber,
%   style=numeric,
%   maxnames=2,
%   minnames=1,
%   maxbibnames=99,
%   firstinits,
%   uniquename=init,
%   backref=true]{biblatex} % TODO: adapt citation style
%   %maxcitenames=2,
%
% \AtEveryBibitem{%
%   \ifentrytype{onlinex}{%
%   }{%
%     \clearfield{url}%
%     \clearfield{urldate}%
%   }%
% }




\usepackage[labelsep=period]{caption}

\usepackage{cspsymb}

\usetikzlibrary{calc}
\usetikzlibrary{fit}
\usetikzlibrary{positioning}
\usepgfplotslibrary{units}
\pgfplotsset{compat=newest}
\usetikzlibrary{decorations.pathmorphing}
\usetikzlibrary{decorations.markings}
\usetikzlibrary{arrows}
\usetikzlibrary{shapes.geometric}

\usepackage{ifthen}
\pgfkeys{
  /sevenseg/.is family, /sevenseg,
  slant/.estore in      = \sevensegSlant,     % vertical slant in degrees
  size/.estore in       = \sevensegSize,      % length of a segment
  shrink/.estore in     = \sevensegShrink,    % avoids overlapping of segments
  line width/.estore in = \sevensegLinewidth, % thickness of the segments
  line cap/.estore in   = \sevensegLinecap,   % end cap style rect, round, butt
  oncolor/.estore in    = \sevensegOncolor,   % color of an ON segment
  offcolor/.estore in   = \sevensegOffcolor,  % color of an OFF segment
}

\pgfkeys{
  /sevenseg,
  default/.style={
    slant = 0,
    size = 1em,
    shrink = 0.2,
    line width = 0.3em,
    line cap = butt,
    oncolor = green!50!black,
    offcolor = white!75!black
  }
}
\newcommand{\sevenseg}[2][]% options values
{%
\pgfkeys{/sevenseg, default, #1}%
\def\sevensegarray{#2}%
  \begin{tikzpicture}%
    % first define the position of the 6 corner points
    \path (0,0) ++(0,0)                             coordinate (P1);
    \path (0,0) ++(\sevensegSize,0)                 coordinate (P2);
    \path (0,0) ++(90-\sevensegSlant:\sevensegSize) coordinate (P3);
    \path (P2)  ++(90-\sevensegSlant:\sevensegSize) coordinate (P4);
    \path (P3)  ++(90-\sevensegSlant:\sevensegSize) coordinate (P5);
    \path (P4)  ++(90-\sevensegSlant:\sevensegSize) coordinate (P6);
    % then step through the 1/0 values in the segment array
    \foreach \i in {0,...,6}%
    {
      \pgfmathparse{\sevensegarray[\i]}
      \ifthenelse{\equal{\pgfmathresult}{1}}%
        {\let\mycolor=\sevensegOncolor}%  segment is on
        {\let\mycolor=\sevensegOffcolor}% segment is off
      \tikzstyle{segstyle} = [draw=\mycolor, line width = \sevensegLinewidth,
                              line cap = \sevensegLinecap]
      %-----------------------
      \ifthenelse{\equal{\i}{0}}{\path[segstyle]
        (${1-\sevensegShrink}*(P5)+\sevensegShrink*(P6)$)
        -- ($\sevensegShrink*(P5)+{1-\sevensegShrink}*(P6)$);}{} % a
      \ifthenelse{\equal{\i}{1}}{\path[segstyle]
        (${1-\sevensegShrink}*(P6)+\sevensegShrink*(P4)$)
        -- ($\sevensegShrink*(P6)+{1-\sevensegShrink}*(P4)$);}{} % b
      \ifthenelse{\equal{\i}{2}}{\path[segstyle]
        (${1-\sevensegShrink}*(P4)+\sevensegShrink*(P2)$)
        -- ($\sevensegShrink*(P4)+{1-\sevensegShrink}*(P2)$);}{} % c
      \ifthenelse{\equal{\i}{3}}{\path[segstyle]
        (${1-\sevensegShrink}*(P1)+\sevensegShrink*(P2)$)
        -- ($\sevensegShrink*(P1)+{1-\sevensegShrink}*(P2)$);}{} % d
      \ifthenelse{\equal{\i}{4}}{\path[segstyle]
        (${1-\sevensegShrink}*(P1)+\sevensegShrink*(P3)$)
        -- ($\sevensegShrink*(P1)+{1-\sevensegShrink}*(P3)$);}{} % e
      \ifthenelse{\equal{\i}{5}}{\path[segstyle]
        (${1-\sevensegShrink}*(P3)+\sevensegShrink*(P5)$)
        -- ($\sevensegShrink*(P3)+{1-\sevensegShrink}*(P5)$);}{} % f
      \ifthenelse{\equal{\i}{6}}{\path[segstyle]
        (${1-\sevensegShrink}*(P3)+\sevensegShrink*(P4)$)
        -- ($\sevensegShrink*(P3)+{1-\sevensegShrink}*(P4)$);}{} % g
    }
  \end{tikzpicture}%
}

\newcommand{\sevensegnum}[2][]% sample characvters
{%
  \ifthenelse{\equal{#2}{0}}{\sevenseg[#1]{{1,1,1,1,1,1,0,}}}{%
  \ifthenelse{\equal{#2}{1}}{\sevenseg[#1]{{0,1,1,0,0,0,0,}}}{%
  \ifthenelse{\equal{#2}{2}}{\sevenseg[#1]{{1,1,0,1,1,0,1,}}}{%
  \ifthenelse{\equal{#2}{3}}{\sevenseg[#1]{{1,1,1,1,0,0,1,}}}{%
  \ifthenelse{\equal{#2}{4}}{\sevenseg[#1]{{0,1,1,0,0,1,1,}}}{%
  \ifthenelse{\equal{#2}{5}}{\sevenseg[#1]{{1,0,1,1,0,1,1,}}}{%
  \ifthenelse{\equal{#2}{6}}{\sevenseg[#1]{{1,0,1,1,1,1,1,}}}{%
  \ifthenelse{\equal{#2}{7}}{\sevenseg[#1]{{1,1,1,0,0,0,0,}}}{%
  \ifthenelse{\equal{#2}{8}}{\sevenseg[#1]{{1,1,1,1,1,1,1,}}}{%
  \ifthenelse{\equal{#2}{9}}{\sevenseg[#1]{{1,1,1,1,0,1,1,}}}{%
  \ifthenelse{\equal{#2}{A}}{\sevenseg[#1]{{1,1,1,0,1,1,1,}}}{%
  \ifthenelse{\equal{#2}{B}}{\sevenseg[#1]{{0,0,1,1,1,1,1,}}}{%
  \ifthenelse{\equal{#2}{C}}{\sevenseg[#1]{{0,0,0,1,1,0,1,}}}{%
  \ifthenelse{\equal{#2}{D}}{\sevenseg[#1]{{0,1,1,1,1,0,1,}}}{%
  \ifthenelse{\equal{#2}{E}}{\sevenseg[#1]{{1,0,0,1,1,1,1,}}}{%
  \ifthenelse{\equal{#2}{F}}{\sevenseg[#1]{{1,0,0,0,1,1,1,}}}{%
  {\sevenseg[#1]{{0,0,0,0,0,0,0,}}}}}}}}}}}}}}}}}}}%
}

\tikzset{
  myarrow/.style={
    draw=black,
    thick,
    ->,
    shorten <=3pt,
    shorten >=3pt,
  },
  mycircle/.style={
    draw=black,
    shape=circle,
    very thick,
    inner sep=3pt,
    inner ysep=5pt,
    text width=0.75cm,
    align=center,
    minimum size=0.75cm,
    rounded corners,
  },
  mytriangle/.style={
    draw=black,
    regular polygon,
    regular polygon sides=3,
    align=center,
    rounded corners,
    very thick,
    inner sep=3pt,
  },
  myrectangle/.style={
    draw=black,
    shape=rectangle,
    very thick,
    rounded corners,
    align=center,
    inner sep=7pt,
    inner ysep=7pt,
    text width=2.1cm,
    minimum size=0.5cm,
    minimum height=1.5cm,
    font=\footnotesize
  },
  mysquare/.style={
    draw=black,
    shape=rectangle,
    very thick,
    rounded corners,
    align=center,
    inner sep=7pt,
    inner ysep=7pt,
    font=\footnotesize
  },
  main node/.style={
  circle,
  align=center,
  draw,
  text width=.7cm,
  minimum size=.7cm,
  inner sep=7pt,
  font=\footnotesize
  },
  mythinsquare/.style={
  draw=black,
  shape=rectangle,
  rounded corners,
  align=center,
  inner sep=4pt,
  % inner ysep=7pt,
  font=\footnotesize
  },
}

\pgfplotsset{
  every axis plot post/.style={/pgf/number format/fixed}
}



% \newcommand\todo[1]{\textcolor{red}{#1}}
\newcommand{\cspm}{CSP$_M$}



\usepackage{fancyhdr}
\pagestyle{fancy}




%%%%%%%%%%%%%%%%%%%%%%
% Bibliography Setup %
%%%%%%%%%%%%%%%%%%%%%%
% \usepackage[backend=biber]{biblatex}
% \addbibresource{library.bib}


%%%%%%%%%%%%%%%%%%%%%%%%%%%
% Header and Footer Setup %
%%%%%%%%%%%%%%%%%%%%%%%%%%%

\lhead{Alberte Thegler}
% \chead{}
\rhead{\leftmark}

\renewcommand{\headrulewidth}{0.4pt} % thickness of line at header
\renewcommand{\footrulewidth}{0.4pt} % thickness of line at footer
\setlength{\belowcaptionskip}{-10pt} % space below captions


%%%%%%%%%%%%%%%%%%%%
% Front-page Setup %
%%%%%%%%%%%%%%%%%%%%
%
\assignment{Master's Thesis}
\author{Alberte Thegler - alberte@thegler.dk}
\title{Towards Automatic Program Specification \\ From SME Models}
\subtitle{Department of Computer Science}
\date{Handed in: November 21, 2018}
\advisor{Supervisors: Brian Vinter and Kenneth Skovhede}


\begin{document}

\maketitle

\pagenumbering{roman}


\begin{abstract}
\begin{doublespace}
This thesis introduces a method to simplify hardware modeling and verification
thereof in order for software programmers to, more easily, meet the demands of
the growing embedded device industry. The system TAPS is presented, which is a
system for transpiling from the new SME Implementation Language (SMEIL) into the
machine-readable CSP language, \cspm{}.
The translated \cspm{} code is augmented with assertion statements, using
observed values from the SME model. Utilising the formal verification
properties of the FDR4 refinement tool, the assertion statements of the
translated \cspm{} code are formally verified.
An extension to the initial version of TAPS is introduced in the thesis. The extension provides the possibility of modeling a global synchronous clock in
\cspm{}, providing the possibility of verifying the internal state of the
\cspm{} code separately for each clock cycle.
A small example consisting of a
seven segment display clock network is presented, to introduce how to verify the
widths of channels in the network. A small cyclic network is presented, to
show the possibilities of verifying separate clock cycles, using the extended
version of TAPS.
\end{doublespace}
\end{abstract}


\newpage
\tableofcontents

\newpage
\pagenumbering{arabic}

%%%%%%%%%%%%%%%%%%%
% Include Content %
%%%%%%%%%%%%%%%%%%%

% TODO: Add somewhere some stuff about referencing your own paper and what sectons use stuff from the paper.
% \noindent A paper based on this thesis has been submitted for publication as
%
% \begin{center}
% \begin{minipage}{0.8\textwidth}
%     A. Thegler, M. Larsen, K. Skovhede, and B. Vinter. ”Towards automatic program specification using sme models”. In: {\itshape Proceedings of Communicating Process Architectures
%     2018} (2018)
% \end{minipage}
% \end{center}

% \chapter{Introduction}
% The Internet of Things, computerised medical implants, and the omnipresent growth in robotics, bring with them an increased demand for programmers, that are able to develop software for these devices. While this observation may not in itself appear to present a new challenge, many other areas have previously presented a need for more programmers. The new challenge is that these new growth areas are all focused on small size, low power consumption, and high reliability. This means that traditional software engineering methods, and thus traditionally trained programmers, are often not sufficiently qualified to work with these technologies.
In previous decades, such systems have been developed by electronic engineers that apply far more rigid development approaches. Especially for hardware solutions like Very-Large-Scale Integration (VLSI) and Field-Programmable Gate Array (FPGA), correctness has always been favored over productivity due to a more rigid environment, than traditional software developers are used to.
While tools have obviously improved and methods refined, the VLSI process is still mostly the same as presented in~\cite{Agrawal:1985:VDP:320599.322570}. The primary workflow from~\cite{Agrawal:1985:VDP:320599.322570} is shown in Figure~\ref{fig:Agrawal}; note the focus on verification in each step.
\begin{figure}[!ht]
  \centering
  \begin{tikzpicture}[auto, scale=0.8, every node/.style={scale=0.8}]
    \node[myrectangle] (synthesis)                            {Synthesis and test generation};
    \node[myrectangle] (layout)    [right=2.5cm of synthesis] {Layout};
    \node[myrectangle] (wafer)     [right=2.5cm of layout]    {Wafer\\fabrication and packaging};

    \node[myrectangle] (verification1) [below=0.5cm of synthesis] {Verification};
    \node[myrectangle] (verification2) [below=0.5cm of layout]    {Verification};
    \node[myrectangle] (verification3) [below=0.5cm of wafer]     {Testing};

    \node (input) [left=1.5cm of synthesis] {};
    \draw[myarrow] (input) -- node[near start] {\scriptsize Requirements} (synthesis);

    \node (output) [right=1.5cm of wafer] {};
    \draw[myarrow] (wafer) -- node[near end] {\scriptsize VLSI devices} (output);

    \draw[myarrow] (synthesis) -- node[text width=2cm, align=center, midway] {\scriptsize Logic design and test data} (layout);
    \draw[myarrow] (layout)    -- node[text width=2cm, align=center, midway] {\scriptsize Mask and test data} (wafer);

    \draw[myarrow] (synthesis)     to[out=345, in=15]  (verification1);
    \draw[myarrow] (verification1) to[out=165, in=195] (synthesis);

    \draw[myarrow] (layout)        to[out=345, in=15]  (verification2);
    \draw[myarrow] (verification2) to[out=165, in=195] (layout);

    \draw[myarrow] (wafer)         to[out=345, in=15]  (verification3);
    \draw[myarrow] (verification3) to[out=165, in=195] (wafer);

    \draw[myarrow] (verification3) -- node[text width=3cm, align=center, midway, below=2mm] {\scriptsize Timing, testability, quality, reliability, and yield problems} (verification2);
    \draw[myarrow] (verification2) -- node[text width=3cm, align=center, midway, below=2mm] {\scriptsize Timing, testability, quality, reliability, and yield problems} (verification1);
  \end{tikzpicture}
  \caption{VLSI process workflow.}
  \label{fig:Agrawal}
\end{figure}

While the VLSI community is fundamentally following this 1980's design approach, more high-level tools and abstractions have been introduced. Philippe et al.~\cite{ coussy2009introduction} show a workflow (reproduced in Figure~\ref{fig:coussy}) where the important part is the verification, that has been partly automated by basing the development on a formal specification of the solution.

There is no denying that the subjectively slow and rigid development process in the VLSI world~\cite{kepner2004hpc} is highly successful in producing correct and reliable circuits. At the same time, conventional software development is highly focused on productivity and time-to-market, for example, smartphone applications are often developed for continuous release, where bug patches and new features are rolled out daily. This is of course not possible with hardware.

Thus, a growing chasm exists between the way most programmers are trained, and the competencies that are needed to support the growth in mission critical embedded devices.\\
\begin{figure}[!ht]
  \centering
  \begin{tikzpicture}[auto]
    \node[myrectangle, text width=3cm, minimum height=1cm, inner sep=5pt, inner ysep=5pt] (specification) {Specification};
    \node[myrectangle, text width=3cm, minimum height=1cm, inner sep=5pt, inner ysep=5pt] (compilation) [right=0.5cm of specification] {Compilation};
    \node[myrectangle, text width=3cm, minimum height=1cm, inner sep=5pt, inner ysep=5pt] (formalmodel) [right=0.5cm of compilation] {Formal model};

    \node[myrectangle, text width=3cm, minimum height=1cm, inner sep=5pt, inner ysep=5pt] (behavioral) [right=0.5cm of formalmodel] {Behavioral synthesis};

    \node[myrectangle, text width=3cm, minimum height=1cm, inner sep=5pt, inner ysep=5pt] (generation) [below=1cm of specification] {Generation};
    \node[myrectangle, text width=3cm, minimum height=1cm, inner sep=5pt, inner ysep=5pt] (rtl) [right=0.5cm of generation] {RTL architecture};
    \node[myrectangle, text width=3cm, minimum height=1cm, inner sep=5pt, inner ysep=5pt] (logic) [right=0.5cm of rtl] {Logic synthesis};

    \node[] (dotdotdot) [right=0.5cm of logic] {...};

    \draw[myarrow] (specification) -- (compilation);
    \draw[myarrow] (compilation) -- (formalmodel);
    \draw[myarrow] (formalmodel) -- (behavioral);

    \draw[myarrow] (behavioral) |-([shift={(0mm,-5mm)}]behavioral.south west) -- ([shift={(0mm,5mm)}]generation.north east)-| (generation);

    \draw[myarrow] (generation) -- (rtl);
    \draw[myarrow] (rtl) -- (logic);
    \draw[myarrow] (logic) -- (dotdotdot);
  \end{tikzpicture}
  \caption{Reproduced workflow from Philippe et al.~\cite{coussy2009introduction}.}
  \label{fig:coussy}
\end{figure}


In this thesis, I propose a tool to help bridge the gap between available programmer profiles and the required competencies for developing embedded devices. My approach is based on generating a specification from a software implementation and test-suite observations. The overarching goal is to reach a level where a conventional software programmer can write a solution in the new hardware design model, Synchronous Message Exchange (SME)~\cite{Vinter2014, Vinter2015, Skovhede}, and develop a conventional test suite in the software engineering tradition. SME will be further introduced in Chapter \ref{chap:background}. By combining this implementation with the \emph{observed} values of internal states in an SME based system implementation, we can produce a formal specification of the system, which will be introduced further in Chapter \ref{chap:background}.
This specification can be fed into a formal verification tool and thus improve the correctness guarantees from what is covered by the individual test vectors to the entire space that is spawned by the set of test vectors.\\

We approach the task by transpiling\footnote{Source-to-source compile.} the new SME Implementation Language (SMEIL)~\cite{smeil} for SME into \cspm{}~\cite{Scattergood1998} and verify the formal properties of this version with a tool like FDR4~\cite{fdr}.
\section{Motivation}
As explained, hardware testing is not as simple and reusable as software testing ,and it is not possible to deploy automatic bug patches for physical hardware. Therefore, it is cheaper for the hardware companies to have their newly developed hardware model tested as extensively as possible before production, than shipping physical hardware with a critical error. Although the testing can be extensive, it is difficult to achieve complete code coverage accounting for all corner cases using hand written test cases. There are several examples throughout history where more verification could have saved both lives and resources.
\subsection{Ariane 501 failure}
The Ariane 5 lifting rocket\cite{InquiryBoard1996} was designed to launch large payloads into Earth orbit, such as communications satellites, etc. It was owned by the European Space Agency (ESA) and the French space agency Centre National d'\'Etudes Spatiales (CNES), and manufactured by Airbus Defence and Space.\\

Ariane 5 was the follow-up to the successful Ariane 4 launchers. On June 4th, 1996, the Ariane 5 rocket was first flight tested. At about 30 seconds after successful lift-off, the rocket exploded midair, resulting in an approximately \$500 million loss for ESA and CNES. The failure could have been avoided, if there had been more focus on verification or simulation within the systems of Ariane 5. Luckily the rocket was unmanned, but this type of failure could happen in any other critical system. This launch failure is acknowledged as one of the most expensive human errors in history. \\

The failure of Ariane 5 was partially caused by a bug in the Inertial Reference System (SRI) that measures the attitude of the launcher and its movements in space. The SRI comprises an internal computer which calculates angles and velocities.
The failure occurred due to an unexpected high value of an internal alignment function result called Horizontal Bias (BH). The BH value is related to the horizontal velocity and was represented as a 64-bit floating point number. This value was converted to a 16-bit signed integer within the SRI and sent to the On-Board Computer (OBC), which controls the direction of the nozzles of the solid boosters.\\

The BH value had also been calculated in the Ariane 4 systems, so it was not thought of as insecure. However, because the trajectory of Ariane 5 was considerably different than that of Ariane 4, this value was much higher. Due to this high BH value, the 64-bit floating point number could not be represented by a 16-bit signed integer, so the conversion caused an overflow.
The result of the overflow was interpreted as actual flight data by the OBC and according to this misinterpreted flight data, the rocket was off course. The OBC counteracted what it thought was a wrong angle of the rocket by turning the nozzles to change course. Within a few seconds, the forces of aerodynamics ripped the solid boosters apart from the core stage. This caused the self-destruct mechanism to trigger, and the rocket self-destructed in a giant explosion shortly afterwards. The backup SRI, that should take over when errors occur in the active SRI, was executing the same code as the active SRI and had failed, for the same reasons, just before the active SRI.\\

On the Ariane 4 rocket, the BH value was necessary for a short while after lift-off, but this had been changed in the Ariane 5 where the data was only crucial to the rocket at lift-off. Afterwards, the data should not have had any influence on the rocket. but unfortunately, it had not been predicted that the conversion could overflow and so the functionality of the BH value had been kept. If this functionality and the consequences of a different trajectory had been properly considered for the Ariane 5 rocket, this failure might have been avoided.
\subsection{Therac-25 failure}
In the 80's the company Atomic Energy of Canada Limited (AECL) manufactured a revolutionary radiation therapy machine, the Therac-25\cite{Leveson1993}, which could provide two different kinds of radiation treatment. At that time, hospitals would typically have two different machines to be able to perform both of the treatments that a single Therac-25 machine could perform.\\
Radiation is used to kill cancer cells and so a patient is exposed to a beam of particles, or radiation, in specific doses that are designed to kill the specific type of cancer cells. Because cancer cells are more sensitive to radiation than normal cells, the radiation will kill the cancer cells but cause relatively minor damage to the normal tissue. However, radiation is damaging to normal tissue as well and it is, therefore, essential to specify the exact amount of radiation needed in a specific area to minimise the damage to the healthy cells.\\

The first type of treatment was a electron-beam treatment, which kills shallow tissue, like skin cancer. The second treatment option of the Therac-25 was a beam of higher-energy X-ray photons, which travels further into the tissue and are therefore better suited for cancer in deeper-lying tissue. \\

The Therac-25 was based on the previous Therac-20 and Therac-6, which had been very successful. The Therac-20 and Therac-6 both had hardware safety interlocks to avoid failures but unfortunately, this had been removed in the Therac-25 in favor of software-based security. AECL put more faith in software reliability than on hardware. \\

After the Therac-25 had been operational for a couple of years, a series of incidents happened, where patients were overexposed to radiation, leading to six cases of serious injury, resulting in death for some of them. Friz Hager, a physicist at East Texas Cancer Center, tested the Therac-25 rigorously to reproduce the errors they had experienced. He was able to demonstrate the error showing that if the user selected the X-Ray mode on the Therac-25, the machine began setting up for high-powered X-rays, which would take about 8 seconds. If the user switched to Electron-beam mode before the machine finished setting up for X-ray mode, i.e within 8 seconds, the turntable that controlled the amount of radiation, would not switch to the correct position causing an enormous amount of radiation to reach the patient.\\
After solving the problem and releasing a new version of the Therac-25, another problem emerged where a patient was overexposed. This time it turned out to be a counter overflow within the system and if a command was sent at the exact moment the counter overflowed, the machine would not set up properly, again resulting in an overexposure of radiation for the patient. \\
After this incident with the Therac-25, it was found that some of the same software problems were found in Therac-20, but due to the hardware precautions, the problems never occurred.
This example shows how important it is for critical systems to be well-designed, as well as well-tested or verified.

\subsection{The patriot missile failure}
During the Persian Gulf war, on February 25, 1991, an American Patriot missile failed to intercept an incoming Iraqi Scud missile, which caused the Scud to hit an American Army barracks injuring around 100 people and killing 28 soldiers~\cite{patriot1991}.\\

The Patriot missile failed due to an error in converting an integer, representing time since last boot, to a floating-point number using a 24-bit register. As time since last boot increased, the limited 24-bit register did not represent enough precision and so the chopping error increased. At the time of the incident, the Patriot missile battery had been on for approximately 100 hours, which caused the chopping inaccuracy to be around 0.34 seconds. The Scud travels at around 1.676 meters per second, and therefore in the 0.34 seconds, it travels more than a half kilometer. This caused the Patriot missile to wrongly predict the location of the Scud.
The inaccuracy of the 24-bit representation caused the Patriot missile to perform inaccurate calculations, and the consequence of this was that the Patriot missile missed its target Scud missile. This example shows how it is impossible to test 100\% of any system and how that can cause horrible failures. If the Patriot missile systems had been verified or had been subject to long-term testing, this accident might have been avoided. An example, similar to the Patriot missile failure, will be introduced in this thesis and will be verified to show potentiel failures.
%Intels-division bug \\
%Toyota bremse-fejl\\
\section{Learning goals}
The learning goals accepted for this thesis are:
\begin{itemize}
\item Reflect on the set of SME expressible problems, that are verifiable with FDR4.
\item Reason about efficient code transformation from an executable format to a verifiable format
\item Reason about design choices and their consequences for execution performance.
\item Demonstrate efficient constraint transfer from SME to FDR4.
\item Reason about SME program size and time to verification.
\item Reflect on the generality of a generic verification template.
\item Disseminate project results to a professional audience.
\end{itemize}


% \section{Limitations}




% \chapter{Notations and Abbreviations}
% \input{chapters/conventions.tex}
%
% \chapter{Related Work}
% \label{chap:related_work}
% The concepts of formal verification was first expressed in 1954 when Martin Davis created the first computer generated mathematical proof that the product of two even numbers, is even. First-order theorem provers were applied to verification problems in Pascal, Ada and Java, in the late 1960s.  One of these verification systems was the Stanfort Pascal Verifier\cite{Verifier1979} which was developed by David Luckham at Stanford University.
At Stanford, in 1972, Sir Robin Milner had success building the original LCF system for proof checking and his work in automated reasoning have been the foundation for a lot of other theorem provers, like the proof assistant HOL (Higher Order Logic) by Mike Gordon, which was originally developed for reasoning about hardware. The formal proof management system Coq is a descendent of LCF. \\
Also in 1972, Robert S. Boyer and J. Strother Moore was successful in building a machine-based prover, called Nqthm which became the basis for ACL2 which is a programming language and a theorem prover. Theorem provers have proved very valuable over the time, but one problem with them was, that if they found a problem in a theorem, they could not tell why it could not prove the theorem. It was not possible to create a counter example or any other explanation as to why it was not possible to prove this theorem. \\\\


In 1967, Robert W. Floyd was published with the paper \textit{Assigning meaning to programs}\cite{Floyd1967}. Floyd provided a basis for the formal definitions of the meaning of programs which can be used for proving correctness, equivalence and termination. By using flowcharts, he argued that when a command is reached, all previous commands will have been true as well.\\ C.A.R Hoare was inspired by Floyd and in 1969 his paper \textit{An axiomtic basis for computer programming}\cite{Hoare1969} was published. The logic he presented there (later known as \textit{Hoare logic}), was build on Floyd's ideas and proposed the notation \textit{Partial correctness specification}; $\{P\} C \{Q\}$. Here, $C$ is a command and $P$ and $Q$ are conditions on the program variables in $C$. Hoare showed that whenever $C$ is executed in a state that satisfies the condition $P$, and if the execution terminates, then the state that $C$ terminates in, will satisfy $Q$. Hoares logic have been the basis of a lot of different formal languages and have contributed to the continuous work on formal verification. \\
Since the original Hoares logic was not originially thought as to model concurrent programs, L. Lamport extended Hoare's logic in his paper \textit{The 'Hoare logic' of concurrent programs}\cite{Lamport1980} in 1980. Here, he discuss why Hoare's logic, as proposed by C.A.R Hoare, does not work for concurrent programs and proposes a "generalized Hoare's logic" that takes concurrency into account. \\\\
In 1978 Hoares paper \textit{Communicating Sequential Processes} was published and with it, CSP was born. It have been widely used in many different types of work and have also been expanded since Hoare initially described it in 1978\cite{Abdallah2005}. The first version of CSP was a simple programming language that had quite a different syntax than todays CSP. In 1984, Brookes, Hoare and Roscoe published their continued work on CSP with the paper \textit{A Theory of Communicating Sequential Processes}\cite{Brookes1984}, and created the modern process algebra it is today. Only a few minor changes have been made to CSP since then, and they are described in Roscoe's \textit{The Theory and Practice of Concurrency}\cite{Roscoe1997}.Now, several different variations of CSP exists today which all specialize in different areas of formal descriptions.\\
A number of tools have been created in order to analyse, verify and understand systems written in CSP. Since CSP was mostly a blackboard language and difficult to use on larger scale, different types of machine-readable CSP syntaxes have been created over the years in order to make it easier to use CSP on a larger scale. Most of todays CSP tools use a version of machine-readble CSP called \cspm{} which was created by Scattergood\cite{Scattergood1998}. Scattergood created \cspm{} as a combination of the standard CSP algebra and a functional programming language which provided a better baseline for tools to work with CSP.\\
Here is a subset of the different CSP tools:
\begin{itemize}
\item One of the most known CSP tools is the Failure-Divergence Refinement tool (FDR), build by Formal Systems (Europe) Ltd., and is currently at version 4.2.3\cite{fdr}. FDR4 is a refinement checker and the newer version of FDR is able to run in parallel as well as do state compression in order to avoid a very large state space. FDR only work on finite-state processes.
\item ProBE (Process Behaviour Explorer)\cite{probe} is a tool to animate CSP in order to explore the state space of CSP processes. It can handle infinite state and is based on the same \cspm{} version as FDR4 is. ProBE was also created by Formal Systems (Europe) Ltd and ProBE is integrated into the current version of FDR4.
\item At Adelaide University, The Adelaide Refinement Checker (ARC)\cite{Parashkevov1996} was created as an automatic verification tool for CSP. It uses Ordered Binary Decision Diagrams (OBDDs) to represent the internal representation of data structures. This lessen the state explosion problem that other model checker tools have had. \todo{which language does it use?}
\item The ProB project\cite{ProB}\cite{Leuschel2003} was originally created as an animation and model checker tool for the B-Method\cite{Abrial1988} but it also supports other languages like Z and \cspm{}. Newer versions of ProB can do refinement checking of \cspm{} scripts but does not have the full functionality that FDR does. \todo{research this more}
\item J. Sun, Y.Liu, J.Dong et al. presented the Process Analysis Toolkit (PAT) in their 2009 paper\cite{Sun2009}. PAT is a CSP analysis tool that can perform Linear Temporal Logic (LTL) model checking, refinement checking and simulation of CSP processes. \todo{research this more}
\item CSP-Prover\cite{Isobe2005} is a theorem prover for CSP and based on the theorem prover Isabelle. It is an entirely different way to check programs than model checking. It attempts to prove some general results based on specific theory. It is better at proving general results where model checkers are better at proving combinatorial problems. \todo{make this more clear}
\end{itemize}
The programming language Occam\cite{Occam1995}, which was first released in 1983, is a concurrent programming language that builds on the CSP process algebra. Occam was continuouly in development during the years and the Kent Retargetable occam Compiler (KRoC) team at Kent University created the Occam-$\pi$\cite{UniveristyofKent} variant of the Occam programming language. It is a version that extends the ideas of CSP in the original Occam language but adding mobility features from pi-calculus. In the paper \textit{The symbiosis of concurrency and verification: teaching and case studies}\cite{Pedersen2018} Pedersen and Welch uses Occam-$\pi$ along with \cspm in order to reason about the logic behind \cspm and FDR. By using an executable language like Occam-$\pi$ which is based on the concurrency model of CSP it becomes easier to understand the logic of \cspm and thereby verify the program with FDR4.\\\\


SPIN\cite{spin} is a verification tool that uses process interactions to prove correctness for a system. The systems are described in the formal language \texttt{PROMELA}(PROcess MEta LAnguage)\cite{Holzmann1991} and the correctness properties are spcified in Linear Temporal Logic (LTL)\cite{Pnueli1977}. In the paper \textit{Reasoning About Infinite Computations}\cite{Vardi1994}, Vardi and Wolper showed that all LTL formulas can be translated into a B\"uchi automata which SPIN makes use of and thus converting the given LTL into a B\"uchi automaton. Spin performs verification on concurrent software and does not perform verification on hardware circuits. \\
Spin was developed at Bell Labs, starting in 1980. Gerard J. Holzmann gives an introduction to the theoretical foundations, the design and structure and examples of applications in the paper \textit{The model checker SPIN}\cite{Holzmann1997}. SPIN, as well as other model checker tools, have been build on the pioneering work on logic model checking by Clarke and Emerson\cite{Clarke1981}, as well as Sifakis and Queille\cite{Queille1982}. \todo{Should I add more info here? } Vardi and Wolper extended their work with an automata-theoretic approach to automatically verify programs\cite{Vardi1986}.\\\\
Another verification tool was developed as a collaboration between the Department of Information Technology at Uppsala University (UPP) in Sweden and the Department of Computer Science at Aalborg University (AAL) in Denmark. Larsen et al. first proposed the ideas for UPPAAL\cite{Larsen1995} in 1995 and further introduced it in the paper \textit{UPPAAL - a Tool Suite for Automatic Verifcation of Real-Time Systems}\cite{Bengtsson1995}.
UPPAAL is a verification tool for modelling, simulating and verifying real-time systems.
It is based on the theory of timed automata\cite{Hopcroft2001}\cite{Alur1990} and the typical systems to gain advantage of UPPAAL are systems where timing aspects are critical and where the communication goes through channels or shared variables.
As other model checkers, UPPAAL have a modelling language, wherein the system is specified, and a query language that is used to specify the properties to check against the system. The query language is a subset of CTL (computational tree logic) that work for real-time systems\cite{Henzinger1994} \cite{Larsen1995}. The model checking is done by checking the state-space by making a reachability analysis. The current version of UPPAAL is called UPPAAL2K and was released in 1999\cite{Amnell2001}. \\\\
In 1981, Edmund M. Clarke and E. Allen Emerson managed to combine temporal logic with the state-space exploration in order to provide the first automated model checking algorithm\cite{Clarke1981}. It was capable of proving properties of programs as well as producing counter examples.
In the mid 1980s it was shown how model checking could be applied to hardware verification. However, it quickly became clear that model checking on hardware was very limited due to the state-space explosion that occurs especially on hardware. \\
Randall Bryant from the CMU electrical engineering department invented ordered Binary decision diagrams (OBDDs). Later on, J. Burch, E. Clarke, K. McMillan et al.\cite{Burch1992} used OBDDs and created \textit{symbolic model checking} which represents the state space symbolically. The symbolic model checking can verify systems with an extremely large number of states and thus creating a solution to the problems of state space explosion.\\
Because of the state-space explosion problem and the increasing complexity of digital electronic circuits, there was a need to be able to model the timing and data flow of a ciruit with a certain amount of abstraction. This became Hardware Description Languages (HDL) \\\\
VHDL (VHSIC Hardware Description Language) was initially ordered by the United States Department of Defence in 1981 to help with the growing problem of hardware life cycles. It is based on the Ada programming language and have been the base Hardware Description language that was used to model hardware. In 1987 it became an IEEE standard, known as VHDL-87. After a major modification in 1993 it was known as VHDL-93. VHDL ...
% TODO: write something more
\\

Verilog was published by Gateway Design Automation in 1985 and along side VHDL are the two main HDL's used for modelling circuits. Cadence Design Systems received the rights to Verilog-XL which is the HDL simulator that would end up being the de-facto standard Verilog simulator.


However, VHDL and Verilog share many of the same limitations: neither is suitable for analog or mixed-signal circuit simulation; neither possesses language constructs to describe recursively-generated logic structures. Specialized HDLs (such as Confluence) were introduced with the explicit goal of fixing specific limitations of Verilog and VHDL, though none were ever intended to replace them. (From WIKI)
(From WIKI): Essential to HDL design is the ability to simulate HDL programs. Simulation allows an HDL description of a design (called a model) to pass design verification, an important milestone that validates the design's intended function (specification) against the code implementation in the HDL description. It also permits architectural exploration. The engineer can experiment with design choices by writing multiple variations of a base design, then comparing their behaviour in simulation. Thus, simulation is critical for successful HDL design.


% TODO:  Look at functional verification

% TODO: Look at Property Specification Language also look at SVA (two property languages that are derived from LTL) (used for Hardware)


% HDL include explicit notation for expressing concurrency as well as a notion of time.
% HDLs are used to write executable specifications for hardware.
% Because HDLs can be executed it gives the illusion of programming languages even though it is more of a specification language or modelling language.
% First HDLs in late 60's. C.Gordon Bell and Allan Newells text "Computer Structures" in 1971 - first to give a hdl with lasting effect.

% (from
% http://www.techdesignforums.com/practice/guides/formal-verification-guide/) "Equivalence checking has been used for more than a decade to check that RTL and gate-level descriptions of a design represent the same design"


%
% Take a look at Temporal logic model checking (As it is mentioned in the formal verification - evolution article)
% - Clarke et. al. CMU 1981
% - Sifakis et. al. Grenoble 1982
% and also look at
% Symbolic model checking
% McMillan 1991
% SMV

WRIGHT\cite{Allen1997}\cite{Allen1997a} % TODO: It would be worth to read more about this! They have done a bit of the same that I am to do in my thesis with auto generating \cspm}
is an architecture description language which was developed at Carnegie Mellon University. They can auto generate \cspm code from WRIGHT and from there they can confirm certain properties by using FDR. http://www.cs.cmu.edu/~able/wright/



Both theorem provers and model checkers have been, and are still, widely used for both software and hardware. There is a third form of formal verification that is also being used more often now. This is equivalence checking, which compares two models of a design and produces an outcome that either shows that they are equal or provides a counter-example to show when they disagree. It is beginning to become common practice for hardware designers to use equvalence checking to compare the design of an optimized digital design and an unoptimized digital design. This way it is possible for the designer to check that the optimizations did not change the functionality of the design.


%
\chapter{Approach}
%!TEX root = ../main.tex
% some explanation about what I am actually doing in this project.
%
% "Now I have shown what others have done with verification, now I need to be explicit about what I am going to do"

% Tegne et billede af måden jeg gerne vil løse det her problem på.


As explained in Chapter \ref{chap:related_work} several attempts have been made to translate programs written in CSP into a hardware description language. But even a good implementation of this type of system would require that the developer manually models the specification of the \cspm{} network, which can be very tedious especially for a novice.
The complexity of CSP is most likely leading to fewer developers utilising the functionalities and advantages of the CSP process algebra.\\

What I aime to achieve is to create a translation reversed from has have previously been done. I wish to provide a solution where the developer can model the network first and then, using the system, automatically generate the specification for that exact network. On top of this, I want to be able to formally verify specific properties of this specification model. This can provide valuable insights into the possible pitfalls of the hardware model that a standard test bench cannot provide.\\

In this thesis, I introduce the system TAPS, a transpiler that provides translation from hardware models to specification models while also introducing specific properties for verification within the generated specification model.\\

The language used to model the hardware is the SME Interpretation Language (SMEIL) and is based on the Synchronous Message Exchange (SME) model. SMEIL resembles a standard general-purpose programming language while still providing all the necessary elements of hardware modeling. The programming structure of SMEIL enables the traditional developer to model hardware models in a simple but efficient manner.\\

TAPS provides translation to the machine-readable version of the CSP process algebra, \cspm{}. The generated \cspm{} code will not only be equivalent to the SMEIL code, but it will also include assertion statements to formally verify properties within the hardware model. These properties can be verified with the \cspm{} refinement checker tool FDR4, described in Chapter \ref{chap:related_work}.


% TODO: Maybe I should write something about why I chose sme and cspm and FDR?

%
% \chapter{Background Technologies}
% \label{chap:background}
% %!TEX root = ../main.tex
In this chapter we briefly introduce SME as well as CSP and in more detail, we introduce SMEIL and \cspm along with the FDR4 tool and its cababilities.

\section{Synchronous Message Exchange}
Since SMEIL is based on the SME model, we give a brief introduction to SME to familiarize the reader with the SME model before introducing SMEIL.
\\

The development of SME was mainly driven by the need to provide a simple framework for programming a Field Programmable Gate
Array, FPGA, since FPGAs can achieve the same performance as a General Purpose Graphical Processing Units, GPGPU, but with much less energy consumption. GPGPUs have been extensively researched and different environments have been implemented to give programmers the posibility of utilizing the cababilities of it, but FPGAs are a far better choice when it comes to energy sensitive applications. The FPGA has not been the subject of as much research as the GPGPU, and when using FPGAs, the developer need to design an integrated circuit on the gate level, which is difficult and not common knowledge these days.
Unfortunately designing hardware is often a tedious task but with SME it becomes more accessable to design and implement hardware models.

SMEs main target is to give software developers a tool which provide the opportunity for the developer to program hardware but with a distance from the hardware details, so in a way, the development resembles the structures and semantics that is known from software development.
Thus the framework is based of a top-down approach rather than a bottom-up approach that is common in the current hardware frameworks.
% Global synchronicity is fundamental when modeling hardware, but CSP does not support this. Even though CSP does enforce strict synchrony between communicating processes, there is no such semantic beging enforced on a global level. % TODO: Do I want to include this?

SME was first introduced in 2014 and after several iterations~\cite{Vinter2014, Vinter2015, Skovhede} now presents as a programming model, a simulation library, and VHDL code generators. %TODO: Add reference!  Different reference in paper than in mendeley. Figure it out. The original idea was conceived following an attempt to create hardware descriptions from a vector processor model, modeled in PyCSP~\cite{bjorndalen2007pycsp},
% TODO: Figure out what the correct reference to this is.
a Communicating Sequential Processes (CSP)~\cite{hoare1978communicating} library for Python.\\

The work was initially presented in the paper \textit{BPU Simulator}~\cite{Rehr2013}. This paper only introduced a high abstraction level simulation and therefore the subject was explored more in detail in the master thesis project \textit{Generation of FPGA Hardware
Specifications from PyCSP Networks}~\cite{Skaarup14}. From the results of the master thesis it was clear that PyCSP could be used to model hardware however the need to enforce global synchrony to the circuit resulted in an explosion in the number of channels for controlling the progress and for simulating the clock, and even simple circuits would become overwhelmingly large.\\

The design approach of the master thesis was to implement a clock process that would drive the circuit. This meant that each process must read the clock signal and in order to avoid race conditions the system had to be implemented with a two-way clock, the so called \textit{tick} and \textit{tock} signals. Since deadlocks can happen in CSP it was important to implement deadlock prevention, which was done by adding channels with a single buffer element. This way, no processes would end up in a deadlock. Figure \ref{fig:sme:clock_latch} shows how a simple CSP network would be modelled in the synchronous PyCSP model. It is clear how trying to use PyCSP for modelling synchronous hardware would result in extremely large networks.
\begin{figure}[h!]
\centering
\includegraphics[width=10.0cm]{figures/clocked.pdf}
\caption{To enforce global synchrony on a simple reader-writer network in CSP, the complexity increase to include the clock and latch. Figure from \cite{Vinter2014}}
\label{fig:sme:clock_latch}
\end{figure}
\\\\
The advantage of using Python for this kind of processor simulation is the flexibility and simplicity of the language. It is easy to experiment with various ideas and versions but with the increased complexity of the added clock process and and clock-signal propagation, the advantages of Python seemed to diminish. Therefore the conclusion from the master thesis was that using PyCSP alone for building synchronous processor simulators was not feasible since the global clock was forced onto the CSP model. It was also concluded that external choice, which is a powerful and essential part of CSP, was not utilized in the globally synchronous approach. They did, however, conclude that several of other CSP concepts was fitting well with the hardware concepts, such as shared-nothing in CSP matches the structure of hardware communication.\\

After this attempt, it became clear that the structure of CSP was poorly suited for modeling clocked systems, and therefore it was decided to create the Synchronous Message Exchange framework, based on the CSP algebra. The idea was to only use the subset of the CSP algebra that provided beneficial functionality to hardware modeling which, most importantly, meant that external choice was omitted. However, the shared-nothing property of CSP showed to be very useful, since the network state could only be changed by process communication.
\\
% In SME the introduction of an implicit clock that eliminates the complexity caused by the model, described above, was imperative and is one of the key concepts hereof. The implicit clock removed the problems found by forcing the globally synchronous clock onto the CSP model. \\
In SME, a network is a combination of processes that are connected through buses. The processes communicate through a collection of signals in a bus, instead of CSP's synchronous rendezvous model, but retains the shared-nothing trait of CSP.
SME uses the term \texttt{bus} instead of \texttt{channel} to enforce the semantic correlation between the SME bus and a physical hardware signal bus.
The process communication is handled by a hidden clock which eliminates the complexity that arose from adding synchronicity to a CSP network. The combination of the hidden clock and the synchronous message passing between processes means that the SME model provides hardware-like signal propagation.

An SME clock cycle consists of three phases: it reads, executes, and writes as can be seen in Figure~\ref{fig:sme_process_flow}. The process is activated on the rising clock edge where it reads from the bus and it reads, executes and writes to the bus in one clock cycle. Just before the rising edge of the clock, all signals are propagated on all buses which means, that all communication happens simultaneously. Because of this structure, if a value is written by a process in cycle $i$, it is read by the receiving process in cycle $i+1$.

SME is able to detect read/write conflicts where multiple writes are performed to a single bus within the same clock cycle as well as reads from a signal that has not been written to in the previous clock-cycle.\\
All data, that are written to a bus can be logged for each clock cycle which means that these logs can be saved in a Comma-Separated Values, CSV, format and used for validating the VHDL implementation with a VHDL tool. This would eliminate the need to write seperate VHDL tests which will improve developer productivity immensely.

SME also supports clock-multipliers which means that a clocked network can have components inside the network being clocked at a different rate than its own clock. The rate can be an integer multiplier from its own clock. This means that all components are clocked relative to its parent network and therefore the clock will not need any extra coordination. It does however mean that the inner components can only fun faster than their parents.
% TODO: Do I really want this included?

Since an SME network is a network that can be represented as a graph, a diagram tool have also been implemented for SME. It utilizes the Graphviz tool to generate a visual graph of the network. This means that the SME networks can be generated and drawn as a graph which can help with debugging.
%TODO: Also, it this something I really want to include?


\begin{figure}[!ht]
  \centering
  \begin{tikzpicture}[auto]
    \node[mycircle, text width=2cm, shape=rectangle] (read) {Read};
    \node[mycircle, text width=2cm, shape=rectangle] (execute) [below=0.5cm of read] {Execute};
    \node[mycircle, text width=2cm, shape=rectangle] (write)  [below=0.5cm of execute] {Write};

    \draw [myarrow] (read) -- (execute);
    \draw [myarrow] (execute) -- (write);
    \draw [myarrow] (write) |-([shift={(5mm,-5mm)}]write.south east)-- ([shift={(5mm,5mm)}]read.north east)-| (read);
  \end{tikzpicture}
  \caption{SME process flow for one clock cycle.}
  \label{fig:sme_process_flow}
\end{figure}
Since SME is based on CSP, all SME models have a
corresponding CSP model, and because of this property, we are able to create a transpiler translating SME models to \cspm{}.
The SME model is currently implemented as libraries for the general-purpose languages C\#~\cite{Skovhede}, C++~\cite{asheim2015}, and Python~\cite{asheim2016vhdl}. The Python and C\# libraries both have code generators for VHDL as well.
\newpage
\subsection{SMEIL}
\label{SMEIL-section}
With the different SME implementations, a need arose for a common intermediate language. SME Implementation Language (SMEIL) was developed as a Domain Specific Language (DSL) for SME, usable both as an Intermediate Language, IL, and as an independent implementation language. It is accompanied with the implementation, \texttt{LIBSME}. %TODO: Find the reference in truls paper.
SMEIL has a C-like syntax with a specialized type system that makes hardware modeling simple. In spite of its simplicity, SMEIL still provides hardware-specific functionality that is more difficult to create with general-purpose languages.

Programs written in SMEIL are run by using the libsme library. This can be done either by using the command line utility or through the provided API.
The different methods of using SMEIL are:\\
\begin{itemize}
    \item \textbf{Pure SMEIL simulation:} A pure SMEIL program is a network which do not depend on outside influence. This means that the network  generates the data itself and the only communication are between the processes defined in the network.
    \item \textbf{Co-simulation of SMEIL:} The original intent with SMEIL was to create an intermediate language which could be used together with the general-purpose language implementations of SME, which is called co-simulation. With co-simulation, a test bench can be generated as well as VHDL code.
    \item \textbf{Direct code generation:} A SMEIL program which does not need to simulate before generating VHDL. In this case all types are constrained and all information needed for the code generation are in place. Because the simulation step is not executed, a VHDL test bench is also not automatically generated, since this happens in the simulation step.
\end{itemize}
The base entity of an SMEIL program is the module which corresponds to a file. The module consists of the actual SMEIL program as well as import statements. SMEIL programs can be seperated into modules which can then be importet into other SMEIL programs, creating a library-like structure and reusable components.\\

The SME model supports both synchronous and asynchronous processes. A synchronous process are run during every clock-cycle and an asynchronous process are only run when receiving a signal on the input bus, but unfortunately SMEIL does not currently support asynchronous processes. %TODO: Ask truls if it is implemented in other sme implementations.
\\

The two fundamental components of an SMEIL program is \texttt{process} and \texttt{network}. The process consists of variable and bus definitions, as well as the statements that are evaluated once for each clock cycle. The purpose of the \texttt{network} declaration is to define the relations between each entity in the program. A small example of process and network syntax can be seen in Listing~\ref{lst:smeil_small_syntax_example} and a further introduction to the language can be found in Section \ref{chap:analysis}.\\
\begin{listing}
\begin{minted}[escapeinside=||, mathescape=true]{smeil_lexer.py:SMEILLexer -x}
proc addone (in inbus)
    bus outbus {
        val: int;
    };
{
    outbus.val = inbus.val + 1;
}

    |$\vdots$|

network net() {
    instance a of addone(b.outbus);
    instance b of ..
    |$\vdots$|
}
\end{minted}
\caption{Small example of process and network syntax in SMEIL.}
\label{lst:smeil_small_syntax_example}
\end{listing}

\subsubsection{SMEIL type system}
The SMEIL language is stongly, statically typed and has a simple type system that is checked at compile-time. The purpose of SMEIL is to be able to model hardware, and since hardware is static, it was important to create a type system that was cabable of enforcing as many static invariants as possible. Therefore only between signed and unsigned integers is type coercion performed in SMEIL, which also means that only booleans can be used in conditionals.
It is easy to transform a statically typed language to a dynamically typed language, but not the other way around, therefore SMEIL will be simple to translate to various different target languages.

The SMEIL type system differs from standard general-purpose languages mainly on the support for bit-precise types. In general-purpose languages it is typical targeted a CPU which consists of fixed-width registers which typically means that it can not work with data smaller than a byte. This is, however, an important part of modeling hardware to be able to define the exact widths of the wires in order to optimise the hardware implementation. SMEIL supports unlimited-size integers as well as integers constained to a specific bit-length. SMEIL also supports booleans, double and single precision floating point numbers as well as strings. Currenly, floating-point numbers are not supported in the SMEIL hardware-translations. %TODO Venter på svar omkring hvad det her egentligt betyder? Og tjek lige at jeg ikke også skriver om det i analyse.
Arrays of a fixed length of the types, mentioned above, are also supported.

In SMEIL buses are represented as channel names and their types, and since processes or networks can accept buses passed as parameters it is important to ensure that no process or network is instantiated with a bus that does not contain the expected channels.
In the same way, it is essential to ensure that the directionality of the bus is enforced. When buses are passed as parameters, they are explicitly declared as either input or output bus. However, when defining a bus within a process it is not possible to define the directionality of the bus. Therefore the bus are defined as input or output based on their first use.
The type system of SMEIL is enforcing a set of rules that define how two buses are unified and the directionality of the bus, which ensures that problems such as the ones described above, does not happen.
All types of declarations in SMEIL are private exept bus declarations. As mentioned above, they are used for establishing communication between processes and therefore needs to be a part of the public interface of a process or network.

\subsubsection{Simulating SMEIL programs}
When translating SMEIL to hardware models, it is necessary to have all types in the program constrained to a specific width. However, as mentioned before, SMEIL does support unlimited size integers. Since it can be difficult to define an optimal width when writing the SME model, there was a need to provide both the support of the unlimited size integers while still being able to translate to hardware descriptions.
Therefore libsme provides a way to re-type the SME network based on the values that where observed during the simulation of the program.
During the simulation, the observed minimun and maximum value for all variables and channel are captured and saved along with the original value.
After the simulation, these minimum and maximum values are then converted into SMEIL types large enough to hold the range.
The new SMEIL program with the updated types and observed ranges are then passed through the type checker, in order to ensure that all constraints from the original program is still respected.
% An example of this can be seen in Figure \ref{fig:smeil_restricted_types_and_ranges} where a unlimited size integer is re-typed to an integer of fixed size after the simulation. In Figure \ref{fig:invalid} an example of a violation can be seen. Here the value c is contrained to \texttt{i10} but if the channel \texttt{b.chan} observe values that are 16-bit long then the restrictions on the c variable are violated.

% \begin{figure}
%   \centering
%   \begin{subfigure}[t]{0.25\linewidth}
%       \begin{listing}
%       \begin{minted}[escapeinside=||, mathescape=true]{smeil_lexer.py:SMEILLexer -x}
%           proc A ()
%            bus b {
%             chan: ?{\bfseries\underline{int}}?;
%            };
%            var c: i10;
%           {
%            c = b.chan;
%           }
%       \end{minted}
%       \end{listing}
%     \caption{Unconstrained types.}
%     \label{fig:oktype}
%   \end{subfigure}
%   \begin{subfigure}[t]{0.35\linewidth}
%   \begin{listing}
%   \begin{minted}[escapeinside=||, mathescape=true]{smeil_lexer.py:SMEILLexer -x}
%       proc A ()
%           bus b {
%               chan: ?{\bfseries\underline{i6 range 0 to 29}}?;
%           };
%           var c: i10;
%       {
%        c = b.chan;
%       }
%   \end{minted}
%   \end{listing}
%     \caption{Valid.}
%     \label{fig:non-violated}
%   \end{subfigure}
%   \begin{subfigure}[t]{0.35\linewidth}
%   \begin{listing}
%   \begin{minted}[escapeinside=||, mathescape=true]{smeil_lexer.py:SMEILLexer -x}
% proc A ()
% bus b {
% chan: ?{\bfseries\underline{i16 range 0 to 30717}}?;
% };
% var c: i10;
% {
% c = b.chan;
% }
%   \end{minted}
%   \end{listing}
%     \caption{Invalid.}
%     \label{fig:violated}
%   \end{subfigure}
%   \caption{Shows a process entering the simulator with an unconstrained type (a)
%     and examples of two possible resulting programs (b, c). The type changing
%     between the examples is underlined.}
%   \label{fig:simtyping}
% \end{figure}
% NOTE: THis figure does not compile. It is taken from Truls article and I should change the captions and also mention where I got it from.

% In Figure \ref{fig:smeil_restricted_types_and_ranges} the two last programs, showed in Figure x and y, will be the result of the re-typing of the program after simulation. That is, the programs resulting from the simulation of the program in Figure z.


\subsubsection{Co-simulation}
Often when modeling hardware in Hardware Description Languages (HDLs) like VHDL or Verilog, code for testing and verifying are often written in the same language as the design itself. Unfortunately, the HDLs often does not have the functionality for generating proper simulation input. Using general-purpose languages for testing hardware models are useful since the range of available libraries are much larger.
Therefore the SMEIL simulator provides a simple language-independent API which enables SME implementations written for general-purpose languages to communicate with SME networks written in SMEIL, so-called co-simulation.
The big advantage of the SMEIL approach to co-simulation is that SME is used on both sides of the co-simulation and therefore both sides acts as a single entity.
The PySME %TODO: Reference
library have been extended to support co-simulation with SMEIL, thereby providing the posibility of writing networks in PySME which can interact with SME networks written in SMEIL. Currently, this extension have only been implemented in the PySME implementation, but is expected to be implemented in the other SME implementations as well.\\
Another functionality of the libsme compiler is, when simulating the SMEIL network, the compiler can record a trace of all communication between processes in the network. This trace file can then be used as a source for the VHDL test bench which can be used to verify the generated VHDL code.

% Some parts of the SMEIL grammar is not implemented in SMEIL yet and therefore we are also not supporting these.
In Figure~\ref{fig:smeil_transpiler} the SMEIL transpiler structure can be seen.

\begin{figure}[!ht]
  \centering
  \begin{tikzpicture}[auto]
    \node[mycircle, minimum size=1.75cm, align=center, text width=1.75cm, font=\footnotesize]    (smeil)                                       {SMEIL};
    \node[myrectangle, text width=1.5cm, minimum height=1.0cm, inner sep=5pt, inner ysep=5pt] (csme)  [above left=-0.25cm and 1.5cm of smeil] {C\#SME};
    \node[myrectangle, text width=1.5cm, minimum height=1.0cm, inner sep=5pt, inner ysep=5pt] (pysme) [below left=-0.25cm and 1.5cm of smeil] {PySME};
    \node[myrectangle, text width=1.5cm, minimum height=1.0cm, inner sep=5pt, inner ysep=5pt] (vhdl)  [right=1.0cm of smeil]                {VHDL};

    \draw[myarrow] (csme)  -- (smeil);
    \draw[myarrow] (pysme) -- (smeil);
    \draw[myarrow] (smeil) -- (vhdl);
  \end{tikzpicture}
  \caption{SMEIL transpiler structure.}
  \label{fig:smeil_transpiler}
\end{figure}




\newpage
\section{CSP}
Communicating Sequential Processes (CSP) \todo{reference} is a process algebra that provides a way to express interaction between processes of concurrent systems.
CSP had a lot of influence in the design of the programming language Occam as well as the Go programming languages and several other. As described in Chapter \ref{chap:related-work}, it was first itroduced in 1976 by C.A.R Hoare but only later on did it develop into a process algebra. CSP is still the subject of research and due to the tools available today, it is increasingly becomming a more accessable tool for the interested user.

CSP provides the posibility of describing systems in terms of message passing communication between independently operating processes. The \textit{sequential} part of the CSP name is not exactly applicable anymore, since the current version of CSP allows not only sequential processes, but also processes consisting of parallel compositions of other primitive processes.
By using message passing between processes the language avoids certain problems that arise with the use of e.g shared variables.
With the use of the process algebra of CSP, it is possible to describe complex parallel structures with a few simple elements of the algebra.

The two fundamental parts of CSP are \textit{Events} and \textit{Primitive processes}. \textit{Events} are the communication between processes. They are instantanious and can be simple names, like \textit{clock} or compositioned names, like \textit{add.kill} or it could be input/output events, like \textit{add ? x} or \textit{hours ! y}. %TODO: This section is a bit too much like the wikipedia version, this should be rewritten
The \textit{pimitive processes} are consisting of processes which have basic behaviour such as \textit{STOP} or \textit{SKIP}. The \textit{STOP} process is a process that simply does nothing, not even terminate, and it is also called deadlock. The \textit{SKIP} process is the process that terminates successfully.


% ------
%TODO: Figure out if this should be here or somewhere else in the report
CSPs different algebraic operators are used to define the different relationships between processes in CSP.
\begin{itemize}
    \item \textbf{Prefix:} The prefix operator link an event and a process.
    $a \rightarrow P$ defines a process which will communicate $a$ and then behaves like the proces $P$. The process will wait indefinitely until communicating $a$.
    \item \textbf{Deterministic choice:} Also called external choice is the operator which offers the environment the choice between two different processes. $(a \rightarrow P) \square (b \rightarrow Q)$ in this example the environment choose to either perform event $a$ and then behave as the process $P$ or perform event $b$ and then behave as the process $Q$.
    \item \textbf{Nondeterministic choice:} Also called internal choice, defines the choice between two processes but is not affected by the envionment to make the decision.
    $(a \rightarrow P) \sqcap (b \rightarrow Q)$ can perform event $a$ or $b$ and then behave as the corresponding process, but it does not have to accept either. It is only required to accept one if the environment offers both of them.
    \item \textbf{Interleaving:} This operator represents concurrent activity between two independent processes. $ P  |||  Q$ defines a process behaving like $P$ and $Q$ simultaneously.
     \item \textbf{Generalised parallel:} This operator represents two processes concurrent activity but where the processes are required to syncronise on the set of events, defined in the operator. $ P  |[\{ a \}]|  Q$ defines the process where $P$ and $Q$ must syncronise on event $a$ before the event can occur. All events not defined within the synchronisation can happen at any given time.
     \item \textbf{Hide:} This operator represents a process which performs any event from the set defined, but the event is hidden and becomming an internal event, a \textit{tau}.
\end{itemize}

% -----
% TODO: Written a lot like the wikipedia page. This require a rewrite!
There are three denotanional semantics models in CSP, which are the \textit{traces} model, the \textit{failures} model, and the \textit{failures divergences} model. The \textit{trace} model define a process expression as a set of sequential events, traces, that the process can perform. $traces(STOP) = {<>}$ means that the trace of the process \textit{STOP} is the empty set, since the process performs no events. $traces(a \rightarrow b \rightarrow STOP = \{<>, <a>, <a,b>\})$ means that the trace of the process defined above can either be no events performed, the event $a$ performed or both event $a$ and then event $b$.\\
The failures model is similar to the trace model, but consists of a set of refusals, which are the set of events that a process can refuse to perform. A failure consists of both a trace and a set of refusals, which identifies the events that the process can refuse when it has executed the trace. $failures((a \rightarrow STOP) \square (b \rightarrow Q) = \{(<>, \emptyset), (<a>, \{a,b\}), (<b>, \{a.b\})\})$\\
The last model, the \textit{failures divergences} is en extension of the failures model. The model is defined as a pair... %TODO: Explain better, the wikipedia site is weird and I dont really understand it. I should understand what I write!

\subsection{\cspm{}}
%CSPm was devised by Bryan Scattergood as a machine-readable dialect of CSP  - se the paper \textit{The Semantics and Implementation of Machine-Readable CSP}\\\\
\cspm{} is a formal language that combines CSP with a functional programming language in order to make it easier for the programmer to model the systems and then use the code on tools that can animate, verify or similar.


\subsection{FDR4}
%" FDR2 is often described as a model checker, but is technically a refinement checker, in that it converts two CSP process expressions into Labelled Transition Systems (LTSs), and then determines whether one of the processes is a refinement of the other within some specified semantic model (traces, failures, or failures/divergence)" (from Wikipedia - se paper \textit{Model-checking CSP - af Roscoe} \\\\
FDR (Failures Divergence Refinement) tool is a refinement checker for

In the paper \textit{A primer on model checking}\cite{Ben-ari2010} Mordechai Ben-Ari explains a concurrent problem that he had used for many years, to teach his students about concurrency. ... % TODO: write this when I have read the article again


We not only want to transpile SMEIL to \cspm{}, we also want to be able to verify different properties in \cspm{} in order to prove correctness. Today, there exists several tools for formal verification, both in academia and in the industry. One of the currently most favored tools is the Failures-Divergences Refinement tool (FDR4). This tool is a CSP refinement checker that can analyze programs written in the machine-readable version of CSP; \cspm{}.
It provides a parallel refinement-checking engine that can scale up linearly with the number of cores. This means that it can handle processes with a large number of states in a reasonable time. FDR4 can handle several different types of assertions, deadlocks being the most used. However, due to the structure of SMEIL, we use FDR4 in a different way than is typical. Since the SME model cannot have cyclic-wait we have no need to verify the system in this manner.

For our current implementation of the transpiler, we can assert the ranges of the channel inputs, for example, we can automatically assert that the observed ranges, provided by the SMEIL simulation, and the possible input on the \cspm{} channels are not conflicting.
In hardware, we would typically want to verify that the communication on a bus does not exceed a certain range or that the sum of multiple signals does not exceed a specific value. A bus might be able to carry other data than needed, and being able to model a circuit that can assert that the bus never carries other data than expected, is of great value.
\\

CSP was not initially developed for hardware modeling, and therefore it is not evident how to handle the clock cycle, which is an essential part of hardware modeling. When we transpile the SME network into \cspm{} the SMEIL simulation have provided the ranges of all values from the simulation and therefore all clock cycles. This means that when FDR4 asserts a property it asserts on all possible communication combinations for all the simulated clock cycles. Therefore, even though we are transpiling from an SME model, where the clock is crucial, we can simply translate ``one-to-one" from the SMEIL program and still get an accurate assertion on the properties.


%
% \chapter{Analysis}
% \label{chap:analysis}
% % This section should contain information about the problem, but not the solution. I am analysing the problem and explaining what will be problematic when translating from SMEIL to CSPm.

% Der er tre dele i SMEIL jeg har interesse i:
% Three components:
% - Behavioral description - hvad den enkelte funktion gør. det oversættes ret nemt. Jeg bekymre mig ikke om variable og sådan nogle ting. Det er allerede gjort før det gøres til SMEIL, og dem kan jeg genbruge i min code generation.  Funktionel indhold af processer/ opførsel af den enkelte process - det er rimelig nemt oversat direkte til CSPm. Her kan man beskrive hvis der er nogle sproglige udfordringer, fx hvis CSPm ikke har loops eller lign.
% - Strukturel information: hvilke proceser er der og hvilke buser er de limet sammen med. hvordan hænger tingene sammen. Det er også forholdsvis nemt fordi FDR har processer på samme måde som SMEIL har. de har en historisk afhængighed ift. SME og CSP.
% - Opserverede værdier - "known limits". Meta information i SMEIL, og det skal oversættes til faktiske processer med faktiske semantic i FDR der også er en del af topologien. (Det er essensen for mig).

% Vigtigt at have de tre dele adskilt i afsnittet. Brug figuren fra side 107 i bujo til at henvise til at jeg har en Magic 8-ball (Mit system) og jeg skal fra SMEIL til FDR og henvise til at det er det flow jeg har brug for.


% Jeg skal ikke fortælle hvordan jeg har gjort det men analyse af problemet og ikke af løsningen. Jeg skal blot beskrive problemstillingerne.


% I SMEIL har jeg et prædefineret konsistent navnerum. Jeg skal ikke lave en symboltabel. Jeg arver direkte fra SMEIL til CSPm. Så her kan jeg snakke om at hvis jeg gjorde det på en anden måde ville jeg ikke kunne gøre det sådan. Det skal jeg skrive om her.

\section{Transpiling SMEIL to \cspm{}} \label{sec:transpiling}
When transpiling from SMEIL to \cspm{} one of the difficult components was to find a generalized method for transpiling, that could be generalized to most problems. We have worked on separation of concerns in order to simplify, but also have a greater chance of being able to match more SMEIL programs.

An SMEIL process consists of bus and variable declarations, the statements to be run per clock cycle as well as the outgoing communication from the process.  Channels within an SMEIL bus can be translated directly to \cspm{} channels. It is, however, important to give channel names that will be unique since a \cspm{} channel is global as opposed to the local channel within each SMEIL bus. An example of an SMEIL process, where the process structure is evident, can be seen in Listing~\ref{lst:range_smeil} and the corresponding \cspm{} code in Listing~\ref{lst:channel_range_cspm}.

In order to keep the outwards communication and the arithmetic statements together within each process in \cspm{}, we generate \cspm{} processes with a \texttt{let within} statement. The arithmetic statements go into the \texttt{let} section and the communications go into the \texttt{within} section. This gives us the possibility of separating the outwards communication and arithmetic statements while still keeping them within the same \cspm{} process. In Listing~\ref{lst:channel_range_cspm}, an example of the \texttt{let within} statement can be seen in lines 7-14. This structure will work as a general translation structure from SMEIL processes to \cspm{} processes.

The network in an SMEIL program is the crucial part which ties all the processes and communication together. We can standardize the network generation by creating a two-step communication part. Instead of having the actual processes receive the incoming data, they receive the data by their process parameter. The process parameter is then set by the network process which receives the communication from the channels and provides the process with the communicated value.
This ensures that we can generate the processes easily without having to traverse the network in the SMEIL program beforehand to find out which channel provides input for which process. An example of this is shown in Listing~\ref{lst:cspm} in the appendix on lines 61 to 66.

% %
% \chapter{Designing TAPS}
% \label{chap:design}
% %!TEX root = ../main.tex
% Description of the actual solutions
% TODO: Write something here
The goal of automatic translation is to be able to create a general solution which can fit different types of problems and therefore it is necessary to generalise the different aspects of the translation and find a solution that fits all.
\section{Designing TAPS}
In this section we describe all the design decisions included in creating TAPS so that it addresses all the challenges of the translation between SMEIL and \cspm{}. The section is divided in a similar way as Chapter \ref{chap:analysis}, and we will go through all the challenges described there.
\subsection{Behavioral}
%% - Behavioral description - hvad den enkelte funktion gør. det oversættes ret nemt. Jeg bekymre mig ikke om variable og sådan nogle ting. Det er allerede gjort før det gøres til SMEIL, og dem kan jeg genbruge i min code generation.  Funktionel indhold af processer/ opførsel af den enkelte process - det er rimelig nemt oversat direkte til CSPm. Her kan man beskrive hvis der er nogle sproglige udfordringer, fx hvis CSPm ikke har loops eller lign.

% NOTE: This version of the design chapter is only for the non-clocked version. I have to start somewhere and it seems silly not to talk about this solution at all. This way I will also be able to dicuss the differences i guess. However, it will require more writing and thereby lenghten the overall thesis. But I think I should be able to reuse most of the non-cloked version and then add some sections about the clocked structure and maybe add a bit about how the processes are changed.
\subsubsection{Processes}
In order to translate the SMEIL process to a \cspm{} process we had to create a general process structure in \cspm{}. We know that the SME model enforce that each process reads, calculates and writes, in that order, for each clock cycle, so we needed to create a \cspm{} structure that could support this. First of all we wished to have one \cspm{} process per SMEIL process, since it would simplify the translation and a simpler solution typically results in a less errornous solution.

What first comes to mind in \cspm{} when we want to read and then write, is a very simple process structure using prefix and communication operators.
\begin{minted}[linenos=false, escapeinside=||, mathescape=true]{cspm_lexer.py:CSPmLexer -x}
Proc(x) = c ? x -> d ! x-> SKIP
\end{minted}
This is the simplest \cspm{} process that match the SME model, however with this structure it is not possible to include all the possible calculations that SMEIL support. It turned out that the tricky part of translating a general SMEIL process to \cspm{} was to be able to include the calculations properly.

By using the \texttt{let within} structure in \cspm{} we are able to keep the communication together with the arithmetics in one \cspm{} process while keeping a simple structure. The \cspm{} process does not do the actual read, but instead received the value as a parameter, then all arithmetics are performed inside the \texttt{let} clause while the writing will be put in the \texttt{within} part.
In Listing~\ref{lst:channel_range_cspm}, an example of the \texttt{let within} statement can be seen. This structure will work as a general translation structure from SMEIL processes to \cspm{} processes. Since we know that all reads and calculations must be done before writing this structure should always work for a well-structured SMEIL process.

\begin{listing}
\begin{minted}[escapeinside=||, mathescape=true]{cspm_lexer.py:CSPmLexer -x}
channel seconds_out_first_digit : {0..7}
channel seconds_out_second_digit : {0..15}

    |$\vdots$|

Seconds(seconds_in) =
let
    seconds = seconds_in % 60
    seconds_first_temp = seconds / 10
    seconds_second_temp = seconds % 10
within
    seconds_out_first_digit ! seconds_first_temp ->
    seconds_out_second_digit ! seconds_second_temp ->
    SKIP
\end{minted}
\caption{Example of the \texttt{Seconds} process from the generated \cspm{} code in the seven segment display example. See full example in Listing~\ref{lst:cspm} in the appendix.}
\label{lst:channel_range_cspm}
\end{listing}
% TODO: Maybe this example should be shortened a bit (we simply want to show the let within statements). Also change the label to match what it is

Since \cspm{} does not need us to declare variables beforehand, we can ignore the variable declarations in the SMEIL process and simply translate all the arithmetic statements, with the variables, directly. The input bus from the SMEIL process is transformed to a variable containing a value instead but this does not change the structure of the arithmetic statements.
% Constants are simply defined in the \cspm program, seperate from the process. Since the SMEIL programs must be well-formed, we know that only the processes that define the constant will use it, and therefore it is not a problem that
% NOTE Constants are currently not implemented in TAPS
% TODO: How to handle variables with predefined values

The types and ranges of variables and constants are not translated, and unless we need the information for verification, it is completely ignored. We can do this because we know the SMEIL program is well-formed and since the variables are not used for verification, it does not matter what types and values the SMEIL program expects of these.

All assignments that are not communication are simply translated directly into \cspm{} without much change, however if the assignment is to or from a bus, then TAPS have to differentiate and handle these assignments differently, which will be explained later in this section.
If-satements are translated into the \cspm{} version of an if statement, however since \cspm{} does not support \texttt{elif}, the if statements are nested to form these expressions. This quickly becomes very complex and hard to read, but since it is auto generated, it is not a problem to create.
% NOTE: If-statements are currently not implemented in TAPS

Traces and assertions are, as explained in Chapter \ref{chap:analysis} not useful in the \cspm{} program and therefore we either throw them away or keep them as comments for the sake of the overview of the code. Currently TAPS throw them away, but it would be a simple task to change this and add them as comments.
% NOTE: Assertions are not curently implemented in TAPS. Trace is.

Most expressions of SMEIL can be directly translated, like \texttt{+}, \texttt{-}, \texttt{/} and \texttt{\%} etc. however there are a few differences in the presendence
% TODO: Be exactly sure about this and write more.

% TODO: Write about the expressions not possible in \cspm{}


% TODO: Write more when I have more implemented for the processes (like arrays and stuff)
------------------------------------------------------------------------------\\
------------------------------------------------------------------------------\\
------------------------------------------------------------------------------\\
------------------------------------------------------------------------------\\

\subsubsection{Generating data}
%%% Generator processes
It is important to make sure that TAPS can handle the different kind of data generation possible in SMEIL. A data generator process in SMEIL does not read any input value, no matter if it is through a process parameter or by using the channels hierarchical name. \\

In \cspm it is necessary to define the space of data that FDR4 search through. It is a possibility to create a process in \cspm with the same functionality as the SMEIL data generator process. However, in that case, it would be necessary to syncronise the data process with the processes receiving the data, otherwise FDR4 would simply evaluate all values within the defined range of the channels, making the data generator process obsolete. This extra syncronisation will increase the complexity of the system and it might also increase the runtime of the verification, since the \cspm{} network would include more states.

\begin{figure}
    \centering
    \begin{tikzpicture}
       \node[main node] (1) {\small \texttt{data}};
       \node[main node] (2) [right = 4cm of 1] {\texttt{calc}};
       \draw[fill] (0.7,0) circle [radius=0.07];

       \path[draw,thick, ->]
       (1) edge node {} (2);

       \node[align=center, below, text width=1.7cm] at (3.27,0.83){\footnotesize\texttt{channel c : \{0..100\}}};
       \node[align=center, below, text width=1.7cm] at (1.3,0){\footnotesize\texttt{output o : \{0..10\}}};
   \end{tikzpicture}
    \caption{A \cspm{} network with two processes. The output \texttt{o} of the process \texttt{data} is within the range 0 through 10, and the channel \texttt{c} is defined for the range 0 through 100.}
    \label{fig:csp_data_generator_process}
\end{figure}
An figure of this can be seen in Figure \ref{fig:csp_data_generator_process}. Here, the channel \texttt{c} between the \texttt{data} process and the \texttt{calc} process is defined for range of \texttt{\{0..100\}}. If the two processes are not syncronised on this channel, the two processes do not have to agree on communication and therefore the search space for FDR4 includes all the values from 0 through 100. This means that if we just have the channel \texttt{c} as an input channel for the process \texttt{calc}, then we get the same result, since the processes are not syncronised on the channel.
However, if the processes are syncronised, FDR4 will still allocate all 100 posibilities, but it only continues the search on the values actually communicated on the channel. In this case, the only values FDR4 would actually continue searching will be from 0 through 10.
When adding a data generator process as well as syncronisation, the data generator process would be able to define specific data for the search space and that might prove to be an advantage when interested in more complex data.

So when generating data in \cspm{} from a SMEIL data generator process, the two possibilities are either to define the data in \cspm by a single channel or by a data generator process and syncronisation, just like in SMEIL.
Currently TAPS only support one of the two options, which are to generate an input channel from the SMEIL network.
% NOTE: Only one type of data generator process are supported right now.
Listing \ref{lst:clock_data_generation_example_cspm} is the translated version of the SMEIL network from Listing \ref{lst:clock_data_generation_example_smeil}. How TAPS translate the values of the channels are described later in this section. 
\begin{listing}
\begin{minted}[escapeinside=||, mathescape=true]{cspm_lexer.py:CSPmLexer -x}
channel clock : ...
channel minutes_output_val : ...

Minutes(input) =
let
    from_clock = input / 60
within
    output_val ! from_clock ->
    SKIP
\end{minted}
\caption{Example of the translated \texttt{Minutes} process defined in Listing \ref{lst:clock_data_generation_example_smeil}.}
\label{lst:channel_range_cspm}
\end{listing}


In the case where the SMEIL network does not have a data generator process but are instantiated with constants or internal values, the \cspm processes would also have to be instantiated with values as their parameters.


%%% Generator processes - input bus with const declared
If there is an input bus in all processes, the job of locating the data or generator process becomes more difficult. In Figure %TODO: Create a figure with an example of a process like AddOne, where there is an input bus an a const. Also add how it should be translated in a subfigure next to it.
an example of a process with an input bus, but where the process also works as the data instantiation point. In this case the constant that are communicated to the process from the network instance declaration indicates what value the network should start with and the process then communicates it on its output bus and then acts as an integral part of the communication of the network for the rest of the clock cycles. This means that the process has a task to perform after the instatiation of the data and therefore the transpiler cannot simply translate it into a \cspm channel, because that would mean that the task the process performs after the data initialization, will not be possible afterwards. One way of solving this problem would be for the programmer to avoid these "two-task" processes, however that is also not a neat solution and it will be a hindrance for the programmer to structure the network in a certain way in order to verify the code later on. However, in the case of the \texttt{AddOne} example, the easy solution would simply be for the programmer to create an extra process which initialized the data and then communicated it once to the process which then would loop for each clock cycle.
%TODO: Figure out how I can handle this in a better way than simply ask the programmer not to create processes like this, and write about it.

%% Generator processes - More than one process
If there are several processes in the SMEIL program that acts as a data generator process, the transpiler will not handle the processes different than if there were one data generator process. If the processes consists only of generating data, then the processes will be translated into a \cspm channel for each process, and the comunication to the rest of the network will still be kept intact since all the communicated are specified by the SMEIL network structure.
%TODO: Find an example with more than one data generation process.
If the processes are not simple data generator processes, then the transpiler will have to handle it differently %TODO: Figure out how I can handle this in a better way than simply ask the programmer not to create processes like this, and write about it. - Same as with one process but if it becomes more complicated with more than one, then write it here!






% Buses
As explained above, the input for the process is recieved as a process parameter in \cspm{} and therefore TAPS will have to recognise the input parameter. Since the bus parameters for a SMEIL process is the bus name, the process itself must reference the specific channel within the bus. This is applied whether it is reads or writes.
This means that TAPS can recognise communication simply by the structure of the assignment. If one of the elements in an assignment, right or left, contains a dot, then we can assume that this is communication.






\subsection{Structural}
% - Strukturel information: hvilke proceser er der og hvilke buser er de limet sammen med. hvordan hænger tingene sammen. Det er også forholdsvis nemt fordi FDR har processer på samme måde som SMEIL har. de har en historisk afhængighed ift. SME og CSP.

% TODO: Write about two networks and how it will be translated to a network and therefore we should be able to have them seperated easily and the communication is what combines them, and that is the same in cspm. TODO: But ask Truls if this would even make sense. To have to networks.

% TODO: Write about how one process can be defined in SMEIL several times and how it does not matter, since the process will be defined the same in CSPm but that the network will simply be generated twice with two different names and inputs or similar. (Does that also work if it is different output?)

% (moved from analysis: )
%  % None of these two posssible usecases are currently implemented in TAPS, and thus the keyword \texttt{exposed} will cause an error in the transpiler. %TODO: Will it cause an error? And should I add more info to this?



% %%%% Generating the network
We can standardize the network generation by creating a two-step communication part. Instead of having the actual processes receive the incoming data, they receive the data by their process parameter. The process parameter is then set by the network process which receives the communication from the channels and provides the process with the communicated value.
This ensures that we can generate the processes easily without having to traverse the network in the SMEIL program beforehand to find out which channel provides input for which process. An example of this is shown in Listing~\ref{lst:cspm} in the appendix on lines 61 to 66.

% %%%% Instances and two networks.
% TODO: Add an example showing two different networks and how it translates.
When translating the instances of a SMEIL network, the transpiler does not need to keep the instances together in a certain way. Each instance will be translated into a seperate \cspm network and even though the networks might have communication in common, they do not need to be connected in \cspm. If the networks/instances have communication in common, it will be handled by the communication in \cspm. When translating from an instance into a network in \cspm it is simply to map the communication from the correct channels and to the correct channels in \cspm. If the process is connected to a monitor process this is also in the network that the monitor process is added to the communication it will be listening in on.
In a SMEIL program it is possible to have several networks in one program. There is not much of a point in doing so, since it will result in the same as having all instances in one network %TODO: Am I 100\% sure that two networks is the same as one? ask Truls
, but never the less, if there is two networks it will not matter in the \cspm translation. One SMEIL instance represents one \cspm network, and all these networks or instances are combined via communications on channels or buses. The only thing we are interested in when translating the SMEIL instances are the communication defined in each instance. And that is what creates the network. This means that we can generate the \cspm networks without looking at what SMEIL network that specific SMEIL instance came from, because the important thing, the communication, is defined in the instance and therefore all data about how the SMEIL network is combined together we still keep intact even though we translate the instances without considering the rest of that SMEIL network.
% %%%% Instances with the same process twice
% TODO: Add an example showing one process added twice and hwo it translates.
When defining instances in the SMEIL network, it is possible to define the same process several times and have it receive different input or send different output or constants. When translating this to a \cspm network, the process itself will be generated as all other processes, but simply the network will be two seperate network, as if it was two SMEIL instances calling two different SMEIL processes. The only difference between the two networks are that the input channel will be different, if that is the difference between the two SMEIL instances.



% %%%% Channels %%%%%
When generating the channels that represents bus channels in SMEIL, each SMEIL channel will be generated into an \cspm channel.
% TODO: write more here above. Why can we do this. Add something about ranges.
% TODO: What if I need to translate a bus with letters, or something else? what kind of channel does that become?

% %%%% Channel names
The naming will be created by concatinating the channel name, bus name and process name along with underscore in order to generate human readable code. It would also be possible simply to generate a unique string, which might give more security than using a concatinated version, but in this case we decided to make it easier for humans to read and understand the generated code, since the system is still in a "new state".
% %%%% Calling the channel in the bus
If the bus is generated in the network, then the naming will be the channel, bus and network name instead.
% TODO: add an example of naming in SMEIL and then in CSPm. Both with process and network defined buses
In the generated \cspm code, the reference to the each channel, whether it is defined in a process or in a network declaration, will be the specific name generated from the channel, bus and process/network name, as mentioned above. This means that all calls using the syntax \texttt{bus.channel} in SMEIL will be translated to \texttt{bus\_channel} in \cspm.
% %%%% Bus only defined as parameter
If the bus is only defined as a parameter in the process, the process will still have to call the bus channels in order to receive or send data on the bus. This is up to the programmer to make sure that the channel name used, are the channel names that the bus have been defined with, no matter where in the program the bus is defined. When generating the \cspm code, the calls to the channels will be the same, no matter where the buses was defined in the original SMEIL program. The channels will all be defined as global channels and the references will be the same as mentioned above (maybe add reference)
% TODO: Maybe add a refernece to a figure here.
% %%%% Bus used as input bus with more that one channel
If a bus is used as input bus in a process, the process will have to call the specific channel of the bus in order to access the data, comunicated on that specific bus. If the channel have more that one channel, the method of calling does not change, and it is up to the programmer to make sure that the channel called, are actually a channel within that specific input bus.



% channels from SMEIL to CSPm
Channels within an SMEIL bus can be translated directly to \cspm{} channels. Since a SMEIL bus consist of channels, the translation is quite simple. It is, however, important to give channel names that will be unique since a \cspm{} channel is global as opposed to the local channel within each SMEIL bus. Because there are several different ways to define a bus in SMEIL %TODO: see the analysis chapter,
the translation will have to recognize the different types and generate the \cspm channels no matter how they where defined.


\subsection{Meta and verification}
% - Opserverede værdier - "known limits". Meta information i SMEIL, og det skal oversættes til faktiske processer med faktiske semantic i FDR der også er en del af topologien. (Det er essensen for mig).

% %%%% Monitor processes %%%%%
When creating the assertions, we decided to create separate assert functions to keep the code structure clean. We know that for each \cspm{} channel there must be an assertion, except for the input channel.
% TODO: Only in the case where we assert channel ranges.. maybe add something about that.
Consequently, we create a \textit{monitor} process for each channel and its only job is to listen in on the channel communication and assert the values communicated there. The monitor process is a process that we add specifically for asserting legal communication values in FDR4 and it does not affect the original SME network.
In Figure~\ref{fig:assertion_process} the outline of this kind of structure can be seen and we expect that this structure can be used for several different types of problems and thereby ensure a cleaner code structure.

% The monitor process asserts the observed values of the \cspm{} channels and in Listing~\ref{lst:monitor_range_cspm} the two monitor processes for the Seconds \texttt{time} process can be seen. The values used for these statements are the observed values from the SMEIL simulation, as can be seen at the end of lines 2 and 3 in Listing~\ref{lst:range_smeil}. In Listing~\ref{lst:monitor_range_cspm} the ranges are used to assert that the only values communicated on the channels are within 0 and 5, and 0 and 9 respectively.

\begin{figure}[!ht]
  \centering
  \begin{tikzpicture}[auto]
    \node[mycircle] (P) at (-1.5, 0.0) {$P$};
    \node[mycircle] (Q) at ( 2.5, 0.0) {$Q$};
    \node[mycircle, shape=rectangle] (M) at ( 0.5, 1.5) {$M$};

    \node[draw, shape=circle, inner sep=0pt, minimum size=5pt] (m) at (0.5, 0.0) {};


    \draw (M) -- (P -| M) [black!50];
    \draw [myarrow] (P) -- (Q);
  \end{tikzpicture}
  \caption{The monitor process \textit{M} listens in on the communication between \textit{P} and \textit{Q} in order to assert the communicated values.}
  \label{fig:assertion_process}
\end{figure}

% \begin{listing}
% \begin{minted}[escapeinside=||, mathescape=true]{cspm_lexer.py:CSPmLexer -x}
% Seconds_out_first_digit_monitor(c) =
%     c ? x -> if 0 <= x and x <= 5 then SKIP else STOP
% Seconds_out_second_digit_monitor(c) =
%     c ? x -> if 0 <= x and x <= 9 then SKIP else STOP
% \end{minted}
% \caption{Example of the \texttt{Seconds} monitor processes from the generated \cspm{} code in the seven segment display example. See full example in Listing~\ref{lst:cspm} in the appendix.}
% \label{lst:monitor_range_cspm}
% \end{listing}

% %%%% Monitor processes: More that one monitor process for a channel
It is possible to have as many monitor processes as needed in the network. Since they dont change the functionality of the system, they can be added without problems. In both \cspm and SMEIL a channel has an input and an output, but the output is not for a specific process. all processes can acces the data on a bus in SMEIL, as long as the network describes the communication, and the same in \cspm.
% %%%% Monitor processes: Monitor observed ranges

% %%%% Monitor processes: monitor something else that ranges





% It is important to mention that the FDR version of the SMEIL program are represented as one clock cycle and therefore we do not have to handle implicit clock cycle issues. we can just translate one-to-one, because FDR models one clock cycle and the input represents all possible input in one clock cycle.


\section{Clock cycle problem}
%!TEX root = ../main.tex
After developing the initial version of TAPS, I realised that the solution was not broad enough in terms of what type of systems it could verify. The type of networks that it not possible to verify in the initial version of TAPS are network which keeps internal states between clock cycles. In the seven-segment example, no internal results had any influence on the result of other clock cycle results and therefor it could be correctly translated using the structures described in the previous chapters. \\

The initial idea was to avoid modeling a global synchronous clock in \cspm{} because, as described in Chapter \ref{chap:background}, the results from the master's thesis \textit{Generation of FPGA Hardware
Specifications from PyCSP Networks}~\cite{Skaarup14} by E. Skaarup and A. Frisch established how much the complexity of the network would increase when trying to model this. After creating the original system I wanted to model the \texttt{addone} example from \cite{smeil} but of course, because of the cyclic structure it does not fit into the structure of the original TAPS system. In this chapter I will introduce the \texttt{addone} example and the approach for extending TAPS with clocked systems.



% (From design)
% \section{Clock cycle problem}
% % %!TEX root = ../main.tex
After developing the initial version of TAPS, I realised that the solution was not broad enough in terms of what type of systems it could verify. The type of networks that it not possible to verify in the initial version of TAPS are network which keeps internal states between clock cycles. In the seven-segment example, no internal results had any influence on the result of other clock cycle results and therefor it could be correctly translated using the structures described in the previous chapters. \\

The initial idea was to avoid modeling a global synchronous clock in \cspm{} because, as described in Chapter \ref{chap:background}, the results from the master's thesis \textit{Generation of FPGA Hardware
Specifications from PyCSP Networks}~\cite{Skaarup14} by E. Skaarup and A. Frisch established how much the complexity of the network would increase when trying to model this. After creating the original system I wanted to model the \texttt{addone} example from \cite{smeil} but of course, because of the cyclic structure it does not fit into the structure of the original TAPS system. In this chapter I will introduce the \texttt{addone} example and the approach for extending TAPS with clocked systems.



% (From design)
% \section{Clock cycle problem}
% % \input{chapters/clock_cycle_problem}
%
% CSP was not initially developed for hardware modeling, and therefore it is not evident how to handle the clock cycle, which is an essential part of hardware modeling. When we transpile the SME network into \cspm{}, the SMEIL simulation have provided the ranges of all values from the simulation and therefore all clock cycles. This means that when FDR4 asserts a property it asserts on all possible communication combinations for all the simulated clock cycles. Therefore, even though we are transpiling from an SME model, where the clock is crucial, we can simply translate ``one-to-one" from the SMEIL program and still get an accurate assertion on the properties.
%
% % It is important to mention that the FDR version of the SMEIL program are represented as one clock cycle and therefore we do not have to handle implicit clock cycle issues. we can just translate one-to-one, because FDR models one clock cycle and the input represents all possible input in one clock cycle.
%
%

%




\section{Initial Addone Example}
% TODO: remember to remove the addone example in the analysis chapter. Both the original one and the other ones that are using the names. Use something from seven segment example instead.
The \texttt{addone} network is a simple network that consists of two processes communicating with one another. The \texttt{add} process receives a value and increments it by a value passes a as a constant parameter. The \texttt{id} process only receives the value and passes it along on its output bus.
The network is a two process loop and it is therefore essential that there is a way to initialise the loop as well as terminating it properly.
A figure of the network can be seen in Listing \ref{fig:addone_unclocked}.\\

\begin{figure}
    \centering
    \begin{tikzpicture}
       \node[main node, text width=.5cm] (1) {\small \texttt{add}};
       \node[main node, text width=.5cm] (2) [right = 3cm of 1] {\texttt{id}};
       % \draw[fill] (0.7,0) circle [radius=0.07];

       \path[draw,thick, ->, bend right=30]
       (1) edge node {} (2);
       \path[draw,thick, ->, bend right=30]
       (2) edge node {} (1);

       \node[align=center, below, text width=1.7cm] at (2.2,1.3){\footnotesize\texttt{channel d}};
       \node[align=center, below, text width=1.7cm] at (2.2,-0.9){\footnotesize\texttt{channel c}};
   \end{tikzpicture}
    \caption{The \texttt{addone} network. The network have two proceses which communicate to each other on the two buses.}
    \label{fig:addone_unclocked}
\end{figure}
The network is simple to model in SMEIL as can be seen in Listing \ref{lst:addone_smeil_example} but when translated with TAPS the generated \cspm{} code did not model the network correctly. When translating the \texttt{addone} network with the original version of TAPS, the generated \cspm{} code will only be able to simulate one clock cycle. As previously explained%TODO: Make sure I introduce this somewhere before this
this initial version of TAPS will verify all possible input values for the system, but what the \texttt{addone} network shows, is that to enlarge the set of problems possible to verify with TAPS, it is necessary to extend TAPS to support these types of networks. \\

The solution to this problem is to extend the translation to model a global synchronous structure in \cspm{} instead of the simple model that is the initial version of TAPS. As E. Skaarup and A. Frisch already learned, enforcing a global synchronous model onto CSP is not simple and even simple network become very complex. The advantage I have, compared to the previous attempt to model global synchronicity with CSP, is that TAPS will auto generate the \cspm{} code and therefore the complexity and size of the correspoding \cspm{} network is not an issue in terms of creating the network. The extra complexity might, however, become a problem when verifying with FDR4. It is possible that the added complexity requires more of FDR4 and that the size of problems verifiable with FDR4, becomes smaller which this solution.
\begin{listing}
\begin{minted}{smeil_lexer.py:SMEILLexer -x}
proc add (in input, const constant)
    bus output {val: u4 = 0 range 0 to 10;};
{
    output.val = input.val + constant;
}


proc id (in input)
    var from_add: u4 range 0 to 10;
    bus output {val: u4 = 0 range 0 to 10;};
{
    from_add = input.val;
    trace("Wrote value {}", input.val);
    output.val = from_add;
}


network addone_network ()
{
    instance id of id(add.output);
    instance add of add(id.output, constant: 1);
}
\end{minted}
\caption{The simulated SMEIL network \texttt{addone\_network} with two processes. The example is similar to the Addone example in \cite{smeil}.}
\label{lst:addone_smeil_example}
\end{listing}
\section{Clocked Networks}
As can be seen in the examples of the seven-segment \cspm{} code %TODO: Where can they see the code?
, no proceses are recursive. All processes run once and, unless errors occured, behaves as the \texttt{SKIP} process afterwards. This is part of the reason why the \texttt{addone} example cannot represent more than one clock cycle. To be able to verify more than one clock cycle it is essential that the processes are recursive. As explained in the csp background in Chapter \ref{chap:background}, recursive processes are simply processes which instead of behaving like the \texttt{SKIP} process, behaves like itself. An example of this can be seen in Listing \ref{lst:cspm_recursion}.
\begin{listing}
\begin{minted}{cspm_lexer.py:CSPmLexer -x}
Init = d ! 1 -> A(1)
A(x) = d ! x -> A(x+1)
\end{minted}
\caption{Example of the a recursive \cspm{} process which is initialised by the \texttt{Init} process.}
\label{lst:cspm_recursion}
\end{listing}

In theory an SME processes never stops running, but when simulating the SMEIL network it is of course not possible to simulate endless runtime. Therefore the developer indicates the number of clock cycles to simulate and the results should be seen as a snapshot of the process runtime. \\

In Listing \ref{lst:cspm_recursion} the process \texttt{A} performs an endless loop with no change to terminate. As mentioned in the previous, it is also essential to have a limited range of values for FDR4 to verify to avoid running out of space and if the example in Listing \ref{lst:cspm_recursion} was verified with FDR4, it would eventually run out of space. It is therefore crucial to model a structure that can drive the network and which can ensure the process terminate at the specified time. This is done with the \texttt{Clock} process. The clock process drives the network for a specific number of clock cycles and then terminates, which enforce all other processes to do the same, which will be explained shortly. \\


It, of course, is still necessary for a new version of TAPS to model a \cspm{} network that reflects the SME model and therefore it must adhere to the SME model structure. To model the global synchronicity in \cspm{} it is necessary to enforce a synchronising event where all processes synchronise before continuing. This synchronicity can be emulated by having a \texttt{sync} channel to emulate the rising and falling clock signal. All clocked processes in the network will be synchronised with the \texttt{sync} channel and, as previously introduced, when two processes are synhronised on a channel they must agree on communication. Therefore, all clocked processes must agree to synchronise before any process can continue.


----------------------------------------------------
----------------------------------------------------
----------------------------------------------------


(clock stuff)
initialised with a value and syncronise on the \texttt{clock} channel, which all other processes does as well. When the specified number of clock cycles has passed, the \texttt{Clock} process stops clocking and instead behaves as \texttt{SKIP}. This means that all other processes won't be able to syncronise on the \texttt{clock} channel anymore and therefore they will then instead behave as \texttt{SKIP} and that way the system terminates as planned.\\




The \texttt{clock} channel is used as a two-way clock synchronisation, where the same channel is used for syncronising up as well as down. Thus all processes syncronise before they read and before they write. The \texttt{Clock} process then needs to syncronise twice before incrementing its counter. \\

% (from verification in design)
% TODO: Maybe I should do it with the FD model, but in this case it does not really make sense, since all processes end after one iterations (they all SKIP). But it is worth mentioning it in the new system, because there it becomes relevant

% TODO: What happens if an SMEIL channel have been defined with an initial value?



% TODO: What if several processes write to the same output channel? How to handle the monitor process then?



% TODO: This might just belong in the new version!
% When the smaller process monitor networks have been created TAPS will then be able to synhronise other smaller monitor networks where there is shared communication. TAPS controls that all communication are handled within this new network. (Maybe it makes sense to generate smaller network which can then be synhronised. I mean where each smaller network is a process with a name, so it does not become so large nested.. )
% % TODO: Figure out how to make sure that all data are synchronised. It might be something about needing to have a linked list or something to keep all the data together. both stuff defined as parameter and communication defined by using the formal names.






\section{Clocked Processes}
% % TODO: Write that the read of a process does not make sense in the network and that we move it to be internal inside the process.

% %%% Generator processes - clock cycles
% %TODO: Write something here?
%
%



\section{The Bounds Problem}

When trying to verify that this system terminates as expected in FDR4, we came across an error while FDR4 was compiling the program.
FDR4 is complaining that a value, the system is trying to send, is not a part of the set of values defined for the channel.
The channels, used for communicating the value between the processes and the buffers, are defined for a specific range and since the 'Add' process has to read, calculate and write before it can terminate it will always write a value out that have been incremented with one.\\
For instance, if the channels were defined with the range \{0..5\} and the \texttt{Clock} process would stop after 10 clock cycles. This would mean that on the last clock cycle, before terminating, the 'Add' process would write a 6 onto the channel, which of course is not possible since the channel is defined only for the range \{0..5\}, so FDR4 complains about this.\\
However, what we experienced was that when the ranges of the channels were set to a larger number than the system would ever reach, within the defined number of clock cycles, FDR4 would still fail with the same reason.
This caused some problems since we were not able to verify the system and we were not interested in FDR4 trying to verify a communication that would never occur within the network.
Since we are syncronising the processes on the events in the channels, it seems odd that FDR4 still considers events which should not be possible to reach.  \\\\
After some time working with the problem and trying to understand the reason for FDR4s error message, we found that if we simply add a guard or an if-then-else statement that tests the value to be written, FDR4 will gladly verify the system and when using Probe on the network, it is clear that FDR4 does not consider the trace with the wrong values. \\

So the suggested solution, or fix, to this problem, is to add an if-then-else before all writes in a program. The statement then tests the value to be written against the max value of the range of the channel and if the value is not within the range, then the process behaves as the \texttt{SKIP} process, otherwise, it continues with write.
\begin{minted}[escapeinside=&&, mathescape=true]{cspm_lexer.py:CSPmLexer -x}
channel c : {0..20}

&$\vdots$&

    if (i+1) > 20  -- Check the upper limit of the channel
        then SKIP  -- SKIP if the value is above
        else (c_r ! (i+1) -> Add(i))) -- Otherwise write and continue
\end{minted}
In this case, it is only necessary to add an upper limit test, since the network only increments, but as a general rule, it would be necessary to test for both upper and lower bounds.


% The problem ocurred several times with different versions of the solution. Also, Ohm had the same problem with the Commstime problem. It makes sense why FDR4 wants to check, and maybe it is a way to save verification time: if it checks things in parallel or something.
% However it is not a good solution that we have to put in a upper/lower bound check that is actually never relevant.
% Of course it might be good, in any case, since the programmer then do not need to worry. On the other hand it might make the program look like it is terminating properly when it is actually failing because it tries to write a value that is not allowed. This case might happen, and if all other processes do not notice it and then also SKIP according to their specifications, then the verification passes even though it might be wrong.
% A solution might be to ensure that all processes synchronise before skipping, because then the problem (i think) would not occur, since the failing process would simply skip, then the other processes cannot skip because they have not syncrhonised yet. however, I am not sure this is possible since it might be that they either syncronise or they skip.



\section{Introducing Buffers}

For the 'Add' and 'Id' processes to comply with these states, they would have to read first, then the calculate phase, in which the 'Id' process does nothing, and then they would write the result onto a channel. A problem occurs, since they both have to read first, no one can read because no processes have written anything yet. To solve this, we implement two buffers which for each clock cycle reads the output that the process writes and then writes the value to a channel. Thus the buffer structure is the reverse of a 'normal' process since it will write and then read in a clock cycle. \\
If we give the buffers an initial value, they can begin the clock cycle writing the value which the 'Add' and 'Id' processes can read and thereby they will comply with the SME model structure.
The buffers will be instantiated will a 'dummy' value which is also how it is typically done in hardware. The dummy value is simply to indicate that the system should ignore the first clock cycle and then continue with the systems actual values. \\
Each process is also instantiated with a value, which is then used instead of the dummy value from the buffer process. After this initial cycle, the process loop will continue and the communication will hold according to the description of the network explained above.\\\\

Since the processes all must read, calculate and then write, the buffer processes, as mentioned before, must behave the opposite way. This means that the last write the processes make before they \texttt{SKIP} will be left in the buffers since there are no processes to read the value from the channels. This means that the buffers must be able to either write a value or \texttt{SKIP}.\\


\section{Clocked Addone Example}
\begin{figure}
\centering
\begin{tikzpicture}
   \node[main node, text width=.5cm] (add) {\small \texttt{add}};
   \node[main node, text width=.5cm] (id) [right = 4cm of add] {\texttt{id}};
   \node[mythinsquare] (bufd) at (2.7, 1.7) {$Buf_d$};
   \node[mythinsquare] (bufc) at (2.7, -1.5) {$Buf_c$};
   % \draw[fill] (0.7,0) circle [radius=0.07];

   \path[draw,thick, ->, bend right=25]
   (add) edge node {} (bufc);
   \path[draw,thick, ->, bend right=25]
   (bufc) edge node {} (id);


   \path[draw,thick, ->, bend right=25]
   (id) edge node {} (bufd);
   \path[draw,thick, ->, bend right=25]
   (bufd) edge node {} (add);


   \node[align=center, below, font=\scriptsize] at (1.5,1.3){\texttt{d\_write}};
   \node[align=center, below, font=\scriptsize] at (3.9,1.3){\texttt{d\_read}};
   \node[align=center, below, font=\scriptsize] at (1.5,-0.8){\texttt{c\_write}};
   \node[align=center, below, font=\scriptsize] at (3.9,-0.8){\texttt{c\_read}};
   % \node[align=center, below, text width=1.7cm] at (2.3,-0.9){\footnotesize\texttt{channel c}};
\end{tikzpicture}
\caption{The clocked \texttt{addone} network. The network have two proceses and two buffers which ensure the global synchronicity.}
\label{fig:addone_clocked}
\end{figure}




% Kenneths version, which I believe is how the SME model works, is having a process or bus in the middle og all steps. By using a dependency graph (Explain more?) it is possible to see which processes communicate to witch processes and, more importantly, in which order. For each communication step (or maybe for each communication) a process/bus will receive all writes. In SME a process can write several times to the same channel but only the last one before the clock signal will be written, the others are just overwritten. Since we have the dependency graph, we also know which processes we need communication from, and when the process have written all it has to write, then it sends a ready signal to the "bus" process, which then waits for all the ready signals (because it knows how many it should get. And if it is one process/bus pr. communication then it only needs one of course.). When all ready signals are in, the bus-process change behaviour and it is not writing instead of reading. It writes all possible values out and the processes that are supposed to receive the values (which we know from the dependency graph) will receive the values. And the processes then need to send a ready signal back to the bus process to let it know that it have read all it needed. When all ready signals are received, the bus process when change behaviour again and can read values once again.
% All these steps are intermediate steps within one clock cycle. So at the "end" of the dependency graph, the step looks similar to the others, but it is registered as the clock and the next clock cycle begins. In principal, all these steps could be the clock, since the step is the same, but a step is simply chosen to be the clock, based on the dependency graph.
% By treating the communication like this within a clock cycle, the values can propagate through the network and the internal state of the processes are also kept. The original TAPS version could only verify all input for a system, but if the system was internally affected by values from a previous clock cycle, then the system could not verify it. It is not a problem in the seven segment example, since no values are dependent on previous values. But the Addone network do depend on what happened in the last clock cycle.
% With this solution it is possible to verify a specific number of clock cycles.
%



%
% CSP was not initially developed for hardware modeling, and therefore it is not evident how to handle the clock cycle, which is an essential part of hardware modeling. When we transpile the SME network into \cspm{}, the SMEIL simulation have provided the ranges of all values from the simulation and therefore all clock cycles. This means that when FDR4 asserts a property it asserts on all possible communication combinations for all the simulated clock cycles. Therefore, even though we are transpiling from an SME model, where the clock is crucial, we can simply translate ``one-to-one" from the SMEIL program and still get an accurate assertion on the properties.
%
% % It is important to mention that the FDR version of the SMEIL program are represented as one clock cycle and therefore we do not have to handle implicit clock cycle issues. we can just translate one-to-one, because FDR models one clock cycle and the input represents all possible input in one clock cycle.
%
%

%




\section{Initial Addone Example}
% TODO: remember to remove the addone example in the analysis chapter. Both the original one and the other ones that are using the names. Use something from seven segment example instead.
The \texttt{addone} network is a simple network that consists of two processes communicating with one another. The \texttt{add} process receives a value and increments it by a value passes a as a constant parameter. The \texttt{id} process only receives the value and passes it along on its output bus.
The network is a two process loop and it is therefore essential that there is a way to initialise the loop as well as terminating it properly.
A figure of the network can be seen in Listing \ref{fig:addone_unclocked}.\\

\begin{figure}
    \centering
    \begin{tikzpicture}
       \node[main node, text width=.5cm] (1) {\small \texttt{add}};
       \node[main node, text width=.5cm] (2) [right = 3cm of 1] {\texttt{id}};
       % \draw[fill] (0.7,0) circle [radius=0.07];

       \path[draw,thick, ->, bend right=30]
       (1) edge node {} (2);
       \path[draw,thick, ->, bend right=30]
       (2) edge node {} (1);

       \node[align=center, below, text width=1.7cm] at (2.2,1.3){\footnotesize\texttt{channel d}};
       \node[align=center, below, text width=1.7cm] at (2.2,-0.9){\footnotesize\texttt{channel c}};
   \end{tikzpicture}
    \caption{The \texttt{addone} network. The network have two proceses which communicate to each other on the two buses.}
    \label{fig:addone_unclocked}
\end{figure}
The network is simple to model in SMEIL as can be seen in Listing \ref{lst:addone_smeil_example} but when translated with TAPS the generated \cspm{} code did not model the network correctly. When translating the \texttt{addone} network with the original version of TAPS, the generated \cspm{} code will only be able to simulate one clock cycle. As previously explained%TODO: Make sure I introduce this somewhere before this
this initial version of TAPS will verify all possible input values for the system, but what the \texttt{addone} network shows, is that to enlarge the set of problems possible to verify with TAPS, it is necessary to extend TAPS to support these types of networks. \\

The solution to this problem is to extend the translation to model a global synchronous structure in \cspm{} instead of the simple model that is the initial version of TAPS. As E. Skaarup and A. Frisch already learned, enforcing a global synchronous model onto CSP is not simple and even simple network become very complex. The advantage I have, compared to the previous attempt to model global synchronicity with CSP, is that TAPS will auto generate the \cspm{} code and therefore the complexity and size of the correspoding \cspm{} network is not an issue in terms of creating the network. The extra complexity might, however, become a problem when verifying with FDR4. It is possible that the added complexity requires more of FDR4 and that the size of problems verifiable with FDR4, becomes smaller which this solution.
\begin{listing}
\begin{minted}{smeil_lexer.py:SMEILLexer -x}
proc add (in input, const constant)
    bus output {val: u4 = 0 range 0 to 10;};
{
    output.val = input.val + constant;
}


proc id (in input)
    var from_add: u4 range 0 to 10;
    bus output {val: u4 = 0 range 0 to 10;};
{
    from_add = input.val;
    trace("Wrote value {}", input.val);
    output.val = from_add;
}


network addone_network ()
{
    instance id of id(add.output);
    instance add of add(id.output, constant: 1);
}
\end{minted}
\caption{The simulated SMEIL network \texttt{addone\_network} with two processes. The example is similar to the Addone example in \cite{smeil}.}
\label{lst:addone_smeil_example}
\end{listing}
\section{Clocked Networks}
As can be seen in the examples of the seven-segment \cspm{} code %TODO: Where can they see the code?
, no proceses are recursive. All processes run once and, unless errors occured, behaves as the \texttt{SKIP} process afterwards. This is part of the reason why the \texttt{addone} example cannot represent more than one clock cycle. To be able to verify more than one clock cycle it is essential that the processes are recursive. As explained in the csp background in Chapter \ref{chap:background}, recursive processes are simply processes which instead of behaving like the \texttt{SKIP} process, behaves like itself. An example of this can be seen in Listing \ref{lst:cspm_recursion}.
\begin{listing}
\begin{minted}{cspm_lexer.py:CSPmLexer -x}
Init = d ! 1 -> A(1)
A(x) = d ! x -> A(x+1)
\end{minted}
\caption{Example of the a recursive \cspm{} process which is initialised by the \texttt{Init} process.}
\label{lst:cspm_recursion}
\end{listing}

In theory an SME processes never stops running, but when simulating the SMEIL network it is of course not possible to simulate endless runtime. Therefore the developer indicates the number of clock cycles to simulate and the results should be seen as a snapshot of the process runtime. \\

In Listing \ref{lst:cspm_recursion} the process \texttt{A} performs an endless loop with no change to terminate. As mentioned in the previous, it is also essential to have a limited range of values for FDR4 to verify to avoid running out of space and if the example in Listing \ref{lst:cspm_recursion} was verified with FDR4, it would eventually run out of space. It is therefore crucial to model a structure that can drive the network and which can ensure the process terminate at the specified time. This is done with the \texttt{Clock} process. The clock process drives the network for a specific number of clock cycles and then terminates, which enforce all other processes to do the same, which will be explained shortly. \\


It, of course, is still necessary for a new version of TAPS to model a \cspm{} network that reflects the SME model and therefore it must adhere to the SME model structure. To model the global synchronicity in \cspm{} it is necessary to enforce a synchronising event where all processes synchronise before continuing. This synchronicity can be emulated by having a \texttt{sync} channel to emulate the rising and falling clock signal. All clocked processes in the network will be synchronised with the \texttt{sync} channel and, as previously introduced, when two processes are synhronised on a channel they must agree on communication. Therefore, all clocked processes must agree to synchronise before any process can continue.


----------------------------------------------------
----------------------------------------------------
----------------------------------------------------


(clock stuff)
initialised with a value and syncronise on the \texttt{clock} channel, which all other processes does as well. When the specified number of clock cycles has passed, the \texttt{Clock} process stops clocking and instead behaves as \texttt{SKIP}. This means that all other processes won't be able to syncronise on the \texttt{clock} channel anymore and therefore they will then instead behave as \texttt{SKIP} and that way the system terminates as planned.\\




The \texttt{clock} channel is used as a two-way clock synchronisation, where the same channel is used for syncronising up as well as down. Thus all processes syncronise before they read and before they write. The \texttt{Clock} process then needs to syncronise twice before incrementing its counter. \\

% (from verification in design)
% TODO: Maybe I should do it with the FD model, but in this case it does not really make sense, since all processes end after one iterations (they all SKIP). But it is worth mentioning it in the new system, because there it becomes relevant

% TODO: What happens if an SMEIL channel have been defined with an initial value?



% TODO: What if several processes write to the same output channel? How to handle the monitor process then?



% TODO: This might just belong in the new version!
% When the smaller process monitor networks have been created TAPS will then be able to synhronise other smaller monitor networks where there is shared communication. TAPS controls that all communication are handled within this new network. (Maybe it makes sense to generate smaller network which can then be synhronised. I mean where each smaller network is a process with a name, so it does not become so large nested.. )
% % TODO: Figure out how to make sure that all data are synchronised. It might be something about needing to have a linked list or something to keep all the data together. both stuff defined as parameter and communication defined by using the formal names.






\section{Clocked Processes}
% % TODO: Write that the read of a process does not make sense in the network and that we move it to be internal inside the process.

% %%% Generator processes - clock cycles
% %TODO: Write something here?
%
%



\section{The Bounds Problem}

When trying to verify that this system terminates as expected in FDR4, we came across an error while FDR4 was compiling the program.
FDR4 is complaining that a value, the system is trying to send, is not a part of the set of values defined for the channel.
The channels, used for communicating the value between the processes and the buffers, are defined for a specific range and since the 'Add' process has to read, calculate and write before it can terminate it will always write a value out that have been incremented with one.\\
For instance, if the channels were defined with the range \{0..5\} and the \texttt{Clock} process would stop after 10 clock cycles. This would mean that on the last clock cycle, before terminating, the 'Add' process would write a 6 onto the channel, which of course is not possible since the channel is defined only for the range \{0..5\}, so FDR4 complains about this.\\
However, what we experienced was that when the ranges of the channels were set to a larger number than the system would ever reach, within the defined number of clock cycles, FDR4 would still fail with the same reason.
This caused some problems since we were not able to verify the system and we were not interested in FDR4 trying to verify a communication that would never occur within the network.
Since we are syncronising the processes on the events in the channels, it seems odd that FDR4 still considers events which should not be possible to reach.  \\\\
After some time working with the problem and trying to understand the reason for FDR4s error message, we found that if we simply add a guard or an if-then-else statement that tests the value to be written, FDR4 will gladly verify the system and when using Probe on the network, it is clear that FDR4 does not consider the trace with the wrong values. \\

So the suggested solution, or fix, to this problem, is to add an if-then-else before all writes in a program. The statement then tests the value to be written against the max value of the range of the channel and if the value is not within the range, then the process behaves as the \texttt{SKIP} process, otherwise, it continues with write.
\begin{minted}[escapeinside=&&, mathescape=true]{cspm_lexer.py:CSPmLexer -x}
channel c : {0..20}

&$\vdots$&

    if (i+1) > 20  -- Check the upper limit of the channel
        then SKIP  -- SKIP if the value is above
        else (c_r ! (i+1) -> Add(i))) -- Otherwise write and continue
\end{minted}
In this case, it is only necessary to add an upper limit test, since the network only increments, but as a general rule, it would be necessary to test for both upper and lower bounds.


% The problem ocurred several times with different versions of the solution. Also, Ohm had the same problem with the Commstime problem. It makes sense why FDR4 wants to check, and maybe it is a way to save verification time: if it checks things in parallel or something.
% However it is not a good solution that we have to put in a upper/lower bound check that is actually never relevant.
% Of course it might be good, in any case, since the programmer then do not need to worry. On the other hand it might make the program look like it is terminating properly when it is actually failing because it tries to write a value that is not allowed. This case might happen, and if all other processes do not notice it and then also SKIP according to their specifications, then the verification passes even though it might be wrong.
% A solution might be to ensure that all processes synchronise before skipping, because then the problem (i think) would not occur, since the failing process would simply skip, then the other processes cannot skip because they have not syncrhonised yet. however, I am not sure this is possible since it might be that they either syncronise or they skip.



\section{Introducing Buffers}

For the 'Add' and 'Id' processes to comply with these states, they would have to read first, then the calculate phase, in which the 'Id' process does nothing, and then they would write the result onto a channel. A problem occurs, since they both have to read first, no one can read because no processes have written anything yet. To solve this, we implement two buffers which for each clock cycle reads the output that the process writes and then writes the value to a channel. Thus the buffer structure is the reverse of a 'normal' process since it will write and then read in a clock cycle. \\
If we give the buffers an initial value, they can begin the clock cycle writing the value which the 'Add' and 'Id' processes can read and thereby they will comply with the SME model structure.
The buffers will be instantiated will a 'dummy' value which is also how it is typically done in hardware. The dummy value is simply to indicate that the system should ignore the first clock cycle and then continue with the systems actual values. \\
Each process is also instantiated with a value, which is then used instead of the dummy value from the buffer process. After this initial cycle, the process loop will continue and the communication will hold according to the description of the network explained above.\\\\

Since the processes all must read, calculate and then write, the buffer processes, as mentioned before, must behave the opposite way. This means that the last write the processes make before they \texttt{SKIP} will be left in the buffers since there are no processes to read the value from the channels. This means that the buffers must be able to either write a value or \texttt{SKIP}.\\


\section{Clocked Addone Example}
\begin{figure}
\centering
\begin{tikzpicture}
   \node[main node, text width=.5cm] (add) {\small \texttt{add}};
   \node[main node, text width=.5cm] (id) [right = 4cm of add] {\texttt{id}};
   \node[mythinsquare] (bufd) at (2.7, 1.7) {$Buf_d$};
   \node[mythinsquare] (bufc) at (2.7, -1.5) {$Buf_c$};
   % \draw[fill] (0.7,0) circle [radius=0.07];

   \path[draw,thick, ->, bend right=25]
   (add) edge node {} (bufc);
   \path[draw,thick, ->, bend right=25]
   (bufc) edge node {} (id);


   \path[draw,thick, ->, bend right=25]
   (id) edge node {} (bufd);
   \path[draw,thick, ->, bend right=25]
   (bufd) edge node {} (add);


   \node[align=center, below, font=\scriptsize] at (1.5,1.3){\texttt{d\_write}};
   \node[align=center, below, font=\scriptsize] at (3.9,1.3){\texttt{d\_read}};
   \node[align=center, below, font=\scriptsize] at (1.5,-0.8){\texttt{c\_write}};
   \node[align=center, below, font=\scriptsize] at (3.9,-0.8){\texttt{c\_read}};
   % \node[align=center, below, text width=1.7cm] at (2.3,-0.9){\footnotesize\texttt{channel c}};
\end{tikzpicture}
\caption{The clocked \texttt{addone} network. The network have two proceses and two buffers which ensure the global synchronicity.}
\label{fig:addone_clocked}
\end{figure}




% Kenneths version, which I believe is how the SME model works, is having a process or bus in the middle og all steps. By using a dependency graph (Explain more?) it is possible to see which processes communicate to witch processes and, more importantly, in which order. For each communication step (or maybe for each communication) a process/bus will receive all writes. In SME a process can write several times to the same channel but only the last one before the clock signal will be written, the others are just overwritten. Since we have the dependency graph, we also know which processes we need communication from, and when the process have written all it has to write, then it sends a ready signal to the "bus" process, which then waits for all the ready signals (because it knows how many it should get. And if it is one process/bus pr. communication then it only needs one of course.). When all ready signals are in, the bus-process change behaviour and it is not writing instead of reading. It writes all possible values out and the processes that are supposed to receive the values (which we know from the dependency graph) will receive the values. And the processes then need to send a ready signal back to the bus process to let it know that it have read all it needed. When all ready signals are received, the bus process when change behaviour again and can read values once again.
% All these steps are intermediate steps within one clock cycle. So at the "end" of the dependency graph, the step looks similar to the others, but it is registered as the clock and the next clock cycle begins. In principal, all these steps could be the clock, since the step is the same, but a step is simply chosen to be the clock, based on the dependency graph.
% By treating the communication like this within a clock cycle, the values can propagate through the network and the internal state of the processes are also kept. The original TAPS version could only verify all input for a system, but if the system was internally affected by values from a previous clock cycle, then the system could not verify it. It is not a problem in the seven segment example, since no values are dependent on previous values. But the Addone network do depend on what happened in the last clock cycle.
% With this solution it is possible to verify a specific number of clock cycles.
%




CSP was not initially developed for hardware modeling, and therefore it is not evident how to handle the clock cycle, which is an essential part of hardware modeling. When we transpile the SME network into \cspm{}, the SMEIL simulation have provided the ranges of all values from the simulation and therefore all clock cycles. This means that when FDR4 asserts a property it asserts on all possible communication combinations for all the simulated clock cycles. Therefore, even though we are transpiling from an SME model, where the clock is crucial, we can simply translate ``one-to-one" from the SMEIL program and still get an accurate assertion on the properties.



% TODO: Async vs. sync processes

%%% Generator processes - clock cycles
%TODO: Write something here?
%
% \chapter{Implementation}
% \label{chap:implementation}
% % TODO: Write something here
\section{Transpiling SMEIL statements}
% TODO: Read through and correct
% Constants are simply defined in the \cspm program, seperate from the process. Since the SMEIL programs must be well-formed, we know that only the processes that define the constant will use it, and therefore it is not a problem that
% NOTE Constants are currently not implemented in TAPS
% TODO: How to handle variables with predefined values

The types and ranges of variables are not translated unless TAPS need the information for verification. It is, however, necessary that TAPS looks through to check if any of the internal variables are instantiated with a value, since TAPS will have to keep this information for the translation. If the variables are not instantiated with a predefined value, TAPS simply ignore the variable declaration. We can only do this because we know the SMEIL program must be well-formed and since the variables are not used for verification, it does not matter what types and values the SMEIL program expects of these.

All assignments are translated directly into \cspm{} without much change, however if the assignment is to or from a bus, TAPS have to differentiate and handle these assignments differently, which will be explained later in this section.

%
% If-satements are translated into the \cspm{} version of an if-statement, which is very similar to the SMEIL version. However, since \cspm{} does not support \texttt{elif}, the \cspm{} if-statements can be nested to form these expressions. This quickly becomes very complex and hard to read, but since it is auto generated, it is not a problem to create.
% % NOTE: If-statements are currently not implemented in TAPS


Traces and assertions are, as explained in Chapter \ref{chap:analysis} not useful in the \cspm{} program and therefore we either throw them away or keep them as comments for the sake of understanding the generated code. Currently TAPS throw them away, but it would be a simple task to change this and add them as comments in the generated \cspm{} code.
% NOTE: Assertions are not curently implemented in TAPS. Trace is.

Most expressions of SMEIL can be directly translated, like \texttt{+}, \texttt{-}, \texttt{/} and \texttt{\%} etc. however there are a few differences in the presendence. The unary \texttt{not} operator does not have the same presedence in the SMEIL as in \cspm{}. This means that the programmer needs to be aware of this, and include parentheses when using the operator to ensure the correct translation. The programmar also needs to be aware of the fact that in \cspm{} the equality comparison operators and comparison operators have the same presedence, but they are seperated in the SMEIL grammar.
Bitwise operations does not exist in \cspm{}, so these would have to be transformed to standard arithmetics. % TODO: This is currently not implemented.
% TODO: Write more about how this could be done.
% TODO: Write more when I have more implemented for the processes (like arrays and stuff)
% -------------------------------------

\section{Transpiling Channels}
In SMEIL when a process is reading from a bus channel, it is referencing the bus and channelname within the normal expression. Since the bus parameters for a SMEIL process is the bus name, the process itself must reference the specific channel within that bus. This is applied whether it is reads or writes.
An example of this can be seen in Listing \ref{lst:smeil_input_parameter}, where \texttt{input.val} indicates a read from the bus \texttt{input} and the channel \texttt{val} within that bus.

\begin{listing}
\begin{minted}{smeil_lexer.py:SMEILLexer -x}
proc a (in input)
    bus abus {
        val: uint;
    }
{
    abus.val = input.val + 1;
}
\end{minted}
\caption{Example of a read and a write in SMEIL.}
\label{lst:smeil_input_parameter}
\end{listing}
SMEIL does not seperate communication and calculation in the process statements which is of great advantage in SMEIL, but the translation becomes more difficult since TAPS has to be able to recognise the communication.

Becasue SMEIL has to specify the channel name when reading or writing, TAPS can recognise the communication in an assignment. If one of the elements in an assignment, right or left, contains a dot, then we can assume that it is communication. The original grammar of an assignment like the one in Listing \ref{lst:smeil_input_parameter} can be seen in Listing \ref{lst:smeil_assignment_grammar}. A communication is the \textit{hierarchical accessor} alternative in the \textit{statement} grammar. The communication can either be the right hand side of the assignment, which is a write, or part of the left hand side expression, which will be a read. A write is simple to recognise, since it is not combined with other parts of the grammar. TAPS can search the \textit{name} of a right hand side assignment and see if it contains a dot. The read is a bit more complicated since it can be used like any internal variable in the expressions. This means that TAPS will have to search all names within the nested expression to find a potential read.
% TODO: Write something more about how TAPS actually achieve this.
\begin{listing}
\begin{grammar}
<statement> ::= <name> `=' <expression> `;' (assignment)

<name> ::= <ident>
\alt <name> `.' <name> (hierarchical accessor)
\alt <name> `[' <array-index> `]' (array element access)

<expression> ::= <name>
\alt <literal>
\alt <expression> <bin-op> <expression>
\alt <un-op> <expression>
\alt <name> `(' \{ <expression> \}  `)' (function call)
\alt `(' <expression> `)'


\end{grammar}
\caption{The original assignment, name and expression grammars defined in \cite{Asheim2018}.}
\label{lst:smeil_assignment_grammar}
\end{listing}

When translating an SMEIL bus channel to a \cspm{} channel TAPS must provide channel names that will be unique in \cspm{} just like the formal name of a bus channel in SMEIL is unique.
\subsection{Naming Channels}
The simple way of translating the SMEIL channels into \cspm{} chanels are to use the formal name of the SMEIL bus channel already defined. The naming are created by concatinating the channel name, bus name and process name which will ensure that the generated code are more readable than if I had used random unique strings as naming, which is also a possible solution. A unique string might give more security than using a concatinated version, but in this case I decided to make use the allready defined names. This means that all calls using the syntax \texttt{bus.channel} from the process \texttt{A} in SMEIL will be translated to \texttt{A\_bus\_channel} in \cspm{}.

There is however one situation where this naming might cause troubles. If a process is reused and instantiated twice in the network, the bus channels defined in the two processes would have the same name in SMEIL, but because of the instance declarations in the SMEIL network, the channels are seperated. This is important to handle in \cspm{} so there does not occur problems with having identical channel names.

\begin{listing}
\begin{minted}{smeil_lexer.py:SMEILLexer -x}
network net() {
    instance c of clock();
    instance a of A(c.output, val: 1);
    instance _ of A(c.output, val: 2);
    instance s of src(a.output);
}
\end{minted}
\caption{Example of a network with four instances whereas two are instances of the same process.}
\label{lst:Instance_variations}
\end{listing}
In the example in Listing \ref{lst:Instance_variations} we see a network consisting of four instances where two of the instances are the same process with different constant parameter.
All but one instance is defined by an instance name which can be used for referencing in the instance parameters. It is also possible to have instances without an instance name. However, it is defined in SMEIL that two instances of the same process, cannot both be without an instance name.\\

% NOTE: Does not support!!
With two instances of the same process, the \cspm{} channels generated from the buses of these processes, would have identical names. It is, of course, important that TAPS are nameing the channels so that no \cspm{} channels are identical. Therefore TAPS can look at the instances and append the instance name to the \cspm{} channel name. This, of course, means that we expect the developer to not give the two identical process instances the same instance name. Technically this is possible, but would not make sense.
If the instance has an anonymous instance name, the \cspm{} channel name will not include this.

The restrictions to identical process instances in SMEIL ensures that even if there are channels without instance names in the SMEIL program there will never be two identical named \cspm{} channels in the generated code. This however does mean that the naming of each channel quickly becomes long and chaotic, but it is still more readable than random generated names.
% NOTE! Currently TAPS does not support the naming of identical process busses.

In SMEIL it is also possible to have bus declarations within the network declaration. These bus channels will be named in the same manner as the other channels in \cspm{}, however without a potential instance name appended.

\subsection{\cspm{} Channel Ranges}
When defining the channels in \cspm{} it is important, to define a limited range of values accepted for the channel. If the channel is defined for the entire space of integers, FDR4 would search through all possible integers which would result in the statespace becomming too large and FDR would run out of space. Therefore TAPS need to define some specific range of values for each channel but it must be values relevant to the channel.
As explained in Chapter~\ref{chap:analysis}, all simulated SMEIL programs will include the observed range and restricted types for all channels and variables. The types represent the minimum observed width of the channels in bits, and by calculating the possible range from these types, we can create the corresponding channels in \cspm{}, and thereby avoid having a seemingly endless runtime in FDR4.

The simulated \texttt{seconds} process from the seven segments example can be seen in Listing~\ref{lst:range_smeil}. Notice that the two channels are defined both with a type \texttt{u3} and \texttt{u4} and with a range 0 to 5 and 0 to 9. These are the observed types and value ranges the simulation has tracked for each specific channel. In order to create the \cspm{} channels based on the types, we need to convert \texttt{u3} and \texttt{u4} into the range of values that the types represents. For \texttt{u3} this is 0 through 7 and for \texttt{u4} it is 0 through 15. In Listing~\ref{lst:channel_range_cspm} the calculated ranges are used to define the \cspm{} channels.

\begin{listing}
\begin{minted}{cspm_lexer.py:CSPmLexer -x}
channel seconds_out_first_digit : {0..7}
channel seconds_out_second_digit : {0..15}

Seconds(seconds_in) =
let
    seconds = seconds_in % 60
    seconds_first_temp = seconds / 10
    seconds_second_temp = seconds % 10
within
    seconds_out_first_digit ! seconds_first_temp ->
    seconds_out_second_digit ! seconds_second_temp ->
    SKIP
\end{minted}
\caption{Example of the \texttt{Seconds} process from the generated \cspm{} code in the seven segment display example. See full example in Listing~\ref{lst:cspm} in the appendix.}
\label{lst:channel_range_cspm}
\end{listing}


Since the assertions we wish to make is to verify the widths of the channels, it might seem redundant to create \cspm{} channels with a limited range. FDR4 would always only check the values in the defined channel range and therefore there is no point in asserting if the values go beyond this range. After simulating the SMEIL network, the compiler provides us with both a type and a range of observed values. The type, as explained above, is used to create the restricted range for each \cspm{} channel and the observed values are used for the FDR4 assertions. The range of values that the types represent will always be equal or larger than the range of observed values, and by using these values the assertions becomes valuable.\\

When it comes to transpiling the data generator process into a \cspm{} channel, TAPS also use the type of the bus channel to define the data input channel for the network in \cspm{}. TAPS use this instead of the observed values because it is not possible to guarantee the precise input values of the system. If TAPS used the observed values, the assertions will pass every time, since it will test the values already used to generate the rest of the observed values.

\section{Generating Monitor Processes}
When generating the monitor processes in the \cspm{} program we use the observed values for each channel, as described above.
In the SMEIL grammar defined in \cite{Asheim2018} it is optional to include a range to each bus channel definitions. In Listing \ref{lst:smeil_bus_grammar} the relevant grammar rules from \cite{Asheim2018} can be seen. The square brackets indicates optional parts.

As explained in Chapter \ref{chap:design}, all channels except the input channels are verified in FDR4. This means that there will be a monitor process for almost every channel and since TAPS use the observed values for each channel for the verification, I had to change the original grammar of SMEIL to suppport this. Therefore the grammar shown in Listing \ref{lst:smeil_bus_grammar} is changed to the grammar defined in Listing \ref{lst:smeil_bus_grammar_no_option}. The only difference is that ranges are no longer optional and that the keywork \texttt{exposed} are no longer allowed, since TAPS are not supporting co-simulation, which is what this keyword indicates. As a consequence hereof, the output bus within the data generator process must be defined with a range even though it is not necessary. This is a result of the data processes not being different from other processes in SMEIL.\\
\begin{listing}
    \begin{grammar}
    <bus-decl> ::= [ `exposed' ] `bus' <ident> `\{' <bus-signal-decls> `\}'  `;'

    <bus-signal-decls> ::= <bus-signal-decl> \{ <bus-signal-decl> \}

    <bus-signal-decl> ::= <ident> `:' <type> [ `=' <expression> ] [ <range> ] `;'
    \end{grammar}
    \caption{The bus grammar defined in \cite{Asheim2018}}
    \label{lst:smeil_bus_grammar}
\end{listing}
\begin{listing}
    \begin{grammar}
    <bus-decl> ::= `bus' <ident> `\{' <bus-signal-decls> `\}'  `;'

    <bus-signal-decls> ::= <bus-signal-decl> \{ <bus-signal-decl> \}

    <bus-signal-decl> ::= <ident> `:' <type> [ `=' <expression> ] <range> `;'
    \end{grammar}
    \caption{The bus grammar defined in \cite{Asheim2018} changed to match the demands of the translation.}
    \label{lst:smeil_bus_grammar_no_option}
\end{listing}
The monitor process asserts that the observed values of the \cspm{} channels are actually what is communicated on the channels. In Listing~\ref{lst:monitor_range_cspm} the two monitor processes for the \texttt{Seconds} process from the seven segment example can be seen. The values used for these statements are 0 through 5 for the first digit and 0 through 9 for the second which can be seen defined for each channel in Listing~\ref{lst:range_smeil}.
\begin{listing}
\begin{minted}{cspm_lexer.py:CSPmLexer -x}
Seconds_out_first_digit_monitor(c) =
    c ? x -> if 0 <= x and x <= 5 then SKIP else STOP
Seconds_out_second_digit_monitor(c) =
    c ? x -> if 0 <= x and x <= 9 then SKIP else STOP
\end{minted}
\caption{Example of the \texttt{Seconds} monitor processes from the generated \cspm{} code in the seven segment display example. See full example in Listing~\ref{lst:cspm} in the appendix.}
\label{lst:monitor_range_cspm}
\end{listing}
The naming of the monitor processes are the name of the channel they are asserting concatinated with \texttt{\_monitor}. This is very simple but very usefull since it means we can automatically find the name of the monitor process when we have the channel name, and vice versa. We also know that no two channel names will ever be equal in the generated \cspm{} code and therefore the monitor processes will also not have any duplicates when using this name scheme. \\

This monitor process structure is a simple but general solution which can be reused in many situations and due to the requirement of opserved ranges, TAPS can always provide a monitor process with this structure for a channel.
% TODO: Make sure that examples of the failure in the seven segment example are included somewhere, as well as a working example.



\section{Transpiling Networks}
% TODO: This headline is almost the same at it is in design chapter
When generating the process monitor network as described in Section \ref{sec:design_translating_network}, TAPS first search through all writes within each process and generate a monitor processes for each channel.\\

The process monitor network cosists of three parts. Firstly if the process takes an input value, the value must be read from the channel. From the instance parameters defined in SMEIL, TAPS receive the name of the actual SMEIL process and bus that provides the input bus. Then within the process itself, TAPS get the information of which channel within the input bus it is reading from. All this information is gathered to clarify which \cspm{} channel the process monitor network should read from. If there are more input values, this is of course done several times, using prefixing to include more reads.

% NOTE: TAPS does not currently support more than one input channel
The second part is to add the process name with the expected parameters, both input values, output channels as well as constants and internal values.

In the last part, TAPS will have to synchronise the process with the monitor processes connected to its output buses.
TAPS synhronise the process together with each monitor process over the channel it is asserting.
TAPS adds an extra layer around the structure for each extra monitor processes.
This is quite easy to auto generate since TAPS can simply loop over the list of monitor processes or writes for the process. \\

In Listing \ref{lst:channel_range_cspm} the translated \texttt{Seconds} process is defined, and in Listing \ref{lst:monitor_range_cspm} the two monitor processes for the \texttt{Seconds} process are shown.
In Listing \ref{lst:network_example_cspm} the network generated from these two listing is shown where it is easy to see how the read is performed first in the process before the \texttt{first digit} monitor process is synchronised together with the \texttt{Seconds} process. Adding a parentheses around the first synhronisation to keep the structure, the \texttt{second digit} monitor process is synchronised on the \texttt{second digit} channel which completes the process monitor network.

Lastly the \texttt{assert} operator is used to specify to FDR4 that it should create a refinement check against the specification process \texttt{SKIP} and the implementation process \texttt{N\_seconds} and also that it should use the Failure model.

\begin{listing}
\begin{minted}{cspm_lexer.py:CSPmLexer -x}
N_seconds = clock_out_val ? variable ->
            (Seconds(variable)
            [| {| seconds_out_first_digit|} |]
            Seconds_out_first_digit_monitor(seconds_out_first_digit))
            [| {| seconds_out_second_digit|} |]
            Seconds_out_second_digit_monitor(seconds_out_second_digit)

assert SKIP [F= N_seconds \ Events
\end{minted}
\caption{Example of the \texttt{Seconds} network processes from the generated \cspm{} code in the seven segment display example. See full example in Listing~\ref{lst:cspm} in the appendix.}
\label{lst:network_example_cspm}
\end{listing}


% TODO: This might just belong in the new version!
% When the smaller process monitor networks have been created TAPS will then be able to synhronise other smaller monitor networks where there is shared communication. TAPS controls that all communication are handled within this new network. (Maybe it makes sense to generate smaller network which can then be synhronised. I mean where each smaller network is a process with a name, so it does not become so large nested.. )
% % TODO: Figure out how to make sure that all data are synchronised. It might be something about needing to have a linked list or something to keep all the data together. both stuff defined as parameter and communication defined by using the formal names.


\section{Technologies}

Usually when doing any form of translation either by a compiler or interpreter, the use of a symbol table is often needed in order to keep information about variable names, function names, classes, communication ect. Usually the translation, when reaching a symbol in the translation, use the symbol table to look up a symbol and retrieve its context if it has any information about said symbol. This can be to check that a variable have been declared or for type checking or similar. The symbol table is usually generated in the analysis section and are used for look ups througout the compilation or interpretation. When parsing the code, TAPS gathers all relevant information from the code, and since TAPS expect the SMEIL program to be well-formed, it does not need a symbol table to check if variables have been declared in the process declaration before being used in the body of the process.
TAPS use ANTLR4~\cite{antlr} as the parser generator with Python 2.7 as its source.
For the actual gode generation the templating language Jinja2~\cite{jinja2} have been used.

\subsection{ANTLR4}
For the transpiling between SMEIL source code and \cspm{} source code, I decided to use ANTLR4 for creating a parser and a lexer. ANTLR4 is a Java-based parser generator library that, based on a grammar, can generate parsers in Java or another target language. ANTLR is a well used tool with a lot of documentation and which have been updated to a new and improved version 4 recently. By using ANTLR4 I could easily transform the SMEIL grammar into a parser and lexer that could then immediately be used to transform into \cspm{}.
ANTLR4 supports Extended BNF (EBNF) and it must be defined in \texttt{.g4} file format, which is very similar to other standard grammars. The SMEIL grammar have been changed slightly either because it was more efficient to parse the programs if a grammar rule was removed. For instance the SMEIL grammar does not have a short way of representing "one or more" but only "zero or more". In ANTLR4 it is possible to define the rules as "one or more", and some of these SMEIL rules have been simplified to this. The grammar have also been changes in a few places, simply because there are things within the SMEIL grammar that TAPS does not allow, such as optional ranges, as described earlier.

% A lexical analysis, which is what a lexer , is a process of converting a string of characters into tokens, which is also called tokenization. Each token represents a lexer rule in the grammer, for instance, if the string is "123" and there is a lexer rule "INT: {0-9}+" which means one or more of 0-9 digits. Then the token would be an INT.
%
% A parser...%TODO: write more here
% Should I write something about what a lexer and a parser is? I am pretty sure Brian said no, that it is not a part of the job. The job is to describe how I generate the code, not how I parse it with an automatic parse generator. But I should explain what ANTLR do and how it is an advantage to me.

After parsing the code with ANTLR4, one can decide to traverse the parse tree itself or use either the listener or the visitor methods that ANTLR4 provides. The main difference between the listener and the visitor is that the listener provides methods which are called by the walker object directly, and the visitor methods must call its children explicitly to walk them. Both methods can provide the same results, and it often depends on preference. In my implemenation of TAPS I first tried the listener method, but realising that I do not need data from all parse rules, it was not necessary to walk the entire tree and therefore I shifted to the visitor method where I can be specific about which rules ANTLR4 should walk.

After ANTLR4 have walked the parse tree and collected all relevant information, the data is transformed to match the requirements of the \cspm{} code. For example, the names of a bus channel in SMEIL it transformed into a combined name for a \cspm{} channel. I have defined templates in Jinja2 that match these generalised structures in \cspm{} that I can use to represent the SMEIL code. These templates are then used together with the transformed data to generate the \cspm{} code.

%
% With the visitor we gain the advantage that we decide which part of the parse tree we traverse, instead of having to traverse the entire parse tree as with the listener method. With the visitor, for each parser we have to explicitly call \texttt{visit()} on the children that we wish to visit, which, in some cases can make a tree traversal more complex, but in our case we wish to seperate concerns and this is done by having three different tree traversals which collect three differet pieces of data for the code generation.
% When restructuring the code for the visitor method, we have to implement the top parsers, \texttt{module} and \texttt{entity}, even though they dont have any real value for the data we need to collect. However, because of the visitor method having to call all children, we need to implement these two parsers, simply so they can call their children.
% This seem a little pointless compared to the listener, where we can ignore the parsers we do not care about, but if we use a listener we would have to use a stack system and that would cause problems, because a terminal node that we only want sometimes might push something on the stack that we did not want to. Because every time that terminal node is entered or exited the code is run. With the visitor we can control when the nodes are run which means that we avoid some messy stack problems.
% A visitor can return a value like other standard functions which means that we have directly access to the result from it or its children. However, if there are more than one child the result will have to be assigned in a list or similar, to access.
% If there are two expressions in one alternative, we can get the two seperate results by calling \texttt{self.visit(ctx.expression(0))} and \texttt{self.visit(ctx.expression(1))}. This simplifies the structure since we do not have to handle loops in this case, because we know that the structure would always be two expressions, not more and no less.\\
% Even though we update the data structure \texttt{data} by reference instead of returning it as a result from the tree traversals, it is still necessary to return values from some of the parsers. Parts of the data structure cannot be created before having specific data, and the top parser with all the information will then gather the results from its children and update the \texttt{data} dictionary with the new data.


% TODO: Maybe it is worth mentioning how I remove some sub-"parsers" in the grammer, because it made the parsing more complicated . Such like I removed the processdecl nonterminal because I could then call visit(ctx.busdecl) specificly. Otherwise I could also just do "if ctx.busdecl: ..." but, this gave one less step (however, now I am thinking about it, it might not be a good idea. It gives more complexity and maybe thats not good if other people need to understand the antlr grammar. )

% TODO: Write something about how I should just gather the data and then letting the transformation part do any data transformations. I am, for example, not doing that now since I am adding calculations and communications into two different lists.

% TODO: Write something about how smart it is that I can use labels in the grammar and that I can then much more easily control how to handle different alternatives.


%
% \chapter{Extending TAPS for Clocked Systems}
% \label{chap:clock}
% %!TEX root = ../main.tex
After developing the initial version of TAPS, I realised that the solution was not broad enough in terms of what type of systems it could verify. The type of networks that it not possible to verify in the initial version of TAPS are network which keeps internal states between clock cycles. In the seven-segment example, no internal results had any influence on the result of other clock cycle results and therefor it could be correctly translated using the structures described in the previous chapters. \\

The initial idea was to avoid modeling a global synchronous clock in \cspm{} because, as described in Chapter \ref{chap:background}, the results from the master's thesis \textit{Generation of FPGA Hardware
Specifications from PyCSP Networks}~\cite{Skaarup14} by E. Skaarup and A. Frisch established how much the complexity of the network would increase when trying to model this. After creating the original system I wanted to model the \texttt{addone} example from \cite{smeil} but of course, because of the cyclic structure it does not fit into the structure of the original TAPS system. In this chapter I will introduce the \texttt{addone} example and the approach for extending TAPS with clocked systems.



% (From design)
% \section{Clock cycle problem}
% % %!TEX root = ../main.tex
After developing the initial version of TAPS, I realised that the solution was not broad enough in terms of what type of systems it could verify. The type of networks that it not possible to verify in the initial version of TAPS are network which keeps internal states between clock cycles. In the seven-segment example, no internal results had any influence on the result of other clock cycle results and therefor it could be correctly translated using the structures described in the previous chapters. \\

The initial idea was to avoid modeling a global synchronous clock in \cspm{} because, as described in Chapter \ref{chap:background}, the results from the master's thesis \textit{Generation of FPGA Hardware
Specifications from PyCSP Networks}~\cite{Skaarup14} by E. Skaarup and A. Frisch established how much the complexity of the network would increase when trying to model this. After creating the original system I wanted to model the \texttt{addone} example from \cite{smeil} but of course, because of the cyclic structure it does not fit into the structure of the original TAPS system. In this chapter I will introduce the \texttt{addone} example and the approach for extending TAPS with clocked systems.



% (From design)
% \section{Clock cycle problem}
% % %!TEX root = ../main.tex
After developing the initial version of TAPS, I realised that the solution was not broad enough in terms of what type of systems it could verify. The type of networks that it not possible to verify in the initial version of TAPS are network which keeps internal states between clock cycles. In the seven-segment example, no internal results had any influence on the result of other clock cycle results and therefor it could be correctly translated using the structures described in the previous chapters. \\

The initial idea was to avoid modeling a global synchronous clock in \cspm{} because, as described in Chapter \ref{chap:background}, the results from the master's thesis \textit{Generation of FPGA Hardware
Specifications from PyCSP Networks}~\cite{Skaarup14} by E. Skaarup and A. Frisch established how much the complexity of the network would increase when trying to model this. After creating the original system I wanted to model the \texttt{addone} example from \cite{smeil} but of course, because of the cyclic structure it does not fit into the structure of the original TAPS system. In this chapter I will introduce the \texttt{addone} example and the approach for extending TAPS with clocked systems.



% (From design)
% \section{Clock cycle problem}
% % \input{chapters/clock_cycle_problem}
%
% CSP was not initially developed for hardware modeling, and therefore it is not evident how to handle the clock cycle, which is an essential part of hardware modeling. When we transpile the SME network into \cspm{}, the SMEIL simulation have provided the ranges of all values from the simulation and therefore all clock cycles. This means that when FDR4 asserts a property it asserts on all possible communication combinations for all the simulated clock cycles. Therefore, even though we are transpiling from an SME model, where the clock is crucial, we can simply translate ``one-to-one" from the SMEIL program and still get an accurate assertion on the properties.
%
% % It is important to mention that the FDR version of the SMEIL program are represented as one clock cycle and therefore we do not have to handle implicit clock cycle issues. we can just translate one-to-one, because FDR models one clock cycle and the input represents all possible input in one clock cycle.
%
%

%




\section{Initial Addone Example}
% TODO: remember to remove the addone example in the analysis chapter. Both the original one and the other ones that are using the names. Use something from seven segment example instead.
The \texttt{addone} network is a simple network that consists of two processes communicating with one another. The \texttt{add} process receives a value and increments it by a value passes a as a constant parameter. The \texttt{id} process only receives the value and passes it along on its output bus.
The network is a two process loop and it is therefore essential that there is a way to initialise the loop as well as terminating it properly.
A figure of the network can be seen in Listing \ref{fig:addone_unclocked}.\\

\begin{figure}
    \centering
    \begin{tikzpicture}
       \node[main node, text width=.5cm] (1) {\small \texttt{add}};
       \node[main node, text width=.5cm] (2) [right = 3cm of 1] {\texttt{id}};
       % \draw[fill] (0.7,0) circle [radius=0.07];

       \path[draw,thick, ->, bend right=30]
       (1) edge node {} (2);
       \path[draw,thick, ->, bend right=30]
       (2) edge node {} (1);

       \node[align=center, below, text width=1.7cm] at (2.2,1.3){\footnotesize\texttt{channel d}};
       \node[align=center, below, text width=1.7cm] at (2.2,-0.9){\footnotesize\texttt{channel c}};
   \end{tikzpicture}
    \caption{The \texttt{addone} network. The network have two proceses which communicate to each other on the two buses.}
    \label{fig:addone_unclocked}
\end{figure}
The network is simple to model in SMEIL as can be seen in Listing \ref{lst:addone_smeil_example} but when translated with TAPS the generated \cspm{} code did not model the network correctly. When translating the \texttt{addone} network with the original version of TAPS, the generated \cspm{} code will only be able to simulate one clock cycle. As previously explained%TODO: Make sure I introduce this somewhere before this
this initial version of TAPS will verify all possible input values for the system, but what the \texttt{addone} network shows, is that to enlarge the set of problems possible to verify with TAPS, it is necessary to extend TAPS to support these types of networks. \\

The solution to this problem is to extend the translation to model a global synchronous structure in \cspm{} instead of the simple model that is the initial version of TAPS. As E. Skaarup and A. Frisch already learned, enforcing a global synchronous model onto CSP is not simple and even simple network become very complex. The advantage I have, compared to the previous attempt to model global synchronicity with CSP, is that TAPS will auto generate the \cspm{} code and therefore the complexity and size of the correspoding \cspm{} network is not an issue in terms of creating the network. The extra complexity might, however, become a problem when verifying with FDR4. It is possible that the added complexity requires more of FDR4 and that the size of problems verifiable with FDR4, becomes smaller which this solution.
\begin{listing}
\begin{minted}{smeil_lexer.py:SMEILLexer -x}
proc add (in input, const constant)
    bus output {val: u4 = 0 range 0 to 10;};
{
    output.val = input.val + constant;
}


proc id (in input)
    var from_add: u4 range 0 to 10;
    bus output {val: u4 = 0 range 0 to 10;};
{
    from_add = input.val;
    trace("Wrote value {}", input.val);
    output.val = from_add;
}


network addone_network ()
{
    instance id of id(add.output);
    instance add of add(id.output, constant: 1);
}
\end{minted}
\caption{The simulated SMEIL network \texttt{addone\_network} with two processes. The example is similar to the Addone example in \cite{smeil}.}
\label{lst:addone_smeil_example}
\end{listing}
\section{Clocked Networks}
As can be seen in the examples of the seven-segment \cspm{} code %TODO: Where can they see the code?
, no proceses are recursive. All processes run once and, unless errors occured, behaves as the \texttt{SKIP} process afterwards. This is part of the reason why the \texttt{addone} example cannot represent more than one clock cycle. To be able to verify more than one clock cycle it is essential that the processes are recursive. As explained in the csp background in Chapter \ref{chap:background}, recursive processes are simply processes which instead of behaving like the \texttt{SKIP} process, behaves like itself. An example of this can be seen in Listing \ref{lst:cspm_recursion}.
\begin{listing}
\begin{minted}{cspm_lexer.py:CSPmLexer -x}
Init = d ! 1 -> A(1)
A(x) = d ! x -> A(x+1)
\end{minted}
\caption{Example of the a recursive \cspm{} process which is initialised by the \texttt{Init} process.}
\label{lst:cspm_recursion}
\end{listing}

In theory an SME processes never stops running, but when simulating the SMEIL network it is of course not possible to simulate endless runtime. Therefore the developer indicates the number of clock cycles to simulate and the results should be seen as a snapshot of the process runtime. \\

In Listing \ref{lst:cspm_recursion} the process \texttt{A} performs an endless loop with no change to terminate. As mentioned in the previous, it is also essential to have a limited range of values for FDR4 to verify to avoid running out of space and if the example in Listing \ref{lst:cspm_recursion} was verified with FDR4, it would eventually run out of space. It is therefore crucial to model a structure that can drive the network and which can ensure the process terminate at the specified time. This is done with the \texttt{Clock} process. The clock process drives the network for a specific number of clock cycles and then terminates, which enforce all other processes to do the same, which will be explained shortly. \\


It, of course, is still necessary for a new version of TAPS to model a \cspm{} network that reflects the SME model and therefore it must adhere to the SME model structure. To model the global synchronicity in \cspm{} it is necessary to enforce a synchronising event where all processes synchronise before continuing. This synchronicity can be emulated by having a \texttt{sync} channel to emulate the rising and falling clock signal. All clocked processes in the network will be synchronised with the \texttt{sync} channel and, as previously introduced, when two processes are synhronised on a channel they must agree on communication. Therefore, all clocked processes must agree to synchronise before any process can continue.


----------------------------------------------------
----------------------------------------------------
----------------------------------------------------


(clock stuff)
initialised with a value and syncronise on the \texttt{clock} channel, which all other processes does as well. When the specified number of clock cycles has passed, the \texttt{Clock} process stops clocking and instead behaves as \texttt{SKIP}. This means that all other processes won't be able to syncronise on the \texttt{clock} channel anymore and therefore they will then instead behave as \texttt{SKIP} and that way the system terminates as planned.\\




The \texttt{clock} channel is used as a two-way clock synchronisation, where the same channel is used for syncronising up as well as down. Thus all processes syncronise before they read and before they write. The \texttt{Clock} process then needs to syncronise twice before incrementing its counter. \\

% (from verification in design)
% TODO: Maybe I should do it with the FD model, but in this case it does not really make sense, since all processes end after one iterations (they all SKIP). But it is worth mentioning it in the new system, because there it becomes relevant

% TODO: What happens if an SMEIL channel have been defined with an initial value?



% TODO: What if several processes write to the same output channel? How to handle the monitor process then?



% TODO: This might just belong in the new version!
% When the smaller process monitor networks have been created TAPS will then be able to synhronise other smaller monitor networks where there is shared communication. TAPS controls that all communication are handled within this new network. (Maybe it makes sense to generate smaller network which can then be synhronised. I mean where each smaller network is a process with a name, so it does not become so large nested.. )
% % TODO: Figure out how to make sure that all data are synchronised. It might be something about needing to have a linked list or something to keep all the data together. both stuff defined as parameter and communication defined by using the formal names.






\section{Clocked Processes}
% % TODO: Write that the read of a process does not make sense in the network and that we move it to be internal inside the process.

% %%% Generator processes - clock cycles
% %TODO: Write something here?
%
%



\section{The Bounds Problem}

When trying to verify that this system terminates as expected in FDR4, we came across an error while FDR4 was compiling the program.
FDR4 is complaining that a value, the system is trying to send, is not a part of the set of values defined for the channel.
The channels, used for communicating the value between the processes and the buffers, are defined for a specific range and since the 'Add' process has to read, calculate and write before it can terminate it will always write a value out that have been incremented with one.\\
For instance, if the channels were defined with the range \{0..5\} and the \texttt{Clock} process would stop after 10 clock cycles. This would mean that on the last clock cycle, before terminating, the 'Add' process would write a 6 onto the channel, which of course is not possible since the channel is defined only for the range \{0..5\}, so FDR4 complains about this.\\
However, what we experienced was that when the ranges of the channels were set to a larger number than the system would ever reach, within the defined number of clock cycles, FDR4 would still fail with the same reason.
This caused some problems since we were not able to verify the system and we were not interested in FDR4 trying to verify a communication that would never occur within the network.
Since we are syncronising the processes on the events in the channels, it seems odd that FDR4 still considers events which should not be possible to reach.  \\\\
After some time working with the problem and trying to understand the reason for FDR4s error message, we found that if we simply add a guard or an if-then-else statement that tests the value to be written, FDR4 will gladly verify the system and when using Probe on the network, it is clear that FDR4 does not consider the trace with the wrong values. \\

So the suggested solution, or fix, to this problem, is to add an if-then-else before all writes in a program. The statement then tests the value to be written against the max value of the range of the channel and if the value is not within the range, then the process behaves as the \texttt{SKIP} process, otherwise, it continues with write.
\begin{minted}[escapeinside=&&, mathescape=true]{cspm_lexer.py:CSPmLexer -x}
channel c : {0..20}

&$\vdots$&

    if (i+1) > 20  -- Check the upper limit of the channel
        then SKIP  -- SKIP if the value is above
        else (c_r ! (i+1) -> Add(i))) -- Otherwise write and continue
\end{minted}
In this case, it is only necessary to add an upper limit test, since the network only increments, but as a general rule, it would be necessary to test for both upper and lower bounds.


% The problem ocurred several times with different versions of the solution. Also, Ohm had the same problem with the Commstime problem. It makes sense why FDR4 wants to check, and maybe it is a way to save verification time: if it checks things in parallel or something.
% However it is not a good solution that we have to put in a upper/lower bound check that is actually never relevant.
% Of course it might be good, in any case, since the programmer then do not need to worry. On the other hand it might make the program look like it is terminating properly when it is actually failing because it tries to write a value that is not allowed. This case might happen, and if all other processes do not notice it and then also SKIP according to their specifications, then the verification passes even though it might be wrong.
% A solution might be to ensure that all processes synchronise before skipping, because then the problem (i think) would not occur, since the failing process would simply skip, then the other processes cannot skip because they have not syncrhonised yet. however, I am not sure this is possible since it might be that they either syncronise or they skip.



\section{Introducing Buffers}

For the 'Add' and 'Id' processes to comply with these states, they would have to read first, then the calculate phase, in which the 'Id' process does nothing, and then they would write the result onto a channel. A problem occurs, since they both have to read first, no one can read because no processes have written anything yet. To solve this, we implement two buffers which for each clock cycle reads the output that the process writes and then writes the value to a channel. Thus the buffer structure is the reverse of a 'normal' process since it will write and then read in a clock cycle. \\
If we give the buffers an initial value, they can begin the clock cycle writing the value which the 'Add' and 'Id' processes can read and thereby they will comply with the SME model structure.
The buffers will be instantiated will a 'dummy' value which is also how it is typically done in hardware. The dummy value is simply to indicate that the system should ignore the first clock cycle and then continue with the systems actual values. \\
Each process is also instantiated with a value, which is then used instead of the dummy value from the buffer process. After this initial cycle, the process loop will continue and the communication will hold according to the description of the network explained above.\\\\

Since the processes all must read, calculate and then write, the buffer processes, as mentioned before, must behave the opposite way. This means that the last write the processes make before they \texttt{SKIP} will be left in the buffers since there are no processes to read the value from the channels. This means that the buffers must be able to either write a value or \texttt{SKIP}.\\


\section{Clocked Addone Example}
\begin{figure}
\centering
\begin{tikzpicture}
   \node[main node, text width=.5cm] (add) {\small \texttt{add}};
   \node[main node, text width=.5cm] (id) [right = 4cm of add] {\texttt{id}};
   \node[mythinsquare] (bufd) at (2.7, 1.7) {$Buf_d$};
   \node[mythinsquare] (bufc) at (2.7, -1.5) {$Buf_c$};
   % \draw[fill] (0.7,0) circle [radius=0.07];

   \path[draw,thick, ->, bend right=25]
   (add) edge node {} (bufc);
   \path[draw,thick, ->, bend right=25]
   (bufc) edge node {} (id);


   \path[draw,thick, ->, bend right=25]
   (id) edge node {} (bufd);
   \path[draw,thick, ->, bend right=25]
   (bufd) edge node {} (add);


   \node[align=center, below, font=\scriptsize] at (1.5,1.3){\texttt{d\_write}};
   \node[align=center, below, font=\scriptsize] at (3.9,1.3){\texttt{d\_read}};
   \node[align=center, below, font=\scriptsize] at (1.5,-0.8){\texttt{c\_write}};
   \node[align=center, below, font=\scriptsize] at (3.9,-0.8){\texttt{c\_read}};
   % \node[align=center, below, text width=1.7cm] at (2.3,-0.9){\footnotesize\texttt{channel c}};
\end{tikzpicture}
\caption{The clocked \texttt{addone} network. The network have two proceses and two buffers which ensure the global synchronicity.}
\label{fig:addone_clocked}
\end{figure}




% Kenneths version, which I believe is how the SME model works, is having a process or bus in the middle og all steps. By using a dependency graph (Explain more?) it is possible to see which processes communicate to witch processes and, more importantly, in which order. For each communication step (or maybe for each communication) a process/bus will receive all writes. In SME a process can write several times to the same channel but only the last one before the clock signal will be written, the others are just overwritten. Since we have the dependency graph, we also know which processes we need communication from, and when the process have written all it has to write, then it sends a ready signal to the "bus" process, which then waits for all the ready signals (because it knows how many it should get. And if it is one process/bus pr. communication then it only needs one of course.). When all ready signals are in, the bus-process change behaviour and it is not writing instead of reading. It writes all possible values out and the processes that are supposed to receive the values (which we know from the dependency graph) will receive the values. And the processes then need to send a ready signal back to the bus process to let it know that it have read all it needed. When all ready signals are received, the bus process when change behaviour again and can read values once again.
% All these steps are intermediate steps within one clock cycle. So at the "end" of the dependency graph, the step looks similar to the others, but it is registered as the clock and the next clock cycle begins. In principal, all these steps could be the clock, since the step is the same, but a step is simply chosen to be the clock, based on the dependency graph.
% By treating the communication like this within a clock cycle, the values can propagate through the network and the internal state of the processes are also kept. The original TAPS version could only verify all input for a system, but if the system was internally affected by values from a previous clock cycle, then the system could not verify it. It is not a problem in the seven segment example, since no values are dependent on previous values. But the Addone network do depend on what happened in the last clock cycle.
% With this solution it is possible to verify a specific number of clock cycles.
%



%
% CSP was not initially developed for hardware modeling, and therefore it is not evident how to handle the clock cycle, which is an essential part of hardware modeling. When we transpile the SME network into \cspm{}, the SMEIL simulation have provided the ranges of all values from the simulation and therefore all clock cycles. This means that when FDR4 asserts a property it asserts on all possible communication combinations for all the simulated clock cycles. Therefore, even though we are transpiling from an SME model, where the clock is crucial, we can simply translate ``one-to-one" from the SMEIL program and still get an accurate assertion on the properties.
%
% % It is important to mention that the FDR version of the SMEIL program are represented as one clock cycle and therefore we do not have to handle implicit clock cycle issues. we can just translate one-to-one, because FDR models one clock cycle and the input represents all possible input in one clock cycle.
%
%

%




\section{Initial Addone Example}
% TODO: remember to remove the addone example in the analysis chapter. Both the original one and the other ones that are using the names. Use something from seven segment example instead.
The \texttt{addone} network is a simple network that consists of two processes communicating with one another. The \texttt{add} process receives a value and increments it by a value passes a as a constant parameter. The \texttt{id} process only receives the value and passes it along on its output bus.
The network is a two process loop and it is therefore essential that there is a way to initialise the loop as well as terminating it properly.
A figure of the network can be seen in Listing \ref{fig:addone_unclocked}.\\

\begin{figure}
    \centering
    \begin{tikzpicture}
       \node[main node, text width=.5cm] (1) {\small \texttt{add}};
       \node[main node, text width=.5cm] (2) [right = 3cm of 1] {\texttt{id}};
       % \draw[fill] (0.7,0) circle [radius=0.07];

       \path[draw,thick, ->, bend right=30]
       (1) edge node {} (2);
       \path[draw,thick, ->, bend right=30]
       (2) edge node {} (1);

       \node[align=center, below, text width=1.7cm] at (2.2,1.3){\footnotesize\texttt{channel d}};
       \node[align=center, below, text width=1.7cm] at (2.2,-0.9){\footnotesize\texttt{channel c}};
   \end{tikzpicture}
    \caption{The \texttt{addone} network. The network have two proceses which communicate to each other on the two buses.}
    \label{fig:addone_unclocked}
\end{figure}
The network is simple to model in SMEIL as can be seen in Listing \ref{lst:addone_smeil_example} but when translated with TAPS the generated \cspm{} code did not model the network correctly. When translating the \texttt{addone} network with the original version of TAPS, the generated \cspm{} code will only be able to simulate one clock cycle. As previously explained%TODO: Make sure I introduce this somewhere before this
this initial version of TAPS will verify all possible input values for the system, but what the \texttt{addone} network shows, is that to enlarge the set of problems possible to verify with TAPS, it is necessary to extend TAPS to support these types of networks. \\

The solution to this problem is to extend the translation to model a global synchronous structure in \cspm{} instead of the simple model that is the initial version of TAPS. As E. Skaarup and A. Frisch already learned, enforcing a global synchronous model onto CSP is not simple and even simple network become very complex. The advantage I have, compared to the previous attempt to model global synchronicity with CSP, is that TAPS will auto generate the \cspm{} code and therefore the complexity and size of the correspoding \cspm{} network is not an issue in terms of creating the network. The extra complexity might, however, become a problem when verifying with FDR4. It is possible that the added complexity requires more of FDR4 and that the size of problems verifiable with FDR4, becomes smaller which this solution.
\begin{listing}
\begin{minted}{smeil_lexer.py:SMEILLexer -x}
proc add (in input, const constant)
    bus output {val: u4 = 0 range 0 to 10;};
{
    output.val = input.val + constant;
}


proc id (in input)
    var from_add: u4 range 0 to 10;
    bus output {val: u4 = 0 range 0 to 10;};
{
    from_add = input.val;
    trace("Wrote value {}", input.val);
    output.val = from_add;
}


network addone_network ()
{
    instance id of id(add.output);
    instance add of add(id.output, constant: 1);
}
\end{minted}
\caption{The simulated SMEIL network \texttt{addone\_network} with two processes. The example is similar to the Addone example in \cite{smeil}.}
\label{lst:addone_smeil_example}
\end{listing}
\section{Clocked Networks}
As can be seen in the examples of the seven-segment \cspm{} code %TODO: Where can they see the code?
, no proceses are recursive. All processes run once and, unless errors occured, behaves as the \texttt{SKIP} process afterwards. This is part of the reason why the \texttt{addone} example cannot represent more than one clock cycle. To be able to verify more than one clock cycle it is essential that the processes are recursive. As explained in the csp background in Chapter \ref{chap:background}, recursive processes are simply processes which instead of behaving like the \texttt{SKIP} process, behaves like itself. An example of this can be seen in Listing \ref{lst:cspm_recursion}.
\begin{listing}
\begin{minted}{cspm_lexer.py:CSPmLexer -x}
Init = d ! 1 -> A(1)
A(x) = d ! x -> A(x+1)
\end{minted}
\caption{Example of the a recursive \cspm{} process which is initialised by the \texttt{Init} process.}
\label{lst:cspm_recursion}
\end{listing}

In theory an SME processes never stops running, but when simulating the SMEIL network it is of course not possible to simulate endless runtime. Therefore the developer indicates the number of clock cycles to simulate and the results should be seen as a snapshot of the process runtime. \\

In Listing \ref{lst:cspm_recursion} the process \texttt{A} performs an endless loop with no change to terminate. As mentioned in the previous, it is also essential to have a limited range of values for FDR4 to verify to avoid running out of space and if the example in Listing \ref{lst:cspm_recursion} was verified with FDR4, it would eventually run out of space. It is therefore crucial to model a structure that can drive the network and which can ensure the process terminate at the specified time. This is done with the \texttt{Clock} process. The clock process drives the network for a specific number of clock cycles and then terminates, which enforce all other processes to do the same, which will be explained shortly. \\


It, of course, is still necessary for a new version of TAPS to model a \cspm{} network that reflects the SME model and therefore it must adhere to the SME model structure. To model the global synchronicity in \cspm{} it is necessary to enforce a synchronising event where all processes synchronise before continuing. This synchronicity can be emulated by having a \texttt{sync} channel to emulate the rising and falling clock signal. All clocked processes in the network will be synchronised with the \texttt{sync} channel and, as previously introduced, when two processes are synhronised on a channel they must agree on communication. Therefore, all clocked processes must agree to synchronise before any process can continue.


----------------------------------------------------
----------------------------------------------------
----------------------------------------------------


(clock stuff)
initialised with a value and syncronise on the \texttt{clock} channel, which all other processes does as well. When the specified number of clock cycles has passed, the \texttt{Clock} process stops clocking and instead behaves as \texttt{SKIP}. This means that all other processes won't be able to syncronise on the \texttt{clock} channel anymore and therefore they will then instead behave as \texttt{SKIP} and that way the system terminates as planned.\\




The \texttt{clock} channel is used as a two-way clock synchronisation, where the same channel is used for syncronising up as well as down. Thus all processes syncronise before they read and before they write. The \texttt{Clock} process then needs to syncronise twice before incrementing its counter. \\

% (from verification in design)
% TODO: Maybe I should do it with the FD model, but in this case it does not really make sense, since all processes end after one iterations (they all SKIP). But it is worth mentioning it in the new system, because there it becomes relevant

% TODO: What happens if an SMEIL channel have been defined with an initial value?



% TODO: What if several processes write to the same output channel? How to handle the monitor process then?



% TODO: This might just belong in the new version!
% When the smaller process monitor networks have been created TAPS will then be able to synhronise other smaller monitor networks where there is shared communication. TAPS controls that all communication are handled within this new network. (Maybe it makes sense to generate smaller network which can then be synhronised. I mean where each smaller network is a process with a name, so it does not become so large nested.. )
% % TODO: Figure out how to make sure that all data are synchronised. It might be something about needing to have a linked list or something to keep all the data together. both stuff defined as parameter and communication defined by using the formal names.






\section{Clocked Processes}
% % TODO: Write that the read of a process does not make sense in the network and that we move it to be internal inside the process.

% %%% Generator processes - clock cycles
% %TODO: Write something here?
%
%



\section{The Bounds Problem}

When trying to verify that this system terminates as expected in FDR4, we came across an error while FDR4 was compiling the program.
FDR4 is complaining that a value, the system is trying to send, is not a part of the set of values defined for the channel.
The channels, used for communicating the value between the processes and the buffers, are defined for a specific range and since the 'Add' process has to read, calculate and write before it can terminate it will always write a value out that have been incremented with one.\\
For instance, if the channels were defined with the range \{0..5\} and the \texttt{Clock} process would stop after 10 clock cycles. This would mean that on the last clock cycle, before terminating, the 'Add' process would write a 6 onto the channel, which of course is not possible since the channel is defined only for the range \{0..5\}, so FDR4 complains about this.\\
However, what we experienced was that when the ranges of the channels were set to a larger number than the system would ever reach, within the defined number of clock cycles, FDR4 would still fail with the same reason.
This caused some problems since we were not able to verify the system and we were not interested in FDR4 trying to verify a communication that would never occur within the network.
Since we are syncronising the processes on the events in the channels, it seems odd that FDR4 still considers events which should not be possible to reach.  \\\\
After some time working with the problem and trying to understand the reason for FDR4s error message, we found that if we simply add a guard or an if-then-else statement that tests the value to be written, FDR4 will gladly verify the system and when using Probe on the network, it is clear that FDR4 does not consider the trace with the wrong values. \\

So the suggested solution, or fix, to this problem, is to add an if-then-else before all writes in a program. The statement then tests the value to be written against the max value of the range of the channel and if the value is not within the range, then the process behaves as the \texttt{SKIP} process, otherwise, it continues with write.
\begin{minted}[escapeinside=&&, mathescape=true]{cspm_lexer.py:CSPmLexer -x}
channel c : {0..20}

&$\vdots$&

    if (i+1) > 20  -- Check the upper limit of the channel
        then SKIP  -- SKIP if the value is above
        else (c_r ! (i+1) -> Add(i))) -- Otherwise write and continue
\end{minted}
In this case, it is only necessary to add an upper limit test, since the network only increments, but as a general rule, it would be necessary to test for both upper and lower bounds.


% The problem ocurred several times with different versions of the solution. Also, Ohm had the same problem with the Commstime problem. It makes sense why FDR4 wants to check, and maybe it is a way to save verification time: if it checks things in parallel or something.
% However it is not a good solution that we have to put in a upper/lower bound check that is actually never relevant.
% Of course it might be good, in any case, since the programmer then do not need to worry. On the other hand it might make the program look like it is terminating properly when it is actually failing because it tries to write a value that is not allowed. This case might happen, and if all other processes do not notice it and then also SKIP according to their specifications, then the verification passes even though it might be wrong.
% A solution might be to ensure that all processes synchronise before skipping, because then the problem (i think) would not occur, since the failing process would simply skip, then the other processes cannot skip because they have not syncrhonised yet. however, I am not sure this is possible since it might be that they either syncronise or they skip.



\section{Introducing Buffers}

For the 'Add' and 'Id' processes to comply with these states, they would have to read first, then the calculate phase, in which the 'Id' process does nothing, and then they would write the result onto a channel. A problem occurs, since they both have to read first, no one can read because no processes have written anything yet. To solve this, we implement two buffers which for each clock cycle reads the output that the process writes and then writes the value to a channel. Thus the buffer structure is the reverse of a 'normal' process since it will write and then read in a clock cycle. \\
If we give the buffers an initial value, they can begin the clock cycle writing the value which the 'Add' and 'Id' processes can read and thereby they will comply with the SME model structure.
The buffers will be instantiated will a 'dummy' value which is also how it is typically done in hardware. The dummy value is simply to indicate that the system should ignore the first clock cycle and then continue with the systems actual values. \\
Each process is also instantiated with a value, which is then used instead of the dummy value from the buffer process. After this initial cycle, the process loop will continue and the communication will hold according to the description of the network explained above.\\\\

Since the processes all must read, calculate and then write, the buffer processes, as mentioned before, must behave the opposite way. This means that the last write the processes make before they \texttt{SKIP} will be left in the buffers since there are no processes to read the value from the channels. This means that the buffers must be able to either write a value or \texttt{SKIP}.\\


\section{Clocked Addone Example}
\begin{figure}
\centering
\begin{tikzpicture}
   \node[main node, text width=.5cm] (add) {\small \texttt{add}};
   \node[main node, text width=.5cm] (id) [right = 4cm of add] {\texttt{id}};
   \node[mythinsquare] (bufd) at (2.7, 1.7) {$Buf_d$};
   \node[mythinsquare] (bufc) at (2.7, -1.5) {$Buf_c$};
   % \draw[fill] (0.7,0) circle [radius=0.07];

   \path[draw,thick, ->, bend right=25]
   (add) edge node {} (bufc);
   \path[draw,thick, ->, bend right=25]
   (bufc) edge node {} (id);


   \path[draw,thick, ->, bend right=25]
   (id) edge node {} (bufd);
   \path[draw,thick, ->, bend right=25]
   (bufd) edge node {} (add);


   \node[align=center, below, font=\scriptsize] at (1.5,1.3){\texttt{d\_write}};
   \node[align=center, below, font=\scriptsize] at (3.9,1.3){\texttt{d\_read}};
   \node[align=center, below, font=\scriptsize] at (1.5,-0.8){\texttt{c\_write}};
   \node[align=center, below, font=\scriptsize] at (3.9,-0.8){\texttt{c\_read}};
   % \node[align=center, below, text width=1.7cm] at (2.3,-0.9){\footnotesize\texttt{channel c}};
\end{tikzpicture}
\caption{The clocked \texttt{addone} network. The network have two proceses and two buffers which ensure the global synchronicity.}
\label{fig:addone_clocked}
\end{figure}




% Kenneths version, which I believe is how the SME model works, is having a process or bus in the middle og all steps. By using a dependency graph (Explain more?) it is possible to see which processes communicate to witch processes and, more importantly, in which order. For each communication step (or maybe for each communication) a process/bus will receive all writes. In SME a process can write several times to the same channel but only the last one before the clock signal will be written, the others are just overwritten. Since we have the dependency graph, we also know which processes we need communication from, and when the process have written all it has to write, then it sends a ready signal to the "bus" process, which then waits for all the ready signals (because it knows how many it should get. And if it is one process/bus pr. communication then it only needs one of course.). When all ready signals are in, the bus-process change behaviour and it is not writing instead of reading. It writes all possible values out and the processes that are supposed to receive the values (which we know from the dependency graph) will receive the values. And the processes then need to send a ready signal back to the bus process to let it know that it have read all it needed. When all ready signals are received, the bus process when change behaviour again and can read values once again.
% All these steps are intermediate steps within one clock cycle. So at the "end" of the dependency graph, the step looks similar to the others, but it is registered as the clock and the next clock cycle begins. In principal, all these steps could be the clock, since the step is the same, but a step is simply chosen to be the clock, based on the dependency graph.
% By treating the communication like this within a clock cycle, the values can propagate through the network and the internal state of the processes are also kept. The original TAPS version could only verify all input for a system, but if the system was internally affected by values from a previous clock cycle, then the system could not verify it. It is not a problem in the seven segment example, since no values are dependent on previous values. But the Addone network do depend on what happened in the last clock cycle.
% With this solution it is possible to verify a specific number of clock cycles.
%



%
% CSP was not initially developed for hardware modeling, and therefore it is not evident how to handle the clock cycle, which is an essential part of hardware modeling. When we transpile the SME network into \cspm{}, the SMEIL simulation have provided the ranges of all values from the simulation and therefore all clock cycles. This means that when FDR4 asserts a property it asserts on all possible communication combinations for all the simulated clock cycles. Therefore, even though we are transpiling from an SME model, where the clock is crucial, we can simply translate ``one-to-one" from the SMEIL program and still get an accurate assertion on the properties.
%
% % It is important to mention that the FDR version of the SMEIL program are represented as one clock cycle and therefore we do not have to handle implicit clock cycle issues. we can just translate one-to-one, because FDR models one clock cycle and the input represents all possible input in one clock cycle.
%
%

%




\section{Initial Addone Example}
% TODO: remember to remove the addone example in the analysis chapter. Both the original one and the other ones that are using the names. Use something from seven segment example instead.
The \texttt{addone} network is a simple network that consists of two processes communicating with one another. The \texttt{add} process receives a value and increments it by a value passes a as a constant parameter. The \texttt{id} process only receives the value and passes it along on its output bus.
The network is a two process loop and it is therefore essential that there is a way to initialise the loop as well as terminating it properly.
A figure of the network can be seen in Listing \ref{fig:addone_unclocked}.\\

\begin{figure}
    \centering
    \begin{tikzpicture}
       \node[main node, text width=.5cm] (1) {\small \texttt{add}};
       \node[main node, text width=.5cm] (2) [right = 3cm of 1] {\texttt{id}};
       % \draw[fill] (0.7,0) circle [radius=0.07];

       \path[draw,thick, ->, bend right=30]
       (1) edge node {} (2);
       \path[draw,thick, ->, bend right=30]
       (2) edge node {} (1);

       \node[align=center, below, text width=1.7cm] at (2.2,1.3){\footnotesize\texttt{channel d}};
       \node[align=center, below, text width=1.7cm] at (2.2,-0.9){\footnotesize\texttt{channel c}};
   \end{tikzpicture}
    \caption{The \texttt{addone} network. The network have two proceses which communicate to each other on the two buses.}
    \label{fig:addone_unclocked}
\end{figure}
The network is simple to model in SMEIL as can be seen in Listing \ref{lst:addone_smeil_example} but when translated with TAPS the generated \cspm{} code did not model the network correctly. When translating the \texttt{addone} network with the original version of TAPS, the generated \cspm{} code will only be able to simulate one clock cycle. As previously explained%TODO: Make sure I introduce this somewhere before this
this initial version of TAPS will verify all possible input values for the system, but what the \texttt{addone} network shows, is that to enlarge the set of problems possible to verify with TAPS, it is necessary to extend TAPS to support these types of networks. \\

The solution to this problem is to extend the translation to model a global synchronous structure in \cspm{} instead of the simple model that is the initial version of TAPS. As E. Skaarup and A. Frisch already learned, enforcing a global synchronous model onto CSP is not simple and even simple network become very complex. The advantage I have, compared to the previous attempt to model global synchronicity with CSP, is that TAPS will auto generate the \cspm{} code and therefore the complexity and size of the correspoding \cspm{} network is not an issue in terms of creating the network. The extra complexity might, however, become a problem when verifying with FDR4. It is possible that the added complexity requires more of FDR4 and that the size of problems verifiable with FDR4, becomes smaller which this solution.
\begin{listing}
\begin{minted}{smeil_lexer.py:SMEILLexer -x}
proc add (in input, const constant)
    bus output {val: u4 = 0 range 0 to 10;};
{
    output.val = input.val + constant;
}


proc id (in input)
    var from_add: u4 range 0 to 10;
    bus output {val: u4 = 0 range 0 to 10;};
{
    from_add = input.val;
    trace("Wrote value {}", input.val);
    output.val = from_add;
}


network addone_network ()
{
    instance id of id(add.output);
    instance add of add(id.output, constant: 1);
}
\end{minted}
\caption{The simulated SMEIL network \texttt{addone\_network} with two processes. The example is similar to the Addone example in \cite{smeil}.}
\label{lst:addone_smeil_example}
\end{listing}
\section{Clocked Networks}
As can be seen in the examples of the seven-segment \cspm{} code %TODO: Where can they see the code?
, no proceses are recursive. All processes run once and, unless errors occured, behaves as the \texttt{SKIP} process afterwards. This is part of the reason why the \texttt{addone} example cannot represent more than one clock cycle. To be able to verify more than one clock cycle it is essential that the processes are recursive. As explained in the csp background in Chapter \ref{chap:background}, recursive processes are simply processes which instead of behaving like the \texttt{SKIP} process, behaves like itself. An example of this can be seen in Listing \ref{lst:cspm_recursion}.
\begin{listing}
\begin{minted}{cspm_lexer.py:CSPmLexer -x}
Init = d ! 1 -> A(1)
A(x) = d ! x -> A(x+1)
\end{minted}
\caption{Example of the a recursive \cspm{} process which is initialised by the \texttt{Init} process.}
\label{lst:cspm_recursion}
\end{listing}

In theory an SME processes never stops running, but when simulating the SMEIL network it is of course not possible to simulate endless runtime. Therefore the developer indicates the number of clock cycles to simulate and the results should be seen as a snapshot of the process runtime. \\

In Listing \ref{lst:cspm_recursion} the process \texttt{A} performs an endless loop with no change to terminate. As mentioned in the previous, it is also essential to have a limited range of values for FDR4 to verify to avoid running out of space and if the example in Listing \ref{lst:cspm_recursion} was verified with FDR4, it would eventually run out of space. It is therefore crucial to model a structure that can drive the network and which can ensure the process terminate at the specified time. This is done with the \texttt{Clock} process. The clock process drives the network for a specific number of clock cycles and then terminates, which enforce all other processes to do the same, which will be explained shortly. \\


It, of course, is still necessary for a new version of TAPS to model a \cspm{} network that reflects the SME model and therefore it must adhere to the SME model structure. To model the global synchronicity in \cspm{} it is necessary to enforce a synchronising event where all processes synchronise before continuing. This synchronicity can be emulated by having a \texttt{sync} channel to emulate the rising and falling clock signal. All clocked processes in the network will be synchronised with the \texttt{sync} channel and, as previously introduced, when two processes are synhronised on a channel they must agree on communication. Therefore, all clocked processes must agree to synchronise before any process can continue.


----------------------------------------------------
----------------------------------------------------
----------------------------------------------------


(clock stuff)
initialised with a value and syncronise on the \texttt{clock} channel, which all other processes does as well. When the specified number of clock cycles has passed, the \texttt{Clock} process stops clocking and instead behaves as \texttt{SKIP}. This means that all other processes won't be able to syncronise on the \texttt{clock} channel anymore and therefore they will then instead behave as \texttt{SKIP} and that way the system terminates as planned.\\




The \texttt{clock} channel is used as a two-way clock synchronisation, where the same channel is used for syncronising up as well as down. Thus all processes syncronise before they read and before they write. The \texttt{Clock} process then needs to syncronise twice before incrementing its counter. \\

% (from verification in design)
% TODO: Maybe I should do it with the FD model, but in this case it does not really make sense, since all processes end after one iterations (they all SKIP). But it is worth mentioning it in the new system, because there it becomes relevant

% TODO: What happens if an SMEIL channel have been defined with an initial value?



% TODO: What if several processes write to the same output channel? How to handle the monitor process then?



% TODO: This might just belong in the new version!
% When the smaller process monitor networks have been created TAPS will then be able to synhronise other smaller monitor networks where there is shared communication. TAPS controls that all communication are handled within this new network. (Maybe it makes sense to generate smaller network which can then be synhronised. I mean where each smaller network is a process with a name, so it does not become so large nested.. )
% % TODO: Figure out how to make sure that all data are synchronised. It might be something about needing to have a linked list or something to keep all the data together. both stuff defined as parameter and communication defined by using the formal names.






\section{Clocked Processes}
% % TODO: Write that the read of a process does not make sense in the network and that we move it to be internal inside the process.

% %%% Generator processes - clock cycles
% %TODO: Write something here?
%
%



\section{The Bounds Problem}

When trying to verify that this system terminates as expected in FDR4, we came across an error while FDR4 was compiling the program.
FDR4 is complaining that a value, the system is trying to send, is not a part of the set of values defined for the channel.
The channels, used for communicating the value between the processes and the buffers, are defined for a specific range and since the 'Add' process has to read, calculate and write before it can terminate it will always write a value out that have been incremented with one.\\
For instance, if the channels were defined with the range \{0..5\} and the \texttt{Clock} process would stop after 10 clock cycles. This would mean that on the last clock cycle, before terminating, the 'Add' process would write a 6 onto the channel, which of course is not possible since the channel is defined only for the range \{0..5\}, so FDR4 complains about this.\\
However, what we experienced was that when the ranges of the channels were set to a larger number than the system would ever reach, within the defined number of clock cycles, FDR4 would still fail with the same reason.
This caused some problems since we were not able to verify the system and we were not interested in FDR4 trying to verify a communication that would never occur within the network.
Since we are syncronising the processes on the events in the channels, it seems odd that FDR4 still considers events which should not be possible to reach.  \\\\
After some time working with the problem and trying to understand the reason for FDR4s error message, we found that if we simply add a guard or an if-then-else statement that tests the value to be written, FDR4 will gladly verify the system and when using Probe on the network, it is clear that FDR4 does not consider the trace with the wrong values. \\

So the suggested solution, or fix, to this problem, is to add an if-then-else before all writes in a program. The statement then tests the value to be written against the max value of the range of the channel and if the value is not within the range, then the process behaves as the \texttt{SKIP} process, otherwise, it continues with write.
\begin{minted}[escapeinside=&&, mathescape=true]{cspm_lexer.py:CSPmLexer -x}
channel c : {0..20}

&$\vdots$&

    if (i+1) > 20  -- Check the upper limit of the channel
        then SKIP  -- SKIP if the value is above
        else (c_r ! (i+1) -> Add(i))) -- Otherwise write and continue
\end{minted}
In this case, it is only necessary to add an upper limit test, since the network only increments, but as a general rule, it would be necessary to test for both upper and lower bounds.


% The problem ocurred several times with different versions of the solution. Also, Ohm had the same problem with the Commstime problem. It makes sense why FDR4 wants to check, and maybe it is a way to save verification time: if it checks things in parallel or something.
% However it is not a good solution that we have to put in a upper/lower bound check that is actually never relevant.
% Of course it might be good, in any case, since the programmer then do not need to worry. On the other hand it might make the program look like it is terminating properly when it is actually failing because it tries to write a value that is not allowed. This case might happen, and if all other processes do not notice it and then also SKIP according to their specifications, then the verification passes even though it might be wrong.
% A solution might be to ensure that all processes synchronise before skipping, because then the problem (i think) would not occur, since the failing process would simply skip, then the other processes cannot skip because they have not syncrhonised yet. however, I am not sure this is possible since it might be that they either syncronise or they skip.



\section{Introducing Buffers}

For the 'Add' and 'Id' processes to comply with these states, they would have to read first, then the calculate phase, in which the 'Id' process does nothing, and then they would write the result onto a channel. A problem occurs, since they both have to read first, no one can read because no processes have written anything yet. To solve this, we implement two buffers which for each clock cycle reads the output that the process writes and then writes the value to a channel. Thus the buffer structure is the reverse of a 'normal' process since it will write and then read in a clock cycle. \\
If we give the buffers an initial value, they can begin the clock cycle writing the value which the 'Add' and 'Id' processes can read and thereby they will comply with the SME model structure.
The buffers will be instantiated will a 'dummy' value which is also how it is typically done in hardware. The dummy value is simply to indicate that the system should ignore the first clock cycle and then continue with the systems actual values. \\
Each process is also instantiated with a value, which is then used instead of the dummy value from the buffer process. After this initial cycle, the process loop will continue and the communication will hold according to the description of the network explained above.\\\\

Since the processes all must read, calculate and then write, the buffer processes, as mentioned before, must behave the opposite way. This means that the last write the processes make before they \texttt{SKIP} will be left in the buffers since there are no processes to read the value from the channels. This means that the buffers must be able to either write a value or \texttt{SKIP}.\\


\section{Clocked Addone Example}
\begin{figure}
\centering
\begin{tikzpicture}
   \node[main node, text width=.5cm] (add) {\small \texttt{add}};
   \node[main node, text width=.5cm] (id) [right = 4cm of add] {\texttt{id}};
   \node[mythinsquare] (bufd) at (2.7, 1.7) {$Buf_d$};
   \node[mythinsquare] (bufc) at (2.7, -1.5) {$Buf_c$};
   % \draw[fill] (0.7,0) circle [radius=0.07];

   \path[draw,thick, ->, bend right=25]
   (add) edge node {} (bufc);
   \path[draw,thick, ->, bend right=25]
   (bufc) edge node {} (id);


   \path[draw,thick, ->, bend right=25]
   (id) edge node {} (bufd);
   \path[draw,thick, ->, bend right=25]
   (bufd) edge node {} (add);


   \node[align=center, below, font=\scriptsize] at (1.5,1.3){\texttt{d\_write}};
   \node[align=center, below, font=\scriptsize] at (3.9,1.3){\texttt{d\_read}};
   \node[align=center, below, font=\scriptsize] at (1.5,-0.8){\texttt{c\_write}};
   \node[align=center, below, font=\scriptsize] at (3.9,-0.8){\texttt{c\_read}};
   % \node[align=center, below, text width=1.7cm] at (2.3,-0.9){\footnotesize\texttt{channel c}};
\end{tikzpicture}
\caption{The clocked \texttt{addone} network. The network have two proceses and two buffers which ensure the global synchronicity.}
\label{fig:addone_clocked}
\end{figure}




% Kenneths version, which I believe is how the SME model works, is having a process or bus in the middle og all steps. By using a dependency graph (Explain more?) it is possible to see which processes communicate to witch processes and, more importantly, in which order. For each communication step (or maybe for each communication) a process/bus will receive all writes. In SME a process can write several times to the same channel but only the last one before the clock signal will be written, the others are just overwritten. Since we have the dependency graph, we also know which processes we need communication from, and when the process have written all it has to write, then it sends a ready signal to the "bus" process, which then waits for all the ready signals (because it knows how many it should get. And if it is one process/bus pr. communication then it only needs one of course.). When all ready signals are in, the bus-process change behaviour and it is not writing instead of reading. It writes all possible values out and the processes that are supposed to receive the values (which we know from the dependency graph) will receive the values. And the processes then need to send a ready signal back to the bus process to let it know that it have read all it needed. When all ready signals are received, the bus process when change behaviour again and can read values once again.
% All these steps are intermediate steps within one clock cycle. So at the "end" of the dependency graph, the step looks similar to the others, but it is registered as the clock and the next clock cycle begins. In principal, all these steps could be the clock, since the step is the same, but a step is simply chosen to be the clock, based on the dependency graph.
% By treating the communication like this within a clock cycle, the values can propagate through the network and the internal state of the processes are also kept. The original TAPS version could only verify all input for a system, but if the system was internally affected by values from a previous clock cycle, then the system could not verify it. It is not a problem in the seven segment example, since no values are dependent on previous values. But the Addone network do depend on what happened in the last clock cycle.
% With this solution it is possible to verify a specific number of clock cycles.
%



%
% \chapter{Experiments and results}
% \label{chap:exp}
% %!TEX root = ../main.tex

In this chapter, I first present examples of verification for the seven segment display example as well as the addone example. Secondly, an experiment has been conducted to gain further insight into program size and validation time. The experimental setup and results are introduced with the three properties \texttt{verification time}, \texttt{number of states} and \texttt{maximum resident set size}.
% TODO: Mention how to use taps if i keep it here.

\section{Seven Segments Display Example Validation}
The seven segments example has been presented in different parts throughout this thesis. An illustration of the entire translated unclocked seven segments network can be seen in Figure \ref{fig:cspm-network}. The SMEIL representation of the same network can be seen in Chapter \ref{chap:analysis} in Figure \ref{fig:smeil_network}.
The unclocked \cspm{} network consists of 12 different processes, all created so that not only the network is simulated correctly, but also so the assertions are placed correctly. The input is represented by a triangle, since it transpiles from an SME process to a \cspm{} channel and is not represented as a process in this network. Each of the dotted squares represents the network of synchronizations for each \texttt{time} processes, which in itself is a process in \cspm{}. For each network, we have the \texttt{time} processes and two monitor processes, for example, $H$, $M_{H_1}$ and $M_{H_2}$.
\\

% Errornous example
\begin{listing}
\begin{minted}[escapeinside=||, mathescape=true]{cspm_lexer.py:CSPmLexer -x}
channel clock_out_val : {0..131071}

channel hours_out_first_digit : {0..3}
channel hours_out_second_digit : {0..15}
    |$\vdots$|

Hours(hours_in) =
let
    hours = hours_in / 3600
    |$\vdots$|

Hours_out_first_digit_monitor(c) =
    c ? x -> if 0 <= x and x <= 2 then SKIP else STOP
Hours_out_second_digit_monitor(c) =
    c ? x -> if 0 <= x and x <= 9 then SKIP else STOP

\end{minted}
\caption{Example of an erroneous version of the \texttt{Hours} process from the \cspm{} seven segment display example seen in Listing~\ref{lst:smeil} and in Listing~\ref{lst:cspm} in the appendix.}
\label{lst:cspm_error}
\end{listing}

In order to show that the verification is accurate, the example in Listing~\ref{lst:cspm_error} contains an error that results in FDR4 failing the verification. In Listing~\ref{lst:cspm_error} the example is only able to handle an input that is below 24 hours. This is because the calculation in the \texttt{Hours} process does not handle the wrap around at the 24\textsuperscript{th} hour. This means that if the input represents more than 24 hours, the assertions will fail in FDR4 because one seven segment display suddenly has to display two digits instead of one. An example of such could be the input \texttt{131071}, which represents 36 hours, 24 minutes and 31 seconds, or 1 day, 12 hours, 24 minutes and 31 seconds. When trying to assert the code from Listing~\ref{lst:cspm_error} in FDR4, the assertion fails. The counterexample, provided by FDR4, shows that the number 3 is communicated on \texttt{hours\_out\_first\_digit}, which is not allowed according to the monitor process on lines 12 and 13 in Listing~\ref{lst:cspm_error}.\\

This example of failure shows how verifying the solution with a tool like FDR4 actually catches errors that the programmer might have overseen. In this case, the error is simply corrected by adding \texttt{\% 24} on the end of line 9 in Listing~\ref{lst:cspm_error} and can be seen corrected in Listing~\ref{lst:cspm} in the appendix at line 15. Now when we try to assert the example in FDR4, it passes. By using modulo on the result, we ensure that we still get the accurate time of day, no matter how many full days the input represents.
The full SMEIL and \cspm{} code for the unclocked seven segment display example can be seen in Listing~\ref{lst:smeil} and in Listing~\ref{lst:cspm} in the appendix.

\begin{figure}[!ht]
  \centering
  \begin{tikzpicture}
    \node [mytriangle] (I) at (0, 0) {$I$};

    %%%%

    \node [mycircle, above right=25ex and 25ex of I] (H) {$H$};

    \node [mysquare, above right=1.5ex and 25ex of H] (H_d1) {$D_{H_1}$};
    \node [mysquare, below right=1.5ex and 25ex of H] (H_d2) {$D_{H_2}$};
    \node [mycircle, above right=3ex and 7.5ex of H] (H_m1) {$M_{H_1}$};
    \node [mycircle, below right=3ex and 7.5ex of H] (H_m2) {$M_{H_2}$};
    \node [draw, red, thick, dotted, fit=(H)(H_m1)(H_m2), inner sep=0.5cm] {};
    \node [right=15ex of H, red] {$N_{hours}$};

    \draw [myarrow, smooth] (I) to[out=0, in=180] (H);

    \draw [myarrow, smooth] (H) to[out=0, in=180] coordinate[midway, black!50, draw, shape=circle, inner sep=0pt, minimum size=5pt](H_mp1) (H_d1);
    \draw (H_m1) -- (H_mp1)  [black!50];
    \draw [myarrow, smooth] (H) to[out=0, in=180] coordinate[midway, black!50, draw, shape=circle, inner sep=0pt, minimum size=5pt](H_mp2) (H_d2);
    \draw (H_m2) -- (H_mp2)  [black!50];

    %%%%

    \node [mycircle, right=23.2ex of I] (M) {$M$};

    \node [mysquare, above right=1.5ex and 25ex of M] (M_d1) {$D_{M_1}$};
    \node [mysquare, below right=1.5ex and 25ex of M] (M_d2) {$D_{M_2}$};
    \node [mycircle, above right=3ex and 7.5ex of M] (M_m1) {$M_{M_1}$};
    \node [mycircle, below right=3ex and 7.5ex of M] (M_m2) {$M_{M_2}$};
    \node [draw, red, thick, dotted, fit=(M)(M_m1)(M_m2), inner sep=0.5cm] {};
    \node [right=15ex of M, red] {$N_{minutes}$};

    \draw [myarrow, smooth] (I) to[out=0, in=180] (M);

    \draw [myarrow, smooth] (M) to[out=0, in=180] coordinate[midway, black!50, draw, shape=circle, inner sep=0pt, minimum size=5pt](M_mp1) (M_d1);
    \draw (M_m1) -- (M_mp1)  [black!50];
    \draw [myarrow, smooth] (M) to[out=0, in=180] coordinate[midway, black!50, draw, shape=circle, inner sep=0pt, minimum size=5pt](M_mp2) (M_d2);
    \draw (M_m2) -- (M_mp2)  [black!50];

    %%%%

    \node [mycircle, below right=24.5ex and 24.5ex of I] (S) {$S$};

    \node [mysquare, above right=1.5ex and 25ex of S] (S_d1) {$D_{S_1}$};
    \node [mysquare, below right=1.5ex and 25ex of S] (S_d2) {$D_{S_2}$};
    \node [mycircle, above right=3ex and 7.5ex of S] (S_m1) {$M_{S_1}$};
    \node [mycircle, below right=3ex and 7.5ex of S] (S_m2) {$M_{S_2}$};
    \node [draw, red, thick, dotted, fit=(S)(S_m1)(S_m2), inner sep=0.50cm, inner ysep=0.5cm] {};
    \node [right=15ex of S, red] {$N_{seconds}$};

    \draw [myarrow, smooth] (I) to[out=0, in=180] (S);

    \draw [myarrow, smooth] (S) to[out=0, in=180] coordinate[midway, black!50, draw, shape=circle, inner sep=0pt, minimum size=5pt](S_mp1) (S_d1);
    \draw (S_m1) -- (S_mp1)  [black!50];
    \draw [myarrow, smooth] (S) to[out=0, in=180] coordinate[midway, black!50, draw, shape=circle, inner sep=0pt, minimum size=5pt](S_mp2) (S_d2);
    \draw (S_m2) -- (S_mp2)  [black!50];
  \end{tikzpicture}
  \caption{A seven segment display clock network in \cspm{}. $I$ represents the input channel. $N_{hours}$, $N_{minutes}$ and $N_{seconds}$ represent the network processes with $H$, $M$ and $S$ as the \texttt{time} processes. The results from the \texttt{time} processes are communicated to the displays. The displays are represented by a square since they are not actual \cspm{} processes. Each display communication also has a monitor process which assert the legal communication values.}
  \label{fig:cspm-network}
\end{figure}


\section{Addone Example Validation}
The \texttt{addone} example has been introduced in Chapter \ref{chap:clock} and an illustration of the clocked network with its monitor processes can be seen in Figure \ref{fig:addone_clocked_monitor} in the chapter. As explained, the \texttt{addone} example does not translate well in the initial version of TAPS and therefore the clocked version was created.
The difference between a clocked an unclocked network is that FDR is able to verify different internal states in the clocked version which suits this cyclic network perfectly.
The \texttt{addone} example differs from the seven segment example in different ways. It does not require an input range because the cyclic network is instantiated with the initial values and not a data generator process. The cyclic structure of the \texttt{addone} example causes the values to circulate and increase indefinitely if not restricted. It is not possible to represent an indefinite amount of values on hardware buses and therefore it must be restricted to a specific bit size. If the network is not restricted to specific values, the verification will simply be based on the values from the simulation of the SMEIL program, which can cause an unnecessary failure in the verification. \\

In an unrestricted \texttt{addone} network, if an SMEIL simulation of the \texttt{addone} example resulted in the internal values reached 20 and the FDR4 verification verified more clock cycles than the SMEIL simulation, this would cause the values of the FDR4 verification to exceed the observed values, which would cause FDR4 to fail the verification. It is therefore necessary that the user make an informed choice as to the number of clock cycles to simulate in SMEIL but also to verify in FDR4. The number of simulated clock cycles and FDR4 verified clock cycles do not have to be equal, but in some cases, it might be the obvious choice. \\

As mentioned, the \texttt{addone} example should be restricted and so a \texttt{\% 5} statement has been added to the computation in the \texttt{add} process. Listing \ref{lst:addone_mod_example} shows the simulated \texttt{addone} network with a restriction added in the \texttt{add} process. This example is identical to the example in Listing \ref{lst:addone_smeil_example} in Chapter \ref{chap:clock} besides the restriction. \\

With this enhanced example, the SMEIL simulation provides reasonable observed values which can be used for the translation to \cspm{}. In Listing \ref{lst:cspm_addone_restricted} a subset of the translated \texttt{addone} example can be seen which includes the restriction.
% TODO: Add something about where the entire code can be seen
This restriction ensures that the verification will succeed even if FDR4 verifies more than 10 clock cycles. If the restriction is removed and FDR4 verifies more than 10 clock cycles, the verification fails as expected. \\

This example is somewhat strange to verify with FDR4 since it does not take advantage of the state space exploration that FDR4 provides. The lack of an input range for the system means that for each clock cycle a new state machine is verified, but only the internal values change. In this example, FDR4 only verifies the relation between internal values for each clock cycles with no external influence. The advantage of the clocked structure is that values, which might cause failures after a certain amount of clock cycles, are now possible to verify, which was not possible with the initial version of TAPS. \\

It is easy to see that this example never fails with the added restriction, but the example clearly introduces the clocked version of TAPS and how it is possible to verify clocked networks in FDR4 and therefore it still provides value to this thesis.

\begin{listing}
\begin{minted}{smeil_lexer.py:SMEILLexer -x}

proc id (in input)
    bus output {
        val: u4 = 0 range 0 to 4;
    };
    var from_add: u4 range 0 to 4;
{
    from_add = input.val;
    output.val = from_add;
}

proc add (in input, const constant)
    bus output {
        val: u4 = 0 range 0 to 4;
    };
    var from_id: u4 range 0 to 4;
{
    from_id = (input.val + constant) % 5;
    output.val = from_id;
}

network addone_network ()
{
    instance id of id(add.output);
    instance add of add(id.output, constant: 1);
}
\end{minted}
\caption{The restricted SMEIL network \texttt{addone\_network} similar to the example in Listing \ref{lst:addone_smeil_example}.}
\label{lst:addone_mod_example}
\end{listing}

\begin{listing}
\begin{minted}{cspm_lexer.py:CSPmLexer -x}
channel sync
channel d_read, c_read : { -1..15}
channel d_write, c_write : { -1..15}

DUM_VAL = -1

Add(i, input_channel) =
    (sync ->
     input_channel ? x ->
     sync ->
        if (x == DUM_VAL) -- initial value
            then (
                let
                    var = i
                within
                    var <= 15 &
                        c_read ! var -> Add(i, input_channel))
            else (
                let
                    var = (x + 1) % 5 -- % restriction
                within
                    var <= 15 &
                        c_read ! var -> Add(i, input_channel))
    )
    [] SKIP


c_read_monitor(c) =
    (c ? x ->
    (0 <= x and x <= 5 or x == -1) &
        c_read_monitor(c)
    ) [] SKIP

\end{minted}
\caption{Sections of the translated \texttt{addone} network. The \texttt{Add} process have restrictions included to ensure no values above 5. The monitor process defined this range along with the acceptance of the dummy value -1. This example have been manually translated due to limitations of the clocked version of TAPS.}
\label{lst:cspm_addone_restricted}
\end{listing}
\section{Problem Size Experiments}
The examples presented in this thesis are not very complex, but they provide a suitable introduction to the translation and the verification in FDR4. More complex examples would have required a substantial introduction and it would not be as straightforward to understand the translations. The challenge with verification of a state space is to keep the verification time to a minimum. This does not only apply for FDR4 but for model checkers in general. FDR4 performs different kinds of internal optimisations on the networks to minimise the state space before the refinement checks. FDR4 also provides several compression algorithms to provide further compression of larger problems. \\

I have performed some experiments on the seven segments example to examine the behaviour in FDR4.\\
% TODO: Rewrite if I cannot use the unclocked version!
Both experiments have been run on a (info to come) machine with no other programs running. The experiments consist of measuring three different properties from the FDR4 verification. The first property is \texttt{verification time} which is measured by the \texttt{time} command. Even though all experiments have been performed on the same machine, to avoid potential confusion, the GNU \texttt{time} command was used instead of the built-in \texttt{time} command shells as bash and zsh provide. This property will provide insight into the size of feasible inputs for a \cspm{} network to be verified.\\

The second property is \texttt{number of visited states} which is a piece of information FDR4 provides.
As previously explained, FDR4 performs compression to minimise the state space.
This property provides an insight into the amount of work FDR4 performs when verifying a network and so it is interesting to learn how the number of visited states corresponds with the verification time. This will also give an insight into the inner workings of FDR4 and how the state space compression behaves.
Because the seven segments example is divided into three different assertions, one for each \texttt{time} process, FDR4 provides a separate \texttt{number of visited states} for each verified \texttt{time} network. \\

The last property is \texttt{resident set size} which is also provided by the GNU \texttt{time} command. The resident set size defines the amount of memory the process currently holds, and it will provide an insight into how much memory FDR requires to verify the network and how the memory usage behaves as the input size increases. If FDR4 requires too much memory, it is not feasible to verify larger problems unless compression algorithms can reduce the state space. \\

The experiment has been designed to keep the internal system fixed and only increase the size of the input range for the system. This means that FDR4 will verify increasingly more values, but the network in itself stays the same.
The lower bound of the input range will be fixed at 0 and the upper bound will be increased with 500 for each verification until 15000. The input range \{0..15000\} represents 4 hours and 10 seconds. All three property values are gathered after FDR4 finish the verification.
\subsection{Unclocked Experiment}
The full code for the unclocked seven segments display example can be seen in Listing \ref{lst:cspm} in Appendix. %TODO: Figure out how this should be presented. One appendix or several?
The unclocked seven segments display example consists of three \texttt{time} processes with associated monitor processes. Each \texttt{assert} function verifies the process monitor network described in Chapter \ref{chap:design}.
\paragraph{Number of states}
For each verification, three assertions are performed within the seven segment display example, one for each \texttt{time} process monitor network. With the unclocked example, all three verifications contain the same number of visited states and so this property will not be divided into three different results.\\

Figure \ref{fig:unclocked_states} presents the results of the \texttt{number of states} property. From this graph, it is very clear how the state space increases linearly with the input range. This result means that FDR4 is not able to compress the state space further with the increase of input. A reason for FDR4 not providing any additional state space compression could be that the example is too simple and since no input is repeated, the number of states remain the same. This does, however, show how a problem can quickly become very large within FDR4 which is something a user must consider when choosing the data to verify.
\begin{figure}
    \includegraphics[width=0.98\textwidth]{./figures/15-11-2018/unclocked_number_of_states.jpg}
\caption{y}
\label{fig:unclocked_states}
\end{figure}
\paragraph{Verification time}
In Figure \ref{fig:unclocked_verification} the \texttt{verification time} results can be seen. The graph represents the verification time in seconds for each increase in the input range. As can be seen the verification time increase exponentially with the input values. Since the number of states visited is increasing linearly it can seem odd that the verification time does not follow that same pattern. However, besides the refinement checking of the GLTS which will increase with the number of states, FDR4 must compile the network and generate the GLTS. It is reasonable to expect that the larger the state space, the more effort for FDR4 to complete all the steps of the verification. Therefore these results are consistent with what could be expected.
\begin{figure}
    \includegraphics[width=0.98\textwidth]{./figures/15-11-2018/unclocked_verification_time.jpg}
\caption{x}
\label{fig:unclocked_verification}
\end{figure}
\paragraph{Maximum resident set size}
The result from this property can be seen in Figure \ref{fig:unclocked_resident_size}. These results are not fitted to a perfect line as well as the other two experiment properties. It is clear that the amount of memory used for the verification grows with an increase in the input range. It is also somewhat consistent until around 10000 in upper bound limit. This fluctuation could be caused by some internal structure in FDR4 or it could be a result of other requirements within the machine that is running the verification. Unfortunately, FDR4 does not provide a lot of information about the internal workings and so it can be very difficult to examine these results further. However, the results are overall consistent with the results from both \texttt{number of states} and \texttt{verification time}.
\begin{figure}
    \includegraphics[width=0.98\textwidth]{./figures/15-11-2018/unclocked_maximum_resident_set_size.jpg}
\caption{y}
\label{fig:unclocked_resident_size}
\end{figure}



\subsection{Clocked Experiment}
% TODO when I know if I can use this example
\subsection{Results}
% TODO when I know I can use this example

% TODO: I should also still mention something about how the increase in number of processes was or was not feasible.



\newpage
\section{How to use TAPS}
The required dependencies for generating the ANTLR4 parser, running TAPS and verification with FDR4 are listed below:
\begin{itemize}
    \item ANTLR4
    \item Python 2.7
    \item ANTLR Python runtime
    \item FDR4
\end{itemize}

ANTLR4 and FDR4 can both be downloaded from their websites where installation instructions are also provided.
ANTLR4 can be found at \url{https://www.antlr.org/} and FDR4 at \url{https://www.cs.ox.ac.uk/projects/fdr/}.\\
It is also necessary to download the Python runtime from \url{https://pypi.org/project/antlr4-python2-runtime/}.

Assuming that an ANTLR4 alias has been created as the ANTLR4 installation suggest, the parser and lexer can be generated with ANTLR4 using the command: {\ttfamily antlr4 -Dlanguage=Python2 - visitor -no-listener Smeil.g4.}
This will create all the required parser and lexer files as well as the visitor methods. This step is only necessary if the .g4 grammar file has been modified.\\

When the ANTLR4 files have been generated TAPS can be used directly with a well-formed SMEIL program with the command: {\ttfamily python taps.py input.sme output.csp}. The system requires both an input file as well as an output file. If the output file does not exist it will be created by TAPS. \\

The resulting \cspm{} file can be verified in FDR either by the command line tool or by the FDR4 tool which is a graphical tool. The command line tool can be used by the command \texttt{refines output.csp}. There are several options to adjust the output of the command. The FDR4 command line tool is mostly used to quickly check if a network passes the verification because it is difficult to navigate the counterexamples. The FDR4 graphical tool provides a better visualisation of counterexamples and the ProBE visualiser can be called directly from the FDR graphical tool.


%
% \chapter{Discussion}
% 
\section{Usability of TAPS}
TAPS is still in early development and more examples must be verified and corner cases must be handled. In spite of this, it presents the advantages gained from automatic translation from SMEIL to \cspm{}. To my knowledge, transpiling from a hardware description language to a specification language like CSP have not been done successfully before and the advantages of such a system are many. When using TAPS, it is not necessary to develop a test bench for the system and it is not necessary to develop the specification seperately from the hardware model. The workflow from hardware model to formal verification have been simplifies dramatically with TAPS. I believe that it will increase the usage of hardware description languages to be able to perform formal verification with no extra development steps. \\

There are limitations to TAPS and it is possible for failures to happen with systems verifies with TAPS. The bottlenecks of the verification lies both in the simulation values as well as in the number of verified clock cycles. The values verified in FDR4 is based on the observed values from the simulation and so if the SMEIL simulation is not representing all possible values, the verification will not be able to verify these. It will be difficult to find a balance between how long to simulate the SMEIL network and how critical failures can be. In some systems it will not be possible to know if all corner cases have been found in the simulation but FDR4 might provide insight into some corner cases missing in the simulation. The same balance can be difficult to find between how many clock cycles to simulate and how devestating failures can be. The consequence of choosing too low a number of clock cycles to verify can be failures in the system that FDR4 was not able to verify because it never verified that specific state machine where the failure happens.
The answer to this dilemma will be different in all cases and it will ultimately be up to the user to choose. \\

The reason it is possible to create the relatively simple translations from SMEIl to \cspm{} lies in the fact that SME is defined from the CSP language and therefore, as previosuly mentioned, all SME models will have an equivalent CSP model. Having the same basis on both sides of the translation results in a much smoother translation.

\section{Clocked or Unclocked \cspm{} Systems}
The initial idea of the design of TAPS was to build a system which did not have to enforce a global synchronus clock structure since it would create a much more complex structure. As the implementation of the initial version of TAPS progressed it was clear that the design was too simple. The set of verifiable problems in the initial version of TAPS are too small to have an actual impact.\\

The clocked version of TAPS is still able to verify the set of problems verifiable with the initial version of TAPS. By only verifying one clock cycle, the results are the same as for the unclocked version of TAPS, which is a great result from the design structure of the clocked version of TAPS. \\

The clocked version of TAPS becomes more complex and the user must define the number of clock cycles to verify in TAPS, which provides more uncertainty in the verification. If the user does not choose to verify enough clock cycles potential failures might not be caught by FDR4. The number of clock cycles to verify is a balance between the input range for the system, the time requirement of the verification as well as the complexity of the system. \\

It is a huge advantage that the clocked TAPS system are able to reuse such a large part of the initial design. What actually needed to be done was to extend the system and not rewrite it, which also made it feasible to design within the timeframe of this thesis. It is, however, clear that the set of verifiable problems with the clocked TAPS system are far bigger than the initial version of TAPS and so there exist no doubt that this is the system to contine development on. The clocked version of TAPS do increase the complexity, both in the translated \cspm{} but also in the structures of the translations and the amount of corner cases to handle. It is only feasible to translate a clocked network becasue it is automatically generated and even so it is difficult to ensure all the different SMEIl structures are handled in TAPS.
The clocked \cspm{} network provide a better comparability with the original SMEIL network, which I belive is a huge advantage for the further development. The user do not necessarily need to spend time understanding the \cspm{} network, but when developing and testing new features of TAPS, it is a big advantage to be able to compare the behavior of the SMEIL simulation to that of the FDR4 verification. \\

% Maybe add more here when I know if I can use  the clocked example

The clocked version of TAPS do have its pitfalls which require some ingenuity to ensure proper code genration. The advantages of having a larger set of verifiable problems are without a doubt worth the increased complexity.
\section{From Pure SMEIL to Co-Simulation}
Co-simulation is a big feature of SMEIL and yet I focused the work in this thesis only on pure SMEIL support. The reason behind only working with pure SMEIL was simply to have a simple base to start with so that I could ensure that the translation would be as accurate as possible. Since pure SMEIL does not allow for a lot of functionality, the examples also quickly became small enough for easy inspection of the resulting \cspm{} program. When translating from a language like SMEIL to \cspm{} it very quickly becomes overwhelming. The FDR4 traces very quickly became too complex to understand exactly what was going on and why. Because of this I believe that the decicion to only focus on pure SMEIL structures was a good one.
With the current version of TAPS it would be possible to simulate the input from the exposed buses in a co-simulated program so that it is actually a pure SMEIL program but the input is the same as the co-simulated program would input to the SMEIL part.\\

When TAPS become more advanced it should definitely be extended to also include co-simulated programs. TAPS will not have any real value to the industry or adademia before this happens. However I do believe that the extension to support co-simulated programs is possible. A huge benefit lies in the fact that the SME model is a general model which spans all the different SME implementations. The way TAPS translate the processes into three different sections will always fit the SME model, no matter which language TAPS translate from.
There will of course be a lot of challenges in handling the data and communication that spans across the SME implementations in co-simulation, and even more so, the general network in \cspm{} will also be a great challenge. However, since it is very precisely defined which buses communicate across the two implementations, it might help with generating the \cspm{} network in a co-simulated environment.\\

Focusing on pure SMEIL programs have definitely led to a more manageable set of problems and extending this to co-simulated problems seems like the perfect next step.

\section{Verification FDR4}
FDR4 is a 


%
% \chapter{Future work}
% %!TEX root = ../main.tex

Different SME implementations is in active development and so the SMEIL language will be extended and improved to match the new features of the different SME implementations. As the SMEIL language becomes more comprehensive and supports more features, TAPS should be kept in line with the advancements of SMEIL.\\
The automatic verification provided by TAPS, decrease a mayor workload in testing and verifying hardware models in the SME model and therefore, it is also relevant to structure the SMEIL implementation towards better FDR4 results, while still maintaining the basic SME model structure. \\

The extended version of TAPS provides an extended verification of the different internal states within a hardware model. It was introduced rather late in the project and therefore the development have not been as extensive as the initial version of TAPS. A substantial groundwork have been laid in providing the design structures for the extended version, making further development more straightforward. It is, however, obvious that future work should include providing a full implementation of the extended TAPS system. \\

Besides further development of the extended version of TAPS, more advanced and in-depth examples should be developed in SMEIL in order to understand the limitations of the translation as well as FDR4.\\

FDR4 also provides the possibility of integration with other tools using the FDR4 API which is currently available for C++, Java and Python. The FDR4 API is currently not used in TAPS, but it is an obvious choice to extend TAPS to use the FDR4 API to provide a more clean workflow. Because TAPS have been developed for FDR4 specifically, no other verification tool will match the current translation structures and therfore the current version of TAPS would benefit from the available API.\\

As described in Chapter \ref{chap:background}, SMEIL was mainly developed to provide co-simulation with other SME implementations. TAPS should be augmented to support co-simulation. The optimal solution would be for TAPS to provide translations from other SME implementations as well as SMEIL, in order to translate the entire co-simulated network into \cspm{}. A simpler solution would be to translate the SMEIL code of the co-simulation along with the data representing the communication between the different SME implementations. This solution would not provide verification for the total network but it is very similar to the pure SMEIL translation that TAPS can currently provide.\\


Another point for future work is to extend the different assertion possibilities within TAPS. Currently only channel communication can be verified, but as described in Chapter \ref{chap:implementation} %TODO: Make sure I have written this!
these monitor processes does have its limitations because each monitor process can verify a value, but it is not currently possible to verify the combination of values. Therfore it would be an advantage to extend TAPS to define more advanced assertions to verify values over multiple channels. \\

In future work, it would be an advantage to extend TAPS to support software-hardware co-design. The idea behind software-hardware co-design is that hardware and software is designed in parallel so that both can be implemented on either hardware or software depending on what is most suited. If SMEIL and TAPS was extended to support this, then the communication between software and hardware would be possible to verify with FDR4. \\

When verifying a system in FDR4, it can be crucial for the developer to know what values and states have been verified. It is therefore desirable to have TAPS generate a human-readable report on the ranges and communications that are verified with TAPS. This would also give the developer a possibility of better evaluating the number of clock cycles to verify in FDR4.
This report could become a standard addition to the documentation of the developed system, which would give a programmer an easy overview of a complicated system and would also allow for easier contemplation over the system.
%
% \chapter{Conclusion}
% We have presented a transpiler that transpiles SME intermediate language (SMEIL) into \cspm{} for then to use the Failure-Divergences Refinement tool (FDR4) to assert properties in a \cspm{} network. We provide a simple approach that makes it more accessible for software programmers to program hardware and thereby bridging a gap between software programmers and the needs of the industry.
Instead of having to create advanced test-benches, our tool provides a simple way to verify the hardware model via FDR4s assertion functionalities. We can assert that the observed values of a channel, in a simulated SMEIL program, are in fact the only possible values communicated on that specific channel. We have also shown this to work in an example case of a seven segment display.

%
% \chapter*{Acknowledgements}
% % Thanks to Uwe Zimmermann who made the seven segment example in TikZ on \url{http://www.texample.net/tikz/examples/segment-display/}.


\newpage
citation for citations\cite{Bengtsson1995}


%%%%%%%%%%%%%%%%
% Bibliography %
%%%%%%%%%%%%%%%%

% \clearpage
% \addcontentsline{toc}{chapter}{References}
% \printbibliography[title={References}]

\newpage
\bibliographystyle{abbrv}
\bibliography{library}

%%%%%%%%%%%%%%%%%%%%
% Include Appendix %
%%%%%%%%%%%%%%%%%%%%
% \appendix
% % Appendix
\chapter{How to use TAPS}
The required dependencies for generating the ANTLR4 parser, running TAPS and verification with FDR4 are listed below:
\begin{itemize}
    \item ANTLR4
    \item Python 2.7
    \item ANTLR Python runtime
    \item FDR4
\end{itemize}

ANTLR4 and FDR4 can both be downloaded from their websites, where installation instructions are also provided.
The ANTLR4 project can be found at \url{https://www.antlr.org/} and the FDR4 Project at \url{https://www.cs.ox.ac.uk/projects/fdr/}.
It is also necessary to download the ANTLR4 Python runtime from \url{https://pypi.org/project/antlr4-python2-runtime/}.
Assuming that an ANTLR4 alias has been created as the ANTLR4 installation suggest, the parser and lexer can be generated with ANTLR4 using the command: {\ttfamily antlr4 -Dlanguage=Python2 - visitor -no-listener Smeil.g4.}
This will create all the required parser and lexer files, as well as the visitor methods. This step is only necessary if the .g4 grammar file has been modified.\\

When the ANTLR4 files have been generated, TAPS can be used directly with a well-formed SMEIL program with the command: {\ttfamily python taps.py input.sme output.csp}. The system requires both an input file and an output file. If the output file does not exist, it will be created by TAPS. \\

The resulting \cspm{} file can be verified in FDR4, either by the command line tool or by the FDR4 tool, which is a graphical tool. The command line tool can be used by the command \texttt{refines output.csp}. There are several options to adjust the output of the command. The FDR4 command line tool is mostly used to quickly check if a network passes the verification, because it is difficult to navigate the counterexamples. The FDR4 graphical tool provides a better visualisation of counterexamples, and the ProBE visualiser can be called directly from the FDR4 graphical tool.

\chapter{Seven Segments Display Example Full code}
\section*{SMEIL code}
\begin{minted}{smeil_lexer.py:SMEILLexer -x}
proc clock ()
    bus clock_out {val: u17 range 1 to 86401;};
    var i: u17 = 0 range 0 to 86401;
{
    i = i + 1;
    clock_out.val = i;
}


proc hours (in hours_in)
    bus hours_out {first_digit: u2 range 0 to 2;
                   second_digit: u4 range 0 to 9;};
    var hours: u5 range 0 to 23;
    var hours_first_temp: u2 range 0 to 2;
    var hours_second_temp: u4 range 0 to 9;
{
    hours = hours_in.val / 3600 % 24;
    hours_first_temp = hours / 10;
    hours_second_temp = hours % 10;
    hours_out.first_digit = hours_first_temp;
    hours_out.second_digit = hours_second_temp;
}


proc minutes (in minutes_in)
    bus minutes_out {first_digit: u3 range 0 to 5;
                     second_digit: u4 range 0 to 9;};
    var minutes: u6 range 0 to 59;
    var minutes_first_temp: u3 range 0 to 5;
    var minutes_second_temp: u4 range 0 to 9;

{
    minutes = minutes_in.val / 60 % 60;
    minutes_first_temp = minutes / 10;
    minutes_second_temp = minutes % 10;
    minutes_out.first_digit = minutes_first_temp;
    minutes_out.second_digit = minutes_second_temp;
}


proc seconds (in seconds_in)
    bus seconds_out {first_digit: u3 range 0 to 5;
                     second_digit: u4 range 0 to 9;};
    var seconds: u6 range 0 to 59;
    var seconds_first_temp: u3 range 0 to 5;
    var seconds_second_temp: u4 range 0 to 9;
{
    seconds = seconds_in.val % 60;
    seconds_first_temp = seconds / 10;
    seconds_second_temp = seconds % 10;
    seconds_out.first_digit = seconds_first_temp;
    seconds_out.second_digit = seconds_second_temp;
}


network clock_network ()
{
    instance g of clock();
    instance h of hours(g.clock_out);
    instance m of minutes(g.clock_out);
    instance s of seconds(g.clock_out);
}

\end{minted}
\captionof{listing}{The full SMEIL code used for transpiling in the seven segment display example.\label{lst:smeil}}

\section*{Unclocked \cspm{} code}
\begin{minted}{cspm_lexer.py:CSPmLexer -x}
channel clock_out_val : {0..131071}

channel hours_out_first_digit : {0..3}
channel hours_out_second_digit : {0..15}

channel minutes_out_first_digit : {0..7}
channel minutes_out_second_digit : {0..15}

channel seconds_out_first_digit : {0..7}
channel seconds_out_second_digit : {0..15}


Hours(hours_in) =
let
    hours = hours_in / 3600  % 24
    hours_first_temp = hours / 10
    hours_second_temp = hours % 10
within
    hours_out_first_digit ! hours_first_temp ->
    hours_out_second_digit ! hours_second_temp ->
    SKIP

Hours_out_first_digit_monitor(c) =
    c ? x -> if 0 <= x and x <= 2 then SKIP else STOP
Hours_out_second_digit_monitor(c) =
    c ? x -> if 0 <= x and x <= 9 then SKIP else STOP


Minutes(minutes_in) =
let
    minutes = minutes_in / 60  % 60
    minutes_first_temp = minutes / 10
    minutes_second_temp = minutes % 10
within
    minutes_out_first_digit ! minutes_first_temp ->
    minutes_out_second_digit ! minutes_second_temp ->
    SKIP

Minutes_out_first_digit_monitor(c) =
    c ? x -> if 0 <= x and x <= 5 then SKIP else STOP
Minutes_out_second_digit_monitor(c) =
    c ? x -> if 0 <= x and x <= 9 then SKIP else STOP


Seconds(seconds_in) =
let
    seconds = seconds_in % 60
    seconds_first_temp = seconds / 10
    seconds_second_temp = seconds % 10
within
    seconds_out_first_digit ! seconds_first_temp ->
    seconds_out_second_digit ! seconds_second_temp ->
    SKIP

Seconds_out_first_digit_monitor(c) =
    c ? x -> if 0 <= x and x <= 5 then SKIP else STOP
Seconds_out_second_digit_monitor(c) =
    c ? x -> if 0 <= x and x <= 9 then SKIP else STOP


N_hours = clock_out_val ? variable ->
          (Hours(variable)
          [| {| hours_out_first_digit|} |]
          Hours_out_first_digit_monitor(hours_out_first_digit))
          [| {| hours_out_second_digit|} |]
          Hours_out_second_digit_monitor(hours_out_second_digit)

assert SKIP [F= N_hours \ Events


N_minutes = clock_out_val ? variable ->
            (Minutes(variable)
            [| {| minutes_out_first_digit|} |]
            Minutes_out_first_digit_monitor(minutes_out_first_digit))
            [| {| minutes_out_second_digit|} |]
            Minutes_out_second_digit_monitor(minutes_out_second_digit)

assert SKIP [F= N_minutes \ Events


N_seconds = clock_out_val ? variable ->
            (Seconds(variable)
            [| {| seconds_out_first_digit|} |]
            Seconds_out_first_digit_monitor(seconds_out_first_digit))
            [| {| seconds_out_second_digit|} |]
            Seconds_out_second_digit_monitor(seconds_out_second_digit)

assert SKIP [F= N_seconds \ Events

\end{minted}
\captionof{listing}{The full unclocked \cspm{} code after transpiling the seven segment display example, as seen in Listing~\ref{lst:smeil} in the appendix.\label{lst:cspm}}
\section*{Clocked \cspm{} code}
\begin{minted}{cspm_lexer.py:CSPmLexer -x}
channel hours_hours_out_first_digit : {0..3}
channel hours_hours_out_second_digit : {0..15}
channel minutes_min_out_first_digit : {0..7}
channel minutes_min_out_second_digit : {0..15}
channel seconds_sec_out_first_digit : {0..7}
channel seconds_sec_out_second_digit : {0..15}
channel clock_c_out_val : { 0..1000}
channel sync

Clock(1) = SKIP
Clock(n) =  sync -> sync -> Clock(n+1)

Hours(input_channel) =
    (sync ->
     input_channel ? hours_in ->
     sync ->
        let
            hours = ( hours_in / 3600 ) % 24
            hours_first_temp = hours / 10
            hours_second_temp = hours % 10
        within
            (hours_first_temp <= 3) &
                (hours_hours_out_first_digit ! hours_first_temp ->
                (hours_second_temp <= 15) &
                    (hours_hours_out_second_digit ! hours_second_temp ->
                    Hours(input_channel)
                    )
                )
    ) [] SKIP

hours_hours_out_first_digit_monitor(c) =
    (c ? x ->
    (0 <= x and x <= 2 or x == -1) &
        hours_hours_out_first_digit_monitor(c)
    ) [] SKIP

hours_hours_out_second_digit_monitor(c) =
    (c ? x ->
    (0 <= x and x <= 9 or x == -1) &
        hours_hours_out_second_digit_monitor(c)
    ) [] SKIP


N_hours =
        (
            (
                Hours(clock_c_out_val)
                [|{| hours_hours_out_first_digit |}|]
                hours_hours_out_first_digit_monitor(hours_hours_out_first_digit)
            )
            [|{| hours_hours_out_second_digit |}|]
            hours_hours_out_second_digit_monitor(hours_hours_out_second_digit)
        )
        [|{| sync |}|]
        Clock(0)

assert SKIP [F= N_hours \ Events


Minutes(input_channel) =
    (sync ->
     clock_c_out_val ? min_in ->
     sync ->
        let
            min = ( min_in / 60 ) % 60
            min_first_temp = min / 10
            min_second_temp = min % 10
        within
            (min_first_temp <= 7) &
                (minutes_min_out_first_digit ! min_first_temp ->
                (min_second_temp <= 15) &
                    (minutes_min_out_second_digit ! min_second_temp ->
                    Minutes(input_channel)
                    )
                )
    ) [] SKIP

minutes_min_out_first_digit_monitor(c) =
    (c ? x ->
    (0 <= x and x <= 5 or x == -1) &
        minutes_min_out_first_digit_monitor(c)
    ) [] SKIP

minutes_min_out_second_digit_monitor(c) =
    (c ? x ->
    (0 <= x and x <= 9 or x == -1) &
        minutes_min_out_second_digit_monitor(c)
    ) [] SKIP

N_minutes =
        (
            (
                Minutes(clock_c_out_val)
                [|{| minutes_min_out_first_digit |}|]
                minutes_min_out_first_digit_monitor(minutes_min_out_first_digit)
            )
            [|{| minutes_min_out_second_digit |}|]
            minutes_min_out_second_digit_monitor(minutes_min_out_second_digit)
        )
        [|{| sync |}|]
        Clock(0)

assert SKIP [F= N_minutes \ Events


Seconds(input_channel) =
    (sync ->
     clock_c_out_val ? sec_in ->
     sync ->
        let
            sec = sec_in % 60
            sec_first_temp = sec / 10
            sec_second_temp = sec % 10
        within
            (sec_first_temp <= 7) &
                (seconds_sec_out_first_digit ! sec_first_temp ->
                (sec_second_temp <= 15) &
                    (seconds_sec_out_second_digit ! sec_second_temp ->
                    Seconds(input_channel)
                    )
                )
    ) [] SKIP


sec_sec_out_first_digit_monitor(c) =
    (c ? x ->
    (0 <= x and x <= 5) &
        sec_sec_out_first_digit_monitor(c)
    ) [] SKIP

sec_sec_out_second_digit_monitor(c) =
    (c ? x ->
    (0 <= x and x <= 9) &
        sec_sec_out_second_digit_monitor(c)
    ) [] SKIP

N_seconds =
        (
            (
                Seconds(clock_c_out_val)
                [|{| seconds_sec_out_first_digit |}|]
                sec_sec_out_first_digit_monitor(seconds_sec_out_first_digit)
            )
            [|{| seconds_sec_out_second_digit |}|]
            sec_sec_out_second_digit_monitor(seconds_sec_out_second_digit)
        )
        [|{| sync |}|]
        Clock(0)

assert SKIP [F= N_seconds \ Events
\end{minted}
\captionof{listing}{The full clocked \cspm{} code after transpiling the seven segment display example, as seen in Listing~\ref{lst:smeil} in the appendix.\label{lst:cspm_clocked}}


\chapter{Addone Example Full SMEIL and \cspm code}
\section*{SMEIL code}
\begin{minted}{smeil_lexer.py:SMEILLexer -x}
proc add (in input, const constant)
    bus output {
        val: u4 = 0 range 0 to 10;
    };
{
    output.val = (input.val + constant) % 11; // upper limit + 1
}

proc id (in input)
    bus output {
        val: u4 = 0 range 0 to 10;
    };
    var from_add: u4 range 0 to 10;
{
    from_add = input.val;
    output.val = from_add;
}


network addone_network ()
{
    instance id of id(add.output);
    instance add of add(id.output, constant: 1);
}
\end{minted}
\captionof{listing}{The full SMEIL code used for transpiling in the Addone example.\label{lst:smeil_addone_full}}

\section*{\cspm{} code}
\begin{minted}{cspm_lexer.py:CSPmLexer -x}
channel sync
channel d_read, c_read : { -1..15} -- u4 and initial value
channel d_write, c_write : { -1..15} -- u4 and initial value

DUM_VAL = -1 -- initial value


Add(i, input_channel) =
    (sync ->
     input_channel ? x ->
     sync ->
        if (x == DUM_VAL) -- initial value
            then (
                let
                    var = i
                within
                    var <= 15 & -- upper limit
                        c_read ! var -> Add(i, input_channel))
            else (
                let
                    var = (x + 1) % 11 -- observed value + 1 restriction
                within
                    var <= 15 & -- upper limit
                        c_read ! var -> Add(i, input_channel))
    )
    [] SKIP


Id(i, input_channel) =
    (sync ->
     input_channel ? x ->
     sync ->
        if (x == DUM_VAL) -- initial value
            then (
                i <= 15 & -- upper limit
                    d_read ! i -> Id(i, input_channel))
            else (
                x <= 15 & -- upper limit
                    d_read ! x -> Id(i, input_channel))
    )
    [] SKIP


c_read_monitor(c) =
    (c ? x ->
    (0 <= x and x <= 10 or x == -1) & -- observed values + initial value
        c_read_monitor(c)
    ) [] SKIP

d_read_monitor(c) =
    (c ? x ->
    (0 <= x and x <= 10 or x == -1) & -- observed values + initial value
        d_read_monitor(c)
    ) [] SKIP


Buf_d_read = sync -> ((d_read ? x -> (d_read ? x -> STOP [] Buf_d_write(x))
                    [] sync -> Buf_d_read) [] SKIP)

Writes_d_write(x) = d_write ! x -> ((Writes_d_write(x) [] Buf_d_read) [] SKIP)
Buf_d_write(x) = sync -> (Writes_d_write(x) [] Buf_d_read) [] SKIP


Buf_c_read = sync -> ((c_read ? x -> (c_read ? x -> STOP [] Buf_c_write(x))
                    [] sync -> Buf_c_read) [] SKIP)

Writes_c_write(x) = c_write ! x -> ((Writes_c_write(x) [] Buf_c_read) [] SKIP)
Buf_c_write(x) = sync -> (Writes_c_write(x) [] Buf_c_read) [] SKIP

Clock(21) = SKIP
Clock(n) =  sync -> sync -> Clock(n+1)


System =
        (
            (
                (
                    Add(0, d_write)
                    [{| sync, c_read, d_write |} || {| c_read |}]
                    c_read_monitor(c_read)
                )
                [{| sync, c_read, d_write |} || {| sync, d_read, d_write |}]
                Buf_d_write(DUM_VAL)
            )
            [{| sync, c_read, d_read, d_write |} || {| sync, c_read, c_write, d_read |}]
            (
                (
                    Id(0, c_write)
                    [{| sync, d_read, c_write |} || {| d_read |}]
                    d_read_monitor(d_read)
                )
                [{| sync, c_write, d_read |} || {| sync, c_read, c_write |}]
                Buf_c_write(DUM_VAL)
            )
        )
        [|{| sync |}|]
        Clock(0)

assert SKIP [F= System \ Events
\end{minted}
\captionof{listing}{The full \cspm{} code after transpiling the Addone example, as seen in Listing~\ref{lst:smeil_addone_full} in the appendix. This example have been manually translated. \label{lst:cspm_addone_full}}


\chapter{Towards Automatic Program Specification
Using SME Models}
\noindent A paper, based on this thesis, has been published as

\begin{center}
\begin{minipage}{0.8\textwidth}
    A. Thegler, M. Larsen, K. Skovhede, and B. Vinter. Towards Automatic Program Specification Using SME Models. In: In K. Chalmers, J. Pedersen, M. Smith, K. Skovhede, and P. Welch,
    editors, {\itshape Proceedings of Communicating Process Architectures
    2018}. IOS Press, Amsterdam, The Netherlands,
    August 2018.
\end{minipage}
\end{center}
The following pages contains this paper.
\includepdf[pages=-]{./paper/main.pdf}

\end{document}
