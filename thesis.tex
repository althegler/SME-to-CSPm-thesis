\documentclass[a4paper]{report}

\usepackage[latin1]{inputenc}
\usepackage{palatino}
\usepackage[usenames]{color}

\usepackage{hyperref}

\usepackage{chngpage}
\usepackage{graphicx}

\usepackage{booktabs}
\usepackage{multirow}

\usepackage[]{algorithm2e}
\usepackage{varwidth}

\usepackage{setspace}
\usepackage{hyperref}

\usepackage{todonotes}


\usepackage[babel, en, nat, farve, titelside]{ku-forside}

% (asbjoern commando: nbuf command)
% \newcommand{\nbuf}{\textit{nbuf} }

\newcommand{\cspm}{CSP$_M$}


\usepackage{fancyhdr}
\pagestyle{fancy}

\lhead{Alberte Thegler}
\chead{Master's Thesis}
\rhead{August 2018}

\renewcommand{\headrulewidth}{0.4pt} % thickness of line at header
\renewcommand{\footrulewidth}{0.4pt} % thickness of line at footer
\setlength{\belowcaptionskip}{-10pt} % space below captions


\opgave{Master's Thesis}
\title{Towards formal verification of FDR4}
\undertitel{Department of Computer Science}
\author{Alberte Thegler - alberte@thegler.dk}
\date{August 2018}
\vejleder{Professor Brian Vinter}

\begin{document}

\maketitle

\pagenumbering{roman}
\begin{abstract}
\begin{doublespace}
Bla bla 
bla bla


\end{doublespace}
\end{abstract}



\newpage
\tableofcontents

\newpage
\listoftodos
\newpage
\pagenumbering{arabic}
\chapter{Introduction}
When we create programs, we wish to verify that it is also correct. There are several ways to do this, one commenly used is \texttt{testing} which require that the programmer creates several different scenarios and its expected output, or that the programmer programs a test-generator to create the scenarios and expected output. This, however, is not adequate for (word for important systens). Therefore it is of high interest to create a verification of the system or program.\\
Talk about how verification was first created and how it became to be used for concurrent systems. Then write about how it works and then write about the different systems and formal languages that is used for it. 


In this thesis we look at model checking, that is, verifying that a specific property will always hold for a piece of code.


Formal verification is the process of checking whether a program satisfies specific properties. Different methods have evolved, all having different advantages and disadvantages. FDR is sometimes referred to as a model checker however is it actually a refinement checker.



\textbf{Matematicians tend to reject proofs by exhaustive checking of all cases as being less satisfying than deductive proofs, and with good reason. First, they are not applicable for proving theorems about integers and real numbers, which are infinite domains so that the number of interpretations is infinite and they cannot be exhaustively checked. Second, they offer no insight into why a theorem is true. But computer scientists have more practical concerns. If they can check all computations of a program and show that they all satisfy a correctness property, we will be willing to forego elegance and be more than satisfied that our program has been proven correct. (from "A primer on model checking af Ben-Ari} \cite{Ben-ari2010}
\section{Learning goals}
This is where the learning goals go.
\chapter{Related work}
In this chapter we will discuss previous work that has lead formal verification to what it is today. We will also discuss different tools and languages that are used today and the differences between them. \\\\
In 1967 Robert W. Floyd was published with his paper \textit{Assigning meaning to programs}\cite{Floyd1967}. In his paper he provide a basis for formal definitions of the meaning of programs which can be used for proving correctness, equvalence and termination. He uses flowcharts to argue that when a command is reached all previous commands will have been true as well.\\\\
C.A.R Hoare was inspired by Floyd and in 1969 his paper \textit{An axiomtic basis for computer programming}\cite{Hoare1969} was published. With his \textit{Hoare's logic}, he builds on Floyd's ideas and proposed that program could be viewed as a partial correctness relation between a precondition and a postcondition predicate. This means that if the state the program starts in satisfies the precondition and it terminates, then the final state satisfies the postcondition. L. Lamport expended Hoare's logic to concurrent programs in his paper \textit{The 'Hoare logic' of concurrent programs}\cite{Lamport1980}. He argues why Hoare's logic as proposed by C.A.R Hoare does not work for concurrent programs and proposes a "generalized Hoare's logic" that takes concurrency into account. \todo{Figure out if Hoare used this information/update in his work with CSP. I am not sure if CSP work on Hoare logic? }\\\\
In 1972 his paper \textit{Towards a Theory of Parallel Programming}\todo{download file and make citation - I have not been able to find a free version of this paper} was published and in 1978 his paper \textit{Communicating Sequential Processes} was published. CSP was born and have been widely used and have also been expanded since Hoare initially described it in 1978\todo{Citation to Communicating Sequential Processes: The first 25 years}. The first version of CSP was mostly a concurrent programming language and later, Brookes, Hoare and Roscoe\cite{Brookes1984} continued the work on CSP and created the modern process algebra it is today, only a few minor changes have been made to CSP since then, and they are described in Roscoe's \textit{The Theory and Practice of Concurrency}\cite{Roscoe1997}\\\\
A number of tools have been created to analyse, verify and understand CSP written systems. In order to use these tools along with CSP, different types of machine-readble CSP syntaxes have been created over the years, but most of todays CSP tools use a version of machine-readble CSP calles \cspm which was created by Scattergood\cite{Scattergood1998}. Since CSP was mostly a blackboard language, Scattergood created a combination of the standard CSP and a functional programming language which created a better baseline for tools to work with CSP. One of the most known CSP tool is the Failure-Divergence Refinement (FDR), build by Formal Systems (Europe) Ltd., which is currently at version 4.2.3\cite{fdr}. FDR is a refinement checker which differs from a lot of other CSP tools that are merely model checkers. \todo{Figure out the precise difference between refinement and model checking}.\\ The Adelaide Refinement Checker (ARC)\cite{Parashkevov1996} is a automatic verification tool for untimed CSP. It represents the internal representation by using Ordered Binary Decision Diagrams (OBDDs). This lessen the state explosion problem that other model checkers have with LTS representations. \\ The ProB project\todo{Look at the site on ProB I found - under Thesis bookmarks}

The programming language Occam, which was first released in 1983, is a concurrent programming language that builds on the Communicating Sequential Processes process algebra. Occam developed over the years and the Kent Retargetable occam Compiler (KRoC) team at Kent University created the Occam-$\pi$ variant of the occam programming language. It is a version that extends the idas of CSP in the original occam language but adding mobility features from pi-calculus. On the KRoC webpage they describe the reason to include functionality from pi-calculus; \textit{"Specifically, we want to allow networks of processes to evolve, to change their topologies, to cope with growth and decay without losing semantic or structural integrity. We want to address the mobility of processes, channels and data and understand the relationships between these ideas. We want to retain the ability to reason about such systems, preserving the concept of refinement."}\footnote{https://www.cs.kent.ac.uk/projects/ofa/kroc/ access date: 3/4/18}\\\\
SPIN is a verification system that uses process interactions to prove correctness for a system. The system is described in the formal language \texttt{PROMELA}(PROcess MEta LAnguage)\cite{Holzmann1991} and the correctness properties are spcified in Linear Temporal Logic (LTL)\cite{Pnueli1977}. In the paper \textit{Reasoning About Infinite Computations}\cite{Vardi1994}, Vardi and Wolper showed that all LTL formulas can be translated into a B\"uchi automata which SPIN makes use of and thus converting the given LTL into a B\"uchi automaton.\\
Spin was developed at Bell Labs, starting in 1980. Since 1991 it has been freely available and today it is used by thousands of people worldwide.\\Gerard J. Holzmann gives an introduction to the theoretical foundations, the design and structure and examples of applications in the paper \textit{The model checker SPIN}\cite{Holzmann1997}. SPIN was build on the pioneering work on logic model checking by Clarke and Emerson\cite{Clarke1981}, as well as Sifakis and Queille\cite{Queille1982}. Vardi and Wolper extended their work with an automata-theoretic approach to automatically verify programs\cite{Vardi1986}.\\\\




%\textbf{Related work should be about describing the history of formal verification and formal languages in both software and hardware.}
%
%In 1969, Tony Hoare proposed the Hoare logic\cite{Hoare1969} in his paper \textit{An axiomtic basis for computer programming}. He proposed that a program could be viewed as a  
%partial correctness relation between a precondition and a postcondition predicate. This means that if the state the program starts in satisfies the precondition and it terminates, then the final state satisfies the postcondition.\\ Hoare was much inspired by Robert W. Floyd, who had published his paper \textit{Assigning meaning to programs}\cite{Floyd1967} a few years before. The paper described a method for proving the partial correctness of a program by using flowcharts and Hoare's logic provided a good way of formulating Floyd's method.
%
%A lot of verification systems used today is based on Hoare logic. Hoare was much inpired by Robert W. Floyd, who a few years before had published 
%Several different software verification tools exists today, and they all have different advantages.
%
%
%
%
%SPIN \todo{Add reference to SPIN} is a verification system that uses process interactions to prove correctness for a system. The system is described in the formal language \texttt{PROMELA}(PROcess MEta LAnguage)\todo{add reference to promela} and the correctness properties are spcified in Linear Temporal Logic or LTL \todo{add reference to Linear Temporal Logic}. Spin was developed at Bell Labs, starting in 1980. Since 1991 it has been freely available and today it is used by thousands of people worldwide.\\
%The job is to find other types of models and compare them, as well as the types of languages. so different tools and different languages.. 
%
%Occam-pi is a programming language that came from occam and was merged with pi-calculus. 
%
%
%\textbf{The two most popular methods for automatic formal verification are language containment and model checking. The current version of VIS emphasizes model checking, but it also offers to the user a limited form of language containment (language emptiness)." from https://embedded.eecs.berkeley.edu/research/vis/doc/VisUser/vis\_user/node4.html}
%





\chapter{Theory}
\textbf{This is where the theory go. fx. SME and the correlation between that and CSP.}
CSPm was devised by Bryan Scattergood as a machine-readable dialect of CSP  - se the paper \textit{The Semantics and Implementation of Machine-Readable CSP}\\\\
" FDR2 is often described as a model checker, but is technically a refinement checker, in that it converts two CSP process expressions into Labelled Transition Systems (LTSs), and then determines whether one of the processes is a refinement of the other within some specified semantic model (traces, failures, or failures/divergence)" (from Wikipedia - se paper \textit{Model-checking CSP - af Roscoe} \\\\
\section{Hoare's logic}
\chapter{Method}
\textbf{This is the method section that describes what I did, how and why.}
\chapter{Results and tests (Experiment?)}
\textbf{Does it work? why, why not}
\chapter{Discussion}

\chapter{Conclusion}
\section{Future work}

\newpage
\bibliographystyle{abbrv}
\bibliography{library}

\section{Appendix}
\end{document}
