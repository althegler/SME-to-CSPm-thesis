\documentclass[13pt]{beamer}
\usetheme[nat,greyfoot,footstyle=low,headstyle=institute,style=alternative]{Frederiksberg}

\usepackage[utf8]{inputenc}
\usepackage[english]{babel}
\usepackage{times}

\usepackage{amsmath}
\usepackage{amssymb}

\usepackage{xcolor}
\usepackage{color}
\usepackage[outputdir=.tmp,cache=false]{minted}
% \setminted{frame=lines,linenos,framesep=2mm,fontsize=\small}

\usepackage{url}
\def\CC{{C\nolinebreak[4]\hspace{-.05em}\raisebox{.4ex}{\tiny ++}}}

\usepackage{pgfplots}
\usepgfplotslibrary{units}
\pgfplotsset{compat=newest}

\pgfplotsset{
  every axis plot post/.style={/pgf/number format/fixed}
}
\usepackage{graphicx}
\usepackage[caption=false,font=footnotesize]{subfig}
\newsubfloat{figure}



\normalfont
%\usepackage[T1]{fontenc}
% \renewcommand{\ttdefault}{cmtt}
\usepackage{booktabs, caption, siunitx}

\usepackage{tikz}


\usepackage[labelsep=period]{caption}

\usetikzlibrary{calc}
\usetikzlibrary{fit}
\usetikzlibrary{positioning}
\usepgfplotslibrary{units}
\pgfplotsset{compat=newest}

\usetikzlibrary{decorations.markings}
\usetikzlibrary{arrows}
\usetikzlibrary{shapes.geometric}

\usepackage{ifthen}
\pgfkeys{
  /sevenseg/.is family, /sevenseg,
  slant/.estore in      = \sevensegSlant,     % vertical slant in degrees
  size/.estore in       = \sevensegSize,      % length of a segment
  shrink/.estore in     = \sevensegShrink,    % avoids overlapping of segments
  line width/.estore in = \sevensegLinewidth, % thickness of the segments
  line cap/.estore in   = \sevensegLinecap,   % end cap style rect, round, butt
  oncolor/.estore in    = \sevensegOncolor,   % color of an ON segment
  offcolor/.estore in   = \sevensegOffcolor,  % color of an OFF segment
}

\pgfkeys{
  /sevenseg,
  default/.style={
    slant = 0,
    size = 1em,
    shrink = 0.2,
    line width = 0.3em,
    line cap = butt,
    oncolor = green,
    offcolor = black!75!green
  }
}
\newcommand{\sevenseg}[2][]% options values
{%
\pgfkeys{/sevenseg, default, #1}%
\def\sevensegarray{#2}%
  \begin{tikzpicture}%
    % first define the position of the 6 corner points
    \path (0,0) ++(0,0)                             coordinate (P1);
    \path (0,0) ++(\sevensegSize,0)                 coordinate (P2);
    \path (0,0) ++(90-\sevensegSlant:\sevensegSize) coordinate (P3);
    \path (P2)  ++(90-\sevensegSlant:\sevensegSize) coordinate (P4);
    \path (P3)  ++(90-\sevensegSlant:\sevensegSize) coordinate (P5);
    \path (P4)  ++(90-\sevensegSlant:\sevensegSize) coordinate (P6);
    % then step through the 1/0 values in the segment array
    \foreach \i in {0,...,6}%
    {
      \pgfmathparse{\sevensegarray[\i]}
      \ifthenelse{\equal{\pgfmathresult}{1}}%
        {\let\mycolor=\sevensegOncolor}%  segment is on
        {\let\mycolor=\sevensegOffcolor}% segment is off
      \tikzstyle{segstyle} = [draw=\mycolor, line width = \sevensegLinewidth,
                              line cap = \sevensegLinecap]
      %-----------------------
      \ifthenelse{\equal{\i}{0}}{\path[segstyle]
        (${1-\sevensegShrink}*(P5)+\sevensegShrink*(P6)$)
        -- ($\sevensegShrink*(P5)+{1-\sevensegShrink}*(P6)$);}{} % a
      \ifthenelse{\equal{\i}{1}}{\path[segstyle]
        (${1-\sevensegShrink}*(P6)+\sevensegShrink*(P4)$)
        -- ($\sevensegShrink*(P6)+{1-\sevensegShrink}*(P4)$);}{} % b
      \ifthenelse{\equal{\i}{2}}{\path[segstyle]
        (${1-\sevensegShrink}*(P4)+\sevensegShrink*(P2)$)
        -- ($\sevensegShrink*(P4)+{1-\sevensegShrink}*(P2)$);}{} % c
      \ifthenelse{\equal{\i}{3}}{\path[segstyle]
        (${1-\sevensegShrink}*(P1)+\sevensegShrink*(P2)$)
        -- ($\sevensegShrink*(P1)+{1-\sevensegShrink}*(P2)$);}{} % d
      \ifthenelse{\equal{\i}{4}}{\path[segstyle]
        (${1-\sevensegShrink}*(P1)+\sevensegShrink*(P3)$)
        -- ($\sevensegShrink*(P1)+{1-\sevensegShrink}*(P3)$);}{} % e
      \ifthenelse{\equal{\i}{5}}{\path[segstyle]
        (${1-\sevensegShrink}*(P3)+\sevensegShrink*(P5)$)
        -- ($\sevensegShrink*(P3)+{1-\sevensegShrink}*(P5)$);}{} % f
      \ifthenelse{\equal{\i}{6}}{\path[segstyle]
        (${1-\sevensegShrink}*(P3)+\sevensegShrink*(P4)$)
        -- ($\sevensegShrink*(P3)+{1-\sevensegShrink}*(P4)$);}{} % g
    }
  \end{tikzpicture}%
}

\newcommand{\sevensegnum}[2][]% sample characvters
{%
  \ifthenelse{\equal{#2}{0}}{\sevenseg[#1]{{1,1,1,1,1,1,0,}}}{%
  \ifthenelse{\equal{#2}{1}}{\sevenseg[#1]{{0,1,1,0,0,0,0,}}}{%
  \ifthenelse{\equal{#2}{2}}{\sevenseg[#1]{{1,1,0,1,1,0,1,}}}{%
  \ifthenelse{\equal{#2}{3}}{\sevenseg[#1]{{1,1,1,1,0,0,1,}}}{%
  \ifthenelse{\equal{#2}{4}}{\sevenseg[#1]{{0,1,1,0,0,1,1,}}}{%
  \ifthenelse{\equal{#2}{5}}{\sevenseg[#1]{{1,0,1,1,0,1,1,}}}{%
  \ifthenelse{\equal{#2}{6}}{\sevenseg[#1]{{1,0,1,1,1,1,1,}}}{%
  \ifthenelse{\equal{#2}{7}}{\sevenseg[#1]{{1,1,1,0,0,0,0,}}}{%
  \ifthenelse{\equal{#2}{8}}{\sevenseg[#1]{{1,1,1,1,1,1,1,}}}{%
  \ifthenelse{\equal{#2}{9}}{\sevenseg[#1]{{1,1,1,1,0,1,1,}}}{%
  \ifthenelse{\equal{#2}{A}}{\sevenseg[#1]{{1,1,1,0,1,1,1,}}}{%
  \ifthenelse{\equal{#2}{B}}{\sevenseg[#1]{{0,0,1,1,1,1,1,}}}{%
  \ifthenelse{\equal{#2}{C}}{\sevenseg[#1]{{0,0,0,1,1,0,1,}}}{%
  \ifthenelse{\equal{#2}{D}}{\sevenseg[#1]{{0,1,1,1,1,0,1,}}}{%
  \ifthenelse{\equal{#2}{E}}{\sevenseg[#1]{{1,0,0,1,1,1,1,}}}{%
  \ifthenelse{\equal{#2}{F}}{\sevenseg[#1]{{1,0,0,0,1,1,1,}}}{%
  {\sevenseg[#1]{{0,0,0,0,0,0,0,}}}}}}}}}}}}}}}}}}}%
}

\tikzset{
  myarrow/.style={
    draw=black,
    thick,
    ->,
    shorten <=3pt,
    shorten >=3pt,
  },
  mycircle/.style={
    draw=black,
    shape=circle,
    very thick,
    inner sep=3pt,
    inner ysep=5pt,
    text width=0.75cm,
    align=center,
    minimum size=0.75cm,
    rounded corners,
  },
  mytriangle/.style={
    draw=black,
    regular polygon,
    regular polygon sides=3,
    align=center,
    rounded corners,
    very thick,
    inner sep=3pt,
  },
  myrectangle/.style={
    draw=black,
    shape=rectangle,
    very thick,
    rounded corners,
    align=center,
    inner sep=7pt,
    inner ysep=7pt,
    text width=2.1cm,
    minimum size=0.5cm,
    minimum height=1.5cm,
    font=\footnotesize
  },
  mysquare/.style={
    draw=black,
    shape=rectangle,
    very thick,
    rounded corners,
    align=center,
    inner sep=7pt,
    inner ysep=7pt,
    font=\footnotesize
  }
}

\pgfplotsset{
  every axis plot post/.style={/pgf/number format/fixed}
}

\newsavebox{\smeilexamplecode}
\begin{lrbox}{\smeilexamplecode}
  \begin{minipage}{1.1\textwidth}
    \begin{minted}[fontsize=\scriptsize, frame=none, linenos, framesep=2mm, autogobble, escapeinside=||, mathescape=true]{smeil_lexer.py:SMEILLexer -x}
        proc seconds (in seconds_in)
            bus seconds_out {first_digit: u3 range 0 to 5;
                             second_digit: u4 range 0 to 9;};
            var seconds: u6 range 1 to 59;
            var seconds_first_temp: u3 range 0 to 5;
            var seconds_second_temp: u4 range 0 to 9;
        {
            seconds = seconds_in.val % 60;
            seconds_first_temp = seconds / 10;
            seconds_second_temp = seconds % 10;
            seconds_out.first_digit = seconds_first_temp;
            seconds_out.second_digit = seconds_second_temp;
        }
    \end{minted}
  \end{minipage}
\end{lrbox}

\newsavebox{\smeilchannelexample}
\begin{lrbox}{\smeilchannelexample}
  \begin{minipage}{1.1\textwidth}
    \begin{minted}[fontsize=\scriptsize, frame=none, linenos, framesep=2mm, autogobble, escapeinside=||, mathescape=true]{smeil_lexer.py:SMEILLexer -x}
        proc seconds (in seconds_in)
            bus seconds_out {first_digit: u3 range 0 to 5;
                             second_digit: u4 range 0 to 9;};
    \end{minted}
  \end{minipage}
\end{lrbox}


\newsavebox{\smeilprocessexample}
\begin{lrbox}{\smeilprocessexample}
  \begin{minipage}{1.1\textwidth}
    \begin{minted}[fontsize=\scriptsize, frame=none, linenos, framesep=2mm, autogobble, escapeinside=||, mathescape=true]{smeil_lexer.py:SMEILLexer -x}
        proc seconds (in seconds_in)
            |$\vdots$|
        {
            seconds = seconds_in.val % 60;
            seconds_first_temp = seconds / 10;
            seconds_second_temp = seconds % 10;
            seconds_out.first_digit = seconds_first_temp;
            seconds_out.second_digit = seconds_second_temp;
        }
    \end{minted}
  \end{minipage}
\end{lrbox}



\newsavebox{\cspmchannelexample}
\begin{lrbox}{\cspmchannelexample}
  \begin{minipage}{1.1\textwidth}
        \begin{minted}[fontsize=\scriptsize, frame=none, linenos, framesep=2mm, autogobble, escapeinside=||, mathescape=true]{cspm_lexer.py:CSPmLexer -x}
        channel seconds_out_first_digit : {0..7}
        channel seconds_out_second_digit : {0..15}
        \end{minted}
  \end{minipage}
\end{lrbox}


\newsavebox{\cspmprocessexample}
\begin{lrbox}{\cspmprocessexample}
  \begin{minipage}{1.1\textwidth}
        \begin{minted}[fontsize=\scriptsize, frame=none, linenos, framesep=2mm, autogobble, escapeinside=||, mathescape=true]{cspm_lexer.py:CSPmLexer -x}
        Seconds(seconds_in) =
        let
            seconds = seconds_in % 60
            seconds_first_temp = seconds / 10
            seconds_second_temp = seconds % 10
        within
            seconds_out_first_digit ! seconds_first_temp ->
            seconds_out_second_digit ! seconds_second_temp ->
            SKIP
        \end{minted}
  \end{minipage}
\end{lrbox}


\newsavebox{\cspmmonitorexample}
\begin{lrbox}{\cspmmonitorexample}
  \begin{minipage}{1.1\textwidth}
        \begin{minted}[fontsize=\scriptsize, frame=none, linenos, framesep=2mm, autogobble, escapeinside=||, mathescape=true]{cspm_lexer.py:CSPmLexer -x}
        Seconds_out_first_digit_monitor(c) =
            c ? x -> if 0 <= x and x <= 5 then SKIP else STOP
        Seconds_out_second_digit_monitor(c) =
            c ? x -> if 0 <= x and x <= 9 then SKIP else STOP
        \end{minted}
  \end{minipage}
\end{lrbox}

\newsavebox{\cspmexample}
\begin{lrbox}{\cspmexample}
  \begin{minipage}{1.1\textwidth}
        \begin{minted}[fontsize=\scriptsize, frame=none, linenos, framesep=2mm, autogobble, escapeinside=||, mathescape=true]{cspm_lexer.py:CSPmLexer -x}
        channel seconds_out_first_digit : {0..7}
        channel seconds_out_second_digit : {0..15}

        Seconds(seconds_in) =
        let
            seconds = seconds_in % 60
            seconds_first_temp = seconds / 10
            seconds_second_temp = seconds % 10
        within
            seconds_out_first_digit ! seconds_first_temp ->
            seconds_out_second_digit ! seconds_second_temp ->
            SKIP

        Seconds_out_first_digit_monitor(c) =
            c ? x -> if 0 <= x and x <= 5 then SKIP else STOP
        Seconds_out_second_digit_monitor(c) =
            c ? x -> if 0 <= x and x <= 9 then SKIP else STOP

        N_seconds = clock_out_val ? variable ->
                    (Seconds(variable)
                    [| {| seconds_out_first_digit|} |]
                    Seconds_out_first_digit_monitor(seconds_out_first_digit))
                    [| {| seconds_out_second_digit|} |]
                    Seconds_out_second_digit_monitor(seconds_out_second_digit)

        assert SKIP [F= N_seconds \ Events

        \end{minted}
  \end{minipage}
\end{lrbox}







\title[Towards Automatic Program Specification Using SME Models]{Towards Automatic Program Specification \\ Using SME Models}
\subtitle{\tiny Communicating Process Architectures 2018 -- Technische Universität Dresden}
\author[A. Thegler]
{\textbf{Alberte Thegler},\\
Mads Ohm Larsen,\\
Kenneth Skovhede,\\
and Brian Vinter\\
}
\institute[Niels Bohr Institute]{Niels Bohr Institute, University of Copenhagen, Denmark}
\date[August 21]{21 August 2018}

\newcommand{\cspm}{CSP$_M$}

\begin{document}

\frame[plain]{\titlepage}

%%%%%%%
%%% TOC
% \begin{frame}{Table of Contents}
%   \begin{enumerate}
%     \item Why should we verify hardware?
%     \item What have we done
%     \item What can SME do?
%     \item SMEIL
%     \item Simple example
%     \item SMEIL bus to \cspm{} channel
%     \item \cspm{} process structure
%     \item Monitor process
%     \item Example continued
%     \item Results - time to verify in FDR4?
%     \item Conclusion
%     \item Future work
%   \end{enumerate}
% \end{frame}
%%% /TOC
%%%%%%%%

%%%%%%%%%%%%%%%%%%%%%%
%%% Why should we verify hardware?
% \begin{frame}{Ariane-5}
%   \begin{block}{}
%     Picture?
%   \end{block}
%
% \end{frame}

\begin{frame}{Ariane-5}
  \begin{block}{}
    4th June 1996
  \end{block}

  \pause

  \begin{block}{}
     Total failure on launch
  \end{block}

  \pause

  \begin{block}{}
     Converting a 64-bit floating point number to signed 16-bit integer.
  \end{block}

  \pause

  \begin{block}{}
    Overflow caused the self-destruct mechanism in both primary and backup computer
  \end{block}

  \pause

  \begin{block}{}
     No people where harmed
  \end{block}

\end{frame}


% \begin{frame}{The Patriot Missile Failure}
%   \begin{block}{}
%     Picture?
%   \end{block}
%
% \end{frame}

\begin{frame}{The Patriot Missile Failure}
  \begin{block}{}
    25th February 1991 in the Persian Gulf war
  \end{block}

  \pause

  \begin{block}{}
     A Patriot missile failed to intercept an incomming "Scud".
  \end{block}

  \pause

  \begin{block}{}
     Conversion of time since last boot from an integer to a real number was performed using a 24 bit register.
  \end{block}

  \pause

  \begin{block}{}
     The patriot missile missed the Scud which struck a U.S Army barracks, killing 28 soldiers.
  \end{block}

\end{frame}

\begin{frame}{Why should we verify hardware?}
    \begin{block}{}
        Because, as these examples have shown, the consequences of not verifying can be devastating.
            \vspace{5mm}

        Loss of milions of money
            \vspace{5mm}

        Loss of human life
    \end{block}
\end{frame}
%%%
%%%%%%%%%%%%%%%%%%%%%%%
%
%%%%%%%%%%%%%%%%%%%%%%%%%%%%%%%%%%
%%% What have we done
\begin{frame}{What have we done?}
 \begin{block}{}
   A transpiler which transpiles SMEIL code to \cspm{} in order to verify SME models with FDR4
 \begin{figure}[!ht]
  \centering
  \begin{tikzpicture}[auto]
    \node[myrectangle, text width=1cm, minimum height=0.8cm, inner sep=5pt, inner ysep=5pt] (sme) {SME};
    \node[myrectangle, text width=1cm, minimum height=0.8cm, inner sep=5pt, inner ysep=5pt] (smeil) [right=1cm of sme] {SMEIL};
    \node[mycircle, text width=2cm, inner sep=5pt, inner ysep=5pt] (transpiler) [right=1cm of smeil] {Transpiler};
    \node[myrectangle, text width=1cm, minimum height=0.8cm, inner sep=5pt, inner ysep=5pt] (cspm) [right=1cm of transpiler] {CSP$_M$};

    \draw[myarrow] (sme) -- (smeil);
    \draw[myarrow] (smeil) -- (transpiler);
    \draw[myarrow] (transpiler) -- (cspm);
  \end{tikzpicture}
  \caption{SME to \cspm{} transpiler.}
  \label{fig:sme-to-cspm}
\end{figure}
 \end{block}
\end{frame}
%%%
%%%%%%%%%%%%%%%%%%%%%%%
%
%%%%%%%%%%%%%%%%%%%%%%%%%%%%%%%%%%
%%% What can SME do?
\begin{frame}{How do we use SME?}
 \begin{block}{}
   The SME model builds on the CSP algebra and therefore all SME models have a corresponding CSP model.

 \end{block}

 \pause

  \begin{block}{}
    We transpile not only the SME network, but also all the SME processes and their content.
  \end{block}

 \pause

  \begin{block}{}
    We can translate SME sequentially which simplifies the transpilation.
  \end{block}
\end{frame}
%%
%%%%%%%%%%%%%%%%%%%%%%%%%%%%%%%%%%%%
%
%%%%%%%%%%%%%%%%%%%%%%%%%%%
%% SMEIL
\begin{frame}{How do we use SMEIL?}
 \begin{block}{}
    Introduced by Truls Asheim in the previous presentation
 \end{block}

 \pause

 \begin{block}{}
   We transpile from SMEIL to \cspm{}\\
   And then verify it in FDR4
 \end{block}

 \pause

 \begin{block}{}
   The transpiler currently only works with pure SMEIL programs
   \begin{figure}[!ht]
  \centering
  \begin{tikzpicture}[auto]
    \node[mycircle, minimum size=1.75cm, align=center, text width=1.75cm, font=\footnotesize]    (smeil)                                       {SMEIL};
    \node[myrectangle, text width=1.5cm, minimum height=1.0cm, inner sep=5pt, inner ysep=5pt] (csme)  [above left=-0.25cm and 1.5cm of smeil] {C\#SME};
    \node[myrectangle, text width=1.5cm, minimum height=1.0cm, inner sep=5pt, inner ysep=5pt] (pysme) [below left=-0.25cm and 1.5cm of smeil] {PySME};
    \node[myrectangle, text width=1.5cm, minimum height=1.0cm, inner sep=5pt, inner ysep=5pt] (vhdl)  [right=1.0cm of smeil]                {VHDL};

    \draw[myarrow] (csme)  -- (smeil);
    \draw[myarrow] (pysme) -- (smeil);
    \draw[myarrow] (smeil) -- (vhdl);
  \end{tikzpicture}
  \caption{SMEIL transpiler structure.}
  \label{fig:smeil_transpiler}
\end{figure}
 \end{block}
\end{frame}
%%
%%%%%%%%%%%%%%%%%%%%%%%%%%%%%%%%%%%%
%
%%%%%%%%%%%%%%%%%%%%%%%%%%%
%% Simple example
\begin{frame}{Seven segment display clock}
 \begin{block}{}
   \begin{figure}[!ht]
        \tikz{
          \node[inner sep=5pt, outer sep=2pt, draw=blue, fill=black] {
            \sevensegnum[size=2em, shrink=0.1]{1}
            \sevensegnum[size=2em, shrink=0.1]{2}
          }
        }
        \tikz{
          \node[inner sep=5pt, outer sep=2pt, draw=blue, fill=black] {
            \sevensegnum[size=2em, shrink=0.1]{3}
            \sevensegnum[size=2em, shrink=0.1]{4}
          }
        }
        \tikz{
          \node[inner sep=5pt, outer sep=2pt, draw=blue, fill=black] {
            \sevensegnum[size=2em, shrink=0.1]{5}
            \sevensegnum[size=2em, shrink=0.1]{6}
          }
        }
      \caption{Digital clock with six seven segment displays, displaying 12:34:56.}
      \label{fig:6_displays}
   \end{figure}
 \end{block}

 \pause

 \begin{block}{}
   Seconds since midnight
 \end{block}

 \pause

 \begin{block}{}
   Arithmetics calculate hours, minutes and seconds respectively
 \end{block}

 \pause

 \begin{block}{}
  Two seven segment displays pr. \texttt{time} process
 \end{block}

\end{frame}


\begin{frame}{Seven segment display clock}
 \begin{block}{}
  \begin{figure}[!ht]
  \centering
  \begin{tikzpicture}
    \node [mycircle] (I) at (0,0) {$I$};

    \node [mycircle] (H) at (2.5,  1.50) {$H$};
    \node [mycircle] (M) at (2.5,  0.00) {$M$};
    \node [mycircle] (S) at (2.5, -1.50) {$S$};

    \draw [myarrow] (I) -- (M);

    \draw [myarrow, smooth] (I) to[out=0, in=180] (H);
    \draw [myarrow, smooth] (I) to[out=0, in=180] (S);

    % Output arrows without processes
    \draw [myarrow] (3.125,  1.625) -- (4.000,  1.750);
    \draw [myarrow] (3.125,  1.375) -- (4.000,  1.250);
    \draw [myarrow] (3.125,  0.125) -- (4.000,  0.250);
    \draw [myarrow] (3.125, -0.125) -- (4.000, -0.250);
    \draw [myarrow] (3.125, -1.375) -- (4.000, -1.250);
    \draw [myarrow] (3.125, -1.625) -- (4.000, -1.750);
  \end{tikzpicture}
  \caption{SMEIL network for a seven segment display clock. Each SMEIL process is represented by a cicle with a letter corresponding to the processes Input, Hours, Minutes and Seconds respectively.}
  \label{fig:smeil_network}
\end{figure}
 \end{block}
\end{frame}

\begin{frame}{What are we verifying?}
 \begin{block}{}
   One seven segment display can only display the numbers 0-9. \\
   4 bits can represent 0-15, which is more than needed.
 \end{block}

 \pause

 \begin{block}{}
   We can verify that the values communicated to all the seven segment displays does not exceed the expected values.
 \end{block}

  \pause

  \begin{block}{}
    In this case we can restrict the assertions further. \\
    Hours will never be more than 24, etc.
  \end{block}


 \pause

 \begin{block}{}
   In general, we verify the values comnmunicated on \cspm{} channels
 \end{block}
\end{frame}

\begin{frame}{Seven Segment display clock}
 \begin{block}{}
  \texttt{SMEIL} code:
    \vspace{5mm}

     \scalebox{0.8}{\usebox{\smeilexamplecode}}
 \end{block}
\end{frame}
%%
%%%%%%%%%%%%%%%%%%%%%%%%%%%%
%
%%%%%%%%%%%%%%%%%%%%%%%%%%%%%%
%% Now for the transpiling
\begin{frame}{The transpiling}
 \begin{block}{}
   SMEIL bus to \cspm{} channel
 \end{block}

 \pause

 \begin{block}{}
   \cspm{} process structure
 \end{block}

 \pause

 \begin{block}{}
   The monitor process
 \end{block}
\end{frame}

%%
%%%%%%%%%%%%%%%%%%%%%%%%%%%%
%
%%%%%%%%%%%%%%%%%%%%%%%%%%%%%%
%% SMEIL bus to \cspm{} channel
\begin{frame}{SMEIL bus to \cspm{} channel}
 \begin{block}{}
   \texttt{SMEIL} code:
     \vspace{5mm}

      \scalebox{0.8}{\usebox{\smeilchannelexample}}
 \end{block}
 \pause
 \begin{block}{}
    \texttt{\cspm{}} code:
      \vspace{5mm}

       \scalebox{0.8}{\usebox{\cspmchannelexample}}
 \end{block}
\end{frame}

%%
%%%%%%%%%%%%%%%%%%%%%%%%%%%%
%
%%%%%%%%%%%%%%%%%%%%%%%%%%%%%%
%% \cspm{} process structure
\begin{frame}{\cspm{} process structure}
 \begin{block}{}
   \texttt{SMEIL} code:
     \vspace{5mm}

      \scalebox{0.8}{\usebox{\smeilprocessexample}}
 \end{block}
 \pause
 \begin{block}{}
    \texttt{\cspm{}} code:
      \vspace{5mm}

       \scalebox{0.8}{\usebox{\cspmprocessexample}}
 \end{block}

\end{frame}
%%
%%%%%%%%%%%%%%%%%%%%%%%%%%%%%%%
%
%%%%%%%%%%%%%%%%%%%%%%%%%%%%
%% Monitor process
\begin{frame}{The monitor process}
 \begin{block}{}
     \begin{figure}[!ht]
      \centering
      \begin{tikzpicture}[auto]
        \node[mycircle] (P) at (-1.5, 0.0) {$P$};
        \node[mycircle] (Q) at ( 2.5, 0.0) {$Q$};
        \node[mycircle, shape=rectangle] (M) at ( 0.5, 1.5) {$M$};

        \node[draw, shape=circle, inner sep=0pt, minimum size=5pt] (m) at (0.5, 0.0) {};


        \draw (M) -- (P -| M) [black!50];
        \draw [myarrow] (P) -- (Q);
      \end{tikzpicture}
      \caption{The monitor process \textit{M} listens in on the communication between \textit{P} and \textit{Q} in order to assert the communicated values.}
      \label{fig:assertion_process}
    \end{figure}
 \end{block}
\end{frame}
\begin{frame}{The monitor process}
 \begin{block}{}
   \texttt{SMEIL} code:
     \vspace{5mm}

      \scalebox{0.8}{\usebox{\smeilchannelexample}}
 \end{block}
 \pause
 \begin{block}{}
    \texttt{\cspm{}} code:
      \vspace{5mm}

       \scalebox{0.8}{\usebox{\cspmmonitorexample}}
 \end{block}

\end{frame}
%%
%%%%%%%%%%%%%%%%%%%%%%%%%%%%%
%
%%%%%%%%%%%%%%%%%%%%%%%
%% Example continued
\begin{frame}{Seven segment display clock}
 \begin{block}{}
  \texttt{\cspm{}} code:
    \vspace{3mm}

     \scalebox{0.7}{\usebox{\cspmexample}}
 \end{block}
\end{frame}
%%
%%%%%%%%%%%%%%%%%%%%%%%%%%%%%
%
%%%%%%%%%%%%%%%%%%%%%%%
%% Results - time to verify in FDR4?
\begin{frame}{Results - time to verify in FDR4?}
 \begin{block}{}
     The seven segment example have been run on a Intel(R) Xeon(R) CPU E5-2698 v4 @ 2.20GHz.

     \vspace{5mm}

   The example were run x times and the average was measured.
 \end{block}
\end{frame}
%%
%%%%%%%%%%%%%%%%%%%%%%%%
%
%%%%%%%%%%%%%
%% Conclusion
\begin{frame}{Conclusion}
 \begin{block}{}
  With this system we can transpile hardware models to \cspm{}.
 \end{block}

 \pause

 \begin{block}{}
  Verify values on the \cspm{} channels.
 \end{block}

 \pause

 \begin{block}{}
  Verify the original hardware model.
 \end{block}

 \pause

 \begin{block}{}
  Extract specification.
 \end{block}

\end{frame}
%% /Conclusion
%%%%%%%%%%%%%%
%
%%%%%%%%%%%%%%
%% Future work
\begin{frame}{Future work}
 \begin{block}{}
     Hardware/software co-simulation
 \end{block}

 \pause

 \begin{block}{}
     Creating more extensive examples to show the possibilities of the system
 \end{block}

\end{frame}
%% /Future work
%%%%%%%%%%%%%%%
%
%%%%%%%%%%%%%%%
%%%% Questions?
\begin{frame}{Questions?}
	\begin{block}{}
		Thank you!\\
            \vspace{5mm}
        Feel free to ask anything.
	\end{block}
	% \begin{block}{}
	% 	Contact:
    %     tpq587@alumni.ku.dk
    %     alberte@thegler.dk
	% \end{block}
\end{frame}
%%%% /Questions?
%%%%%%%%%%%%%%%%

\end{document}
