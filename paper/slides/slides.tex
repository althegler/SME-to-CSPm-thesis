\documentclass[13pt]{beamer}
\usetheme[nat,greyfoot,footstyle=low,headstyle=institute,style=alternative]{Frederiksberg}

\usepackage[utf8]{inputenc}
\usepackage[english]{babel}

\usepackage{xcolor}
\usepackage{color}
\usepackage{minted}
\usemintedstyle{emacs}
\usepackage{url}
\def\CC{{C\nolinebreak[4]\hspace{-.05em}\raisebox{.4ex}{\tiny ++}}}

\usepackage{pgfplots}
\usepgfplotslibrary{units}
\pgfplotsset{compat=newest}
\usetikzlibrary{decorations.pathmorphing}

\pgfplotsset{
  every axis plot post/.style={/pgf/number format/fixed}
}

% Math stuff
%\usepackage{amssymb, amsfonts, dsfont, amsthm, amsmath, wasysym, marvosym}

\usepackage{graphicx}
\usepackage[caption=false,font=footnotesize]{subfig}
\newsubfloat{figure}

\title[Towards Automatic Program Specification \\ Using SME Models]{Towards Automatic Program Specification Using SME Models}
\subtitle{\tiny Communicating Process Architectures 2018 -- Technische Universität Dresden}
\author[A. Thegler]{Alberte Thegler}
\institute[Niels Bohr Institute]{Niels Bohr Institute, University of Copenhagen, Denmark}
\date[August 21]{21 August 2018}



\begin{document}

\frame[plain]{\titlepage}

%%%%%%%
%%% TOC
% \begin{frame}{Table of Contents}
%   \begin{enumerate}
%     \item Why should we verify hardware?
%     \item What can SME do?
%     \item Simple example
%     \item CSPm process structure
%     \item Monitor process
%     \item Example continued
%     \item Results - time to verify in FDR4?
%     \item Conclusion
%     \item Future work
%   \end{enumerate}
% \end{frame}
%%% /TOC
%%%%%%%%

%%%%%%%%%%%%%%%%%%%%%%
%%% Why use libraries?
\begin{frame}{Why should we verify hardware?}
  \begin{block}{}
    ...
  \end{block}

  \pause

  \begin{block}{}
     ...
  \end{block}

  \pause

  \begin{block}{}
    ...
  \end{block}
\end{frame}
%%% /Why use libraries?
%%%%%%%%%%%%%%%%%%%%%%%

%%%%%%%%%%%%%%%%%%%%%%%%%%%%%%%%%%
% BLAS, Accelerate, clBLAS, LAPACK
%\begin{frame}{cBLAS, Accelerate, clBLAS, LAPACK}
%  \begin{block}{Why can't we?}
%    So why not just use one of these specialized libraries all the time?
%  \end{block}
%
%  \pause
%
%  \begin{block}{}
%    There exist many different libraries, for many different purposes/architectures/OSes.
%  \end{block}
%\end{frame}
%
%\begin{frame}{cBLAS, Accelerate, clBLAS, LAPACK}
%  \begin{block}{Use the best?}
%    We have \texttt{cBLAS}, Accelerate, \texttt{clBLAS}, \textsc{lapack} and many many more.
%  \end{block}
%
%  \pause
%
%  \begin{block}{}
%    ``Best'' is hard to define.
%    No one of the above is ``best''.
%    They all are ``best'' in their own way.
%  \end{block}
%\end{frame}
%
%\begin{frame}{Coding \textsc{blas}}
%  \begin{block}{Code}
%    \texttt{cBLAS} code:
%    \vspace{5mm}
%
%    \scalebox{0.8}{\usebox{\cblassgemm}}
%  \end{block}
%\end{frame}
%
%\begin{frame}{Coding \textsc{blas}}
%  \begin{block}{Code}
%    Python code:
%    \vspace{5mm}
%
%    \scalebox{0.8}{\usebox{\pythonmatmul}}
%  \end{block}
%\end{frame}
%% /BLAS, clBLAS, Accelerate, LAPACK
%%%%%%%%%%%%%%%%%%%%%%%%%%%%%%%%%%%%
%
%%%%%%%%%%%%%%%%%%%%%%%%%%%
%% Python/NumPy vs. Bohrium
%\begin{frame}{Python/NumPy}
%  \begin{block}{NumPy}
%    NumPy already uses \textsc{blas} for calls like \texttt{matmul}.
%    \\
%    Problem solved?
%  \end{block}
%
%  \pause
%
%  \begin{block}{}
%    No. Python/NumPy is ``slow'' (single threaded) and cannot utilize GPGPUs or other accelerators out-of-the-box.
%  \end{block}
%\end{frame}
%
%\begin{frame}{Python/NumPy vs. Bohrium}
%  \begin{block}{Bohrium}
%    Bohrium can use GPGPUs, but does not support \textsc{blas}.
%  \end{block}
%
%  \pause
%
%  \begin{block}{}
%    Let us make it support these library methods, such as \textsc{blas}.
%  \end{block}
%\end{frame}
%% /Python/NumPy vs. Bohrium
%%%%%%%%%%%%%%%%%%%%%%%%%%%%
%
%%%%%%%%%%%%%%%%%%%%%%%%%%%%%%
%% Code-generation for Bohrium
%\begin{frame}{Code-generation for Bohrium}
%  \begin{block}{Compile}
%    When you compile/install Bohrium, \texttt{CMake} can look for present libraries to link with. NumPy does the same when you compile or install it.
%  \end{block}
%
%  \pause
%
%  \begin{block}{}
%    If we find \textsc{blas} we want to link with it.
%  \end{block}
%
%  \pause
%
%  \begin{block}{}
%    However, if we find \texttt{clBLAS} we would also link with that.
%  \end{block}
%\end{frame}
%
%\begin{frame}{Code-generation for Bohrium}
%  \begin{block}{Choose}
%    With automatic code inclusion, we can choose which library we want to use on compile- and run-time!
%  \end{block}
%
%  \pause
%
%  \begin{block}{}
%    We want Bohrium to link to both \textsc{blas} and choose the correct one later.
%  \end{block}
%\end{frame}
%
%\begin{frame}{Code-generation for Bohrium}
%  \begin{block}{Implementing}
%    We can implement all the \textsc{blas} calls ourselves.
%  \end{block}
%
%  \pause
%
%  \begin{block}{}
%    Tedious. Let's generate it instead!
%  \end{block}
%\end{frame}
%% /Code-generation for Bohrium
%%%%%%%%%%%%%%%%%%%%%%%%%%%%%%%
%
%%%%%%%%%%%%%%%%%%%%%%%%%%%%
%% JSON, template, generate!
%\begin{frame}{JSON, template, generate!}
%  \begin{block}{JSON}
%    All of the \textsc{blas} methods follow a similar pattern.
%  \end{block}
%
%  \pause
%
%  \begin{block}{}
%    Let's use that to our advantage.
%  \end{block}
%\end{frame}
%
%\begin{frame}{JSON, template, generate!}
%  \begin{block}{JSON}
%    \scalebox{0.8}{\usebox{\json}}
%  \end{block}
%\end{frame}
%
%\begin{frame}{JSON, template, generate!}
%  \begin{block}{\CC}
%    \scalebox{0.7}{\usebox{\generated}}
%  \end{block}
%\end{frame}
%% /JSON, template, generate!
%%%%%%%%%%%%%%%%%%%%%%%%%%%%%
%
%%%%%%%%%%%%%%%%%%%%%%%
%% Examples and results
%\begin{frame}{Examples and results}
%  \begin{block}{Run}
%    All of the following CPU examples have been run on a Intel(R) Core(TM) i7-3770 CPU @ 3.40GHz.
%    \vspace{5mm}
%
%    The GPGPU examples are run on GeForce GTX 680 (OpenCL C 1.1).
%    \vspace{5mm}
%
%    All examples were run 10 times and the average was measured.
%  \end{block}
%\end{frame}
%
%\begin{frame}{Examples and results}
%  \begin{block}{}
%    \begin{figure}[!ht]
%      \begin{tikzpicture}[scale=0.7]
%        \begin{axis}[
%          ybar=6pt,
%          bar width=12pt,
%          x=3cm,
%          y=5cm,
%          ymin=0,
%          axis on top,
%          xtick=data,
%          legend pos={outer north east},
%          yticklabel=\pgfmathprintnumber{\tick}\,s,
%          ylabel={Time [s]},
%          enlarge x limits=0.2,
%          symbolic x coords={A},
%          xticklabels={,,},
%          visualization depends on=rawy\as\rawy, % Save the unclipped values
%          after end axis/.code={ % Draw line indicating break
%              \draw [ultra thick, white, decoration={snake, amplitude=1pt}, decorate] (rel axis cs:0,1.05) -- (rel axis cs:1,1.05);
%          },
%          nodes near coords={%
%              \tiny\pgfmathprintnumber{\rawy} % Print unclipped values
%          },
%          axis lines*=left,
%          clip=false,
%          restrict y to domain*=0:1.1, % Cut values off
%          ymax=1.0 % Must be less than cut-off value
%        ]
%          \addplot coordinates {(A,  0.468026)}; % OpenBLAS
%          \addplot coordinates {(A,  0.466213)}; % NumPy
%          \addplot coordinates {(A, 10.910540)}; % Bohrium
%          \addplot coordinates {(A,  0.460929)}; % Bohrium w/EXT
%          \addplot coordinates {(A,  0.080284)}; % clBLAS
%          \addplot [color=green!20!black, fill=green!50!black] coordinates {(A, 17.119442)}; % Bohrium w/OpenCL
%          \addplot coordinates {(A,  0.098525)}; % Bohrium w/OpenCL w/EXT
%
%          \legend{\texttt{OpenBLAS} (C), NumPy, Bohrium, Bohrium w/Ext, \texttt{clBLAS} (\CC), Bohrium w/OpenCL, Bohrium w/OpenCL w/Ext}
%        \end{axis}
%      \end{tikzpicture}
%      \caption{General matrix multiplication ($\mathbf{C} = \mathbf{AB}$) for multiple implementations. Calculating $2000 \times 3000$ multiplied with $3000 \times 4000$ floating point elements.}
%      \label{fig:gemm}
%    \end{figure}
%  \end{block}
%\end{frame}
%
%\begin{frame}{Examples and results}
%  \begin{block}{}
%    The C version (\texttt{OpenBLAS}) is 76 lines of code.
%  \end{block}
%
%  \pause
%
%  \begin{block}{}
%    The \CC version (\texttt{clBLAS}) is 265 lines of code
%  \end{block}
%
%  \pause
%
%  \begin{block}{}
%    The NumPy version is \textbf{11 lines of code} -- and Bohrium is just another library import.
%    Using Bohrium with OpenCL is just an environment variable.
%  \end{block}
%\end{frame}
%
%\begin{frame}{Examples and results}
%  \begin{block}{}
%    \begin{figure}[!ht]
%      \begin{tikzpicture}[scale=0.7]
%        \begin{axis}[
%          ybar=6pt,
%          bar width=12pt,
%          x=2cm,
%          y=5cm,
%          ymin=0,
%          axis on top,
%          xtick=data,
%          legend pos={outer north east},
%          yticklabel=\pgfmathprintnumber{\tick}\,s,
%          ylabel={Time [s]},
%          enlarge x limits=0.2,
%          symbolic x coords={A},
%          xticklabels={,,},
%          visualization depends on=rawy\as\rawy, % Save the unclipped values
%          after end axis/.code={ % Draw line indicating break
%              \draw [ultra thick, white, decoration={snake, amplitude=1pt}, decorate] (rel axis cs:0,1.05) -- (rel axis cs:1,1.05);
%          },
%          nodes near coords={%
%              \tiny\pgfmathprintnumber{\rawy} % Print unclipped values
%          },
%          axis lines*=left,
%          clip=false,
%          restrict y to domain*=0:0.7, % Cut values off
%          ymax=0.6 % Must be less than cut-off value
%        ]
%          \addplot coordinates {(A,   0.022598)}; % OpenCV
%          \addplot coordinates {(A, 10.8134171)}; % NumPy
%          \addplot coordinates {(A,  0.4857039)}; % Bohrium
%          \addplot coordinates {(A,  0.0299198)}; % Bohrium w/EXT
%
%          \legend{OpenCV, NumPy, Bohrium, Bohrium w/Ext}
%        \end{axis}
%      \end{tikzpicture}
%      \caption{Erode for multiple implementations. We use a $10,000 \times 10,000$ pixel random binary image.}
%      \label{fig:erode}
%    \end{figure}
%  \end{block}
%\end{frame}
%% /Examples and results
%%%%%%%%%%%%%%%%%%%%%%%%
%
%%%%%%%%%%%%%
%% Conclusion
%\begin{frame}{Conclusion}
%  \begin{block}{Productivity and performance}
%    We can now utilize the same libraries in Bohrium as in NumPy.
%  \end{block}
%
%  \pause
%
%  \begin{block}{}
%    ... and they work just as fast.
%  \end{block}
%
%  \pause
%
%  \begin{block}{}
%    ... and even faster in cases, where we can utilize GPGPUs.
%  \end{block}
%\end{frame}
%% /Conclusion
%%%%%%%%%%%%%%
%
%%%%%%%%%%%%%%
%% Future work
%\begin{frame}{Future work}
%  \begin{block}{DSL}
%    A more generic DSL-like interface for creating these method inclusions:
%    \vspace{5mm}
%
%    \scalebox{0.7}{\usebox{\future}}
%  \end{block}
%\end{frame}
%% /Future work
%%%%%%%%%%%%%%%
%
%%%%%%%%%%%%%%%
%%%% Questions?
%\begin{frame}{Questions?}
%	\begin{block}{Comments?}
%		Feel free to ask anything.
%	\end{block}
%\end{frame}
%%%% /Questions?
%%%%%%%%%%%%%%%%

\end{document}
