
\usepackage[english, science, titlepage]{ku-frontpage}

\usepackage[latin1]{inputenc}
\usepackage{palatino}
% \usepackage[usenames]{color}

\usepackage{hyperref}

\usepackage{chngpage}
\usepackage{graphicx}

\usepackage{booktabs}
\usepackage{multirow}
\usepackage[nounderscore]{syntax}
\setlength{\grammarparsep}{0.25cm}
\setlength{\grammarindent}{3.5cm}

\usepackage[]{algorithm2e}
\usepackage{varwidth}

\usepackage{subcaption}

\usepackage{setspace}


\usepackage{todonotes}


\usepackage{listings}
\usepackage[outputdir=.tmp]{minted}
% \usepackage[cache=false]{minted}
\setminted{frame=lines,linenos=true,framesep=2mm,fontsize=\small}


% \usepackage[outputdir=.tmp,chapter,newfloat=true,cache=true]{minted}
% \setminted{fontsize=\scriptsize, frame=single, framesep=2mm, autogobble}
% \usemintedstyle{default}
\usepackage{times}

\usepackage{amsmath}
\usepackage{amssymb}
\normalfont
%\usepackage[T1]{fontenc}
\renewcommand{\ttdefault}{cmtt}
\usepackage{xcolor}
\usepackage{booktabs,siunitx}
\usepackage[font=small]{caption}

\usepackage{parcolumns}

% NOTE: The margins become wider just by including the geometry package
\usepackage{geometry}
\geometry{
%  % a4paper,
%  % total={170mm,257mm},
%  % left=20mm,
 top=30mm,
 bottom=40mm
 }

% \usepackage{url}
\usepackage{pgfplots}
\usepackage{tikz}

\usepackage[labelsep=period]{caption}

\usepackage{cspsymb}

\usetikzlibrary{calc}
\usetikzlibrary{fit}
\usetikzlibrary{positioning}
\usepgfplotslibrary{units}
\pgfplotsset{compat=newest}
\usetikzlibrary{decorations.pathmorphing}
\usetikzlibrary{decorations.markings}
\usetikzlibrary{arrows}
\usetikzlibrary{shapes.geometric}

\usepackage{ifthen}
\pgfkeys{
  /sevenseg/.is family, /sevenseg,
  slant/.estore in      = \sevensegSlant,     % vertical slant in degrees
  size/.estore in       = \sevensegSize,      % length of a segment
  shrink/.estore in     = \sevensegShrink,    % avoids overlapping of segments
  line width/.estore in = \sevensegLinewidth, % thickness of the segments
  line cap/.estore in   = \sevensegLinecap,   % end cap style rect, round, butt
  oncolor/.estore in    = \sevensegOncolor,   % color of an ON segment
  offcolor/.estore in   = \sevensegOffcolor,  % color of an OFF segment
}

\pgfkeys{
  /sevenseg,
  default/.style={
    slant = 0,
    size = 1em,
    shrink = 0.2,
    line width = 0.3em,
    line cap = butt,
    oncolor = green!50!black,
    offcolor = white!75!black
  }
}
\newcommand{\sevenseg}[2][]% options values
{%
\pgfkeys{/sevenseg, default, #1}%
\def\sevensegarray{#2}%
  \begin{tikzpicture}%
    % first define the position of the 6 corner points
    \path (0,0) ++(0,0)                             coordinate (P1);
    \path (0,0) ++(\sevensegSize,0)                 coordinate (P2);
    \path (0,0) ++(90-\sevensegSlant:\sevensegSize) coordinate (P3);
    \path (P2)  ++(90-\sevensegSlant:\sevensegSize) coordinate (P4);
    \path (P3)  ++(90-\sevensegSlant:\sevensegSize) coordinate (P5);
    \path (P4)  ++(90-\sevensegSlant:\sevensegSize) coordinate (P6);
    % then step through the 1/0 values in the segment array
    \foreach \i in {0,...,6}%
    {
      \pgfmathparse{\sevensegarray[\i]}
      \ifthenelse{\equal{\pgfmathresult}{1}}%
        {\let\mycolor=\sevensegOncolor}%  segment is on
        {\let\mycolor=\sevensegOffcolor}% segment is off
      \tikzstyle{segstyle} = [draw=\mycolor, line width = \sevensegLinewidth,
                              line cap = \sevensegLinecap]
      %-----------------------
      \ifthenelse{\equal{\i}{0}}{\path[segstyle]
        (${1-\sevensegShrink}*(P5)+\sevensegShrink*(P6)$)
        -- ($\sevensegShrink*(P5)+{1-\sevensegShrink}*(P6)$);}{} % a
      \ifthenelse{\equal{\i}{1}}{\path[segstyle]
        (${1-\sevensegShrink}*(P6)+\sevensegShrink*(P4)$)
        -- ($\sevensegShrink*(P6)+{1-\sevensegShrink}*(P4)$);}{} % b
      \ifthenelse{\equal{\i}{2}}{\path[segstyle]
        (${1-\sevensegShrink}*(P4)+\sevensegShrink*(P2)$)
        -- ($\sevensegShrink*(P4)+{1-\sevensegShrink}*(P2)$);}{} % c
      \ifthenelse{\equal{\i}{3}}{\path[segstyle]
        (${1-\sevensegShrink}*(P1)+\sevensegShrink*(P2)$)
        -- ($\sevensegShrink*(P1)+{1-\sevensegShrink}*(P2)$);}{} % d
      \ifthenelse{\equal{\i}{4}}{\path[segstyle]
        (${1-\sevensegShrink}*(P1)+\sevensegShrink*(P3)$)
        -- ($\sevensegShrink*(P1)+{1-\sevensegShrink}*(P3)$);}{} % e
      \ifthenelse{\equal{\i}{5}}{\path[segstyle]
        (${1-\sevensegShrink}*(P3)+\sevensegShrink*(P5)$)
        -- ($\sevensegShrink*(P3)+{1-\sevensegShrink}*(P5)$);}{} % f
      \ifthenelse{\equal{\i}{6}}{\path[segstyle]
        (${1-\sevensegShrink}*(P3)+\sevensegShrink*(P4)$)
        -- ($\sevensegShrink*(P3)+{1-\sevensegShrink}*(P4)$);}{} % g
    }
  \end{tikzpicture}%
}

\newcommand{\sevensegnum}[2][]% sample characvters
{%
  \ifthenelse{\equal{#2}{0}}{\sevenseg[#1]{{1,1,1,1,1,1,0,}}}{%
  \ifthenelse{\equal{#2}{1}}{\sevenseg[#1]{{0,1,1,0,0,0,0,}}}{%
  \ifthenelse{\equal{#2}{2}}{\sevenseg[#1]{{1,1,0,1,1,0,1,}}}{%
  \ifthenelse{\equal{#2}{3}}{\sevenseg[#1]{{1,1,1,1,0,0,1,}}}{%
  \ifthenelse{\equal{#2}{4}}{\sevenseg[#1]{{0,1,1,0,0,1,1,}}}{%
  \ifthenelse{\equal{#2}{5}}{\sevenseg[#1]{{1,0,1,1,0,1,1,}}}{%
  \ifthenelse{\equal{#2}{6}}{\sevenseg[#1]{{1,0,1,1,1,1,1,}}}{%
  \ifthenelse{\equal{#2}{7}}{\sevenseg[#1]{{1,1,1,0,0,0,0,}}}{%
  \ifthenelse{\equal{#2}{8}}{\sevenseg[#1]{{1,1,1,1,1,1,1,}}}{%
  \ifthenelse{\equal{#2}{9}}{\sevenseg[#1]{{1,1,1,1,0,1,1,}}}{%
  \ifthenelse{\equal{#2}{A}}{\sevenseg[#1]{{1,1,1,0,1,1,1,}}}{%
  \ifthenelse{\equal{#2}{B}}{\sevenseg[#1]{{0,0,1,1,1,1,1,}}}{%
  \ifthenelse{\equal{#2}{C}}{\sevenseg[#1]{{0,0,0,1,1,0,1,}}}{%
  \ifthenelse{\equal{#2}{D}}{\sevenseg[#1]{{0,1,1,1,1,0,1,}}}{%
  \ifthenelse{\equal{#2}{E}}{\sevenseg[#1]{{1,0,0,1,1,1,1,}}}{%
  \ifthenelse{\equal{#2}{F}}{\sevenseg[#1]{{1,0,0,0,1,1,1,}}}{%
  {\sevenseg[#1]{{0,0,0,0,0,0,0,}}}}}}}}}}}}}}}}}}}%
}

\tikzset{
  myarrow/.style={
    draw=black,
    thick,
    ->,
    shorten <=3pt,
    shorten >=3pt,
  },
  mycircle/.style={
    draw=black,
    shape=circle,
    very thick,
    inner sep=3pt,
    inner ysep=5pt,
    text width=0.75cm,
    align=center,
    minimum size=0.75cm,
    rounded corners,
  },
  mytriangle/.style={
    draw=black,
    regular polygon,
    regular polygon sides=3,
    align=center,
    rounded corners,
    very thick,
    inner sep=3pt,
  },
  myrectangle/.style={
    draw=black,
    shape=rectangle,
    very thick,
    rounded corners,
    align=center,
    inner sep=7pt,
    inner ysep=7pt,
    text width=2.1cm,
    minimum size=0.5cm,
    minimum height=1.5cm,
    font=\footnotesize
  },
  mysquare/.style={
    draw=black,
    shape=rectangle,
    very thick,
    rounded corners,
    align=center,
    inner sep=7pt,
    inner ysep=7pt,
    font=\footnotesize
  },
  main node/.style={
  circle,
  align=center,
  draw,
  text width=.7cm,
  minimum size=.7cm,
  inner sep=7pt,
  font=\footnotesize
  },
  mythinsquare/.style={
  draw=black,
  shape=rectangle,
  rounded corners,
  align=center,
  inner sep=4pt,
  % inner ysep=7pt,
  font=\footnotesize
  },
}

\pgfplotsset{
  every axis plot post/.style={/pgf/number format/fixed}
}

\newcommand{\codemargins}[0]{\paperwidth-2cm}

\newenvironment{widefigure}[1][]{
  \begin{figure}[#1]
  \centering
  \begin{minipage}[t]{\codemargins}
  }{\end{minipage}
   \end{figure}
}



% \newcommand\todo[1]{\textcolor{red}{#1}}
\newcommand{\cspm}{CSP$_M$}



\usepackage{fancyhdr}
\pagestyle{fancy}

