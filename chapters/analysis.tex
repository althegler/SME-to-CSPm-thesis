%!TEX root = ../main.tex
In this section, I will analyse the differences between the SMEIL model and the \cspm{} language. I will also introduce important parts of the SMEIL structure and grammar to provide an understanding of the different challenges that lie in translating a hardware model to a specification model.
\section{Verification}
\label{sec:analysis_verification}
The reason for translating SMEIL to \cspm{} is to use the refinement checking properties of the CSP language. By loading a \cspm{} program into FDR4, it is possible to utilise these properties on the SMEIL network. Before I can design the transpiler and generate the \cspm{} code, it is essential to examine what kind of properties would be beneficial to verify in hardware. It is also relevant to explore how these properties can be verified with FDR4. As mentioned in Chapter \ref{chap:background}, FDR4 provides refinement checking with different models, like the traces and failures models. FDR4 is often used for deadlock checking, but since the SME model guarantees deadlock freedom, it is unnecessary to use this property within FDR4. \\
% NOTE: I can't do this because of the trace of SKIP
% It might, however, be interesting to perform deadlock checking to verify that the translation was done correctly. Since the SMEIL model cannot deadlock, the generated \cspm{} code should also be deadlock free.\\

In hardware, it is typically relevant to verify that the communication on a bus does not exceed a certain range, or that the sum of multiple signals do not exceed a specific value. A bus might be able to carry unexpected data, and being able to formally verify that this never happens, is of great value. Therefore it would be interesting to verify that certain values are never communicated on a specific channel, or that they are the only values communicated on the channel. When designing hardware, each component is designed separately, and usually several components are combined to create a larger network with more functionality. When modeling a small component, it can be difficult to predict the possible input values for the component, especially when parallel structures are involved. It is therefore interesting to be able to verify that a specific range of input values will not cause failures in the parallel model.\\

To provide an understanding of the type of problem that this kind of verification would be able to solve, an example is introduced in Section \ref{sec:example-seven_segment_intro}, which will be referred to several times throughout this thesis.
\subsection{Seven Segment Display Example}
\label{sec:example-seven_segment_intro}
A seven segment display is an electronic display device which is used in digital clocks or other devices that display numerals. An example of a typical digital clock display can be seen in Figure~\ref{fig:6_displays}. As the name states, each display consists of seven segments which can be lit up in patterns to display symbols, like numerals.
When a digit has been determined for a seven segment display, it is encoded to a bitstream that represents the digit in the correctly activated display segments.
\begin{figure}
  \begin{center}
    \tikz{
      \node[inner sep=5pt, outer sep=2pt, draw=blue] {
        \sevensegnum[size=2em, shrink=0.1]{1}
        \sevensegnum[size=2em, shrink=0.1]{2}
      }
    }
    \tikz{
      \node[inner sep=5pt, outer sep=2pt, draw=blue] {
        \sevensegnum[size=2em, shrink=0.1]{3}
        \sevensegnum[size=2em, shrink=0.1]{4}
      }
    }
    \tikz{
      \node[inner sep=5pt, outer sep=2pt, draw=blue] {
        \sevensegnum[size=2em, shrink=0.1]{5}
        \sevensegnum[size=2em, shrink=0.1]{6}
      }
    }
  \end{center}
  \caption{Digital clock with six seven segment displays, displaying 12:34:56.}
  \label{fig:6_displays}
\end{figure}
In this example, we wish to model a typical digital clock that is able to calculate and display the current time in hours, minutes, and seconds. Listing~\ref{lst:python} shows a similar example written in Python. Like \texttt{time\_since\_midnight} in Listing~\ref{lst:python}, the network must have some input of data. The input value represents seconds since midnight, and in order to calculate hours, minutes, and seconds we model three different processes, called the \texttt{time} processes in this example.\\

When writing hardware models in pure SMEIL, one of the possible ways to generate input for the network is to create a data generator process. This process, called the \texttt{clock} process in our example, is instantiated with a start time and is incremented by 1 for each simulation cycle, representing a one second increase. The result of the \texttt{clock} process is communicated on the process output bus where the three \texttt{time} processes are listening. These \texttt{time} processes receive the number and by the use of simple integer arithmetic, calculate the hours, minutes, and seconds since midnight respectively. It is obvious that at some point in time, each \texttt{time} process will calculate a two-digit result, for example at 12 hours or 42 seconds. However, a single seven segment display can only show one digit between 0 and 9. Therefore we need two seven segment displays for each \texttt{time} process in order to show the correct time in a 24-hour interval. Each \texttt{time} process has an output bus with two individual channels that represent the communication to each different display. The number representing either hours, minutes, or seconds are separated into first and second digit, by $\lfloor \frac{x}{10} \rfloor$ and $(x \text{ mod } 10)$. These six different results are then communicated onto the six different channels which represent the six different seven segment displays. The outline of this network can be seen in Figure~\ref{fig:smeil_network} where the network consists of four processes. The data generator process, \textit{I}, which creates the input that is broadcast out on the network. The three \texttt{time} processes, hours (\textit{H}), minutes (\textit{M}), and seconds (\textit{S}) are the processes described above, which calculate each part of the current time. The outputs are communicated on the six outgoing channels.

\begin{listing}
\begin{minted}{python}
from math import floor

def time(time_since_midnight):
    hours   = floor(time_since_midnight / 3600)
    minutes = floor((time_since_midnight - hours * 3600) / 60)
    seconds = time_since_midnight - hours * 3600 - minutes * 60
    return [hours, minutes, seconds]

print(time( 57100)) # =>  [15,51,40] or 15:51:40
print(time(  3601)) # =>  [01,00,01] or 01:00:01
print(time( 66666)) # =>  [18,31,06] or 18:31:06
\end{minted}
\caption{A Python implementation of the seven segment display example.}
\label{lst:python}
\end{listing}
\begin{figure}[!ht]
  \centering
  \begin{tikzpicture}
    \node [mycircle] (I) at (0,0) {$I$};

    \node [mycircle] (H) at (2.5,  1.50) {$H$};
    \node [mycircle] (M) at (2.5,  0.00) {$M$};
    \node [mycircle] (S) at (2.5, -1.50) {$S$};

    \draw [myarrow] (I) -- (M);

    \draw [myarrow, smooth] (I) to[out=0, in=180] (H);
    \draw [myarrow, smooth] (I) to[out=0, in=180] (S);

    % Output arrows without processes
    \draw [myarrow] (3.125,  1.625) -- (4.000,  1.750);
    \draw [myarrow] (3.125,  1.375) -- (4.000,  1.250);
    \draw [myarrow] (3.125,  0.125) -- (4.000,  0.250);
    \draw [myarrow] (3.125, -0.125) -- (4.000, -0.250);
    \draw [myarrow] (3.125, -1.375) -- (4.000, -1.250);
    \draw [myarrow] (3.125, -1.625) -- (4.000, -1.750);
  \end{tikzpicture}
  \caption{SMEIL network for a seven segment display clock. Each SMEIL process is represented by a cicle with a letter corresponding to the processes \texttt{input}, \texttt{hours}, \texttt{minutes} and \texttt{seconds} respectively.}
  \vspace{-3mm}
  \label{fig:smeil_network}
\end{figure}
In this example, the properties we wish to assert with FDR4, are the width of the channels. That is, we want to prove that certain values will never be communicated on certain channels. One could imagine that 4 bits can be communicated between the \texttt{time} processes and the seven segment displays. But 4 bits can represent the numbers 0 through 15, and our seven segment displays can only display the numbers 0 through 9. Therefore we wish to assert that even though the channels can carry 4 bits, the actual communication on the six output channels does not exceed 9. In general, the displays will always be able to display 0 through 9, but since the example is a clock showing a 24-hour interval, the displays will of course not be able to show minutes and seconds above 59 and hours above 23. The full SMEIL code for this example can be seen in Listing~\ref{lst:smeil} in the appendix.
\section{Transpiling from SMEIL to \cspm{}}
\label{sec:transpiling}
Since I am translating code from SMEIL to \cspm{}, the challenge is to find \cspm{} structures that correspond to the SMEIL structures. The ultimate goal is to find methods for transpiling, that can be generalised to most problems. From the introduction to both SMEIL and \cspm, it is clear that the main intention of each language is different and therefore the transpiling of a process in SMEIL to a \cspm{} process might not be completely trivial.\\

When analysing the general structure of SMEIL programs, there are three structures that are particularly interesting in terms of translating from SMEIL to \cspm{}. These three structures are:
\begin{enumerate}
    \item \textbf{Behavioral:} The general behaviour of each process.
    \item \textbf{Structural:} How the circuit is connected, i.e which buses connect to which processes.
    \item \textbf{Verification data:} All data from the SMEIL network that could be useful for the verification of the hardware model.
\end{enumerate}
\subsection{Behavioral}
The processes in an SMEIL program is describing the basic behaviour of the SMEIL program. It is necessary to analyse the different process structures in SMEIL in order to understand how it can be translated into \cspm{}.
\subsubsection{Processes}
An SMEIL process is defined by the \texttt{proc} keyword and consists of an identifier, process parameters, bus, and variable declarations. The body of the process, the process statements, consists of sequential statements such as communications and calculations that are to be evaluated once for each clock cycle. A small example of an SMEIL process has been presented in Listing \ref{lst:smeil_small_syntax_example} in Chapter \ref{chap:background}. \\

A process is initiated in the network of an SMEIL program, which will be explained further in Section \ref{sec:analysis_structural}. A process can be instantiated with a set of parameters. These parameters can be a mix of input and output buses and constants, and similar parameters can be defined for a \cspm{} process. In \cspm{} the process behaviour also represents functionality, but the behaviour is often more focused on communicative behaviour between processes.\\

A typical process in \cspm{} could look like the one in Listing~\ref{lst:dining_philosopher_cspm} which represents a philosopher process from the dining philosophers problem. The process behaviour shows the actions of a philosopher process when communicating with other processes. As can be seen, the process also conduct some simple calculations, but the main structure consists of communication, which is the main purpose of \cspm{}, to be able to describe communication between processes.
\begin{listing}
\begin{minted}{cspm_lexer.py:CSPmLexer -x}
PHIL(i) = thinks.i -> sits!i -> picks!i!i -> picks!i!((i+1)%N) ->
          eats!i -> putsdown!i!((i+1)%N) -> putsdown!i!i -> getsup!i -> PHIL(i)

\end{minted}
\caption{A dining philosopher process from the dining philosophers problem example file provided at the FDR4 webpage~\cite{fdr_example}.}
\label{lst:dining_philosopher_cspm}
\end{listing}

In the SMEIL grammar, it is possible to declare a process to be synchronous or asynchronous by the keywords \texttt{sync} and \texttt{async}. A \texttt{sync} process is run during each clock cycle and an \texttt{async} process is only run when it receives a signal on the input bus. The current implementation of SMEIL does not support \texttt{async} processes, so all SMEIL process are synchronised and behave like described in Chapter \ref{chap:background}, where they read, compute and write once in each clock cycle. The synchronicity of the processes is an essential part of the SMEIL network and therefore it is crucial that the generated code can model this correctly. In \cspm{} a process does not communicate until it receives an input, which matches the asynchronous process of SMEIL. However, since only the synchronous processes are supported in SMEIL, it is necessary to introduce a structure to the \cspm{} network that simulates the synchronous structure of the SMEIL process.\\

In an SMEIL process, the declarative part of the process can consist of different variables and constants as well as bus definitions. Next, variables and constants are introduced, whereas buses are described in section \ref{sec:analysis_structural}.
\paragraph{Variables and Constants}
It is not possible to declare variables, constants, or communication buses inside the process statements of an SMEIL process. Therefore all of these must be declared in the declarative part of the process, which should simplify the translation to \cspm{}, since it will only be necessary to search the declarative part of the process to find all variable and bus definitions.\\

In SMEIL, constants consist of an identifier, a type, and an expression. The expression could be an integer assigned to the constant, for instance:
\begin{minted}{smeil_lexer.py:SMEILLexer -x}
const hours: uint = 24;
\end{minted}
Variables are very similar to constants but simply hold mutable values within the process. In SMEIL processes, variables preserve their values between clock cycles, which mean that it is possible for a process to save a value or result, and reuse it in the next clock cycle. The difference between variable and constants in SMEIL are that variables can contain an optional expression and an optional range of values, for example:
\begin{minted}{smeil_lexer.py:SMEILLexer -x}
var minutes: u6 = 0 range 0 to 59;
\end{minted}
The assignment \texttt{u6 = 0} is providing an initial value to the variable, but both initial values and ranges are optional. The range declaration defines the range of expected values for the variable. \\

In \cspm, constants can be declared as we know it from many other languages and variables can be used without declaring them first, for example in a process declaration within a local definition.
\begin{minted}{cspm_lexer.py:CSPmLexer -x}
N = 5 -- Constant declaration

Proc(x) =
    let
        m = x % 60 -- Variable without declaration
    within
        channel ! m -> SKIP
\end{minted}
A variable is also defined when communicating specific values over channels in \cspm. For example in the example below, the received value is assigned to the variable $x$ which is then written to the \texttt{output} channel.
\begin{minted}{cspm_lexer.py:CSPmLexer -x}
P = input ? x -> output ! x -> STOP
\end{minted}

In SMEIL, the type of a variable, a constant or a bus channel, must always be defined. A type can represent either unlimited size integers, \texttt{int}, and \texttt{uint}, or be restricted to a specific bit-size. The prefixes \texttt{i} and \texttt{u} refer to signed and unsigned integers followed by a number determining the bit-length. For instance, \texttt{i4} represents a signed 4-bit integer. The unlimited size integers are not practical to have in \cspm{} because when verifying a program, FDR4 will look at the entire possible state space and with unlimited integers the state space will be unbounded. FDR4 restricts its integer types to signed 32-bit~\cite{fdrmanual}. The LIBSME compiler provides, because of its type system, a warning if the values of a constant or variable are above or below the range of the type bit size. In the example below the number 59 cannot be represented by 4 bits and therefore it would cause a warning from the compiler.\\
\begin{minted}{smeil_lexer.py:SMEILLexer -x}
var minutes_wrong_type: u4 = 0 range 0 to 59;
\end{minted}

Translating constants and variables from SMEIL to \cspm{} will not prove to be difficult since the structures are quite similar.
A potential challenge lies in deciding how to define a variable that has been declared with an initial value in SMEIL. The \cspm{} translation will have to define the variable before it is used within the calculations, which might increase the complexity of the translation.\\

The second part of an SMEIL process is the statements where the actual behaviour of the process is defined. The semantics of a statement in SMEIL corresponds to what we typically see in C-like languages, some are mentioned here below.
\paragraph{Assignments}
An assignment in SMEIL consists of a name and an expression. It can be used in two different ways within the SMEIL process statements: assigning to a variable and assigning to a bus channel. In SMEIL, the compiler will always be able to recognise what is being assigned by looking at the type of the object and therefore SMEIL does not differentiate between these two assignments. This property will cause a challenge when translating to \cspm{} since communication and variable assignments can be two different things in a \cspm{} program. Thus TAPS must recognise the type of the assignment, in order to create the correct translation.
\paragraph{if-statements}
\texttt{if} statements in SMEIL are structured similarly to most other programming languages with the keywords \texttt{if}, \texttt{then}, \texttt{elif} and \texttt{else}. \texttt{if} statements can also be defined in \cspm, but \cspm{} does not support the \texttt{elif} expressions, so the translation must restructure the \texttt{elif} expression to a nested \texttt{if then} expression. These \texttt{if} statements can quickly become quite complex, but it should not be difficult to auto-generate.\\

To translate processes from SMEIL to \cspm{} is a challenge because they cannot be translated directly. The challenges lie in structuring the \cspm{} processes so that the \cspm{} code provides the same functionality as the SMEIL model while still keeping the \cspm{} communication behaviour and allowing assertion properties.
\subsubsection{Generating data}
It will always be necessary to generate input for pure SMEIL networks. If the program was not written as a pure SMEIL program, the input for the network would be provided by the surrounding code, but in the case of pure SMEIL, the input data must be generated, which can be done in a few ways.
One way of initialising the data in the SMEIL network is by instantiating the process with a constant given as a parameter or hardcode internal values into the process. Another way is to have a separate process that generates data for the network. I call this type of process a data generator process. Examples of both can be seen in Listing \ref{lst:addone_data_generation_example} and in Listing \ref{lst:clock_data_generation_example_smeil}.\\

Listing \ref{lst:addone_data_generation_example} shows an example of the network \texttt{addone\_network} with two instantiated processes. The \texttt{add} process reads an input value and a constant, add the two values together, and writes it out onto the output bus. The \texttt{id} process reads an input and writes it to the output bus immediately. Networks in SMEIL are introduced further in Section \ref{sec:analysis_structural}, but the network in this example is defining the input bus of the \texttt{add} process to be the output bus of the \texttt{id} process, as well as defining that the constant value in the \texttt{add} process is 1.
The \texttt{addone} example will be introduced further in Chaper \ref{chap:clock}.\\

\begin{listing}
\begin{minted}{smeil_lexer.py:SMEILLexer -x}
proc add (in inbus, const constant)
    bus outbus {
        val: uint;
    };
{
    outbus.val = inbus.val + constant;
}

proc id (in inbus)
    bus outbus {
        val: uint;
    }
    var from_add : uint = 0;
{
    from_add = inbus.val;
    outbus.val = from_add;
}
network add_network() {
    instance a of add(i.outbus, constant: 1);
    instance i of id(a.outbus)
}
\end{minted}
\caption{The SMEIL network \texttt{add\_network} with two processes. The \texttt{add} process is instantiated with a value which is constant and used once for each clock cycle. The example is similar to the Addone example in \cite{smeil}.}
\label{lst:addone_data_generation_example}
\end{listing}
Listing \ref{lst:clock_data_generation_example_smeil} shows another way to instantiate data into the SMEIL network. Here the process \texttt{clock} is a data generator process. It does not read data from any input bus, so it can only generate data to write to the network. The example shows the \texttt{minutes} process calculating the number of minutes that have passed since the simulation started by dividing the input with 60.\\

\begin{listing}
\begin{minted}{smeil_lexer.py:SMEILLexer -x}
proc clock()
    bus outbus {
        val: uint;
    };
    var i: uint = 0;
{
    i = i + 1;
    outbus.val = i;
}

proc minutes (in inbus)
    bus outbus {
        val: uint;
    };
    from_clock : uint;
{
    from_clock = inbus.val / 60;
    outbus.val = from_clock
}


network minutes_net() {
    instance c of clock();
    instance m of minutes(c.outbus)
}
\end{minted}
\caption{The SMEIL network \texttt{minutes\_net}, with a data generator process and a calculation process that calculates minutes since simulation start.}
\label{lst:clock_data_generation_example_smeil}
\end{listing}

The data generator process has the same structure as any other process in SMEIL and it is therefore crucial that TAPS can recognise a data generator process from other processes. An SMEIL process that does not read any input is obviously a data generation process, however, in SMEIL it is possible to communicate to or from buses directly in the process body without defining them as input parameters. This will be explained further in Section \ref{bus_and_channels}. \\

An SMEIL network can consist of different permutations of data generator processes and processes with initial values. This might increase the complexity of the translation but it should be feasible as long as TAPS can separate the data generator processes from other processes.
\subsection{Structural}
\label{sec:analysis_structural}
The backbone of an SMEIL network is the communication between the processes. Analysing the different communication structures in SMEIL will provide the insight necessary for designing the translation structures to create equivalent communication in the generated \cspm{} network.
\subsubsection{Network}
A network in an SMEIL program is a crucial part that connects all processes together with communication. In an SMEIL network, processes are instantiated using the \texttt{instance} keyword within the network declaration. This instance declaration instantiates the process with a set of parameters defined for the specific process.
Defining the network from process instances also has the advantage that one process can be instantiated with different parameters several times within the same network, providing the possibility of reusing the processes for different purposes. An example of an SMEIL network can be seen in Listing \ref{lst:smeil_two_instantiations_example}, where the \texttt{add} process from Listing \ref{lst:addone_data_generation_example} is instantiated twice with different constant values.\\
\begin{listing}
    \begin{minted}{smeil_lexer.py:SMEILLexer -x}
network add_net() {
    instance a1 of add(i.outbus, constant: 1);
    instance a2 of add(i.outbus, constant: 2);
    instance i of id(a.outbus)
}
    \end{minted}
    \caption{Example of two instantiations of the \texttt{add} process from Listing \ref{lst:addone_data_generation_example}.}
    \label{lst:smeil_two_instantiations_example}
\end{listing}

In \cspm{} there is no network structure, but \cspm{} does have a structure for declaring parallel communication which resembles the instance declarations in SMEIL. The parallel operators define communication between processes in \cspm{}, which is described in Chapter \ref{chap:background}.
While it is possible to model equivalent communication by using these operators it quickly becomes complex, because the operators only syncronise two processes at a time. It is a challenge to ensure that an SMEIL network is translated to an equivalent \cspm{} structure, and it requires careful planning to provide a complete translation of all the relevant information. However it is possible due to the simple network structure in SMEIL.
\subsubsection{Buses and Channels}
\label{bus_and_channels}
As previously explained, an SMEIL bus defines a collection of channels which are used for all communication between the processes. Each channel has a type describing the communicated data and can be initialised with an initial value. The syntax of a read or write in SMEIL is \texttt{bus.channel} where \texttt{bus} specifies the bus and \texttt{channel} specifies the channel within the bus. An example of this can be seen in the \texttt{add} process in Listing \ref{lst:addone_data_generation_example} where \texttt{inbus.val} refers to the \texttt{inbus} bus and \texttt{val} refers to the channel within the bus.
A bus in itself does not have a type or values, but all channels within a bus have an identifier, types, and values, however the same types do not have to apply to all channels in a single bus. An SMEIL bus does have an identifier which is used for referencing the bus. All channels within a bus are connected to the process at the same time, and it is up to the developer to call the correct channel within the bus for either a read or a write.
An example of a bus definition in SMEIL can be seen in Listing \ref{lst:smeil_three_channels}. The bus is identified with the name \texttt{day} and it consists of three channels, \texttt{hours}, \texttt{minutes}, and \texttt{seconds}. Each channel is defined with a type, \texttt{u5} or \texttt{u6} and a range, 0 through 23 or 0 through 59.
\begin{listing}
\begin{minted}{smeil_lexer.py:SMEILLexer -x}
bus day {
    hours:   u5 range 0 to 23;
    minutes: u6 range 0 to 59;
    seconds: u6 range 0 to 59;
};
\end{minted}
\caption{Example of an SMEIL bus channel with three channels.}
\label{lst:smeil_three_channels}
\end{listing}
A single SMEIL bus channel corresponds to a single \cspm{} channel and since the channel is a more general structure than the bus structure, it should be relatively simple to translate.
\paragraph{SMEIL Input Bus}
There are two different methods for accessing an input bus in an SMEIL process. The name of the input bus can be given as a process parameter, which the process can use to read from the actual bus channel, as can be seen in the \texttt{add} and \texttt{id} processes in Listing \ref{lst:addone_data_generation_example}. In this example, the processes reads the data from the channel \texttt{val} in the bus \texttt{inbus} and writes the data to the channel \texttt{val} on the \texttt{outbus} bus.\\

It is also possible to use the formal identifier of the bus in the process body, thereby not accessing it through the input parameter. An example of this can be seen in Listing \ref{lst:smeil_example_formal_channel_name}. Here the \texttt{add} process reads the input by accessing the formal name of the bus channel \texttt{val} within the bus \texttt{outbus} defined in process \texttt{id}. As a result of this, the process can be instantiated several times in the network but it will always read from the channel \texttt{id.outbus.val}.
% TODO: Can this be the clock example in stead?
\begin{listing}
\begin{minted}[escapeinside=&&, mathescape=true]{smeil_lexer.py:SMEILLexer -x}
proc add ()
    bus outbus {
        val: uint;
    }
{
    outbus.val = id.outbus.val;
}

proc id ()
    bus outbus {
        val: uint;
    }
{
    outbus.val = add.outbus.val;
}

network add_id_net ()
{ instance _ of add();
  instance _ of id();
}
\end{minted}
\caption{Example showing how processes can read from a channel using its formal identifier.}
\label{lst:smeil_example_formal_channel_name}
\end{listing}
\paragraph{SMEIL Output Bus}
As with the input bus in SMEIL, there are several methods for defining and accessing the output bus in SMEIL.
As seen in Listing \ref{lst:smeil_example_formal_channel_name}, the output bus can be defined inside the process itself using the keyword \texttt{bus} along with channel definitions. As with input buses, it is also possible to reference the formal name of the channel directly inside the process body in the same way as can be seen in Listing \ref{lst:smeil_example_formal_channel_name}. Lastly, it is possible to give the output bus as a process parameter.
An example of this can be seen in Listing \ref{lst:smeil_example_output_parameter} where the process \texttt{add} reads from its input bus and writes to its output bus, both defined as parameters. The input bus for the \texttt{add} process is the \texttt{outbus} defined in the \texttt{constant} process. The output bus of the \texttt{add} process is a bus defined inside the network called \texttt{netbus}. In this example, the network calculates \texttt{1 + 1} in each clock cycles.\\

\begin{listing}
\begin{minted}{smeil_lexer.py:SMEILLexer -x}
proc add (in inbus, out outbus)
{
    outbus.val = inbus.val + 1;
}

proc constant ()
    bus outbus {
        val: uint;
    }
{
    outbus.val = 1;
}

network add_const_net() {
    bus netbus {
        val: uint;
    }
    instance const of constant();
    instance add of add(const.outbus, netbus);

}
\end{minted}
\caption{SMEIL example of a process \texttt{add} having both input and output bus as parameters.}
\label{lst:smeil_example_output_parameter}
\end{listing}

SMEIL was designed to be very flexible. Therefore, each process can have any number and combinations of input buses, output buses, and constants as parameters as well as communication directly inside the process body. TAPS will have to be able to translate all these different methods of communication.

\paragraph{\cspm{} Channels}
Like in SMEIL, it is possible to communicate either by using a process parameter name or by communicating directly using the channels global name. These similarities between SMEIL and \cspm{} should provide a relatively simple translation of both communication methods in TAPS.\\

An example of a \cspm{} process, communicating both via its process parameter and via a global name, can be seen in Listing \ref{lst:cspm_communication_methods}. The process \texttt{P} reads a value from the \texttt{input} channel, which is the channel \texttt{d}. The process then writes the value on the \texttt{c} channel and terminates.
\begin{listing}
\begin{minted}{cspm_lexer.py:CSPmLexer -x}
channel c : {0..10}
channel d : {0..10}

P(input) = input ? x -> c ! x -> SKIP

Network = P(d) -> SKIP
\end{minted}
\caption{Example of different communication methods in \cspm{}.}
\label{lst:cspm_communication_methods}
\end{listing}
\subsection{Verification data}
As described in Section \ref{sec:analysis_verification}, one of the goals of this thesis is to verify data communicated in an SMEIL network. For example, in the seven segment example, it would be highly valuable to verify that the values communicated to the displays are never outside the range of values defined. For TAPS to create the assertions in \cspm{}, it is essential to examine which values should be allowed to be communicated, and which should not.
As introduced in Chapter \ref{chap:background} during the simulation of an SMEIL program, all observed values on each channel are tracked and turned into a range of the maximum and minimum values for that specific channel. During the simulation, the type representing the values of the channel will also be restricted to the shortest representation possible. For example, if a channel was originally set to be \texttt{int} (unbounded), and the observed values from the simulation showed that it did not take on other values than the range -110 to 110, the type could be changed to an \texttt{i8} instead, which represents a signed 8-bit integer with a range of -128 to 127.\\

In Listing \ref{lst:range_smeil} an example of the simulated \texttt{seconds} process from the seven segment example can be seen. As explained previously, the seven segment displays can only display digits from 0 through 9, but in this example, the simulation of the system results in a more restricted range of values. As explained in Section \ref{sec:analysis_verification}, the \texttt{seconds} process cannot write values over 5 for the first digit and 9 for the second. This is also the result from the simulation, as can be seen in Listing \ref{lst:range_smeil}. It is crucial that TAPS gather the type and range of observed values for each channel in the SMEIL program, in order to create the proper assertions for FDR4.\\
\begin{listing}
\begin{minted}{smeil_lexer.py:SMEILLexer -x}
proc seconds (in seconds_in)
    bus seconds_out {first_digit: u3 range 0 to 5;
                     second_digit: u4 range 0 to 9;};
    var seconds: u6 range 1 to 59;
    var seconds_first_temp: u3 range 0 to 5;
    var seconds_second_temp: u4 range 0 to 9;
{
    seconds = seconds_in.val % 60;
    seconds_first_temp = seconds / 10;
    seconds_second_temp = seconds % 10;
    seconds_out.first_digit = seconds_first_temp;
    seconds_out.second_digit = seconds_second_temp;
}
\end{minted}
\caption{Example of the simulated \texttt{seconds} process from the SMEIL seven segment display example. See full example in Listing~\ref{lst:smeil} in the appendix.}
\label{lst:range_smeil}
\end{listing}

The main challenge lies in constructing \cspm{} structures that can provide simple assertions of the communicated values while still being applicable to the CSP refinement models.

