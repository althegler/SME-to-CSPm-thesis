% This section should contain information about the problem, but not the solution. I am analysing the problem and explaining what will be problematic when translating from SMEIL to CSPm.

% Der er tre dele i SMEIL jeg har interesse i:
% Three components:
% - Behavioral description - hvad den enkelte funktion gør. det oversættes ret nemt. Jeg bekymre mig ikke om variable og sådan nogle ting. Det er allerede gjort før det gøres til SMEIL, og dem kan jeg genbruge i min code generation.  Funktionel indhold af processer/ opførsel af den enkelte process - det er rimelig nemt oversat direkte til CSPm. Her kan man beskrive hvis der er nogle sproglige udfordringer, fx hvis CSPm ikke har loops eller lign.
% - Strukturel information: hvilke proceser er der og hvilke buser er de limet sammen med. hvordan hænger tingene sammen. Det er også forholdsvis nemt fordi FDR har processer på samme måde som SMEIL har. de har en historisk afhængighed ift. SME og CSP.
% - Opserverede værdier - "known limits". Meta information i SMEIL, og det skal oversættes til faktiske processer med faktiske semantic i FDR der også er en del af topologien. (Det er essensen for mig).

% Vigtigt at have de tre dele adskilt i afsnittet. Brug figuren fra side 107 i bujo til at henvise til at jeg har en Magic 8-ball (Mit system) og jeg skal fra SMEIL til FDR og henvise til at det er det flow jeg har brug for.


% Jeg skal ikke fortælle hvordan jeg har gjort det men analyse af problemet og ikke af løsningen. Jeg skal blot beskrive problemstillingerne.


% I SMEIL har jeg et prædefineret konsistent navnerum. Jeg skal ikke lave en symboltabel. Jeg arver direkte fra SMEIL til CSPm. Så her kan jeg snakke om at hvis jeg gjorde det på en anden måde ville jeg ikke kunne gøre det sådan. Det skal jeg skrive om her.

\section{Transpiling SMEIL to \cspm{}} \label{sec:transpiling}
When transpiling from SMEIL to \cspm{} one of the difficult components was to find a generalized method for transpiling, that could be generalized to most problems. We have worked on separation of concerns in order to simplify, but also have a greater chance of being able to match more SMEIL programs.

An SMEIL process consists of bus and variable declarations, the statements to be run per clock cycle as well as the outgoing communication from the process.  Channels within an SMEIL bus can be translated directly to \cspm{} channels. It is, however, important to give channel names that will be unique since a \cspm{} channel is global as opposed to the local channel within each SMEIL bus. An example of an SMEIL process, where the process structure is evident, can be seen in Listing~\ref{lst:range_smeil} and the corresponding \cspm{} code in Listing~\ref{lst:channel_range_cspm}.

In order to keep the outwards communication and the arithmetic statements together within each process in \cspm{}, we generate \cspm{} processes with a \texttt{let within} statement. The arithmetic statements go into the \texttt{let} section and the communications go into the \texttt{within} section. This gives us the possibility of separating the outwards communication and arithmetic statements while still keeping them within the same \cspm{} process. In Listing~\ref{lst:channel_range_cspm}, an example of the \texttt{let within} statement can be seen in lines 7-14. This structure will work as a general translation structure from SMEIL processes to \cspm{} processes.

The network in an SMEIL program is the crucial part which ties all the processes and communication together. We can standardize the network generation by creating a two-step communication part. Instead of having the actual processes receive the incoming data, they receive the data by their process parameter. The process parameter is then set by the network process which receives the communication from the channels and provides the process with the communicated value.
This ensures that we can generate the processes easily without having to traverse the network in the SMEIL program beforehand to find out which channel provides input for which process. An example of this is shown in Listing~\ref{lst:cspm} in the appendix on lines 61 to 66.
