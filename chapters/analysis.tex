\chapter{Analysis}
\section{Transpiling SMEIL to \cspm{}} \label{sec:transpiling}
When transpiling from SMEIL to \cspm{} one of the difficult components was to find a generalized method for transpiling, that could be generalized to most problems. We have worked on separation of concerns in order to simplify, but also have a greater chance of being able to match more SMEIL programs.

An SMEIL process consists of bus and variable declarations, the statements to be run per clock cycle as well as the outgoing communication from the process.  Channels within an SMEIL bus can be translated directly to \cspm{} channels. It is, however, important to give channel names that will be unique since a \cspm{} channel is global as opposed to the local channel within each SMEIL bus. An example of an SMEIL process, where the process structure is evident, can be seen in Listing~\ref{lst:range_smeil} and the corresponding \cspm{} code in Listing~\ref{lst:channel_range_cspm}.

In order to keep the outwards communication and the arithmetic statements together within each process in \cspm{}, we generate \cspm{} processes with a \texttt{let within} statement. The arithmetic statements go into the \texttt{let} section and the communications go into the \texttt{within} section. This gives us the possibility of separating the outwards communication and arithmetic statements while still keeping them within the same \cspm{} process. In Listing~\ref{lst:channel_range_cspm}, an example of the \texttt{let within} statement can be seen in lines 7-14. This structure will work as a general translation structure from SMEIL processes to \cspm{} processes.

The network in an SMEIL program is the crucial part which ties all the processes and communication together. We can standardize the network generation by creating a two-step communication part. Instead of having the actual processes receive the incoming data, they receive the data by their process parameter. The process parameter is then set by the network process which receives the communication from the channels and provides the process with the communicated value.
This ensures that we can generate the processes easily without having to traverse the network in the SMEIL program beforehand to find out which channel provides input for which process. An example of this is shown in Listing~\ref{lst:cspm} in the appendix on lines 61 to 66.
