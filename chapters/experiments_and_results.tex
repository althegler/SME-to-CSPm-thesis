%!TEX root = ../main.tex

In this chapter, I first present examples of verification for the seven segment display example as well as the addone example. Secondly, an experiment has been conducted to gain further insight into program size and validation time. The experimental setup and results are introduced with the three properties number of visited states, verification time, and maximum resident set size.
\section{Seven Segment Display Example Validation}
The seven segment display example has been presented in different sections throughout this thesis. An illustration of the entire translated unclocked seven segment display network can be seen in Figure \ref{fig:cspm-network}. The SMEIL representation of the same network can be seen in Chapter \ref{chap:analysis}, in Figure \ref{fig:smeil_network}.
The unclocked \cspm{} network consists of 12 different processes, all created so that not only the network is simulated correctly, but also so the monitor processes are placed correctly. The input is represented by a triangle, since it transpiles from an SME process to a \cspm{} channel, it is not represented as a process in this network. Each of the dotted squares represents the network of synchronisations for each \texttt{time} process, which in itself is a process in \cspm{}. For each network, there is the \texttt{time} process and two monitor processes, for example, $H$, $M_{H_1}$ and $M_{H_2}$.
\\

% Errornous example
\begin{listing}
\begin{minted}[escapeinside=||, mathescape=true]{cspm_lexer.py:CSPmLexer -x}
channel clock_out_val : {0..131071}

channel hours_out_first_digit : {0..3}
channel hours_out_second_digit : {0..15}
    |$\vdots$|

Hours(hours_in) =
    let
        hours = hours_in / 3600
        hours_first_temp = hours / 10
        hours_second_temp = hours % 10
    within
        hours_out_first_digit ! hours_first_temp ->
        hours_out_first_digit ! hours_second_temp ->
        SKIP

Hours_out_first_digit_monitor(c) =
    c ? x ->
    (0 <= x and x <= 2) & SKIP
Hours_out_second_digit_monitor(c) =
    c ? x ->
    (0 <= x and x <= 9) & SKIP

\end{minted}
\caption{Example of an erroneous version of the \texttt{Hours} process from the \cspm{} seven segment display example. A corrected an full version of the code can be seen in Listing~\ref{lst:cspm} in Appendix \ref{app:seven_segments}.}
\label{lst:cspm_error}
\end{listing}

In order to show that the verification is accurate, the example in Listing~\ref{lst:cspm_error} contains an error that results in FDR4 failing the verification. In Listing~\ref{lst:cspm_error}, the example is only able to handle an input that is less than 24 hours. This is because the calculation in the \texttt{Hours} process does not perform the wrap-around at the 24\textsuperscript{th} hour. An example of such could be the input \texttt{131071}, which represents 36 hours, 24 minutes and 31 seconds. When trying to assert the code from Listing~\ref{lst:cspm_error} in FDR4, the assertion fails. The counterexample, provided by FDR4, shows that the number 3 is communicated on \texttt{hours\_out\_first\_digit}, which is not allowed according to the monitor process in Listing~\ref{lst:cspm_error}.\\

This example of failure shows how verifying the solution with a tool like FDR4 actually catches errors that the developer might have overlooked. In this case, the error is simply corrected by adding \texttt{\% 24} on the end of line 9 in Listing~\ref{lst:cspm_error} and can be seen corrected in Listing~\ref{lst:cspm} in Appendix \ref{app:seven_segments}, at line 15. When trying to verify the example in FDR4 now, it passes. Using modulo on the result, ensures that I still get the accurate time of day, no matter how many full days the input represents.
The full SMEIL and \cspm{} code for the unclocked seven segment display example can be seen in Listing~\ref{lst:smeil} and in Listing~\ref{lst:cspm} in Appendix \ref{app:seven_segments}.

\begin{figure}[!ht]
  \centering
  \begin{tikzpicture}
    \node [mytriangle] (I) at (0, 0) {$I$};

    %%%%

    \node [mycircle, above right=25ex and 25ex of I] (H) {$H$};

    \node [mysquare, above right=1.5ex and 25ex of H] (H_d1) {$D_{H_1}$};
    \node [mysquare, below right=1.5ex and 25ex of H] (H_d2) {$D_{H_2}$};
    \node [mycircle, above right=3ex and 7.5ex of H] (H_m1) {$M_{H_1}$};
    \node [mycircle, below right=3ex and 7.5ex of H] (H_m2) {$M_{H_2}$};
    \node [draw, red, thick, dotted, fit=(H)(H_m1)(H_m2), inner sep=0.5cm] {};
    \node [right=15ex of H, red] {$N_{hours}$};

    \draw [myarrow, smooth] (I) to[out=0, in=180] (H);

    \draw [myarrow, smooth] (H) to[out=0, in=180] coordinate[midway, black!50, draw, shape=circle, inner sep=0pt, minimum size=5pt](H_mp1) (H_d1);
    \draw (H_m1) -- (H_mp1)  [black!50];
    \draw [myarrow, smooth] (H) to[out=0, in=180] coordinate[midway, black!50, draw, shape=circle, inner sep=0pt, minimum size=5pt](H_mp2) (H_d2);
    \draw (H_m2) -- (H_mp2)  [black!50];

    %%%%

    \node [mycircle, right=23.2ex of I] (M) {$M$};

    \node [mysquare, above right=1.5ex and 25ex of M] (M_d1) {$D_{M_1}$};
    \node [mysquare, below right=1.5ex and 25ex of M] (M_d2) {$D_{M_2}$};
    \node [mycircle, above right=3ex and 7.5ex of M] (M_m1) {$M_{M_1}$};
    \node [mycircle, below right=3ex and 7.5ex of M] (M_m2) {$M_{M_2}$};
    \node [draw, red, thick, dotted, fit=(M)(M_m1)(M_m2), inner sep=0.5cm] {};
    \node [right=15ex of M, red] {$N_{minutes}$};

    \draw [myarrow, smooth] (I) to[out=0, in=180] (M);

    \draw [myarrow, smooth] (M) to[out=0, in=180] coordinate[midway, black!50, draw, shape=circle, inner sep=0pt, minimum size=5pt](M_mp1) (M_d1);
    \draw (M_m1) -- (M_mp1)  [black!50];
    \draw [myarrow, smooth] (M) to[out=0, in=180] coordinate[midway, black!50, draw, shape=circle, inner sep=0pt, minimum size=5pt](M_mp2) (M_d2);
    \draw (M_m2) -- (M_mp2)  [black!50];

    %%%%

    \node [mycircle, below right=24.5ex and 24.5ex of I] (S) {$S$};

    \node [mysquare, above right=1.5ex and 25ex of S] (S_d1) {$D_{S_1}$};
    \node [mysquare, below right=1.5ex and 25ex of S] (S_d2) {$D_{S_2}$};
    \node [mycircle, above right=3ex and 7.5ex of S] (S_m1) {$M_{S_1}$};
    \node [mycircle, below right=3ex and 7.5ex of S] (S_m2) {$M_{S_2}$};
    \node [draw, red, thick, dotted, fit=(S)(S_m1)(S_m2), inner sep=0.50cm, inner ysep=0.5cm] {};
    \node [right=15ex of S, red] {$N_{seconds}$};

    \draw [myarrow, smooth] (I) to[out=0, in=180] (S);

    \draw [myarrow, smooth] (S) to[out=0, in=180] coordinate[midway, black!50, draw, shape=circle, inner sep=0pt, minimum size=5pt](S_mp1) (S_d1);
    \draw (S_m1) -- (S_mp1)  [black!50];
    \draw [myarrow, smooth] (S) to[out=0, in=180] coordinate[midway, black!50, draw, shape=circle, inner sep=0pt, minimum size=5pt](S_mp2) (S_d2);
    \draw (S_m2) -- (S_mp2)  [black!50];
  \end{tikzpicture}
  \caption{A seven segment display clock network in \cspm{}. $I$ represents the input channel. $N_{hours}$, $N_{minutes}$ and $N_{seconds}$ represent the network processes with $H$, $M$ and $S$ as the \texttt{time} processes. The results from the \texttt{time} processes are communicated to the displays $D$. The displays are represented by a square, since they are not actual \cspm{} processes. Each display communication also has a monitor process $M$, which assert the legal communication values.}
  \label{fig:cspm-network}
\end{figure}
\subsection{Clocked Seven Segment Display Example}
The internal state does not change between clock cycles in the seven segment display example, so it can be verified with the unclocked version of TAPS. However, it is interesting to see how the network can be translated into a clocked structure. The full \cspm{} code of the clocked seven segment display example can be seen in Listing \ref{lst:cspm_clocked} in Appendix \ref{app:seven_segments}.
The example is very similar to the unclocked version, however, the processes all synchronise on the \texttt{sync} process as described in Chapter \ref{chap:clock}. Each \texttt{time} process synchronises, then reads a value from the input range, then synchronise again before performing the computation. Then, the value is written to the output channel, that the monitor process reads from, just like in the unclocked seven segment display example. This clocked example is a bit different than the \texttt{addone} example introduced in Chapter \ref{chap:clock}, because there are no buffers in the clocked seven segment display network. This is because there is only one synchronised process communicating, and therefore there is no need for buffers. This means that the network is not as complex as it would have been if buffers were added.
Because the seven segment display network does not persist a state between each clock cycle, there is no need to verify more than one clock cycle. When the \texttt{Clock} process terminates after one clock cycle, the network will behave the same as the unclocked seven segment display network. By looking through the FDR4 traces and the ProBE visualisations, it is clear that the two networks are equivalent in terms of verification and failures.
\section{Addone Example Validation}
The \texttt{addone} example was introduced in Chapter \ref{chap:clock}, and an illustration of the clocked network with its monitor processes can be seen in Figure \ref{fig:addone_clocked_monitor} in the same chapter. As explained, the \texttt{addone} example does not translate well in the initial version of TAPS and therefore the clocked version was created.
The difference between verifying the \texttt{addone} network with a clocked and unclocked network, is that FDR4 is only able to verify one internal state in the unclocked version, whereas it is able to verify multiple internal states in the clocked version. The possibility of verifying different internal states suits this cyclic network perfectly.\\

The \texttt{addone} example differs from the seven segment display example in that it does not require an input range, because the cyclic network is instantiated with initial values instead of a data generator process. The cyclic structure of the \texttt{addone} example causes the values to circulate and increase indefinitely if not restricted. It is not possible to represent an indefinite amount of values on hardware buses and therefore it must be restricted to a specific bit size. If the network is not restricted to specific values, the verification will be based on the values from the simulation of the SMEIL network, even though the network does not fail if the values increase further.  This can cause an unnecessary failure in the verification.
If an SMEIL simulation of an unrestricted \texttt{addone} network results in the internal values reaching 10 and the FDR4 verification verifies more clock cycles than the SMEIL simulation, this would cause the values of the FDR4 verification to exceed the observed values, which would cause FDR4 to fail the verification, even though it would not cause a failure in the SMEIL network. It is, therefore, necessary that the user makes an informed choice as to the number of clock cycles to simulate in SMEIL but also to verify in FDR4. The number of simulated clock cycles and FDR4 verified clock cycles do not have to be equal, but in most cases, it might be the obvious choice. \\

As mentioned, the \texttt{addone} example should be restricted, so a \texttt{\% 11} was added to the value-incrementing expression in the \texttt{add} process to avoid the value becoming larger than 10. Listing \ref{lst:addone_mod_example} shows the simulated \texttt{addone} network with the restriction added in the \texttt{add} process. Besides the restriction, this example is identical to the example in Listing \ref{lst:addone_smeil_example} in Chapter \ref{chap:clock}.
With this enhanced example, the SMEIL simulation provides reasonable observed values which can be used for the verification in FDR4. In Listing \ref{lst:cspm_addone_restricted}, a subset of the translated \texttt{addone} example can be seen which includes the restriction. The full \cspm{} example can be seen in Listing~\ref{lst:cspm_addone_full} in Appendix \ref{app:addone}.\\

After 21 clock cycles, of the example in Listing \ref{lst:addone_mod_example}, the value becomes larger than 10, but the restriction ensures that the verification will succeed even if verified for more than 21 clock cycles. The value will simply wrap around and continue at 0. If the restriction is removed and FDR4 verifies more than 21 clock cycles, the verification fails as expected. \\

This example is somewhat awkward to verify with FDR4, since it does not take advantage of the state space exploration that FDR4 provides. The lack of an input range for the system means that for each clock cycle a new state machine is verified, but only the internal values change. In this example, FDR4 only verifies the relation between internal values for each clock cycle with no external influence. The advantage of the clocked structure is that values, which might cause failures after a certain amount of clock cycles, are now possible to verify, which was not possible with the initial version of TAPS. \\

It is easy to see that this example never fails with the added restriction, but the example clearly introduces the clocked version of TAPS and how it is possible to verify clocked networks in FDR4 and therefore it still provides value to this thesis.

\begin{listing}
\begin{minted}{smeil_lexer.py:SMEILLexer -x}
proc add (in inbus, const constant)
    bus outbus {
        val: u4 = 0 range 0 to 10;
    };
{
    outbus.val = (inbus.val + constant) % 11; //upper limit + 1
}

proc id (in inbus)
    bus outbus {
        val: u4 = 0 range 0 to 10;
    };
    var from_add: u4 range 0 to 10;
{
    from_add = inbus.val;
    outbus.val = from_add;
}

network addone_network ()
{
    instance id of id(add.outbus);
    instance add of add(id.outbus, constant: 1);
}
\end{minted}
\caption{The restricted SMEIL network \texttt{addone\_network} similar to the example in Listing \ref{lst:addone_smeil_example}.}
\label{lst:addone_mod_example}
\end{listing}
\begin{listing}
\begin{minted}{cspm_lexer.py:CSPmLexer -x}
channel sync
channel d_read, c_read : { -1..15} -- u4 and initial value
channel d_write, c_write : { -1..15} -- u4 and initial value

DUM_VAL = -1 -- initial value

Add(i, inbus_channel) =
    (sync ->
     inbus_channel ? x ->
     sync ->
        if (x == DUM_VAL) -- initial value
            then (
                let
                    var = i
                within
                    var <= 15 & -- upper limit
                        c_read ! var -> Add(i, inbus_channel))
            else (
                let
                    var = (x + 1) % 11 -- (observed value + 1) restriction
                within
                    var <= 15 & -- upper limit
                        c_read ! var -> Add(i, inbus_channel))
    )
    [] SKIP


c_read_monitor(c) =
    (c ? x ->
    (0 <= x and x <= 10 or x == -1) & -- observed values and initial value
        c_read_monitor(c)
    ) [] SKIP

\end{minted}
\caption{Sections of the translated \texttt{addone} network. The \texttt{Add} process has a restriction included to ensure no values above 10. The monitor process defines this range along with the acceptance of the dummy value -1. This example has been manually translated due to implementation limitations in the clocked version of TAPS. See the full code in Listing~\ref{lst:cspm_addone_full} in Appendix \ref{app:addone}.}
\label{lst:cspm_addone_restricted}
\end{listing}
\section{Problem Size Experiments}
The examples presented in this thesis provide a suitable introduction to the translation and the verification in FDR4. They consist of straightforward structures that are simple to grasp, so the functionality of TAPS can be proved without confusion.
The challenge with verification via exploration of a state space, is to keep the verification time to a minimum. This does not only apply for FDR4, but for model checking tools in general. FDR4 performs different kinds of internal optimisations on the networks, to minimise the state space before the refinement check. FDR4 also provides several compression algorithms, to provide further compression of larger problems. \\

I have performed some experiments on the seven segment display example, to examine the behaviour in FDR4.
Both experiments have been run on an
AMD Ryzen Threadripper 1950X processor with 16 cores at 3.4 GHz, with 128 GB 2933 MHz DDR4 ram.
The experiments consist of measuring three different properties from the FDR4 verification.
The first property is number of visited states which is a piece of information provided by FDR4.
As previously presented, FDR4 performs compression to minimise the state space.
This property provides an insight into the amount of work FDR4 performs when verifying a network, so it is interesting to learn how the number of visited states corresponds with the verification time. This will also give an insight into the inner workings of FDR4 and how the state space compression behaves.
Because the seven segment display example is divided into three different assertions, one for each \texttt{time} process, FDR4 provides a separate number of visited states for each verified \texttt{time} network. \\

The second property is verification time which is measured by the \texttt{time} command. Even though all experiments have been performed on the same machine, to ensure uniformity, the GNU \texttt{time} command was used instead of the built-in \texttt{time} command that shells such as \texttt{bash} and \texttt{zsh} provide. This property will provide insight into the size of feasible inputs for a \cspm{} network.\\

The last property is maximum resident set size, which is also provided by the GNU \texttt{time} command. The maximum resident set size describes the amount of memory the process holds. It will provide an insight into how much memory FDR4 requires to verify the network, and how the memory usage behaves as the input size increase. If FDR4 requires too much memory, it is not feasible to verify larger problems unless compression algorithms can reduce the state space. \\

The experiment has been designed to keep the internal system fixed and only increase the size of the input range for the system. This means that FDR4 will verify increasingly more values, but the network in itself stays the same.
The lower bound of the input range will be fixed at 0 and the upper bound will be increased with in steps of 500, up to 15000. The input range \{0..15000\} represents 4 hours and 10 seconds. All three property values are gathered after FDR4 finishes the verification.
\subsection{Unclocked Experiment}
The full code for the unclocked seven segment display example can be seen in Listing \ref{lst:cspm} in Appendix \ref{app:seven_segments}.
The unclocked seven segment display example consists of three \texttt{time} processes with associated monitor processes. Each \texttt{assert} function verifies the process monitor network described in Chapter \ref{chap:design}.
\paragraph{Number of visited states}
For each verification, three assertions are performed within the seven segment display example, one for each \texttt{time} process network. With the unclocked example, all three verifications contain the same number of visited states, so this property will not be divided into three different results.\\

Figure \ref{fig:unclocked_states} presents the results of the number of visited states property. From this graph, it is very clear how the state space increases linearly with the input range. This result means that FDR4 is not able to compress the state space further with the increase of input. A reason for FDR4 not providing any additional state space compression could be that because no input is repeated, it is not possible for FDR4 to provide further compression, so the number of visited states remains the same. This does, however, show how a problem quickly can become very large within FDR4, which is something a user must consider when choosing the data to verify.
\begin{figure}
    \centering
    \includegraphics[scale=0.6]{./figures/plots/unclocked_states.png}
\caption{Graph of the number of visited states property from the unclocked seven segment display experiment.}
\label{fig:unclocked_states}
\end{figure}
\paragraph{Verification time}
In Figure \ref{fig:unclocked_verification} the verification time results can be seen. The graph represents the verification time in seconds for each increase in the input range. As can be seen, the verification time increases exponentially with the input values. Since the number of visited states is increasing linearly, it can seem odd that the verification time does not follow that same pattern. However, besides the refinement checking of the Generalised LTS (GLTS), which will increase with the number of visited states, FDR4 must compile the network and generate the GLTS. It is reasonable to expect that the larger the state space, the more effort for FDR4 to complete all the steps of the verification. Therefore these results are consistent with what could be expected.
\begin{figure}
    \centering
    \includegraphics[scale=0.6]{./figures/plots/unclocked_verification_time.png}
\caption{Graph of the verification time property from the unclocked seven segment display experiment.}
\label{fig:unclocked_verification}
\end{figure}
\paragraph{Maximum resident set size}
The result from this property can be seen in Figure \ref{fig:unclocked_resident_size}. These results are not fitted to a simple function as well as the other two experiment properties. It is clear that the amount of memory used for the verification grows with an increase in the input range. It is also somewhat consistent until around 10000 in upper bound limit. This fluctuation could be caused by some internal structure in FDR4 or it could be a result of other processes running on the machine that is running the verification. Unfortunately, FDR4 does not provide a lot of information about the internal workings, so it can be very difficult to examine these results further. However, the results are overall consistent with the results from both number of visited states and verification time.
\begin{figure}
    \centering
    \includegraphics[scale=0.6]{./figures/plots/unclocked_size.png}
\caption{Graph of the maximum resident set size property from the unclocked seven segment display experiment.}
\label{fig:unclocked_resident_size}
\end{figure}

\subsection{Clocked Experiment}
The full code for the clocked seven segment display example can be seen in Listing \ref{lst:cspm_clocked} in Appendix \ref{app:seven_segments}.
As in the unclocked seven segment display example, the clocked example also consists of three \texttt{time} processes with associated monitor processes.
In order to make the clocked seven segment display experiment equivalent with the unclocked seven segment display example, the \texttt{Clock} process in the clocked network only verifies one clock cycle and therefore the two networks should be equivalent in the verification.
\paragraph{Number of visited states}
In Figure \ref{fig:clocked_states}, the results of the clocked number of visited states property can be seen for each \texttt{time} verification. As can be seen, they differ quite a lot from each other.
The results of the experiment, show an increase in the number of visited states for \texttt{Hours} in input values between 3500 and 4000. Further investigation show that the number of visited states increase precisely every 3600 increase.
This corresponds to the number of seconds in an hour, meaning that the number of visited states increases linearly with the number of hours.
The number of visited states for \texttt{Minutes} increase linearly until between 3500 and 4000. Also in this case, further investigation show that the number of visited states increase until exactly 3539 where it levels out and stays at 134 for the rest of the verifications.
It is very clear from this result that the number of visited states reaches its maximum when the input range represents the maximum amount of minutes. The input range \{0..3539\} represents 59 minutes which is maximum. The number of visited states for \texttt{Seconds} is constant at 134, however if the input range was decreased to \{0..59\} the same result could be seen as with the \texttt{Minutes} graph. These results make it clear that FDR4 is able to decrease the number of states to the number of \texttt{Hours}, \texttt{Minutes} and \texttt{Seconds} represented by the input.
\begin{figure}
    \centering
    \includegraphics[scale=0.6]{./figures/plots/clocked_states.png}
\caption{Graphs of the three number of visited states properties from the clocked seven segment display experiment.}
\label{fig:clocked_states}
\end{figure}
\paragraph{Verification time}
In Figure \ref{fig:clocked_verification}, the clocked verification time
results can be seen.
This graph is very similar to the verification time results for the unclocked experiment, but with a slightly different increase. However, the values also increase exponentially with the input which, as explained above, is to be expected. What is important to notice is that even though the graph is similar to the unclocked verification time graph, the number of seconds is very different than in the first experiment. The reason for this is probably the decrease in number of visited states which will be discussed further below.
\begin{figure}
    \centering
    \includegraphics[scale=0.6]{./figures/plots/clocked_verification_time.png}
\caption{Graph of the verification time property from the clocked seven segment display experiment.}
\label{fig:clocked_verification}
\end{figure}
\paragraph{Maximum resident set size}
Figure \ref{fig:clocked_resident_size} shows the clocked maximum resident set size. This graph shows the same situation as the verification time graph. The graph is similar to that of the unclocked maximum resident set size with a slightly more flattened increase and not nearly as much fluctuation. The big difference with this graph is, as with the verification time property, the low values compared to the unclocked maximum resident set size values. This is also discussed further below.
\begin{figure}
    \centering
    \includegraphics[scale=0.6]{./figures/plots/clocked_size.png}
\caption{Graph of the maximum resident set size property from the clocked seven segment display experiment.}
\label{fig:clocked_resident_size}
\end{figure}
\subsection{Results}
As explained above, the number of visited states property from the unclocked seven segment display experiment showed to be equal for all three \texttt{time} verifications whereas, for the clocked experiment, the number of visited states varied between the three verifications. Figure \ref{fig:combined_states} represents all four graphs, but the unclocked values are several order of magnitude larger than the clocked values, making the clocked values hard to discern in the graph.\\

In Figure \ref{fig:combined_verification} the combined verification time graph can be seen. Here, the difference also shows very clearly. The clocked experiment verifies much faster than the unclocked. Again it is difficult to see the actual values of the clocked version in the graph because they are significantly lower. The same case can be seen in Figure \ref{fig:combined_resident_size} which represents the combined maximum resident set size. \\

Both verification time and resident set size seem to be affected by the number of visited states property. This is evident because the
number of visited states is the only internal FDR4 property of the experiment. Unfortunately, it can be quite difficult to investigate why the clocked experiment performs so much better than the unclocked. When looking at the trace of both networks within FDR4, both verify the full input range, as expected. The visualiser ProBE also shows equivalent traces and the two networks behave identically on failures. The only difference between the two networks is that in the clocked system all processes are recursive and synchronise together before continuing. The unclocked processes will always terminate after one clock cycle. However, since the clocked network only simulates one clock cycle, the two networks should still be equivalent. These differences do not affect the actual values verified, or the result of the verification. \\

As previously stated when introducing the clocked version of TAPS in Chapter \ref{chap:clock}, the increase in complexity might increase the verification time in FDR4, but in this case, it is the opposite.
As the two networks seem equivalent in both input range and verification, the reason for the difference in performance might lie in what the number of visited states property shows. It might be that FDR4 is able to compress the state space in the clocked experiment while not being able to perform the same compression in the unclocked experiment. \\

The FDR4 tool provides a machine structure viewer which can provide information about how FDR4 represents the processes and how effective the compression is. When looking at the two different results from FDR4, it becomes clear why the clocked experiment performs so much better than the unclocked experiment. In Figure \ref{fig:unclocked_compression} and Figure \ref{fig:clocked_compression}, the output of the machine structure viewer can be seen. In the unclocked result in Figure \ref{fig:unclocked_compression}, the unclocked network
\texttt{N\_hours} is represented by a high-level machine with 12 formats, 77 rules, and 34 leaf machines. A high-level machine means a process with subprocesses and the only information needed about formats and rules are that the more there are, the more complex the machine. \\

Because all events are hidden in the network, as explained in Chapter \ref{chap:design}, the subprocesses are \texttt{Unknown}. The second to last line shows that the compression algorithm \texttt{sbisim} has been applied to \texttt{Hours(0)} but it also shows that it was not able to reduce the number of states or transitions. \\

Figure \ref{fig:clocked_compression} represents the clocked network and, as can be seen, the \texttt{N\_hours} process includes compression information. It provides an estimate that the reduced machine is around 17\% the size of the original machine. On the second to last line, it can be seen that the \texttt{sbisim} compression algorithm is not applied to \texttt{Hours(0)} as in the unclocked experiment, but to \texttt{Hours(..)}, which indicates that FDR4 represents the two networks differently. The \texttt{sbisim} compression algorithm is able to compress the state space from 36 states to 6 states. It clearly shows that because FDR4 is able to perform compression on the clocked system, the performance also increases dramatically.\\

It is quite surprising that the clocked experiment has such a drastic difference and that it actually performs better than the unclocked version. As the network increases in complexity, the state space should also increase, and it is because of the compression that the clocked version performs so well.
In this aspect, FDR4 is a black box, and it is therefore not possible to learn exactly why FDR4 is able to perform compression on the clocked network and not on the unclocked. It is, of course, interesting to experiment further to find patterns in the \cspm{} structures that suits the compression algorithm best.
\begin{figure}
    \includegraphics[width=0.98\textwidth]{./figures/unclocked_compression.jpg}
\caption{Screen dump of the results of the unclocked seven segment display network in the FDR4 machine structure viewer.}
\label{fig:unclocked_compression}
\end{figure}
\begin{figure}
    \includegraphics[width=1.02\textwidth]{./figures/clocked_compression.jpg}
\caption{Screen dump of the results of the clocked seven segment display network in the FDR4 machine structure viewer.}
\label{fig:clocked_compression}
\end{figure}

\begin{figure}
    \centering
    \includegraphics[scale=0.6]{./figures/plots/combined_states.png}
\caption{Graphs of the number of visited states properties combined from both the unclocked and clocked seven segment display experiment.}
\label{fig:combined_states}
\end{figure}

\begin{figure}
    \centering
    \includegraphics[scale=0.6]{./figures/plots/combined_verification_time.png}
\caption{Graphs of the verification time properties combined from both the unclocked and clocked seven segment display experiment.}
\label{fig:combined_verification}
\end{figure}

\begin{figure}
    \centering
    \includegraphics[scale=0.6]{./figures/plots/combined_size.png}
\caption{Graphs of the maximum resident size set properties combined from both the unclocked and clocked seven segment display experiment.}
\label{fig:combined_resident_size}
\end{figure}