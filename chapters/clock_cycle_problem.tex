When trying to model a clocked hardware system in \cspm{}, we want to create a synchronous network where the system has a controlled termination, and where it can be verified that all processes end, by behaving like the \texttt{SKIP} process.

In this example, we wish to create a network called 'Addone'. The network consists of two processes where one process, the 'Add' process, increments a variable by one and sents it along to the other process, the 'Id' process, which then sents the value back to the 'Add' process.
% TODO: Add a picture of the network, like Truls example.
The network is a two process loop and it is therefore essential that there is a way to initialise the loop as well as terminating it properly.\\

We wish to model a \cspm{} network which reflects the SME model and therefore we have to adhere to the SME model structure. There is three different states for each clock cycle; the read state, the calculation statem, and the write state. A process must read before writing in a clock cycle.\\
For the 'Add' and 'Id' processes to comply with these states, they would have to read first, then the calculate phase, in which the 'Id' process does nothing, and then they would write the result onto a channel. A problem occurs, since they both have to read first, no one can read because no processes have written anything yet. To solve this, we implement two buffers which for each clock cycle reads the output that the process writes and then writes the value to a channel. Thus the buffer structure is the reverse of a 'normal' process since it will write and then read in a clock cycle. \\
If we give the buffers an initial value, they can begin the clock cycle writing the value which the 'Add' and 'Id' processes can read and thereby they will comply with the SME model structure.
The buffers will be instantiated will a 'dummy' value which is also how it is typically done in hardware. The dummy value is simply to indicate that the system should ignore the first clock cycle and then continue with the systems actual values. \\
Each process is also instantiated with a value, which is then used instead of the dummy value from the buffer process. After this initial cycle, the process loop will continue and the communication will hold according to the description of the network explained above.\\\\


We have to limit the number of verified clock cycles, otherwise, FDR4 would never stop. Even though the standard SME process also never stops, when simulating the system in SMEIL, it is only simulating a finite number of clock cycles. Therefore we need to implement a way to have all processes of the \cspm{} network terminate in a timely manner. This is done with the \texttt{Clock} process. The clock process is initialised with a value and syncronise on the \texttt{clock} channel, which all other processes does as well. When the specified number of clock cycles has passed, the \texttt{Clock} process stops clocking and instead behaves as \texttt{SKIP}. This means that all other processes won't be able to syncronise on the \texttt{clock} channel anymore and therefore they will then instead behave as \texttt{SKIP} and that way the system terminates as planned.\\
Since the processes all must read, calculate and then write, the buffer processes, as mentioned before, must behave the opposite way. This means that the last write the processes make before they \texttt{SKIP} will be left in the buffers since there are no processes to read the value from the channels. This means that the buffers must be able to either write a value or \texttt{SKIP}.\\
The \texttt{clock} channel is used as a two-way clock synchronisation, where the same channel is used for syncronising up as well as down. Thus all processes syncronise before they read and before they write. The \texttt{Clock} process then needs to syncronise twice before incrementing its counter. \\

When trying to verify that this system terminates as expected in FDR4, we came across an error while FDR4 was compiling the program.
FDR4 is complaining that a value, the system is trying to send, is not a part of the set of values defined for the channel.
The channels, used for communicating the value between the processes and the buffers, are defined for a specific range and since the 'Add' process has to read, calculate and write before it can terminate it will always write a value out that have been incremented with one.\\
For instance, if the channels were defined with the range \{0..5\} and the \texttt{Clock} process would stop after 10 clock cycles. This would mean that on the last clock cycle, before terminating, the 'Add' process would write a 6 onto the channel, which of course is not possible since the channel is defined only for the range \{0..5\}, so FDR4 complains about this.\\
However, what we experienced was that when the ranges of the channels were set to a larger number than the system would ever reach, within the defined number of clock cycles, FDR4 would still fail with the same reason.
This caused some problems since we were not able to verify the system and we were not interested in FDR4 trying to verify a communication that would never occur within the network.
Since we are syncronising the processes on the events in the channels, it seems odd that FDR4 still considers events which should not be possible to reach.  \\\\
After some time working with the problem and trying to understand the reason for FDR4s error message, we found that if we simply add a guard or an if-then-else statement that tests the value to be written, FDR4 will gladly verify the system and when using Probe on the network, it is clear that FDR4 does not consider the trace with the wrong values. \\

So the suggested solution, or fix, to this problem, is to add an if-then-else before all writes in a program. The statement then tests the value to be written against the max value of the range of the channel and if the value is not within the range, then the process behaves as the \texttt{SKIP} process, otherwise, it continues with write.
\begin{minted}[linenos=false, escapeinside=||, mathescape=true]{cspm_lexer.py:CSPmLexer -x}
channel c : {0..20}

|$\vdots$|

    if (i+1) > 20  -- Check the upper limit of the channel
        then SKIP  -- SKIP if the value is above
        else (c_r ! (i+1) -> Add(i))) -- Otherwise write and continue
\end{minted}
In this case, it is only necessary to add an upper limit test, since the network only increments, but as a general rule, it would be necessary to test for both upper and lower bounds.


% The problem ocurred several times with different versions of the solution. Also, Ohm had the same problem with the Commstime problem. It makes sense why FDR4 wants to check, and maybe it is a way to save verification time: if it checks things in parallel or something.
% However it is not a good solution that we have to put in a upper/lower bound check that is actually never relevant.
% Of course it might be good, in any case, since the programmer then do not need to worry. On the other hand it might make the program look like it is terminating properly when it is actually failing because it tries to write a value that is not allowed. This case might happen, and if all other processes do not notice it and then also SKIP according to their specifications, then the verification passes even though it might be wrong.
% A solution might be to ensure that all processes synchronise before skipping, because then the problem (i think) would not occur, since the failing process would simply skip, then the other processes cannot skip because they have not syncrhonised yet. however, I am not sure this is possible since it might be that they either syncronise or they skip.

% Kenneths version, which I believe is how the SME model works, is having a process or bus in the middle og all steps. By using a dependency graph (Explain more?) it is possible to see which processes communicate to witch processes and, more importantly, in which order. For each communication step (or maybe for each communication) a process/bus will receive all writes. In SME a process can write several times to the same channel but only the last one before the clock signal will be written, the others are just overwritten. Since we have the dependency graph, we also know which processes we need communication from, and when the process have written all it has to write, then it sends a ready signal to the "bus" process, which then waits for all the ready signals (because it knows how many it should get. And if it is one process/bus pr. communication then it only needs one of course.). When all ready signals are in, the bus-process change behaviour and it is not writing instead of reading. It writes all possible values out and the processes that are supposed to receive the values (which we know from the dependency graph) will receive the values. And the processes then need to send a ready signal back to the bus process to let it know that it have read all it needed. When all ready signals are received, the bus process when change behaviour again and can read values once again.
% All these steps are intermediate steps within one clock cycle. So at the "end" of the dependency graph, the step looks similar to the others, but it is registered as the clock and the next clock cycle begins. In principal, all these steps could be the clock, since the step is the same, but a step is simply chosen to be the clock, based on the dependency graph.
% By treating the communication like this within a clock cycle, the values can propagate through the network and the internal state of the processes are also kept. The original TAPS version could only verify all input for a system, but if the system was internally affected by values from a previous clock cycle, then the system could not verify it. It is not a problem in the seven segment example, since no values are dependent on previous values. But the Addone network do depend on what happened in the last clock cycle.
% With this solution it is possible to verify a specific number of clock cycles.
%
% It is important to note that SME processes usually do not end, but continue running forever. However, when we simulate with SME, the simulation end after some specific number of clock cycles, and this is what we wish to verify.
% It might be interesting to use the "running forever" thing to verify stuff on it. But at the moment I cant see what we would verify. Deadlocks would never happen in SME, so it would be possible to use it to check that the translation have been successful, but this is not so relevant for other people.
% It might be that the stuff we wish to verify is not possible in a limited clock cycle range and that it is necessary to run it "forever". I am not sure.



% (From design)
% \section{Clock cycle problem}
% % When trying to model a clocked hardware system in \cspm{}, we want to create a synchronous network where the system has a controlled termination, and where it can be verified that all processes end, by behaving like the \texttt{SKIP} process.

In this example, we wish to create a network called 'Addone'. The network consists of two processes where one process, the 'Add' process, increments a variable by one and sents it along to the other process, the 'Id' process, which then sents the value back to the 'Add' process.
% TODO: Add a picture of the network, like Truls example.
The network is a two process loop and it is therefore essential that there is a way to initialise the loop as well as terminating it properly.\\

We wish to model a \cspm{} network which reflects the SME model and therefore we have to adhere to the SME model structure. There is three different states for each clock cycle; the read state, the calculation statem, and the write state. A process must read before writing in a clock cycle.\\
For the 'Add' and 'Id' processes to comply with these states, they would have to read first, then the calculate phase, in which the 'Id' process does nothing, and then they would write the result onto a channel. A problem occurs, since they both have to read first, no one can read because no processes have written anything yet. To solve this, we implement two buffers which for each clock cycle reads the output that the process writes and then writes the value to a channel. Thus the buffer structure is the reverse of a 'normal' process since it will write and then read in a clock cycle. \\
If we give the buffers an initial value, they can begin the clock cycle writing the value which the 'Add' and 'Id' processes can read and thereby they will comply with the SME model structure.
The buffers will be instantiated will a 'dummy' value which is also how it is typically done in hardware. The dummy value is simply to indicate that the system should ignore the first clock cycle and then continue with the systems actual values. \\
Each process is also instantiated with a value, which is then used instead of the dummy value from the buffer process. After this initial cycle, the process loop will continue and the communication will hold according to the description of the network explained above.\\\\


We have to limit the number of verified clock cycles, otherwise, FDR4 would never stop. Even though the standard SME process also never stops, when simulating the system in SMEIL, it is only simulating a finite number of clock cycles. Therefore we need to implement a way to have all processes of the \cspm{} network terminate in a timely manner. This is done with the \texttt{Clock} process. The clock process is initialised with a value and syncronise on the \texttt{clock} channel, which all other processes does as well. When the specified number of clock cycles has passed, the \texttt{Clock} process stops clocking and instead behaves as \texttt{SKIP}. This means that all other processes won't be able to syncronise on the \texttt{clock} channel anymore and therefore they will then instead behave as \texttt{SKIP} and that way the system terminates as planned.\\
Since the processes all must read, calculate and then write, the buffer processes, as mentioned before, must behave the opposite way. This means that the last write the processes make before they \texttt{SKIP} will be left in the buffers since there are no processes to read the value from the channels. This means that the buffers must be able to either write a value or \texttt{SKIP}.\\
The \texttt{clock} channel is used as a two-way clock synchronisation, where the same channel is used for syncronising up as well as down. Thus all processes syncronise before they read and before they write. The \texttt{Clock} process then needs to syncronise twice before incrementing its counter. \\

When trying to verify that this system terminates as expected in FDR4, we came across an error while FDR4 was compiling the program.
FDR4 is complaining that a value, the system is trying to send, is not a part of the set of values defined for the channel.
The channels, used for communicating the value between the processes and the buffers, are defined for a specific range and since the 'Add' process has to read, calculate and write before it can terminate it will always write a value out that have been incremented with one.\\
For instance, if the channels were defined with the range \{0..5\} and the \texttt{Clock} process would stop after 10 clock cycles. This would mean that on the last clock cycle, before terminating, the 'Add' process would write a 6 onto the channel, which of course is not possible since the channel is defined only for the range \{0..5\}, so FDR4 complains about this.\\
However, what we experienced was that when the ranges of the channels were set to a larger number than the system would ever reach, within the defined number of clock cycles, FDR4 would still fail with the same reason.
This caused some problems since we were not able to verify the system and we were not interested in FDR4 trying to verify a communication that would never occur within the network.
Since we are syncronising the processes on the events in the channels, it seems odd that FDR4 still considers events which should not be possible to reach.  \\\\
After some time working with the problem and trying to understand the reason for FDR4s error message, we found that if we simply add a guard or an if-then-else statement that tests the value to be written, FDR4 will gladly verify the system and when using Probe on the network, it is clear that FDR4 does not consider the trace with the wrong values. \\

So the suggested solution, or fix, to this problem, is to add an if-then-else before all writes in a program. The statement then tests the value to be written against the max value of the range of the channel and if the value is not within the range, then the process behaves as the \texttt{SKIP} process, otherwise, it continues with write.
\begin{minted}[linenos=false, escapeinside=||, mathescape=true]{cspm_lexer.py:CSPmLexer -x}
channel c : {0..20}

|$\vdots$|

    if (i+1) > 20  -- Check the upper limit of the channel
        then SKIP  -- SKIP if the value is above
        else (c_r ! (i+1) -> Add(i))) -- Otherwise write and continue
\end{minted}
In this case, it is only necessary to add an upper limit test, since the network only increments, but as a general rule, it would be necessary to test for both upper and lower bounds.


% The problem ocurred several times with different versions of the solution. Also, Ohm had the same problem with the Commstime problem. It makes sense why FDR4 wants to check, and maybe it is a way to save verification time: if it checks things in parallel or something.
% However it is not a good solution that we have to put in a upper/lower bound check that is actually never relevant.
% Of course it might be good, in any case, since the programmer then do not need to worry. On the other hand it might make the program look like it is terminating properly when it is actually failing because it tries to write a value that is not allowed. This case might happen, and if all other processes do not notice it and then also SKIP according to their specifications, then the verification passes even though it might be wrong.
% A solution might be to ensure that all processes synchronise before skipping, because then the problem (i think) would not occur, since the failing process would simply skip, then the other processes cannot skip because they have not syncrhonised yet. however, I am not sure this is possible since it might be that they either syncronise or they skip.

% Kenneths version, which I believe is how the SME model works, is having a process or bus in the middle og all steps. By using a dependency graph (Explain more?) it is possible to see which processes communicate to witch processes and, more importantly, in which order. For each communication step (or maybe for each communication) a process/bus will receive all writes. In SME a process can write several times to the same channel but only the last one before the clock signal will be written, the others are just overwritten. Since we have the dependency graph, we also know which processes we need communication from, and when the process have written all it has to write, then it sends a ready signal to the "bus" process, which then waits for all the ready signals (because it knows how many it should get. And if it is one process/bus pr. communication then it only needs one of course.). When all ready signals are in, the bus-process change behaviour and it is not writing instead of reading. It writes all possible values out and the processes that are supposed to receive the values (which we know from the dependency graph) will receive the values. And the processes then need to send a ready signal back to the bus process to let it know that it have read all it needed. When all ready signals are received, the bus process when change behaviour again and can read values once again.
% All these steps are intermediate steps within one clock cycle. So at the "end" of the dependency graph, the step looks similar to the others, but it is registered as the clock and the next clock cycle begins. In principal, all these steps could be the clock, since the step is the same, but a step is simply chosen to be the clock, based on the dependency graph.
% By treating the communication like this within a clock cycle, the values can propagate through the network and the internal state of the processes are also kept. The original TAPS version could only verify all input for a system, but if the system was internally affected by values from a previous clock cycle, then the system could not verify it. It is not a problem in the seven segment example, since no values are dependent on previous values. But the Addone network do depend on what happened in the last clock cycle.
% With this solution it is possible to verify a specific number of clock cycles.
%
% It is important to note that SME processes usually do not end, but continue running forever. However, when we simulate with SME, the simulation end after some specific number of clock cycles, and this is what we wish to verify.
% It might be interesting to use the "running forever" thing to verify stuff on it. But at the moment I cant see what we would verify. Deadlocks would never happen in SME, so it would be possible to use it to check that the translation have been successful, but this is not so relevant for other people.
% It might be that the stuff we wish to verify is not possible in a limited clock cycle range and that it is necessary to run it "forever". I am not sure.



% (From design)
% \section{Clock cycle problem}
% % When trying to model a clocked hardware system in \cspm{}, we want to create a synchronous network where the system has a controlled termination, and where it can be verified that all processes end, by behaving like the \texttt{SKIP} process.

In this example, we wish to create a network called 'Addone'. The network consists of two processes where one process, the 'Add' process, increments a variable by one and sents it along to the other process, the 'Id' process, which then sents the value back to the 'Add' process.
% TODO: Add a picture of the network, like Truls example.
The network is a two process loop and it is therefore essential that there is a way to initialise the loop as well as terminating it properly.\\

We wish to model a \cspm{} network which reflects the SME model and therefore we have to adhere to the SME model structure. There is three different states for each clock cycle; the read state, the calculation statem, and the write state. A process must read before writing in a clock cycle.\\
For the 'Add' and 'Id' processes to comply with these states, they would have to read first, then the calculate phase, in which the 'Id' process does nothing, and then they would write the result onto a channel. A problem occurs, since they both have to read first, no one can read because no processes have written anything yet. To solve this, we implement two buffers which for each clock cycle reads the output that the process writes and then writes the value to a channel. Thus the buffer structure is the reverse of a 'normal' process since it will write and then read in a clock cycle. \\
If we give the buffers an initial value, they can begin the clock cycle writing the value which the 'Add' and 'Id' processes can read and thereby they will comply with the SME model structure.
The buffers will be instantiated will a 'dummy' value which is also how it is typically done in hardware. The dummy value is simply to indicate that the system should ignore the first clock cycle and then continue with the systems actual values. \\
Each process is also instantiated with a value, which is then used instead of the dummy value from the buffer process. After this initial cycle, the process loop will continue and the communication will hold according to the description of the network explained above.\\\\


We have to limit the number of verified clock cycles, otherwise, FDR4 would never stop. Even though the standard SME process also never stops, when simulating the system in SMEIL, it is only simulating a finite number of clock cycles. Therefore we need to implement a way to have all processes of the \cspm{} network terminate in a timely manner. This is done with the \texttt{Clock} process. The clock process is initialised with a value and syncronise on the \texttt{clock} channel, which all other processes does as well. When the specified number of clock cycles has passed, the \texttt{Clock} process stops clocking and instead behaves as \texttt{SKIP}. This means that all other processes won't be able to syncronise on the \texttt{clock} channel anymore and therefore they will then instead behave as \texttt{SKIP} and that way the system terminates as planned.\\
Since the processes all must read, calculate and then write, the buffer processes, as mentioned before, must behave the opposite way. This means that the last write the processes make before they \texttt{SKIP} will be left in the buffers since there are no processes to read the value from the channels. This means that the buffers must be able to either write a value or \texttt{SKIP}.\\
The \texttt{clock} channel is used as a two-way clock synchronisation, where the same channel is used for syncronising up as well as down. Thus all processes syncronise before they read and before they write. The \texttt{Clock} process then needs to syncronise twice before incrementing its counter. \\

When trying to verify that this system terminates as expected in FDR4, we came across an error while FDR4 was compiling the program.
FDR4 is complaining that a value, the system is trying to send, is not a part of the set of values defined for the channel.
The channels, used for communicating the value between the processes and the buffers, are defined for a specific range and since the 'Add' process has to read, calculate and write before it can terminate it will always write a value out that have been incremented with one.\\
For instance, if the channels were defined with the range \{0..5\} and the \texttt{Clock} process would stop after 10 clock cycles. This would mean that on the last clock cycle, before terminating, the 'Add' process would write a 6 onto the channel, which of course is not possible since the channel is defined only for the range \{0..5\}, so FDR4 complains about this.\\
However, what we experienced was that when the ranges of the channels were set to a larger number than the system would ever reach, within the defined number of clock cycles, FDR4 would still fail with the same reason.
This caused some problems since we were not able to verify the system and we were not interested in FDR4 trying to verify a communication that would never occur within the network.
Since we are syncronising the processes on the events in the channels, it seems odd that FDR4 still considers events which should not be possible to reach.  \\\\
After some time working with the problem and trying to understand the reason for FDR4s error message, we found that if we simply add a guard or an if-then-else statement that tests the value to be written, FDR4 will gladly verify the system and when using Probe on the network, it is clear that FDR4 does not consider the trace with the wrong values. \\

So the suggested solution, or fix, to this problem, is to add an if-then-else before all writes in a program. The statement then tests the value to be written against the max value of the range of the channel and if the value is not within the range, then the process behaves as the \texttt{SKIP} process, otherwise, it continues with write.
\begin{minted}[linenos=false, escapeinside=||, mathescape=true]{cspm_lexer.py:CSPmLexer -x}
channel c : {0..20}

|$\vdots$|

    if (i+1) > 20  -- Check the upper limit of the channel
        then SKIP  -- SKIP if the value is above
        else (c_r ! (i+1) -> Add(i))) -- Otherwise write and continue
\end{minted}
In this case, it is only necessary to add an upper limit test, since the network only increments, but as a general rule, it would be necessary to test for both upper and lower bounds.


% The problem ocurred several times with different versions of the solution. Also, Ohm had the same problem with the Commstime problem. It makes sense why FDR4 wants to check, and maybe it is a way to save verification time: if it checks things in parallel or something.
% However it is not a good solution that we have to put in a upper/lower bound check that is actually never relevant.
% Of course it might be good, in any case, since the programmer then do not need to worry. On the other hand it might make the program look like it is terminating properly when it is actually failing because it tries to write a value that is not allowed. This case might happen, and if all other processes do not notice it and then also SKIP according to their specifications, then the verification passes even though it might be wrong.
% A solution might be to ensure that all processes synchronise before skipping, because then the problem (i think) would not occur, since the failing process would simply skip, then the other processes cannot skip because they have not syncrhonised yet. however, I am not sure this is possible since it might be that they either syncronise or they skip.

% Kenneths version, which I believe is how the SME model works, is having a process or bus in the middle og all steps. By using a dependency graph (Explain more?) it is possible to see which processes communicate to witch processes and, more importantly, in which order. For each communication step (or maybe for each communication) a process/bus will receive all writes. In SME a process can write several times to the same channel but only the last one before the clock signal will be written, the others are just overwritten. Since we have the dependency graph, we also know which processes we need communication from, and when the process have written all it has to write, then it sends a ready signal to the "bus" process, which then waits for all the ready signals (because it knows how many it should get. And if it is one process/bus pr. communication then it only needs one of course.). When all ready signals are in, the bus-process change behaviour and it is not writing instead of reading. It writes all possible values out and the processes that are supposed to receive the values (which we know from the dependency graph) will receive the values. And the processes then need to send a ready signal back to the bus process to let it know that it have read all it needed. When all ready signals are received, the bus process when change behaviour again and can read values once again.
% All these steps are intermediate steps within one clock cycle. So at the "end" of the dependency graph, the step looks similar to the others, but it is registered as the clock and the next clock cycle begins. In principal, all these steps could be the clock, since the step is the same, but a step is simply chosen to be the clock, based on the dependency graph.
% By treating the communication like this within a clock cycle, the values can propagate through the network and the internal state of the processes are also kept. The original TAPS version could only verify all input for a system, but if the system was internally affected by values from a previous clock cycle, then the system could not verify it. It is not a problem in the seven segment example, since no values are dependent on previous values. But the Addone network do depend on what happened in the last clock cycle.
% With this solution it is possible to verify a specific number of clock cycles.
%
% It is important to note that SME processes usually do not end, but continue running forever. However, when we simulate with SME, the simulation end after some specific number of clock cycles, and this is what we wish to verify.
% It might be interesting to use the "running forever" thing to verify stuff on it. But at the moment I cant see what we would verify. Deadlocks would never happen in SME, so it would be possible to use it to check that the translation have been successful, but this is not so relevant for other people.
% It might be that the stuff we wish to verify is not possible in a limited clock cycle range and that it is necessary to run it "forever". I am not sure.



% (From design)
% \section{Clock cycle problem}
% % When trying to model a clocked hardware system in \cspm{}, we want to create a synchronous network where the system has a controlled termination, and where it can be verified that all processes end, by behaving like the \texttt{SKIP} process.

In this example, we wish to create a network called 'Addone'. The network consists of two processes where one process, the 'Add' process, increments a variable by one and sents it along to the other process, the 'Id' process, which then sents the value back to the 'Add' process.
% TODO: Add a picture of the network, like Truls example.
The network is a two process loop and it is therefore essential that there is a way to initialise the loop as well as terminating it properly.\\

We wish to model a \cspm{} network which reflects the SME model and therefore we have to adhere to the SME model structure. There is three different states for each clock cycle; the read state, the calculation statem, and the write state. A process must read before writing in a clock cycle.\\
For the 'Add' and 'Id' processes to comply with these states, they would have to read first, then the calculate phase, in which the 'Id' process does nothing, and then they would write the result onto a channel. A problem occurs, since they both have to read first, no one can read because no processes have written anything yet. To solve this, we implement two buffers which for each clock cycle reads the output that the process writes and then writes the value to a channel. Thus the buffer structure is the reverse of a 'normal' process since it will write and then read in a clock cycle. \\
If we give the buffers an initial value, they can begin the clock cycle writing the value which the 'Add' and 'Id' processes can read and thereby they will comply with the SME model structure.
The buffers will be instantiated will a 'dummy' value which is also how it is typically done in hardware. The dummy value is simply to indicate that the system should ignore the first clock cycle and then continue with the systems actual values. \\
Each process is also instantiated with a value, which is then used instead of the dummy value from the buffer process. After this initial cycle, the process loop will continue and the communication will hold according to the description of the network explained above.\\\\


We have to limit the number of verified clock cycles, otherwise, FDR4 would never stop. Even though the standard SME process also never stops, when simulating the system in SMEIL, it is only simulating a finite number of clock cycles. Therefore we need to implement a way to have all processes of the \cspm{} network terminate in a timely manner. This is done with the \texttt{Clock} process. The clock process is initialised with a value and syncronise on the \texttt{clock} channel, which all other processes does as well. When the specified number of clock cycles has passed, the \texttt{Clock} process stops clocking and instead behaves as \texttt{SKIP}. This means that all other processes won't be able to syncronise on the \texttt{clock} channel anymore and therefore they will then instead behave as \texttt{SKIP} and that way the system terminates as planned.\\
Since the processes all must read, calculate and then write, the buffer processes, as mentioned before, must behave the opposite way. This means that the last write the processes make before they \texttt{SKIP} will be left in the buffers since there are no processes to read the value from the channels. This means that the buffers must be able to either write a value or \texttt{SKIP}.\\
The \texttt{clock} channel is used as a two-way clock synchronisation, where the same channel is used for syncronising up as well as down. Thus all processes syncronise before they read and before they write. The \texttt{Clock} process then needs to syncronise twice before incrementing its counter. \\

When trying to verify that this system terminates as expected in FDR4, we came across an error while FDR4 was compiling the program.
FDR4 is complaining that a value, the system is trying to send, is not a part of the set of values defined for the channel.
The channels, used for communicating the value between the processes and the buffers, are defined for a specific range and since the 'Add' process has to read, calculate and write before it can terminate it will always write a value out that have been incremented with one.\\
For instance, if the channels were defined with the range \{0..5\} and the \texttt{Clock} process would stop after 10 clock cycles. This would mean that on the last clock cycle, before terminating, the 'Add' process would write a 6 onto the channel, which of course is not possible since the channel is defined only for the range \{0..5\}, so FDR4 complains about this.\\
However, what we experienced was that when the ranges of the channels were set to a larger number than the system would ever reach, within the defined number of clock cycles, FDR4 would still fail with the same reason.
This caused some problems since we were not able to verify the system and we were not interested in FDR4 trying to verify a communication that would never occur within the network.
Since we are syncronising the processes on the events in the channels, it seems odd that FDR4 still considers events which should not be possible to reach.  \\\\
After some time working with the problem and trying to understand the reason for FDR4s error message, we found that if we simply add a guard or an if-then-else statement that tests the value to be written, FDR4 will gladly verify the system and when using Probe on the network, it is clear that FDR4 does not consider the trace with the wrong values. \\

So the suggested solution, or fix, to this problem, is to add an if-then-else before all writes in a program. The statement then tests the value to be written against the max value of the range of the channel and if the value is not within the range, then the process behaves as the \texttt{SKIP} process, otherwise, it continues with write.
\begin{minted}[linenos=false, escapeinside=||, mathescape=true]{cspm_lexer.py:CSPmLexer -x}
channel c : {0..20}

|$\vdots$|

    if (i+1) > 20  -- Check the upper limit of the channel
        then SKIP  -- SKIP if the value is above
        else (c_r ! (i+1) -> Add(i))) -- Otherwise write and continue
\end{minted}
In this case, it is only necessary to add an upper limit test, since the network only increments, but as a general rule, it would be necessary to test for both upper and lower bounds.


% The problem ocurred several times with different versions of the solution. Also, Ohm had the same problem with the Commstime problem. It makes sense why FDR4 wants to check, and maybe it is a way to save verification time: if it checks things in parallel or something.
% However it is not a good solution that we have to put in a upper/lower bound check that is actually never relevant.
% Of course it might be good, in any case, since the programmer then do not need to worry. On the other hand it might make the program look like it is terminating properly when it is actually failing because it tries to write a value that is not allowed. This case might happen, and if all other processes do not notice it and then also SKIP according to their specifications, then the verification passes even though it might be wrong.
% A solution might be to ensure that all processes synchronise before skipping, because then the problem (i think) would not occur, since the failing process would simply skip, then the other processes cannot skip because they have not syncrhonised yet. however, I am not sure this is possible since it might be that they either syncronise or they skip.

% Kenneths version, which I believe is how the SME model works, is having a process or bus in the middle og all steps. By using a dependency graph (Explain more?) it is possible to see which processes communicate to witch processes and, more importantly, in which order. For each communication step (or maybe for each communication) a process/bus will receive all writes. In SME a process can write several times to the same channel but only the last one before the clock signal will be written, the others are just overwritten. Since we have the dependency graph, we also know which processes we need communication from, and when the process have written all it has to write, then it sends a ready signal to the "bus" process, which then waits for all the ready signals (because it knows how many it should get. And if it is one process/bus pr. communication then it only needs one of course.). When all ready signals are in, the bus-process change behaviour and it is not writing instead of reading. It writes all possible values out and the processes that are supposed to receive the values (which we know from the dependency graph) will receive the values. And the processes then need to send a ready signal back to the bus process to let it know that it have read all it needed. When all ready signals are received, the bus process when change behaviour again and can read values once again.
% All these steps are intermediate steps within one clock cycle. So at the "end" of the dependency graph, the step looks similar to the others, but it is registered as the clock and the next clock cycle begins. In principal, all these steps could be the clock, since the step is the same, but a step is simply chosen to be the clock, based on the dependency graph.
% By treating the communication like this within a clock cycle, the values can propagate through the network and the internal state of the processes are also kept. The original TAPS version could only verify all input for a system, but if the system was internally affected by values from a previous clock cycle, then the system could not verify it. It is not a problem in the seven segment example, since no values are dependent on previous values. But the Addone network do depend on what happened in the last clock cycle.
% With this solution it is possible to verify a specific number of clock cycles.
%
% It is important to note that SME processes usually do not end, but continue running forever. However, when we simulate with SME, the simulation end after some specific number of clock cycles, and this is what we wish to verify.
% It might be interesting to use the "running forever" thing to verify stuff on it. But at the moment I cant see what we would verify. Deadlocks would never happen in SME, so it would be possible to use it to check that the translation have been successful, but this is not so relevant for other people.
% It might be that the stuff we wish to verify is not possible in a limited clock cycle range and that it is necessary to run it "forever". I am not sure.



% (From design)
% \section{Clock cycle problem}
% % \input{chapters/clock_cycle_problem}
%
% CSP was not initially developed for hardware modeling, and therefore it is not evident how to handle the clock cycle, which is an essential part of hardware modeling. When we transpile the SME network into \cspm{}, the SMEIL simulation have provided the ranges of all values from the simulation and therefore all clock cycles. This means that when FDR4 asserts a property it asserts on all possible communication combinations for all the simulated clock cycles. Therefore, even though we are transpiling from an SME model, where the clock is crucial, we can simply translate ``one-to-one" from the SMEIL program and still get an accurate assertion on the properties.
%
%
%
%
% % It is important to mention that the FDR version of the SMEIL program are represented as one clock cycle and therefore we do not have to handle implicit clock cycle issues. we can just translate one-to-one, because FDR models one clock cycle and the input represents all possible input in one clock cycle.
%
%
%
% % TODO: Async vs. sync processes
%
% %%% Generator processes - clock cycles
% %TODO: Write something here?
%
%
% % TODO: Write that the read of a process does not make sense in the network and that we move it to be internal inside the process.


% (from verification in design)
% TODO: Maybe I should do it with the FD model, but in this case it does not really make sense, since all processes end after one iterations (they all SKIP). But it is worth mentioning it in the new system, because there it becomes relevant
%
% CSP was not initially developed for hardware modeling, and therefore it is not evident how to handle the clock cycle, which is an essential part of hardware modeling. When we transpile the SME network into \cspm{}, the SMEIL simulation have provided the ranges of all values from the simulation and therefore all clock cycles. This means that when FDR4 asserts a property it asserts on all possible communication combinations for all the simulated clock cycles. Therefore, even though we are transpiling from an SME model, where the clock is crucial, we can simply translate ``one-to-one" from the SMEIL program and still get an accurate assertion on the properties.
%
%
%
%
% % It is important to mention that the FDR version of the SMEIL program are represented as one clock cycle and therefore we do not have to handle implicit clock cycle issues. we can just translate one-to-one, because FDR models one clock cycle and the input represents all possible input in one clock cycle.
%
%
%
% % TODO: Async vs. sync processes
%
% %%% Generator processes - clock cycles
% %TODO: Write something here?
%
%
% % TODO: Write that the read of a process does not make sense in the network and that we move it to be internal inside the process.


% (from verification in design)
% TODO: Maybe I should do it with the FD model, but in this case it does not really make sense, since all processes end after one iterations (they all SKIP). But it is worth mentioning it in the new system, because there it becomes relevant
%
% CSP was not initially developed for hardware modeling, and therefore it is not evident how to handle the clock cycle, which is an essential part of hardware modeling. When we transpile the SME network into \cspm{}, the SMEIL simulation have provided the ranges of all values from the simulation and therefore all clock cycles. This means that when FDR4 asserts a property it asserts on all possible communication combinations for all the simulated clock cycles. Therefore, even though we are transpiling from an SME model, where the clock is crucial, we can simply translate ``one-to-one" from the SMEIL program and still get an accurate assertion on the properties.
%
%
%
%
% % It is important to mention that the FDR version of the SMEIL program are represented as one clock cycle and therefore we do not have to handle implicit clock cycle issues. we can just translate one-to-one, because FDR models one clock cycle and the input represents all possible input in one clock cycle.
%
%
%
% % TODO: Async vs. sync processes
%
% %%% Generator processes - clock cycles
% %TODO: Write something here?
%
%
% % TODO: Write that the read of a process does not make sense in the network and that we move it to be internal inside the process.


% (from verification in design)
% TODO: Maybe I should do it with the FD model, but in this case it does not really make sense, since all processes end after one iterations (they all SKIP). But it is worth mentioning it in the new system, because there it becomes relevant
%
% CSP was not initially developed for hardware modeling, and therefore it is not evident how to handle the clock cycle, which is an essential part of hardware modeling. When we transpile the SME network into \cspm{}, the SMEIL simulation have provided the ranges of all values from the simulation and therefore all clock cycles. This means that when FDR4 asserts a property it asserts on all possible communication combinations for all the simulated clock cycles. Therefore, even though we are transpiling from an SME model, where the clock is crucial, we can simply translate ``one-to-one" from the SMEIL program and still get an accurate assertion on the properties.
%
%
%
%
% % It is important to mention that the FDR version of the SMEIL program are represented as one clock cycle and therefore we do not have to handle implicit clock cycle issues. we can just translate one-to-one, because FDR models one clock cycle and the input represents all possible input in one clock cycle.
%
%
%
% % TODO: Async vs. sync processes
%
% %%% Generator processes - clock cycles
% %TODO: Write something here?
%
%
% % TODO: Write that the read of a process does not make sense in the network and that we move it to be internal inside the process.


% (from verification in design)
% TODO: Maybe I should do it with the FD model, but in this case it does not really make sense, since all processes end after one iterations (they all SKIP). But it is worth mentioning it in the new system, because there it becomes relevant