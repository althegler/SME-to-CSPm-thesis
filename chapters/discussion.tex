
\section{Usability of TAPS}

% Snak om at TAPS er super og at det er revolutionerende at man ikke skal til at lave de her testing selv og at man ikke skal lave specifikationerne selv. Snak om det brian sagde med at jeg automatisere at lave test hvilket også simplificere arbejdsgangen samtidigt med at det er meget mere sikkert at have det formelt verificeret end at have det testet.
% Something about how TAPS can be used and the limitations in the simulation and clock cycles and such

% Noget med at det er kun fordi smeil er et subset af csp at det kan lade sig gøre, og at det er essentielt at der kan laves generelle strukture i cspm.

The flexibility of SMEIL also increase the complexity for the system, but TAPS is able to recognise and take action of all the different SMEIL structures. T
% (from analysis)
% In the SME model, processes never terminate, but when simulating an SMEIL program, it is necessary to terminate the simulation at some point, but it should simply be seen as a snapshot of the process and not that the process terminates when the simulation ends.
% The number of clock cycles, the simulation is run for, is specified via the command line tool.
% There is, of course, limitations to the simulation, and it is important to note that the simulation is a simple tool to simulate the running system, however it does necessary provide all possible results of the system. If a system fails at a 1000 clock cycles, but the system is only simulated for 100 clock cycles, then we would not know about the failure. If the translation use the observed values from a simulation, the user need to know the limitations of the simulation.
% % If the simulation is not passing through enough clock cycles, the verification might be inadequate. Since the verification builds on the observed values, the simulation needs to be long enough such that the whole possible range of input values is exhausted. %TODO: Where should I add this? Removed it from the background section

\section{Clocked or Unclocked \cspm{} Systems}
The initial idea of the design of TAPS was to build a system which did not have to enforce a global synchronus clock structure since it would create a much more complex structure. As the implementation of the initial version of TAPS progressed it was clear that the design was too simple. The set of verifiable problems in the initial version of TAPS are too small to have an actual impact.\\

The clocked version of TAPS is still able to verify the set of problems verifiable with the initial version of TAPS. By only verifying one clock cycle, the results are the same as for the unclocked version of TAPS, which is a great result from the design structure of the clocked version of TAPS. \\

The clocked version of TAPS becomes more complex and the user must define the number of clock cycles to verify in TAPS, which provides more uncertainty in the verification. If the user does not choose to verify enough clock cycles potential failures might not be caught by FDR4. The number of clock cycles to verify is a balance between the input range for the system, the time requirement of the verification as well as the complexity of the system. \\

It is a huge advantage that the clocked TAPS system are able to reuse such a large part of the initial design. What actually needed to be done was to extend the system and not rewrite it, which also made it feasible to design within the timeframe of this thesis. It is, however, clear that the set of verifiable problems with the clocked TAPS system are far bigger than the initial version of TAPS and so there exist no doubt that this is the system to contine development on. The clocked version of TAPS do increase the complexity, both in the translated \cspm{} but also in the structures of the translations and the amount of corner cases to handle. It is only feasible to translate a clocked network becasue it is automatically generated and even so it is difficult to ensure all the different SMEIl structures are handled in TAPS.
The clocked \cspm{} network provide a better comparability with the original SMEIL network, which I belive is a huge advantage for the further development. The user do not necessarily need to spend time understanding the \cspm{} network, but when developing and testing new features of TAPS, it is a big advantage to be able to compare the behavior of the SMEIL simulation to that of the FDR4 verification. \\

% Maybe add more here when I know if I can use  the clocked example

The clocked version of TAPS do have its pitfalls which require some ingenuity to ensure proper code genration. The advantages of having a larger set of verifiable problems are without a doubt worth the increased complexity.
\section{From Pure SMEIL to Co-Simulation}
Co-simulation is a big feature of SMEIL and yet I focused the work in this thesis only on pure SMEIL support. The reason behind only working with pure SMEIL was simply to have a simple base to start with so that I could ensure that the translation would be as accurate as possible. Since pure SMEIL does not allow for a lot of functionality, the examples also quickly became small enough for easy inspection of the resulting \cspm{} program. When translating from a language like SMEIL to \cspm{} it very quickly becomes overwhelming. The FDR4 traces very quickly became too complex to understand exactly what was going on and why. Because of this I believe that the decicion to only focus on pure SMEIL structures was a good one.
With the current version of TAPS it would be possible to simulate the input from the exposed buses in a co-simulated program so that it is actually a pure SMEIL program but the input is the same as the co-simulated program would input to the SMEIL part.\\

When TAPS become more advanced it should definitely be extended to also include co-simulated programs. TAPS will not have any real value to the industry or adademia before this happens. However I do believe that the extension to support co-simulated programs is possible. A huge benefit lies in the fact that the SME model is a general model which spans all the different SME implementations. The way TAPS translate the processes into three different sections will always fit the SME model, no matter which language TAPS translate from.
There will of course be a lot of challenges in handling the data and communication that spans across the SME implementations in co-simulation, and even more so, the general network in \cspm{} will also be a great challenge. However, since it is very precisely defined which buses communicate across the two implementations, it might help with generating the \cspm{} network in a co-simulated environment.\\

Focusing on pure SMEIL programs have definitely led to a more manageable set of problems and extending this to co-simulated problems seems like the perfect next step.

