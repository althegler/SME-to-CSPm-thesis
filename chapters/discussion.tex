
\section{Usability of TAPS}
TAPS is still in early development and more examples must be verified and corner cases must be handled. In spite of this, it presents the advantages gained from automatic translation from SMEIL to \cspm{}. To my knowledge, transpiling from a hardware description language to a specification language like CSP have not been done successfully before and the advantages of such a system are many. When using TAPS, it is not necessary to develop a test bench for the system and it is not necessary to develop the specification seperately from the hardware model. The workflow from hardware model to formal verification have been simplifies dramatically with TAPS. I believe that it will increase the usage of hardware description languages to be able to perform formal verification with no extra development steps. \\

There are limitations to TAPS and it is possible for failures to happen with systems verifies with TAPS. The bottlenecks of the verification lies both in the simulation values as well as in the number of verified clock cycles. The values verified in FDR4 is based on the observed values from the simulation and so if the SMEIL simulation is not representing all possible values, the verification will not be able to verify these. It will be difficult to find a balance between how long to simulate the SMEIL network and how critical failures can be. In some systems it will not be possible to know if all corner cases have been found in the simulation but FDR4 might provide insight into some corner cases missing in the simulation. The same balance can be difficult to find between how many clock cycles to simulate and how devestating failures can be. The consequence of choosing too low a number of clock cycles to verify can be failures in the system that FDR4 was not able to verify because it never verified that specific state machine where the failure happens.
The answer to this dilemma will be different in all cases and it will ultimately be up to the user to choose. \\

The reason it is possible to create the relatively simple translations from SMEIl to \cspm{} lies in the fact that SME is defined from the CSP language and therefore, as previosuly mentioned, all SME models will have an equivalent CSP model. Having the same basis on both sides of the translation results in a much smoother translation.

\section{Clocked or Unclocked \cspm{} Systems}
The initial idea of the design of TAPS was to build a system which did not have to enforce a global synchronus clock structure since it would create a much more complex structure. As the implementation of the initial version of TAPS progressed it was clear that the design was too simple. The set of verifiable problems in the initial version of TAPS are too small to have an actual impact.\\

The clocked version of TAPS is still able to verify the set of problems verifiable with the initial version of TAPS. By only verifying one clock cycle, the results are the same as for the unclocked version of TAPS, which is a great result from the design structure of the clocked version of TAPS. \\

The clocked version of TAPS becomes more complex and the user must define the number of clock cycles to verify in TAPS, which provides more uncertainty in the verification. If the user does not choose to verify enough clock cycles potential failures might not be caught by FDR4. The number of clock cycles to verify is a balance between the input range for the system, the time requirement of the verification as well as the complexity of the system. \\

It is a huge advantage that the clocked TAPS system are able to reuse such a large part of the initial design. What actually needed to be done was to extend the system and not rewrite it, which also made it feasible to design within the timeframe of this thesis. It is, however, clear that the set of verifiable problems with the clocked TAPS system are far bigger than the initial version of TAPS and so there exist no doubt that this is the system to contine development on. The clocked version of TAPS do increase the complexity, both in the translated \cspm{} but also in the structures of the translations and the amount of corner cases to handle. It is only feasible to translate a clocked network becasue it is automatically generated and even so it is difficult to ensure all the different SMEIl structures are handled in TAPS.
The clocked \cspm{} network provide a better comparability with the original SMEIL network, which I belive is a huge advantage for the further development. The user do not necessarily need to spend time understanding the \cspm{} network, but when developing and testing new features of TAPS, it is a big advantage to be able to compare the behavior of the SMEIL simulation to that of the FDR4 verification. \\

% Maybe add more here when I know if I can use  the clocked example

The clocked version of TAPS do have its pitfalls which require some ingenuity to ensure proper code genration. The advantages of having a larger set of verifiable problems are without a doubt worth the increased complexity.
\section{From Pure SMEIL to Co-Simulation}
Co-simulation is a big feature of SMEIL and yet I focused the work in this thesis only on pure SMEIL support. The reason behind only working with pure SMEIL was simply to have a simple base to start with so that I could ensure that the translation would be as accurate as possible. Since pure SMEIL does not allow for a lot of functionality, the examples also quickly became small enough for easy inspection of the resulting \cspm{} program. When translating from a language like SMEIL to \cspm{} it very quickly becomes overwhelming. The FDR4 traces very quickly became too complex to understand exactly what was going on and why. Because of this I believe that the decicion to only focus on pure SMEIL structures was a good one.
With the current version of TAPS it would be possible to simulate the input from the exposed buses in a co-simulated program so that it is actually a pure SMEIL program but the input is the same as the co-simulated program would input to the SMEIL part.\\

When TAPS become more advanced it should definitely be extended to also include co-simulated programs. TAPS will not have any real value to the industry or adademia before this happens. However I do believe that the extension to support co-simulated programs is possible. A huge benefit lies in the fact that the SME model is a general model which spans all the different SME implementations. The way TAPS translate the processes into three different sections will always fit the SME model, no matter which language TAPS translate from.
There will of course be a lot of challenges in handling the data and communication that spans across the SME implementations in co-simulation, and even more so, the general network in \cspm{} will also be a great challenge. However, since it is very precisely defined which buses communicate across the two implementations, it might help with generating the \cspm{} network in a co-simulated environment.\\

Focusing on pure SMEIL programs have definitely led to a more manageable set of problems and extending this to co-simulated problems seems like the perfect next step.

\section{Verification FDR4}
FDR4 is a 

