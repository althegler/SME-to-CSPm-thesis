



\begin{figure}[!ht]
  \centering
  \begin{tikzpicture}
    \node [mysquare] (SME) at (0, 2.5) {$SME$};
    \node [mysquare] (SMEIL) at (0, 0) {$SMEIL$};
    \node [draw, black, thick, rounded corners, dotted, inner sep=0.2cm] (Q) at (2.5, 1.3){$Questions$};
    \draw [myarrow, smooth] (SME) to[out=270, in=90] (SMEIL);
    \draw [myarrow, dotted] (Q) to[out=180, in=90] (SMEIL);


    \node [mysquare] (Parser) at (3, -1.3) {$Parser$};
    \node [mysquare] (Codegen) at (5.5, -1.3) {$Code Gen$};
    \node [draw, red, thick, dotted, fit=(Parser)(Codegen), inner sep=0.5cm] (TAPS) {};
    \node [red] at (4.4, -0.2) {$TAPS$};
    \draw [myarrow, smooth] (SMEIL) to[out=270, in=180] (Parser);
    \draw [myarrow, smooth] (Parser) to[out=0, in=180] (Codegen);

    \node [mysquare] (cspm) at (8.5, 0) {$CSP_M$};
    \draw [myarrow, smooth] (Codegen) to[out=0, in=270] (cspm);
    \node [mysquare] (FDR) at (8.5, 2.5) {$FDR4$};
    \node [draw, black, thick, rounded corners, dotted, inner sep=0.2cm] (A) at (5.5, 2.5){$Answers$};
    \draw [myarrow, smooth] (cspm) to[out=90, in=270] (FDR);
    \draw [myarrow, dotted] (FDR) to[out=180, in=0] (A);

  \end{tikzpicture}
  \caption{System structure.}
  \label{fig:TAPS_network}
\end{figure}

\section{Transpiling Channels}
In SMEIL when a process is reading from a bus channel, it is referencing the bus and channelname within the normal expression. Since the bus parameters for a SMEIL process is the bus name, the process itself must reference the specific channel within that bus. This is applied whether it is reads or writes.
An example of this can be seen in Listing \ref{lst:smeil_input_parameter}, where \texttt{input.val} indicates a read from the bus \texttt{input} and the channel \texttt{val} within that bus.
\begin{listing}
    \begin{minted}[linenos=false, escapeinside=||, mathescape=true]{smeil_lexer.py:SMEILLexer -x}
        proc a (in input)
            bus abus {
                val: uint;
            }
        {
            abus.val = input.val + 1;
        }
    \end{minted}
    \caption{}
    \label{lst:smeil_input_parameter}
\end{listing}
SMEIL does not seperate communication and calculation in the process statements which is of great advantage in SMEIL, but the translation becomes more difficult since TAPS has to be able to recognise the communication.
This means that TAPS can recognise communication by the structure of the assignment. If one of the elements in an assignment, right or left, contains a dot, then we can assume that this is communication. The original grammar of an assignment like the one seen above can be seen in Listing \ref{lst:smeil_assignment_grammar}. A communication is the \textit{hierarchical accessor} alternative in the \textit{statement} grammar. The communication can either be the right hand side of the assignment, which is a write, or part of the left hand side expression, which will then be a read. A write is simple to recognise, since it is not combined with other parts of the grammar. TAPS can search the \textit{name} of a right hand side assignment and see if it contains a dot. The read is a bit more complicated since it can be used like any internal variable in the expressions. This means that TAPS will have to search all names within the nested expression to find a potential read.
% TODO: Write something more about how TAPS actually achieve this.
\begin{listing}
\begin{grammar}
<statement> ::= <name> `=' <expression> `;' (assignment)

<name> ::= <ident>
\alt <name> `.' <name> (hierarchical accessor)
\alt <name> `[' <array-index> `]' (array element access)

<expression> ::= <name>
\alt <literal>
\alt <expression> <bin-op> <expression>
\alt <un-op> <expression>
\alt <name> `(' \{ <expression> \}  `)' (function call)
\alt `(' <expression> `)'


\end{grammar}
\caption{The original assignment, name and expression grammars defined in \cite{Asheim2018}.}
\label{lst:smeil_assignment_grammar}
\end{listing}

When translating an SMEIL bus channel to a \cspm{} channel it give channel names that will be unique in \cspm{} just like the formal name of a bus channel in SMEIL is unique.
\subsection{Naming Channels}
The simple way of translating the SMEIL channels into \cspm{} chanels are to use the formal name of the SMEIL bus channel already defined. The naming are created by concatinating the channel name, bus name and process name which will ensure that the generated code are more readable than if I had used random unique strings as naming, which is also a possible solution. A unique string might give more security than using a concatinated version, but in this case I decided to make use the allready defined names. This means that all calls using the syntax \texttt{bus.channel} in SMEIL will be translated to \texttt{bus\_channel} in \cspm{}.

There is however one situation where this naming might cause troubles. If a process is reused and instantiated twice in the network, the bus channels defined in the processes would have the same name in SMEIL, but because of the instance declarations in the SMEIL network, it is possible to separate the different channels in SMEIL. Therefore it is important to handle this in \cspm{} so there does not occur problems with having identical channel names.

In the example below we see a network consisting of four instances where two of the instances are the same process. Each instance is defined by an instance name which is used for referencing the buses declared as parameters.
\begin{minted}[escapeinside=||, mathescape=true]{smeil_lexer.py:SMEILLexer -x}
network net() {
    instance c of clock();
    instance a of A(c.output, val: 1);
    instance _ of A(c.output, val: 2);
    instance s of src(a.output);
}
\end{minted}
In SMEIL it is also possible to have a instance without an instance name as can be seen above. However, it is defined in SMEIL that two instances of the same process, cannot both be without an instance name.
% NOTE: Does not support!!
TAPS will need to name the channels so that no \cspm{} channels are identical. This can be done by looking at the data from the instances and append the instance name to the \cspm{} channel name. The restrictions to identical process instances in SMEIL ensures that even if there are channels without instance names in the SMEIL program there will never be two identical named  \cspm{} channels in the generated code. This however does mean that the naming of each channel quickly becomes long and chaotic, but it is still more readable than random generated names.
% NOTE! Currently TAPS does not support the naming of identical process busses.

%%% Calling the channel in the bus
It is also possible to have bus declarations within the network declaration in SMEIL. These bus channels will be named in the same manner as the other channels, however without a potential instance name appended.

\subsection{\cspm{} Channel Ranges}
When defining the channels in \cspm{} it is important, to defined a limited space of values accepted for the channel. If the channel is defined for the entire space of integers, which are within a signed or twos-complement 32-bit representation in FDR4, FDR4 would try to verify all possible integers which would result in the statespace becomming too large and FDR would run out of space. Therefore TAPS need to define some specific range of values for each channel but it must be values relevant to the channel and not some random ranges.
As explained in Chapter~\ref{chap:analysis}, all simulated SMEIL programs will include the observed range and restricted types for all channels and variables. The types represent the minimum observed width of the channels in bits, and by calculating the possible range from these types, we can create the corresponding channels in \cspm{}, and thereby avoid having a seemingly endless runtime in FDR4.
\begin{listing}
\begin{minted}[escapeinside=||, mathescape=true]{smeil_lexer.py:SMEILLexer -x}
proc seconds (in seconds_in)
    bus seconds_out {first_digit: u3 range 0 to 5;
                     second_digit: u4 range 0 to 9;};
    var seconds: u6 range 1 to 59;
    var seconds_first_temp: u3 range 0 to 5;
    var seconds_second_temp: u4 range 0 to 9;
{
    seconds = seconds_in.val % 60;
    seconds_first_temp = seconds / 10;
    seconds_second_temp = seconds % 10;
    seconds_out.first_digit = seconds_first_temp;
    seconds_out.second_digit = seconds_second_temp;
}
\end{minted}
\caption{Example of the \texttt{seconds} process from the SMEIL seven segment display example. See full example in Listing~\ref{lst:smeil} in the appendix.}
\label{lst:range_smeil}
\end{listing}


An example of simulated SMEIL code can be seen in Listing~\ref{lst:range_smeil}. Notice that the two channels are defined both with a type \texttt{u3} and \texttt{u4} and with a range 0 to 5 and 0 to 9. These are the observed types and value ranges the simulation has tracked for each specific channel. In order to create the \cspm{} channels based on the types, we need to convert \texttt{u3} and \texttt{u4} into the range of values that the types represents. For \texttt{u3} this is 0 through 7 and for \texttt{u4} it is 0 through 15. In Listing~\ref{lst:channel_range_cspm} the calculated ranges are used to define the \cspm{} channels.

\begin{listing}
\begin{minted}[escapeinside=||, mathescape=true]{cspm_lexer.py:CSPmLexer -x}
channel seconds_out_first_digit : {0..7}
channel seconds_out_second_digit : {0..15}

Seconds(seconds_in) =
let
    seconds = seconds_in % 60
    seconds_first_temp = seconds / 10
    seconds_second_temp = seconds % 10
within
    seconds_out_first_digit ! seconds_first_temp ->
    seconds_out_second_digit ! seconds_second_temp ->
    SKIP
\end{minted}
\caption{Example of the \texttt{Seconds} process from the generated \cspm{} code in the seven segment display example. See full example in Listing~\ref{lst:cspm} in the appendix.}
\label{lst:channel_range_cspm}
\end{listing}


Since the assertion we wish to make is to verify the widths of the channels, it might seem redundant to create \cspm{} channels with a limited range. FDR4 would always only check the values in the defined channel range and therefore there is no point in asserting if the values go beyond this range. After simulating the SMEIL network, the compiler provides us with both a type and a range of observed values. The type, as explained above, is used to create the restricted range for each \cspm{} channel and the observed values are used for the assertion. The range of values that the types represent will always represent equal or more values than the range of observed values, and by using these values the assertions becomes valuable.\\

When it comes to transpiling the data generator process into a \cspm{} channel, we also use the type of the outbus bus channel to define it. TAPS use this instead of the observed values because we cannot guarantee the precise input values of the system. If I used the observed values, the assertions will pass every time, since it will test the values already used to generate the rest of the observed values.



\section{Generating Monitor Processes}
When generating the monitor processes in the \cspm{} program we use the observed values for each channel, as described above.
In the SMEIL grammar defined in \cite{Asheim2018} it is optional to include a range to each bus channel definitions. In Listing \ref{lst:smeil_bus_grammar} the relevant grammar rules from \cite{Asheim2018} can be seen. The square brackets indicates optional parts.
\begin{listing}
    \begin{grammar}
    <bus-decl> ::= [ `exposed' ] `bus' <ident> `\{' <bus-signal-decls> `\}'  `;'

    <bus-signal-decls> ::= <bus-signal-decl> \{ <bus-signal-decl> \}

    <bus-signal-decl> ::= <ident> `:' <type> [ `=' <expression> ] [ <range> ] `;'
    \end{grammar}
    \caption{The bus grammar defined in \cite{Asheim2018}}
    \label{lst:smeil_bus_grammar}
\end{listing}
\begin{listing}
    \begin{grammar}
    <bus-decl> ::= `bus' <ident> `\{' <bus-signal-decls> `\}'  `;'

    <bus-signal-decls> ::= <bus-signal-decl> \{ <bus-signal-decl> \}

    <bus-signal-decl> ::= <ident> `:' <type> [ `=' <expression> ] <range> `;'
    \end{grammar}
    \caption{The bus grammar defined in \cite{Asheim2018} changed to match the demands of the translation.}
    \label{lst:smeil_bus_grammar_no_option}
\end{listing}
As explained in Chapter \ref{chap:design}, all channels except the input channels are verified in FDR4. This means that there will be created a monitor process for each channel and since TAPS use the observed values for each channel for the verification, I had to change the original grammar of SMEIL to suppport this. Therefore the grammar shown in Listing \ref{lst:smeil_bus_grammar} is changed to the grammar defined in Listing \ref{lst:smeil_bus_grammar_no_option}. The only difference is that ranges are no longer optional and that the keywork \texttt{exposed} are no longer allowed, since TAPS are not supporting co-simulation, which is what this keyword indicates. \\

The monitor process asserts that the observed values of the \cspm{} channels are actually what is communicated on the channels. In Listing~\ref{lst:monitor_range_cspm} the two monitor processes for the \texttt{Seconds} process from the seven segment example can be seen. The values used for these statements are 0 through 5 for the first digit and 0 through 9 for the second which can be seen defined for each channel in Listing~\ref{lst:range_smeil}.
\begin{listing}
\begin{minted}[escapeinside=||, mathescape=true]{cspm_lexer.py:CSPmLexer -x}
Seconds_out_first_digit_monitor(c) =
    c ? x -> if 0 <= x and x <= 5 then SKIP else STOP
Seconds_out_second_digit_monitor(c) =
    c ? x -> if 0 <= x and x <= 9 then SKIP else STOP
\end{minted}
\caption{Example of the \texttt{Seconds} monitor processes from the generated \cspm{} code in the seven segment display example. See full example in Listing~\ref{lst:cspm} in the appendix.}
\label{lst:monitor_range_cspm}
\end{listing}

This monitor process structure is a simple but general solution which can be reused in many situations and due to the opserved ranges, TAPS can always provide a monitor process with this structure.
% TODO: Make sure that examples of the failure in the seven segment example are included somewhere, as well as a working example.


------------------------------------------------------------------------------\\
------------------------------------------------------------------------------\\
------------------------------------------------------------------------------\\
------------------------------------------------------------------------------\\

\section{Transpiling Networks}

\section{Technologies}

Usually when doing any form of translation either by a compiler or interpreter, the use of a symbol table is often needed in order to keep information about variable names, function names, classes, communication ect. Usually the translation, when reaching a symbol in the translation, use the symbol table to look up a symbol and retrieve its context if it has any information about said symbol. This can be to check that a variable have been declared or for type checking ect.. The symbol table is usually generated in the analysis section and are used for look ups througout the compilation or interpretation.
....

\subsection{ANTLR4}
For the transpiling between SMEIL source code and \cspm{} source code, we decided to use ANTLR4~\cite{antlr} for creating a parser and a lexer. ANTLR4 is a Java-based parser generator library that, based on a grammar, can generate parsers in Java or another target language. ANTLR4 provided a tool that could easily transform the given grammar into a parser and lexer that could immediately be used to transform into \cspm{}.

ANTLR takes a grammar, defined in .g4 (BKNF?) and create the parser and lexer of the files.

A lexical analysis, which is what the lexer does, is a process of converting a string of characters into tokens, which is also called tokenization. Each token represents a lexer rule in the grammer, for instance, if the string is "123" and there is a lexer rule "INT: {0-9}+" which means one or more of 0-9 digits. Then the token would be an INT.

A parser...%TODO: write more here

Currently, only a subset of the SMEIL grammar have been implemented for translating and parts of the grammar have been changed slightly to match the expectations from a simulated SMEIL program, which varies a bit from a non-simulated SMEIL program. An example of this could be that in the original SMEIL grammar a channel declaration only includes an optional range, however, we are expecting a simulated SMEIL program and therefore the simulation would always have generated ranges of observed values for each channel. Therefore, in the grammar, created for ANTLR4, the range for each channel must always be defined.

The ANTRL4 grammar is provided in a filetype called \texttt{.g4} but the structure of the grammar is similar to standard grammar notation.
After running ANTLR4 and generating a parser and a lexer, one can decide to traverse the parse tree itself or use a listener or visitor that ANTLR4 provides. The main difference between the two is that the methods the listener provides are called by the walker object, which ANTLR4 provides, and the visitor methods must call their children explicitly to walk them.
For our implementation, we used the ANTLR4 listener along with Python. When generating a parser and lexer for another target language than Java, the programmer only has to specify this in the ANTLR4 command in the command-line.\\

% NOTE: This part below have been added mostly as a brain dump. Should be re-written or something

With this implementation, we have used the visitor method, which is a well known traversal method %TODO: Is that true?
With the visitor we gain the advantage that we decide which part of the parse tree we traverse, instead of having to traverse the entire parse tree as with the listener method. With the visitor, for each parser we have to explicitly call \texttt{visit()} on the children that we wish to visit, which, in some cases can make a tree traversal more complex, but in our case we wish to seperate concerns and this is done by having three different tree traversals which collect three differet pieces of data for the code generation.
When restructuring the code for the visitor method, we have to implement the top parsers, \texttt{module} and \texttt{entity}, even though they dont have any real value for the data we need to collect. However, because of the visitor method having to call all children, we need to implement these two parsers, simply so they can call their children.
This seem a little pointless compared to the listener, where we can ignore the parsers we do not care about, but if we use a listener we would have to use a stack system and that would cause problems, because a terminal node that we only want sometimes might push something on the stack that we did not want to. Because every time that terminal node is entered or exited the code is run. With the visitor we can control when the nodes are run which means that we avoid some messy stack problems.
A visitor can return a value like other standard functions which means that we have directly access to the result from it or its children. However, if there are more than one child the result will have to be assigned in a list or similar, to access.
If there are two expressions in one alternative, we can get the two seperate results by calling \texttt{self.visit(ctx.expression(0))} and \texttt{self.visit(ctx.expression(1))}. This simplifies the structure since we do not have to handle loops in this case, because we know that the structure would always be two expressions, not more and no less.\\
Even though we update the data structure \texttt{data} by reference instead of returning it as a result from the tree traversals, it is still necessary to return values from some of the parsers. Parts of the data structure cannot be created before having specific data, and the top parser with all the information will then gather the results from its children and update the \texttt{data} dictionary with the new data.


% TODO: Maybe it is worth mentioning how I remove some sub-"parsers" in the grammer, because it made the parsing more complicated . Such like I removed the processdecl nonterminal because I could then call visit(ctx.busdecl) specificly. Otherwise I could also just do "if ctx.busdecl: ..." but, this gave one less step (however, now I am thinking about it, it might not be a good idea. It gives more complexity and maybe thats not good if other people need to understand the antlr grammar. )

% TODO: Write something about how I should just gather the data and then letting the transformation part do any data transformations. I am, for example, not doing that now since I am adding calculations and communications into two different lists.

% TODO: Write something about how smart it is that I can use labels in the grammar and that I can then much mor easily control how to handle different alternatives.



