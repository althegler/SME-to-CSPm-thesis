%!TEX root = ../main.tex
In this chapter I introduce some of the main challenges in translating from SMEIL to \cspm{} and how they have been solved in TAPS.

The requirements for TAPS are considerable because of the flexibility of SMEIL. Translating the different functionalities of SMEIL have been priorities according to the requirements of the examples used within this thesis and due to time restrictions, the current version of TAPS does not support the entire SMEIL grammar. TAPS expect a well-formed legal SMEIL program as input and does not perform any type-checking. The best way to ensure an accurate translation to \cspm{} is to simulate the SMEIL program before using TAPS to ensure that the program complies with all the rules of the libsme compiler.
\section{Transpiling SMEIL Statements}
SMEIL statements are defining the functionality of the process and can consist of simple arithmetic operations, conditionals, or communication.
The statements are translated into the \texttt{let within} structure as introduced in Chapter \ref{chap:design}.\\

All assignments in SMEIL are translated directly into \cspm{} without much change, however if the assignment is consisting of communication either in or out, TAPS must handle these assignments differently, which will be explained later in this section.\\

When translating a constant from SMEIL to \cspm{} the constant is added to the \cspm{} program separately from the process it was defined within.
The SMEIL programs must be well-formed and thus the constant will only be used by the process it was defined within. It is therefore not a problem to have the constant defined outside of the \cspm{} process. If a constant is defined as a process parameter in the SMEIL program, will be translated as an initial value for the process and not as a separate \cspm{} constant.\\
% NOTE Constants are currently not implemented in TAPS

Variables, in SMEIL, can be instantiated with an initial value and therefore TAPS must search through all the variable declarations in the SMEIL process. If a variable is defined with an initial value this information is gathered so that the initial value of the variable can be added to the generated \cspm{} process.
Variables with no initial value will be ignored by TAPS because they do not need instantiation in \cspm{} and will be translated along with the other statements. With the current version of TAPS only values communicated on channels are verified, and therefore the range and type of a variable have no purpose within the generated \cspm{} code. It would, however, be simple to gather this information if needed.\\

The trace statement in SMEIL is not affecting the behaviour of the process. It prints out the string and arguments that are given, like a printf in C. It is possible to add a print statement in \cspm{} but it has a very different functionality than the \texttt{trace} statement in SMEIL. A print statement in a \cspm{} program will appear in the right-hand side of the FDR4 session window where it can be activated for easy evaluation. It does not support printing out text, but rather calculations and thus the trace statement in SMEIL is not compatible with the print statement \cspm.\\

The assert statement is used internally in SMEIL and evaluated during program execution. If the assert statement is not valid, then the execution is halted and the error message, defined in the assert statement, is printed. An example can be seen below.
\begin{minted}{smeil_lexer.py:SMEILLexer -x}
assert(hour < 23, "hours cannot be greater than 23");
\end{minted}
Even though the purpose of this project is to assert properties in FDR4, the properties that FDR4 can verify are not similar to the properties that the \texttt{assert} statement in SMEIL can assert. As was introduced in Section \ref{sec:background_fdr}, FDR4 is a refinement checker and the assert statements in FDR4 are asserting the refinement checks. In the \cspm language, assertions like \texttt{assert 4 + 4 == 8} are possible, however FDR4 does not support this\cite{fdr4}. The two types of assert statement are therefore not equivalent. It would, however, also not be of any interest to translate the SMEIL assert statements, since a potential error found via an SMEIL assertion, might as well be found and corrected in SMEIL than in FDR4.\\

Neither the SMEIL trace statement or the assert statement provide any functionality to the translation or the network model, so there are two options for how TAPS should handle these. Either TAPS throws them away or keeps them as comments for reference in the generated code. Currently, TAPS throw them away, but it would be a simple task to change this and add them as comments in the generated \cspm{} code.\\
% NOTE: Assertions are not curently implemented in TAPS. Trace is.

Expressions in SMEIL and their precedence rules are similar to C-like languages and they are used for defining operations and naming in SMEIL. All of the standard binary operators behave the same in SMEIL as in \cspm{} as well as relational operators and logical conjunction and disjunction. The unary \texttt{!(not)} operator does not have the same precedence in the SMEIL grammar as in \cspm{}. Also, the equality operators and comparison operators have the same precedence in \cspm{}, but not in the SMEIL grammar. To avoid erroneous computation, TAPS add parentheses around all nested expression. The SMEIl grammar also consists of bitwise operations which FDR4 does not support because \cspm{} only supports integers. It is likely possible to model the bitwise operations using the already supported operators, but this has not been a priority to implement and the current version of TAPS does not support it.\\

As mentioned in Section \ref{sec:background_smeil}, SMEIL supports floating point numbers in the simulator, but it has not been extensively tested. Floating point numbers are not supported by \cspm{}\cite{Scattergood2011} because it is not possible to assert if two floating point numbers are equal and therefore it would never be possible to create refinement checks with floating point numbers. It is possible to create models in \cspm{} that defines a floating point number, but the type itself is not supported in \cspm{} and FDR4. TAPS does not provide any type checking and therefore it does not provide a warning if floating point numbers occur within an SMEIL program. It will be possible to verify the program with FDR4 but the verification will not be accurate since it will round all numbers to the nearest integer.
\section{Transpiling Channels}
 The read and write syntax in SMEIL are used as a variable in SMEIL. In Listing \ref{lst:smeil_input_parameter}, the syntax \texttt{input.val} indicates a read to the channel \texttt{val} within the bus \texttt{input}. As can be seen, the read and write structures are used like they where simple variables representing a value.

\begin{listing}
\begin{minted}{smeil_lexer.py:SMEILLexer -x}
proc a (in input)
    bus abus {
        val: uint;
    }
{
    abus.val = input.val + 1;
}
\end{minted}
\caption{Example of a read and a write in SMEIL.}
\label{lst:smeil_input_parameter}
\end{listing}
SMEIL does not separate the communication and calculation in a process statement as can also be seen in Listing \ref{lst:smeil_input_parameter}. This is of great advantage in SMEIL, but the translation becomes more complex because TAPS has to be able to recognise the communication.
TAPS must be able to recognise the communication in an assignment which is possible because both the bus and the channel must be specified in a read or write in SMEIL. If one of the elements in an assignment, right or left-hand side, contains a dot, TAPS can assume that it is communication. The original grammar of an assignment like the one in Listing \ref{lst:smeil_input_parameter} can be seen in Listing \ref{lst:smeil_assignment_grammar}. A communication is the \textit{hierarchical accessor} alternative in the \textit{statement} rule. The communication can either be the right-hand side of the assignment, which is a write, or part of the left-hand side expression, which is a read. A write is simple to recognise since it is not combined with other parts of the grammar. TAPS can search the \textit{name} of a right-hand side assignment and see if it contains a dot. The read is a bit more complicated since it can be used as an internal variable in the expressions. This means that TAPS will have to search all names within the nested expression to find a potential read.
% TODO: Write something more about how TAPS actually achieve this.
\begin{listing}
\begin{grammar}
<statement> ::= <name> `=' <expression> `;' (assignment)

<name> ::= <ident>
\alt <name> `.' <name> (hierarchical accessor)
\alt <name> `[' <array-index> `]' (array element access)

<expression> ::= <name>
\alt <literal>
\alt <expression> <bin-op> <expression>
\alt <un-op> <expression>
\alt <name> `(' \{ <expression> \}  `)' (function call)
\alt `(' <expression> `)'


\end{grammar}
\caption{The original assignment, name and expression grammars defined in \cite{Asheim2018}.}
\label{lst:smeil_assignment_grammar}
\end{listing}
\subsection{Naming Channels}
When translating an SMEIL bus channel to a \cspm{} channel TAPS must define \cspm{} channel names that will be unique, the same way that the formal name of a bus channel in SMEIL is unique.\\

The simplest way of translating the SMEIL channels into \cspm{} channels is to use the already defined formal name of the SMEIL bus channel. By concatenating the channel name, bus name, and process name into a single channel name in \cspm{} the naming becomes unique.
This means that all reads and writes using the syntax \texttt{bus.channel} from the process \texttt{A} in SMEIL will be translated to \texttt{A\_bus\_channel} in \cspm{}.
It is also possible to use randomly generated strings instead of using the original SMEIL names, but to ensure correct translation in TAPS it was necessary to have recognisable names for the \cspm channels.
\begin{listing}
\begin{minted}{smeil_lexer.py:SMEILLexer -x}
network net() {
    instance c of clock();
    instance a of A(c.output, val: 1);
    instance _ of A(c.output, val: 2);
    instance s of src(a.output);
}
\end{minted}
\caption{Example of a network with four instances whereas two are instances of the same process.}
\label{lst:Instance_variations}
\end{listing}
The example in Listing \ref{lst:Instance_variations} shows a network consisting of four instances where two of them are instantiating the same process with different constant parameters. All but one instance declaration also includes an instance name which can be used when referencing the process in the instance parameters. SMEIL enforce a rule that two instances of the same process, cannot both have an anonymous instance name which is defined as \texttt{\_}.\\

% NOTE: SMEIL does not support adding instance names to the cspm channel name
With two instances of the same process, in SMEIL, the \cspm{} channels of those processes, would have identical names. It is, of course, crucial to avoid duplicate channel names. TAPS look through the instances, in the SMEIL network, and append the instance name to the \cspm{} channel name. This is also assuming that no developer would assign two identical process instances the same instance name. If the instance name is anonymous, a \texttt{\_} is appended to the \cspm{} channel for simplicity. These restrictions to identical process instances in SMEIL ensures that there will never be two identical named \cspm{} channels in the generated code. The naming of each channel quickly becomes long and chaotic which is one of the compromises of generated code. It is also possible to have bus declarations within the network declaration in SMEIL. These bus channels are named in the same manner as other channels in \cspm{}, however without an instance name appended.

\subsection{\cspm{} Channel Range}
When defining channels in \cspm{} it is important to define a restricted range of acceptable communication for each channel. If the channel is defined with an unbounded range of integers, FDR4 would search through all possible integers which would result in the state space becoming too large and FDR4 would run out of space. These restricted ranges must, of course, represent relevant values.

As explained in Chapter~\ref{chap:analysis}, all simulated SMEIL channels and variables are defined with a range of observed values and a restricted type.
Because FDR4 only supports integers, only signed and unsigned types are supported in TAPS. The integer following the type defined the bit size of the observed values. This bit size is used to generate a restricted range for the \cspm{} channel, and thereby avoid having a seemingly endless runtime in FDR4.

In Listing ~\ref{lst:range_smeil} the simulated \texttt{seconds} process from the seven segments display example can be seen. The types of the two channels are \texttt{u3} and \texttt{u4} and the observed values are between 0 and 5 for the first digit and 0 and 9 for the second. TAPS converts the bit size of the type \texttt{u3} to the range \texttt{\{0..7\}} which are then used as the restricted range for the \cspm{} channel representing the first digit. The second digit channel type \texttt{u4} is converted to the range \texttt{\{0..15\}}.  In Listing~\ref{lst:channel_range_cspm} the converted ranges are defining the values of the \cspm{} channels.

\begin{listing}
\begin{minted}{cspm_lexer.py:CSPmLexer -x}
channel seconds_out_first_digit : {0..7}
channel seconds_out_second_digit : {0..15}

Seconds(seconds_in) =
let
    seconds = seconds_in % 60
    seconds_first_temp = seconds / 10
    seconds_second_temp = seconds % 10
within
    seconds_out_first_digit ! seconds_first_temp ->
    seconds_out_second_digit ! seconds_second_temp ->
    SKIP
\end{minted}
\caption{Example of the \texttt{Seconds} process from the generated \cspm{} code in the seven segment display example. See full example in Listing~\ref{lst:cspm} in the appendix.}
\label{lst:channel_range_cspm}
\end{listing}

It might seem redundant to create \cspm{} channels with a limited range when TAPS is also asserting the widths of the channels. FDR4 will always only perform refinement checks using the values defined for each channel in \cspm{} and therefore there seems to be no point in asserting if the values go beyond this range. The method for performing usable refinement checks is to use both the type and the range of observed values, defined for each simulated SMEIL channel. The range of observed values are used within the monitor process to ensure that no values outside the range are communicated on the channel and because the range of the bit size can only be equal or larger than the range of the observed values, the assertions become valuable. \\

As explained previously, an SMEIL data generator process will be translated into a \cspm{} channel. In this case, it is important to use the bit size range for the \cspm{} channel because it is not possible to guarantee the precise input values in the network. If TAPS were to use the range of observed values instead, the monitor process assertions will always pass because the input values are the exact values used for the simulation. These input values are the base of all the observed values for all SMEIL channels and so the assertions would never fail. It is therefore necessary to have a range of input that goes beyond the range of observed values to verify that the model can handle a larger range of input without failures.
\section{Generating Monitor Processes}

It is clear that all channels except input channels must be verified with FDR4 and thus TAPS must generate a monitor process for all channels except the input channels.


As explained above, the observed values for each SMEIL channel are used within the monitor processes to verify the values communicated on the channels. For TAPS to be able to generate the monitor process, it is essential that all the channels are defined with observed values. In the original SMEIL grammar, defined in \cite{Asheim2018}, it is optional to include a range in a bus channel definition as can be seen in Listing \ref{lst:smeil_bus_grammar}.
In order to restrict the SMEIL network to include ranges for each SMEIL bus channel before translating with TAPS, I changed the original grammar of SMEIL. The grammar shown in Listing \ref{lst:smeil_bus_grammar} is slightly changed compared to the grammar defined in Listing \ref{lst:smeil_bus_grammar_no_option}. The difference is that a range is no longer optional and that the keyword \texttt{exposed} is no longer allowed. The keyword \texttt{exposed} can be defined for a bus in SMEIL to indicate buses that communicate with another SME implementation using co-simulation, but since TAPS is not supporting co-simulation, the keyword is not needed. A consequence of this change in the grammar is that the output bus channels within the data generator process must be defined with a range even though the range is not needed for the translation. Because the data generator processes are not defined differently in SMEIL, the same grammar rules apply to it.
\begin{listing}
    \begin{grammar}
    <bus-decl> ::= [ `exposed' ] `bus' <ident> `\{' <bus-signal-decls> `\}'  `;'

    <bus-signal-decls> ::= <bus-signal-decl> \{ <bus-signal-decl> \}

    <bus-signal-decl> ::= <ident> `:' <type> [ `=' <expression> ] [ <range> ] `;'
    \end{grammar}
    \caption{The bus grammar defined in \cite{Asheim2018}. The square brackets indicates an optional nonterminal and curly brackets indicates zero or more nonterminals.}
    \label{lst:smeil_bus_grammar}
\end{listing}
\begin{listing}
    \begin{grammar}
    <bus-decl> ::= `bus' <ident> `\{' <bus-signal-decls> `\}'  `;'

    <bus-signal-decls> ::= <bus-signal-decl> \{ <bus-signal-decl> \}

    <bus-signal-decl> ::= <ident> `:' <type> [ `=' <expression> ] <range> `;'
    \end{grammar}
    \caption{The bus grammar defined in \cite{Asheim2018} changed to match the demands of the translation.}
    \label{lst:smeil_bus_grammar_no_option}
\end{listing}
The monitor process asserts that the observed values from the SMEIL simulation are the only values communicated on the channels. The two monitors of the \texttt{Seconds} process from the seven segments example can be seen in Listing~\ref{lst:monitor_range_cspm}. The values used for these monitors are 0 through 5 for the first digit monitor and 0 through 9 for the second digit monitor. These values have been defined each channel in the simulated SMEIL program in Listing~\ref{lst:range_smeil}.
\begin{listing}
\begin{minted}{cspm_lexer.py:CSPmLexer -x}
Seconds_out_first_digit_monitor(c) =
    c ? x -> if 0 <= x and x <= 5 then SKIP else STOP
Seconds_out_second_digit_monitor(c) =
    c ? x -> if 0 <= x and x <= 9 then SKIP else STOP
\end{minted}
\caption{Example of the two generated \texttt{Seconds} monitor processes from the seven segment display example. See full example in Listing~\ref{lst:cspm} in the appendix.}
\label{lst:monitor_range_cspm}
\end{listing}
Each monitor asserts values on one channel and thus the name of the monitor process is the channel name with \texttt{\_monitor} appended. This is a very simple solution but very useful because it means that TAPS can easily find the name of the monitor process from the channel name, and vice versa. It is clear that no \cspm{} channels will have identical names and so by using this naming method for the monitor processes I also ensure that no monitor processes will be identical.\\

This monitor process structure is a simple solution but it is a general structure that can be reused with different assertion properties.
\section{Generating a \cspm{} Network}
The process monitor network consists of three parts. The first part consists of reading input values for the process. As explained previously all reads will be performed outside the process itself. From the instance parameters, defined in SMEIL, TAPS will lookup the name of the SMEIL process where the input is defined. The read within the SMEIL process defines the specific channel to read from. This information is gathered to clarify which \cspm{} channel the process monitor network should read from. Each read is added using prefixing, so it would also work if there are more than one input values for the process. An example of the first part of the network can be seen in Listing \ref{lst:network_first_part}.
\begin{listing}
\begin{minted}{cspm_lexer.py:CSPmLexer -x}
N_seconds = clock_out_val ? variable -> ...
\end{minted}
\caption{First part of the network generation where an input value is read from the channel \texttt{clock\_out\_val}.}
\label{lst:network_first_part}
\end{listing}
% NOTE: TAPS does not currently support more than one input channel
The second part is to add the process name with the expected parameters, both input values, output channels as well as constants and internal values. The example in Listing \ref{lst:network_second_part} is a continuation of the example in Listing \ref{lst:network_first_part}.
\begin{listing}
\begin{minted}{cspm_lexer.py:CSPmLexer -x}
N_seconds = clock_out_val ? variable -> Seconds(variable) ...
\end{minted}
\caption{The second part added to the network generation.}
\label{lst:network_second_part}
\end{listing}

In the last part, TAPS will synchronise the process with the monitor processes over the channels they are asserting.
For each monitor process TAPS builds a new layer of synchronisation around the network. This is straightforward to auto-generate since TAPS can loop over the list of monitor processes or writes for the process. \\

Listing \ref{lst:network_example_cspm} shows the network generated from the translated \texttt{Seconds} process in Listing \ref{lst:channel_range_cspm} and the corresponding monitor processes in Listing \ref{lst:monitor_range_cspm}. Adding parentheses around each synchronisation, the \texttt{first digit} monitor process is synchronised on the \texttt{first digit} channel and the \texttt{second digit} monitor process is synchronised on the \texttt{second digit} channel which completes the process monitor network.

Lastly the \cspm{} \texttt{assert} operator is specifying the refinement check for FDR4. The assertion defines a refinement check from the specification process \texttt{SKIP} and the implementation process \texttt{N\_seconds} and that the refinement check should be done using the Failures model.\\
\begin{listing}
\begin{minted}{cspm_lexer.py:CSPmLexer -x}
N_seconds = clock_out_val ? variable ->
            (Seconds(variable)
            [| {| seconds_out_first_digit|} |]
            Seconds_out_first_digit_monitor(seconds_out_first_digit))
            [| {| seconds_out_second_digit|} |]
            Seconds_out_second_digit_monitor(seconds_out_second_digit)

assert SKIP [F= N_seconds \ Events
\end{minted}
\caption{Example of the \texttt{Seconds} network processes from the generated \cspm{} code in the seven segment display example. See full example in Listing~\ref{lst:cspm} in the appendix.}
\label{lst:network_example_cspm}
\end{listing}

In SMEIL it is possible to define more than one network within the same SMEIL program. This is not a problem for TAPS to translate as long as the program is still a well-formed SMEIL program. Each instance defines a process and its parameters. This communication will always refer to a specific bus, defined in either a process or a network. This will not change no matter how many networks are defined within the program because communication cannot span across the networks.
In theory, it is possible to create nested networks where a network can instantiate other networks, and thereby allowing communication across networks, but this has not yet been implemented in SMEIL and therefore TAPS does not handle this case.
The only situation where several networks can become a problem for TAPS is if the networks have instance names in common. As mentioned above, it is necessary to include the instance name to the name of a generated \cspm{} channel in order to tell the difference between two \cspm{} channels. If there is more than one network using the same instance name for a specific process the generated code would have identical channel names in one \cspm{} program. To solve this, TAPS could include the network name to the generated \cspm{} channel name to add another layer of separation. This is a corner case, and the current version of TAPS does not support this solution, but since it is a potential problem it is important to address.

\section{The Technologies of TAPS}
The two main parts of TAPS are the parser and the code generator. The parser is created with ANTLR4~\cite{antlr}, which is described further below. The data transformation and code generation have been developed in Python 2.7 while also utilising the templating language Jinja2~\cite{jinja2}.\\

Often when developing any form of translation either by a compiler or interpreter, the use of a symbol table is often necessary in order to keep information about variable names, function names, classes, communication etc. During the translation when reaching a symbol the system will use the symbol table to look up the symbol and to retrieve its context. The symbol table is usually generated in the analysis section and is used for lookups throughout the compilation or interpretation. \\

When walking the parse tree after the parsing, TAPS gathers all relevant information from the SMEIL code. TAPS expect the SMEIL program to be well-formed and does not provide any checking of the SMEIL code and therefore it is not necessary for TAPS to include a symbol table to check if variables have been properly declared in the process declaration before being used in the body of the process.
\subsection{ANTLR4}
For the transpiling from SMEIL to \cspm{}, I decided to use a parser generator tool to generate a parser for the SMEIL language.

I choose a parser generator library because it would simplify the parsing process so I could focus more on the development of correct code generation. Parser generators are tools that provide an automatically generated parser, interpreter or compiler from a specified grammar. Using a parser generator was an obvious choice since the SMEIL grammar was already in a well defined and simple EBNF format. It was also possible to create a parser from scratch but there seemed to be no reason to spend the time writing and debugging my own parser when I could generate it directly. \\

I decided to use the ANTLR4 tool after looking at several other parser generator tools. ANTLR4 is a Java-based parser generator library that, based on a grammar, can generate parsers in Java or another target language. ANTLR is a well-used tool which has recently been updated to the new and improved ANTLR4 version. Companies like Twitter and Oracle are using ANTLR\ref{Parr2012} for different parse tasks. ANTLR4 also has an active user base and there is plenty of documentation and guides for the new ANTLR4 online. A lot of parser generator tools exists today and some of the most known parser generators are Yacc or Bison but unfortunately, a lot of these tools seemed to be a bit outdated and not very user friendly in terms of error messages or simplicity in the grammar structure. ANTLR4 seemed to be the right choice for this simple grammar.

By using ANTLR4 it was simple to transform the SMEIL grammar into a parser and lexer using the already defined SMEIL grammar. ANTLR4 supports Extended BNF (EBNF) grammers which must be defined in \texttt{.g4} file format. This is quite similar to other standard grammars. ANTLR4 also provides the possibility of generating a tree walker which can be used to traverse the parse tree. There are three options for tree traversal, either using the listener or visitor method provided by ANTLR4 or manually traversing the parse tree.
The listener methods are called directly by the walker object and are analogous to XML document handler objects because they respond to SAX events triggered by XML parsers. The visitor method in ANTLR is similar to the standard visitor patterns seen in other parser structures. In ANTLR each visitor must call \texttt{visit()} on its children to walk the subtrees. ANTLR4 also provides the possibility of adding labels to alternatives in grammar rules which creates a visitor method for each labeled alternative. This ensures a more manageable parsing and less complex visitor functions. \\

Both the visitor method and the listener method can provide the same results. In the first version of TAPS I used the listener method, but I realised that it is not necessary to traverse the entire parse tree in every parse and therefore TAPS was rewritten to use the visitor method that only visits the subtree needed.\\

After TAPS have walked the parse tree and collected all relevant information from the SMEIL program, the data is transformed to match the requirements of the \cspm{} code. For example, the name of a bus and its channels in SMEIL is transformed into combined names for the \cspm{} channels. Because TAPS only collect data during the tree walk and not performing direct translation, it does not matter how the SMEIL code is structured. With this solution, the SMEIL network declaration can be defined first or last in the program without any difference to the translation.\\

The templating language Jinja2 have been used to define the standard \cspm{} structures which the SMEIL program will be translated into. Together with the transformed data the templates are used to generate the \cspm{} code.



%
% With the visitor we gain the advantage that we decide which part of the parse tree we traverse, instead of having to traverse the entire parse tree as with the listener method. With the visitor, for each parser we have to explicitly call \texttt{visit()} on the children that we wish to visit, which, in some cases can make a tree traversal more complex, but in our case we wish to separate concerns and this is done by having three different tree traversals which collect three differet pieces of data for the code generation.
% When restructuring the code for the visitor method, we have to implement the top parsers, \texttt{module} and \texttt{entity}, even though they dont have any real value for the data we need to collect. However, because of the visitor method having to call all children, we need to implement these two parsers, simply so they can call their children.
% This seem a little pointless compared to the listener, where we can ignore the parsers we do not care about, but if we use a listener we would have to use a stack system and that would cause problems, because a terminal node that we only want sometimes might push something on the stack that we did not want to. Because every time that terminal node is entered or exited the code is run. With the visitor we can control when the nodes are run which means that we avoid some messy stack problems.
% A visitor can return a value like other standard functions which means that we have directly access to the result from it or its children. However, if there are more than one child the result will have to be assigned in a list or similar, to access.
% If there are two expressions in one alternative, we can get the two separate results by calling \texttt{self.visit(ctx.expression(0))} and \texttt{self.visit(ctx.expression(1))}. This simplifies the structure since we do not have to handle loops in this case, because we know that the structure would always be two expressions, not more and no less.\\
% Even though we update the data structure \texttt{data} by reference instead of returning it as a result from the tree traversals, it is still necessary to return values from some of the parsers. Parts of the data structure cannot be created before having specific data, and the top parser with all the information will then gather the results from its children and update the \texttt{data} dictionary with the new data.


% TODO: Maybe it is worth mentioning how I remove some sub-"parsers" in the grammer, because it made the parsing more complicated . Such like I removed the processdecl nonterminal because I could then call visit(ctx.busdecl) specificly. Otherwise I could also just do "if ctx.busdecl: ..." but, this gave one less step (however, now I am thinking about it, it might not be a good idea. It gives more complexity and maybe thats not good if other people need to understand the antlr grammar. )

% TODO: Write something about how I should just gather the data and then letting the transformation part do any data transformations. I am, for example, not doing that now since I am adding calculations and communications into two different lists.

% TODO: Write something about how smart it is that I can use labels in the grammar and that I can then much more easily control how to handle different alternatives.

