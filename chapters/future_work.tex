With this work, we have taken a small step towards creating a simpler method for software developers to model hardware as well as verify properties within this model. In future work, we would like to extend this to software-hardware co-design, with which we would be able to assert deadlocks.

It would be desirable to be able to automatically create a human-readable report on the ranges and communications that are used within the system. This could become a standard addition to the documentation of the system, which would give a programmer an easy overview of a complicated system and would also allow for easier contemplation over the system.

Another, more complex idea for future work, is to implement support for multi-channel invariants. This is not something that can easily be simulated and therefore it would require some work, but it would provide the ability to express more complex assertions.


% "Hardware-software co-design is an area that is actively researched. The idea is that
% specialized hardware is designed in parallel with the corresponding software such that
% each part of the design can be implemented on either hardware or software depending
% on what is best suited. Such heterogeneous designs require code for setting up the
% communication between the hardware and software parts. We should therefore extend
% our co-simulation approach to also allow SME networks to be distributed across several
% devices." (From Truls thesis)