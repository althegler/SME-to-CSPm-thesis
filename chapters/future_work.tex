%!TEX root = ../main.tex

SMEIL is not a complete implementation, and as other SME implementations are improved, so should the SMEIL implementation. As the SMEIL language becomes more comprehensive and supports more features, TAPS should be kept in line with these advancements.\\
The automatic verification provided by TAPS lightens a major workload in testing and verifying hardware models. Therefore, it might also be relevant to structure the SMEIL implementation towards better FDR4 results, while still maintaining the basic SME model structure. \\

The extended version of TAPS provides verification of the different internal states within a hardware model. It was introduced rather late in the project, and therefore the implementation is not as extensive as the initial version of TAPS. Substantial groundworks have been laid in providing the design structures for the extended version, making further development more straightforward. It is, however, obvious that future work should include providing a full implementation of the extended TAPS system. \\

Besides further development of the extended version of TAPS, more advanced and in-depth examples should be developed in SMEIL, in order to understand the limitations of the translation as well as FDR4. This will also provide a better understanding of the compression FDR4 provides. It is important to learn if the compression could be performed in the experiments because of the general clocked network structure, or if it is only performed in that specific network. \\

FDR4 also provides the possibility for integration with other tools using the FDR4 API, which is currently available for C++, Java, and Python. The FDR4 API is currently not used in TAPS, but it is an obvious choice to extend TAPS with the FDR4 API to provide a cleaner workflow. Because TAPS has been developed for FDR4 specifically, no other verification tool will match the current translation structures, and therefore the current version of TAPS would benefit from using FDR4 directly via the API.\\

As described in Chapter \ref{chap:background}, SMEIL was mainly developed to provide an intermediate language between the existing SME implementations.
To provide support for the entire SMEIL functionality, TAPS should, of course, be augmented to support co-simulation as well. As mentioned in Chapter \ref{chap:discussion}, it might be possible to make use of external communication trace files, to generate channel ranges for the generated \cspm{} network.\\

Another point for future work is to extend the different assertion possibilities within TAPS. Currently, only channel communication can be verified, but as described in Chapter \ref{chap:implementation}
these monitor processes do have their limitations, as it is not currently possible to verify a combination of values. Therefore it would be an advantage to extend TAPS to define more advanced assertions to verify values over multiple channels. \\

It would also be an advantage to extend TAPS to support software-hardware co-design. The idea behind software-hardware co-design is that hardware and software are designed in parallel, so that both can be implemented on either hardware or software depending on what is most suited. If SMEIL and TAPS were extended to support this, then the communication between software and hardware would be possible to verify with FDR4. \\

When verifying a system in FDR4, it can be crucial for the developer to know what values and states have been verified. It is therefore desirable to have TAPS generate a human-readable report on the ranges and communications that are verified with TAPS. This would also give the developer a possibility of better evaluating the number of clock cycles to verify in FDR4.
This report could become a standard addition to the documentation of the hardware model, which would give a developer an easy overview of a complicated system, and would also allow for reasoning about the system.