%!TEX root = ../main.tex
I have presented TAPS, a transpiler that translates programs written in in the
new SME intermediate language (SMEIL) into the machine-readable CSP language,
\cspm{}. In TAPS, the translated \cspm{} code is augmented with assertion
properties which can be verified with the Failure-Divergences Refinement tool
(FDR4).\\
I have presented an initial system which is able to translate SMEIL networks to
\cspm{} and verify a large range of defined input values. TAPS can assert that
the observed values of a channel, in a simulated SMEIL program, are in fact the
only possible values communicated on that specific channel. \\

An extension to this initial version of TAPS have also been presented in this
thesis. This extension provides TAPS with the support to model a global
synchronous clock in \cspm{}. It have previously been shown in \cite{Skaarup14}
that enforcing a global synchronous clock onto a PyCSP network results in a
state explosion. To avoid having to model networks with this large amount of
states, the SME model was developed, based on the CSP algebra. With the
extended version of TAPS, the state explosion occurs but because TAPS auto
generates the \cspm{} code the state explosion is not a hindrance to the
translation.\\

The translated \cspm{} network can be verified with the FDR4 tool which can
perform refinement checks on \cspm{} networks. I present examples of failures
and solution to these with an example of a Seven Segment Display network and a
cyclic Addone network showing the advantages of both the initial version of TAPS and the clocked version.\\

This system makes it more accessible for software programmers to program
hardware and thereby bridging a gap between software programmers and the needs
of the industry.
Instead of having to create advanced test-benches, TAPS provides a simple
way to verify the hardware model via FDR4s assertion functionalities.
TAPS do not yet provide support for the entire SMEIL grammar. In spite of this,
I am pleased with the possibilities that TAPS present and I am optimistic about its future prospects.



