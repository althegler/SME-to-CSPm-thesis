%!TEX root = ../main.tex
% Description of the actual solutions
% TODO: Write something here
The goal of automatic translation is to be able to create a general solution which can fit different types of problems and therefore it is necessary to generalise the different aspects of the translation and find a solution that fits all.
\section{Designing TAPS}
In this section we describe all the design decisions included in creating TAPS so that it addresses all the challenges of the translation between SMEIL and \cspm{}. The section is divided in a similar way as Chapter \ref{chap:analysis}, and we will go through all the challenges described there.
\subsection{Behavioral}
%% - Behavioral description - hvad den enkelte funktion gør. det oversættes ret nemt. Jeg bekymre mig ikke om variable og sådan nogle ting. Det er allerede gjort før det gøres til SMEIL, og dem kan jeg genbruge i min code generation.  Funktionel indhold af processer/ opførsel af den enkelte process - det er rimelig nemt oversat direkte til CSPm. Her kan man beskrive hvis der er nogle sproglige udfordringer, fx hvis CSPm ikke har loops eller lign.
\subsubsection{Processes}
In order to translate the SMEIL process to a \cspm{} process we had to create a general process structure in \cspm{}. We know that the SME model enforce that each process reads, calculates and writes, in that order, for each clock cycle, so we needed to create a \cspm{} structure that could support this. First of all we wished to have one \cspm{} process per SMEIL process, since it would simplify the translation and a simpler solution typically results in a less errornous solution.

What first comes to mind in \cspm{} when we want to read and then write, is a very simple process structure using prefix and communication operators.
\begin{minted}[linenos=false, escapeinside=||, mathescape=true]{cspm_lexer.py:CSPmLexer -x}
Proc(x) = c ? x -> d ! x-> SKIP
\end{minted}
This is the simplest \cspm{} process that match the SME model, however with this structure it is not possible to include all the possible calculations that SMEIL support. It turned out that the tricky part of translating a general SMEIL process to \cspm{} was to be able to include the calculations properly.

By using the \texttt{let within} structure in \cspm{} we are able to keep the communication together with the arithmetics in one \cspm{} process while keeping a simple structure. The \cspm{} process reads first and then all arithmetics are put inside the \texttt{let} clause while the writing will be put in the \texttt{within} part.

% NOTE: This is where i stopped. The part below is just copied in. I have to figure out if I want to keep the parameter for the process or not, and that depends on if I talk about both the first and the second version or if I only talk about the second version. But it would be great to keep the seven segment example as a reference point. This same process is shown in the analysis part, so the reader just saw it. 


% TODO: Figure out if I should ignore my first version, and only talk about the firnal one or if I should divide into two sections.


\begin{listing}
\begin{minted}[escapeinside=||, mathescape=true]{cspm_lexer.py:CSPmLexer -x}
channel seconds_out_first_digit : {0..7}
channel seconds_out_second_digit : {0..15}

    |$\vdots$|

Seconds(seconds_in) =
let
    seconds = seconds_in % 60
    seconds_first_temp = seconds / 10
    seconds_second_temp = seconds % 10
within
    seconds_out_first_digit ! seconds_first_temp ->
    seconds_out_second_digit ! seconds_second_temp ->
    SKIP
\end{minted}
\caption{Example of the \texttt{Seconds} process from the generated \cspm{} code in the seven segment display example. See full example in Listing~\ref{lst:cspm} in the appendix.}
\label{lst:channel_range_cspm}
\end{listing}



------------------------------------------------------------------------------\\
------------------------------------------------------------------------------\\
------------------------------------------------------------------------------\\
------------------------------------------------------------------------------\\


% Let within statement
In order to keep the outwards communication and the arithmetic statements together within each process in \cspm{}, we generate \cspm{} processes with a \texttt{let within} statement. The arithmetic statements go into the \texttt{let} section and the communications go into the \texttt{within} section. This gives us the possibility of separating the outwards communication and arithmetic statements while still keeping them within the same \cspm{} process. In Listing~\ref{lst:channel_range_cspm}, an example of the \texttt{let within} statement can be seen in lines 7-14. This structure will work as a general translation structure from SMEIL processes to \cspm{} processes.

%%% Generator processes
For the generator processes, we need to make sure the transpiler can handle the different kind of data generation possible in SMEIL.
%%% Generator processes - clock cycles
%TODO: Write something here?
%%% Generator processes - no input bus
The simplest structure is when the process does not have an input bus. When analysing the purpose of these processes, their only task is to feed data into the network via communication, and nothing else. Since this is their only task, we do not need to translate this into a process in \cspm, because we are only interested in the data the process outputs. Therefore we look at the specified observed\textbf{(???)} range and use this to create a \cspm channel with that specific range. By only creating a channel instead of a process in \cspm, we avoid further complications and simplify the generated code. In Figure %TODO: Create a figure with how a generator process becomes a channel.
an example of how a generator process without an input bus can be translated into a \cspm channel and still keep the data for the network instact.

%%% Generator processes - input bus with const declared
If there is an input bus in all processes, the job of locating the data or generator process becomes more difficult. In Figure %TODO: Create a figure with an example of a process like AddOne, where there is an input bus an a const. Also add how it should be translated in a subfigure next to it.
an example of a process with an input bus, but where the process also works as the data instantiation point. In this case the constant that are communicated to the process from the network instance declaration indicates what value the network should start with and the process then communicates it on its output bus and then acts as an integral part of the communication of the network for the rest of the clock cycles. This means that the process has a task to perform after the instatiation of the data and therefore the transpiler cannot simply translate it into a \cspm channel, because that would mean that the task the process performs after the data initialization, will not be possible afterwards. One way of solving this problem would be for the programmer to avoid these "two-task" processes, however that is also not a neat solution and it will be a hindrance for the programmer to structure the network in a certain way in order to verify the code later on. However, in the case of the \texttt{AddOne} example, the easy solution would simply be for the programmer to create an extra process which initialized the data and then communicated it once to the process which then would loop for each clock cycle.
%TODO: Figure out how I can handle this in a better way than simply ask the programmer not to create processes like this, and write about it.

%% Generator processes - More than one process
If there are several processes in the SMEIL program that acts as a data generator process, the transpiler will not handle the processes different than if there were one data generator process. If the processes consists only of generating data, then the processes will be translated into a \cspm channel for each process, and the comunication to the rest of the network will still be kept intact since all the communicated are specified by the SMEIL network structure.
%TODO: Find an example with more than one data generation process.
If the processes are not simple data generator processes, then the transpiler will have to handle it differently %TODO: Figure out how I can handle this in a better way than simply ask the programmer not to create processes like this, and write about it. - Same as with one process but if it becomes more complicated with more than one, then write it here!


% TODO: Write something about even though variables are defined int the declaration, we do not need to use that part and can basically ignore it. This also means that if there is a variable or const that has not been used in the statements, the translation wont know. But also on page 27 of truls master it shows that having unused variables would result in an error from SMEIL.


% An example of an SMEIL process, where the process structure is evident, can be seen in Listing~\ref{lst:range_smeil} and the corresponding \cspm{} code in Listing~\ref{lst:channel_range_cspm}.

\subsection{Structural}
% - Strukturel information: hvilke proceser er der og hvilke buser er de limet sammen med. hvordan hænger tingene sammen. Det er også forholdsvis nemt fordi FDR har processer på samme måde som SMEIL har. de har en historisk afhængighed ift. SME og CSP.

% TODO: Write about two networks and how it will be translated to a network and therefore we should be able to have them seperated easily and the communication is what combines them, and that is the same in cspm. TODO: But ask Truls if this would even make sense. To have to networks.

% TODO: Write about how one process can be defined in SMEIL several times and how it does not matter, since the process will be defined the same in CSPm but that the network will simply be generated twice with two different names and inputs or similar. (Does that also work if it is different output?)

% (moved from analysis: )
%  % None of these two posssible usecases are currently implemented in TAPS, and thus the keyword \texttt{exposed} will cause an error in the transpiler. %TODO: Will it cause an error? And should I add more info to this?

% %%%% Generating the network
We can standardize the network generation by creating a two-step communication part. Instead of having the actual processes receive the incoming data, they receive the data by their process parameter. The process parameter is then set by the network process which receives the communication from the channels and provides the process with the communicated value.
This ensures that we can generate the processes easily without having to traverse the network in the SMEIL program beforehand to find out which channel provides input for which process. An example of this is shown in Listing~\ref{lst:cspm} in the appendix on lines 61 to 66.

% %%%% Instances and two networks.
% TODO: Add an example showing two different networks and how it translates.
When translating the instances of a SMEIL network, the transpiler does not need to keep the instances together in a certain way. Each instance will be translated into a seperate \cspm network and even though the networks might have communication in common, they do not need to be connected in \cspm. If the networks/instances have communication in common, it will be handled by the communication in \cspm. When translating from an instance into a network in \cspm it is simply to map the communication from the correct channels and to the correct channels in \cspm. If the process is connected to a monitor process this is also in the network that the monitor process is added to the communication it will be listening in on.
In a SMEIL program it is possible to have several networks in one program. There is not much of a point in doing so, since it will result in the same as having all instances in one network %TODO: Am I 100\% sure that two networks is the same as one? ask Truls
, but never the less, if there is two networks it will not matter in the \cspm translation. One SMEIL instance represents one \cspm network, and all these networks or instances are combined via communications on channels or buses. The only thing we are interested in when translating the SMEIL instances are the communication defined in each instance. And that is what creates the network. This means that we can generate the \cspm networks without looking at what SMEIL network that specific SMEIL instance came from, because the important thing, the communication, is defined in the instance and therefore all data about how the SMEIL network is combined together we still keep intact even though we translate the instances without considering the rest of that SMEIL network.
% %%%% Instances with the same process twice
% TODO: Add an example showing one process added twice and hwo it translates.
When defining instances in the SMEIL network, it is possible to define the same process several times and have it receive different input or send different output or constants. When translating this to a \cspm network, the process itself will be generated as all other processes, but simply the network will be two seperate network, as if it was two SMEIL instances calling two different SMEIL processes. The only difference between the two networks are that the input channel will be different, if that is the difference between the two SMEIL instances.



% %%%% Channels %%%%%
When generating the channels that represents bus channels in SMEIL, each SMEIL channel will be generated into an \cspm channel.
% TODO: write more here above. Why can we do this. Add something about ranges.
% TODO: What if I need to translate a bus with letters, or something else? what kind of channel does that become?

% %%%% Channel names
The naming will be created by concatinating the channel name, bus name and process name along with underscore in order to generate human readable code. It would also be possible simply to generate a unique string, which might give more security than using a concatinated version, but in this case we decided to make it easier for humans to read and understand the generated code, since the system is still in a "new state".
% %%%% Calling the channel in the bus
If the bus is generated in the network, then the naming will be the channel, bus and network name instead.
% TODO: add an example of naming in SMEIL and then in CSPm. Both with process and network defined buses
In the generated \cspm code, the reference to the each channel, whether it is defined in a process or in a network declaration, will be the specific name generated from the channel, bus and process/network name, as mentioned above. This means that all calls using the syntax \texttt{bus.channel} in SMEIL will be translated to \texttt{bus\_channel} in \cspm.
% %%%% Bus only defined as parameter
If the bus is only defined as a parameter in the process, the process will still have to call the bus channels in order to receive or send data on the bus. This is up to the programmer to make sure that the channel name used, are the channel names that the bus have been defined with, no matter where in the program the bus is defined. When generating the \cspm code, the calls to the channels will be the same, no matter where the buses was defined in the original SMEIL program. The channels will all be defined as global channels and the references will be the same as mentioned above (maybe add reference)
% TODO: Maybe add a refernece to a figure here.
% %%%% Bus used as input bus with more that one channel
If a bus is used as input bus in a process, the process will have to call the specific channel of the bus in order to access the data, comunicated on that specific bus. If the channel have more that one channel, the method of calling does not change, and it is up to the programmer to make sure that the channel called, are actually a channel within that specific input bus.



% channels from SMEIL to CSPm
Channels within an SMEIL bus can be translated directly to \cspm{} channels. Since a SMEIL bus consist of channels, the translation is quite simple. It is, however, important to give channel names that will be unique since a \cspm{} channel is global as opposed to the local channel within each SMEIL bus. Because there are several different ways to define a bus in SMEIL %TODO: see the analysis chapter,
the translation will have to recognize the different types and generate the \cspm channels no matter how they where defined.


\subsection{Meta and verification}
% - Opserverede værdier - "known limits". Meta information i SMEIL, og det skal oversættes til faktiske processer med faktiske semantic i FDR der også er en del af topologien. (Det er essensen for mig).

% %%%% Monitor processes %%%%%
When creating the assertions, we decided to create separate assert functions to keep the code structure clean. We know that for each \cspm{} channel there must be an assertion, except for the input channel.
% TODO: Only in the case where we assert channel ranges.. maybe add something about that.
Consequently, we create a \textit{monitor} process for each channel and its only job is to listen in on the channel communication and assert the values communicated there. The monitor process is a process that we add specifically for asserting legal communication values in FDR4 and it does not affect the original SME network.
In Figure~\ref{fig:assertion_process} the outline of this kind of structure can be seen and we expect that this structure can be used for several different types of problems and thereby ensure a cleaner code structure.

% The monitor process asserts the observed values of the \cspm{} channels and in Listing~\ref{lst:monitor_range_cspm} the two monitor processes for the Seconds \texttt{time} process can be seen. The values used for these statements are the observed values from the SMEIL simulation, as can be seen at the end of lines 2 and 3 in Listing~\ref{lst:range_smeil}. In Listing~\ref{lst:monitor_range_cspm} the ranges are used to assert that the only values communicated on the channels are within 0 and 5, and 0 and 9 respectively.

\begin{figure}[!ht]
  \centering
  \begin{tikzpicture}[auto]
    \node[mycircle] (P) at (-1.5, 0.0) {$P$};
    \node[mycircle] (Q) at ( 2.5, 0.0) {$Q$};
    \node[mycircle, shape=rectangle] (M) at ( 0.5, 1.5) {$M$};

    \node[draw, shape=circle, inner sep=0pt, minimum size=5pt] (m) at (0.5, 0.0) {};


    \draw (M) -- (P -| M) [black!50];
    \draw [myarrow] (P) -- (Q);
  \end{tikzpicture}
  \caption{The monitor process \textit{M} listens in on the communication between \textit{P} and \textit{Q} in order to assert the communicated values.}
  \label{fig:assertion_process}
\end{figure}

% \begin{listing}
% \begin{minted}[escapeinside=||, mathescape=true]{cspm_lexer.py:CSPmLexer -x}
% Seconds_out_first_digit_monitor(c) =
%     c ? x -> if 0 <= x and x <= 5 then SKIP else STOP
% Seconds_out_second_digit_monitor(c) =
%     c ? x -> if 0 <= x and x <= 9 then SKIP else STOP
% \end{minted}
% \caption{Example of the \texttt{Seconds} monitor processes from the generated \cspm{} code in the seven segment display example. See full example in Listing~\ref{lst:cspm} in the appendix.}
% \label{lst:monitor_range_cspm}
% \end{listing}

% %%%% Monitor processes: More that one monitor process for a channel
It is possible to have as many monitor processes as needed in the network. Since they dont change the functionality of the system, they can be added without problems. In both \cspm and SMEIL a channel has an input and an output, but the output is not for a specific process. all processes can acces the data on a bus in SMEIL, as long as the network describes the communication, and the same in \cspm.
% %%%% Monitor processes: Monitor observed ranges

% %%%% Monitor processes: monitor something else that ranges





% \section{Seven Segment Display Clock in SMEIL}\label{sec:example-smeil}
% In order to explain how we can transpile programs from SMEIL to \cspm{}, we have designed an example using a seven segment display clock.
% In this section, the seven segment display example will be explained as well as the SMEIL implementation of the network.
% \\
%
% A seven segment display is an electronic display device which is used in displays such as digital clocks or other types of devices that display numerals. An example of a typical digital clock display can be seen in Figure~\ref{fig:6_displays}. When a digit has been determined for a seven segment display, it is encoded to a bitstream that represents the digit in the correctly activated display segments.
% \begin{figure}[!ht]
%   \begin{center}
%     \tikz{
%       \node[inner sep=5pt, outer sep=2pt, draw=blue] {
%         \sevensegnum[size=2em, shrink=0.1]{1}
%         \sevensegnum[size=2em, shrink=0.1]{2}
%       }
%     }
%     \tikz{
%       \node[inner sep=5pt, outer sep=2pt, draw=blue] {
%         \sevensegnum[size=2em, shrink=0.1]{3}
%         \sevensegnum[size=2em, shrink=0.1]{4}
%       }
%     }
%     \tikz{
%       \node[inner sep=5pt, outer sep=2pt, draw=blue] {
%         \sevensegnum[size=2em, shrink=0.1]{5}
%         \sevensegnum[size=2em, shrink=0.1]{6}
%       }
%     }
%   \end{center}
%   \caption{Digital clock with six seven segment displays, displaying 12:34:56.}
%   \label{fig:6_displays}
% \end{figure}
% In this example, we wish to model a typical digital clock that is able to calculate and display the current time in hours, minutes, and seconds. Listing~\ref{lst:python} shows this example written in Python.
% When creating this model in SMEIL some input must be added to the network, just like \texttt{time\_since\_midnight} in Listing~\ref{lst:python}. The input value represents seconds since midnight, and in order to calculate hours, minutes, and seconds we model three different processes, called the \texttt{time} processes in this example.
%
% When writing hardware models in pure SMEIL, the only way to generate input for the network is to create a data generator process. This process, called the \texttt{clock} process in our example, is instantiated with the start time and is incremented by 1 for each simulation cycle, representing a one second increase. The result is communicated on the process output bus, where the three \texttt{time} processes are listening. These \texttt{time} processes receive the number and by the use of simple integer arithmetic, calculate the hours, minutes, and seconds since midnight respectively. It is obvious that at some point in time, each \texttt{time} process will calculate a two-digit result, for example at 12 hours or 42 seconds. However, a single seven segment display can only show one digit between 0 and 9. Therefore we need two seven segment displays for each \texttt{time} process in order to show the correct time in a 24-hour interval. Each \texttt{time} process has an output bus with two individual channels that represent the communication to each different display. The number representing either hours, minutes, or seconds are separated into first and second digit, by $\lfloor \frac{x}{10} \rfloor$ and $(x \text{ mod } 10)$. These six different results are then communicated onto the six different channels which represent the six different seven segment displays.
% The outline of this network can be seen in Figure~\ref{fig:smeil_network}.
% \begin{listing}
% \begin{minted}[escapeinside=||, mathescape=true]{python}
% from math import floor
%
% def time(time_since_midnight):
%     hours   = floor(time_since_midnight / 3600)
%     minutes = floor((time_since_midnight - hours * 3600) / 60)
%     seconds = time_since_midnight - hours * 3600 - minutes * 60
%     return [hours, minutes, seconds]
%
% print(time( 57100)) # =>  15:51:40
% print(time(  3601)) # =>  01:00:01
% print(time( 66666)) # =>  18:31:06
% \end{minted}
% \caption{A Python implementation of the seven segment display example.}
% \label{lst:python}
% \end{listing}
% \begin{figure}[!ht]
%   \centering
%   \begin{tikzpicture}
%     \node [mycircle] (I) at (0,0) {$I$};
%
%     \node [mycircle] (H) at (2.5,  1.50) {$H$};
%     \node [mycircle] (M) at (2.5,  0.00) {$M$};
%     \node [mycircle] (S) at (2.5, -1.50) {$S$};
%
%     \draw [myarrow] (I) -- (M);
%
%     \draw [myarrow, smooth] (I) to[out=0, in=180] (H);
%     \draw [myarrow, smooth] (I) to[out=0, in=180] (S);
%
%     % Output arrows without processes
%     \draw [myarrow] (3.125,  1.625) -- (4.000,  1.750);
%     \draw [myarrow] (3.125,  1.375) -- (4.000,  1.250);
%     \draw [myarrow] (3.125,  0.125) -- (4.000,  0.250);
%     \draw [myarrow] (3.125, -0.125) -- (4.000, -0.250);
%     \draw [myarrow] (3.125, -1.375) -- (4.000, -1.250);
%     \draw [myarrow] (3.125, -1.625) -- (4.000, -1.750);
%   \end{tikzpicture}
%   \caption{SMEIL network for a seven segment display clock. Each SMEIL process is represented by a cicle with a letter corresponding to the processes Input, Hours, Minutes and Seconds respectively.}
%   \label{fig:smeil_network}
% \end{figure}
%
% In Figure~\ref{fig:smeil_network} the network consists of four processes, the data generator process, \textit{I}, which creates the input that is broadcasted out on the network. The three \texttt{time} processes, hours (\textit{H}), minutes (\textit{M}), and seconds (\textit{S}) are the processes described above, which calculate each part of the current time. The outputs are communicated on the six outgoing channels.
%
% The full SMEIL code for this example can be seen in Listing~\ref{lst:smeil} in  the appendix.
%
% \section{Seven Segment Display Clock Transpiling}
% In the following we use a classic hardware design to illustrate each of the steps in the transpiling, and how the types, constraints, and assertions are carried from the original SMEIL program into the \cspm{} program.
%
% \begin{figure}[!ht]
%   \centering
%   \begin{tikzpicture}[auto]
%     \node[myrectangle, text width=2cm, minimum height=1cm, inner sep=5pt, inner ysep=5pt] (sme) {SME};
%     \node[myrectangle, text width=2cm, minimum height=1cm, inner sep=5pt, inner ysep=5pt] (smeil) [right=1cm of sme] {SMEIL};
%     \node[mycircle, text width=2cm, inner sep=5pt, inner ysep=5pt] (transpiler) [right=1cm of smeil] {Transpiler};
%     \node[myrectangle, text width=2cm, minimum height=1cm, inner sep=5pt, inner ysep=5pt] (cspm) [right=1cm of transpiler] {CSP$_M$};
%
%     \draw[myarrow] (sme) -- (smeil);
%     \draw[myarrow] (smeil) -- (transpiler);
%     \draw[myarrow] (transpiler) -- (cspm);
%   \end{tikzpicture}
%   \caption{SME to \cspm{} transpiler.}
%   \label{fig:sme-to-cspm}
% \end{figure}
%
% We wish to model the network presented in Section~\ref{sec:example-smeil} in SMEIL in order to transpile it to \cspm{} so that we may verify properties in FDR4. In Figure~\ref{fig:sme-to-cspm} the workflow of this system can be seen.
%
% Even though SME buses can contain a series of channels, every single channel is translated into a \cspm{} channel. The properties we will assert with FDR4, are the width of the \cspm{} channels. That is, we want to prove that certain values will never be communicated on certain channels.
% It is easy to imagine that 4 bits can be communicated between the \texttt{time} processes and the seven segment displays. But 4 bits can represent the numbers 0 through 15, and our seven segment displays can only display the numbers 0 through 9. Therefore we wish to assert that even though the channels can carry 4 bits, the actual communication on the six output channels does not exceed 9. In general, the displays will be able to display 0 through 9, but since the example is a clock showing a 24-hour interval, the displays will of course not be able to show minutes and seconds above 59 and hours above 23.
%
% We know that a program in pure SMEIL must have a data generation process, but this is not the case in a CSP network. Since we are only transpiling from pure SMEIL networks, we can be certain that there will always be a process which just contributes an initial value to the rest of the network.
% We also know that a process must either have communication in or out or both.
% Therefore, we can assume that all SMEIL processes with no input bus will be a data generator process of some kind, and therefore must have some outwards communication.
% So when transpiling to \cspm{}, we do not translate the SMEIL process to a \cspm{} process, but simply create a \cspm{} channel that represents the values communicated out of this SMEIL process.
% \\
%
% We assume that the SMEIL programs we transpile only contains channels with types and range annotations. During the simulation, the type will be restricted to the lowest representation possible. For example, if a channel was originally set to be \texttt{int} (unbounded), but the observed values from the simulation show that it could be changed to an \texttt{i8} (signed 8-bit integer with a range of -128 to 127), then the simulated output would be \texttt{i8}.
%
% When creating channels in \cspm{}, we need to define its range of possible values. If a channel is only defined by having the integer type, FDR4 would try to verify for all possible integers, which results in a seemingly unbounded runtime. As explained in Section~\ref{SMEIL-section}, all simulated SMEIL programs will include the observed range and restricted types for all channels and variables. The types represent the observed width of the channels in bits, and by calculating the possible range from these types, we can create the corresponding channels in \cspm{}, and thereby avoid having a seemingly endless runtime in FDR4.
%
% Since the assertion we wish to make is to verify the widths of the channels, it might seem redundant to create \cspm{} channels with a limited range. FDR4 would always only check the values in the defined channel range and therefore there is no point in asserting if the values go beyond this range. After simulating the SME network, SMEIL provides us with both a type and a range of observed values. The type is used to create the \cspm{} channel range and the observed values are used for the assertion. The type will always represent equal or more values than the range of observed values, and by using these values the assertions becomes valuable.
% \\
%
% \begin{listing}
% \begin{minted}[escapeinside=||, mathescape=true]{smeil_lexer.py:SMEILLexer -x}
% proc seconds (in seconds_in)
%     bus seconds_out {first_digit: u3 range 0 to 5;
%                      second_digit: u4 range 0 to 9;};
%     var seconds: u6 range 1 to 59;
%     var seconds_first_temp: u3 range 0 to 5;
%     var seconds_second_temp: u4 range 0 to 9;
% {
%     seconds = seconds_in.val % 60;
%     seconds_first_temp = seconds / 10;
%     seconds_second_temp = seconds % 10;
%     seconds_out.first_digit = seconds_first_temp;
%     seconds_out.second_digit = seconds_second_temp;
% }
% \end{minted}
% \caption{Example of the \texttt{seconds} process from the SMEIL seven segment display example. See full example in Listing~\ref{lst:smeil} in the appendix.}
% \label{lst:range_smeil}
% \end{listing}
%
% When it comes to transpiling the data generator process into a \cspm{} channel, we also use the types of the SMEIL simulation to define it. We use this instead of the observed values because we cannot guarantee the precise input values of the system. If we used the observed values, the assertions will pass every time, since it will test the values already used to generate the rest of the observed values.
%
% An example of simulated SMEIL code can be seen in Listing~\ref{lst:range_smeil}. Notice on lines 2 and 3 that the two channels are defined both with a type \texttt{u3} and \texttt{u4} and with a range 0 to 5 and 0 to 9. These are the observed types and value ranges the simulation tracked for the specific channel. In order to create the \cspm{} channels based on the types, we need to convert \texttt{u3} and \texttt{u4} into its corresponding range, which for \texttt{u3} is 0 through 7 and for \texttt{u4} is 0 through 15. In Listing~\ref{lst:channel_range_cspm} on lines 1 and 2, the calculated ranges are used to define the \cspm{} channels.
%
% \begin{listing}
% \begin{minted}[escapeinside=||, mathescape=true]{cspm_lexer.py:CSPmLexer -x}
% channel seconds_out_first_digit : {0..7}
% channel seconds_out_second_digit : {0..15}
%
%     |$\vdots$|
%
% Seconds(seconds_in) =
% let
%     seconds = seconds_in % 60
%     seconds_first_temp = seconds / 10
%     seconds_second_temp = seconds % 10
% within
%     seconds_out_first_digit ! seconds_first_temp ->
%     seconds_out_second_digit ! seconds_second_temp ->
%     SKIP
% \end{minted}
% \caption{Example of the \texttt{Seconds} process from the generated \cspm{} code in the seven segment display example. See full example in Listing~\ref{lst:cspm} in the appendix.}
% \label{lst:channel_range_cspm}
% \end{listing}
%
% When creating the assertions, we decided to create separate assert functions to keep the code structure clean. We know that for each \cspm{} channel there must be an assertion, except for the input channel.
% Consequently, we create a \textit{monitor} process for each channel and its only job is to listen in on the channel communication and assert the values communicated there. The monitor process is a process that we add specifically for asserting legal communication values in FDR4 and it does not affect the original SME network.
% In Figure~\ref{fig:assertion_process} the outline of this kind of structure can be seen and we expect that this structure can be used for several different types of problems and thereby ensure a cleaner code structure.
%
% The monitor process asserts the observed values of the \cspm{} channels and in Listing~\ref{lst:monitor_range_cspm} the two monitor processes for the Seconds \texttt{time} process can be seen. The values used for these statements are the observed values from the SMEIL simulation, as can be seen at the end of lines 2 and 3 in Listing~\ref{lst:range_smeil}. In Listing~\ref{lst:monitor_range_cspm} the ranges are used to assert that the only values communicated on the channels are within 0 and 5, and 0 and 9 respectively.
%
% \begin{figure}[!ht]
%   \centering
%   \begin{tikzpicture}[auto]
%     \node[mycircle] (P) at (-1.5, 0.0) {$P$};
%     \node[mycircle] (Q) at ( 2.5, 0.0) {$Q$};
%     \node[mycircle, shape=rectangle] (M) at ( 0.5, 1.5) {$M$};
%
%     \node[draw, shape=circle, inner sep=0pt, minimum size=5pt] (m) at (0.5, 0.0) {};
%
%
%     \draw (M) -- (P -| M) [black!50];
%     \draw [myarrow] (P) -- (Q);
%   \end{tikzpicture}
%   \caption{The monitor process \textit{M} listens in on the communication between \textit{P} and \textit{Q} in order to assert the communicated values.}
%   \label{fig:assertion_process}
% \end{figure}
%
% \begin{listing}
% \begin{minted}[escapeinside=||, mathescape=true]{cspm_lexer.py:CSPmLexer -x}
% Seconds_out_first_digit_monitor(c) =
%     c ? x -> if 0 <= x and x <= 5 then SKIP else STOP
% Seconds_out_second_digit_monitor(c) =
%     c ? x -> if 0 <= x and x <= 9 then SKIP else STOP
% \end{minted}
% \caption{Example of the \texttt{Seconds} monitor processes from the generated \cspm{} code in the seven segment display example. See full example in Listing~\ref{lst:cspm} in the appendix.}
% \label{lst:monitor_range_cspm}
% \end{listing}
% s
%
% After translating the SMEIL processes and creating the monitor processes, we need to create the network described in the last part of the SMEIL program, see lines 53 to 59 in Listing~\ref{lst:smeil} in the appendix. We wish only to assert the values the \texttt{time} processes are communicating to the monitor processes, and therefore we have to synchronize these processes into a single network in \cspm{}. We create three network processes, one for each part of the network, and we create a nested synchronization, in order to have all monitor processes synchronized with the \texttt{time} process. An example of this network can be seen in on lines 61 to 66 in Listing~\ref{lst:cspm} in the appendix. This network process is also the process that receives the input from the input channel. By not adding the receiving communication in the \texttt{time} processes, we avoid having to specify the name of the input channels before creating the network which simplifies the translation, as described in Section~\ref{sec:transpiling}. In SMEIL, this information is part of the \texttt{network} section, and therefore it fits well within this part of the \cspm{} code.
%
% After creating the network we add the actual assert function calls. For these kinds of assertions, where we want to check a range, the best solution is to assert that the network processes behave as the \texttt{SKIP} process. This is done by having the monitor process running the \texttt{SKIP} process if the value is within the range and the \texttt{STOP} process if not. Two examples can be seen in lines 2 and 4 in Listing~\ref{lst:monitor_range_cspm}. We assert this by using the FDR4 failures model on the the \texttt{SKIP} process along with hiding communication events, which can be seen in lines 68, 78 and 88 in Listing~\ref{lst:cspm} in the appendix.
% \\
%
% The different parts of transpiling the seven segment display example have been presented and in Figure~\ref{fig:cspm-network} the corresponding network of the \cspm{} system is presented.
% The corresponding network in \cspm{} consists of 12 different processes, all created so that not only the network is simulated correctly, but also so the assertions we wish to make, are in place. The input is represented by a triangle, since it transpiles from an SME process to a \cspm{} channel. Each of the dotted squares represents the network of synchronizations for each \texttt{time} processes, which in itself is a process in \cspm{}. For each network, we have the \texttt{time} processes and two monitor processes, for example, $H$, $M_{H_1}$ and $M_{H_2}$.
% \\
%
% % Errornous example
% \begin{listing}
% \begin{minted}[escapeinside=||, mathescape=true]{cspm_lexer.py:CSPmLexer -x}
% channel clock_out_val : {0..131071}
%
% channel hours_out_first_digit : {0..3}
% channel hours_out_second_digit : {0..15}
%     |$\vdots$|
%
% Hours(hours_in) =
% let
%     hours = hours_in / 3600
%     |$\vdots$|
%
% Hours_out_first_digit_monitor(c) =
%     c ? x -> if 0 <= x and x <= 2 then SKIP else STOP
% Hours_out_second_digit_monitor(c) =
%     c ? x -> if 0 <= x and x <= 9 then SKIP else STOP
%
% \end{minted}
% \caption{Example of an erroneous version of the \texttt{Hours} process from the \cspm{} seven segment display example seen in Listing~\ref{lst:smeil} and in Listing~\ref{lst:cspm} in the appendix.}
% \label{lst:cspm_error}
% \end{listing}
%
% In order to show that the verification is accurate, the example in Listing~\ref{lst:cspm_error} contains an error that results in FDR4 failing the verification. In Listing~\ref{lst:cspm_error} the example is only able to handle an input that is below 24 hours. This is because the calculation in the \texttt{Hours} process does not handle the wrap around at the 24\textsuperscript{th} hour. This means that if the input represents more than 24 hours, the assertions will fail in FDR4 because one seven segment display suddenly has to display two digits instead of one. An example of such could be the input \texttt{131071}, which represents 36 hours, 24 minutes and 31 seconds, or 1 day, 12 hours, 24 minutes and 31 seconds. When trying to assert the code from Listing~\ref{lst:cspm_error} in FDR4, the assertion fails. The counterexample shows that the number 3 is communicated on \texttt{hours\_out\_first\_digit}, which is not allowed according to the monitor process on lines 12 and 13 in Listing~\ref{lst:cspm_error}.
%
% This example of failure shows how verifying the solution with a tool like FDR4 actually catches errors that the programmer might have overseen. In this case, the error is simply corrected by adding \texttt{\% 24} on the end of line 9 in Listing~\ref{lst:cspm_error} and can be seen corrected in Listing~\ref{lst:cspm} in the appendix at line 15. Now when we try to assert the example in FDR4, it passes. By using modulo on the result, we ensure that we still get the accurate time of day, no matter how many full days the input represents.
%
% The full SMEIL and \cspm{} code for the seven segment display example can be seen in Listing~\ref{lst:smeil} and in Listing~\ref{lst:cspm} in the appendix.
% % Errornous example end


% It is important to mention that the FDR version of the SMEIL program are represented as one clock cycle and therefore we do not have to handle implicit clock cycle issues. we can just translate one-to-one, because FDR models one clock cycle and the input represents all possible input in one clock cycle.


\section{Clock cycle problem}
When trying to model a clocked hardware system in \cspm{}, we want to create a synchronous network where the system has a controlled termination, and where it can be verified that all processes end, by behaving like the \texttt{SKIP} process.

In this example, we wish to create a network called 'Addone'. The network consists of two processes where one process, the 'Add' process, increments a variable by one and sents it along to the other process, the 'Id' process, which then sents the value back to the 'Add' process.
% TODO: Add a picture of the network, like Truls example.
The network is a two process loop and it is therefore essential that there is a way to initialise the loop as well as terminating it properly.\\

We wish to model a \cspm{} network which reflects the SME model and therefore we have to adhere to the SME model structure. There is three different states for each clock cycle; the read state, the calculation statem, and the write state. A process must read before writing in a clock cycle.\\
For the 'Add' and 'Id' processes to comply with these states, they would have to read first, then the calculate phase, in which the 'Id' process does nothing, and then they would write the result onto a channel. A problem occurs, since they both have to read first, no one can read because no processes have written anything yet. To solve this, we implement two buffers which for each clock cycle reads the output that the process writes and then writes the value to a channel. Thus the buffer structure is the reverse of a 'normal' process since it will write and then read in a clock cycle. \\
If we give the buffers an initial value, they can begin the clock cycle writing the value which the 'Add' and 'Id' processes can read and thereby they will comply with the SME model structure.
The buffers will be instantiated will a 'dummy' value which is also how it is typically done in hardware. The dummy value is simply to indicate that the system should ignore the first clock cycle and then continue with the systems actual values. \\
Each process is also instantiated with a value, which is then used instead of the dummy value from the buffer process. After this initial cycle, the process loop will continue and the communication will hold according to the description of the network explained above.\\\\


We have to limit the number of verified clock cycles, otherwise, FDR4 would never stop. Even though the standard SME process also never stops, when simulating the system in SMEIL, it is only simulating a finite number of clock cycles. Therefore we need to implement a way to have all processes of the \cspm{} network terminate in a timely manner. This is done with the \texttt{Clock} process. The clock process is initialised with a value and syncronise on the \texttt{clock} channel, which all other processes does as well. When the specified number of clock cycles has passed, the \texttt{Clock} process stops clocking and instead behaves as \texttt{SKIP}. This means that all other processes won't be able to syncronise on the \texttt{clock} channel anymore and therefore they will then instead behave as \texttt{SKIP} and that way the system terminates as planned.\\
Since the processes all must read, calculate and then write, the buffer processes, as mentioned before, must behave the opposite way. This means that the last write the processes make before they \texttt{SKIP} will be left in the buffers since there are no processes to read the value from the channels. This means that the buffers must be able to either write a value or \texttt{SKIP}.\\
The \texttt{clock} channel is used as a two-way clock synchronisation, where the same channel is used for syncronising up as well as down. Thus all processes syncronise before they read and before they write. The \texttt{Clock} process then needs to syncronise twice before incrementing its counter. \\

When trying to verify that this system terminates as expected in FDR4, we came across an error while FDR4 was compiling the program.
FDR4 is complaining that a value, the system is trying to send, is not a part of the set of values defined for the channel.
The channels, used for communicating the value between the processes and the buffers, are defined for a specific range and since the 'Add' process has to read, calculate and write before it can terminate it will always write a value out that have been incremented with one.\\
For instance, if the channels were defined with the range \{0..5\} and the \texttt{Clock} process would stop after 10 clock cycles. This would mean that on the last clock cycle, before terminating, the 'Add' process would write a 6 onto the channel, which of course is not possible since the channel is defined only for the range \{0..5\}, so FDR4 complains about this.\\
However, what we experienced was that when the ranges of the channels were set to a larger number than the system would ever reach, within the defined number of clock cycles, FDR4 would still fail with the same reason.
This caused some problems since we were not able to verify the system and we were not interested in FDR4 trying to verify a communication that would never occur within the network.
Since we are syncronising the processes on the events in the channels, it seems odd that FDR4 still considers events which should not be possible to reach.  \\\\
After some time working with the problem and trying to understand the reason for FDR4s error message, we found that if we simply add a guard or an if-then-else statement that tests the value to be written, FDR4 will gladly verify the system and when using Probe on the network, it is clear that FDR4 does not consider the trace with the wrong values. \\

So the suggested solution, or fix, to this problem, is to add an if-then-else before all writes in a program. The statement then tests the value to be written against the max value of the range of the channel and if the value is not within the range, then the process behaves as the \texttt{SKIP} process, otherwise, it continues with write.
\begin{minted}[linenos=false, escapeinside=||, mathescape=true]{cspm_lexer.py:CSPmLexer -x}
channel c : {0..20}

|$\vdots$|

    if (i+1) > 20  -- Check the upper limit of the channel
        then SKIP  -- SKIP if the value is above
        else (c_r ! (i+1) -> Add(i))) -- Otherwise write and continue
\end{minted}
In this case, it is only necessary to add an upper limit test, since the network only increments, but as a general rule, it would be necessary to test for both upper and lower bounds.


% The problem ocurred several times with different versions of the solution. Also, Ohm had the same problem with the Commstime problem. It makes sense why FDR4 wants to check, and maybe it is a way to save verification time: if it checks things in parallel or something.
% However it is not a good solution that we have to put in a upper/lower bound check that is actually never relevant.
% Of course it might be good, in any case, since the programmer then do not need to worry. On the other hand it might make the program look like it is terminating properly when it is actually failing because it tries to write a value that is not allowed. This case might happen, and if all other processes do not notice it and then also SKIP according to their specifications, then the verification passes even though it might be wrong.
% A solution might be to ensure that all processes synchronise before skipping, because then the problem (i think) would not occur, since the failing process would simply skip, then the other processes cannot skip because they have not syncrhonised yet. however, I am not sure this is possible since it might be that they either syncronise or they skip.

% Kenneths version, which I believe is how the SME model works, is having a process or bus in the middle og all steps. By using a dependency graph (Explain more?) it is possible to see which processes communicate to witch processes and, more importantly, in which order. For each communication step (or maybe for each communication) a process/bus will receive all writes. In SME a process can write several times to the same channel but only the last one before the clock signal will be written, the others are just overwritten. Since we have the dependency graph, we also know which processes we need communication from, and when the process have written all it has to write, then it sends a ready signal to the "bus" process, which then waits for all the ready signals (because it knows how many it should get. And if it is one process/bus pr. communication then it only needs one of course.). When all ready signals are in, the bus-process change behaviour and it is not writing instead of reading. It writes all possible values out and the processes that are supposed to receive the values (which we know from the dependency graph) will receive the values. And the processes then need to send a ready signal back to the bus process to let it know that it have read all it needed. When all ready signals are received, the bus process when change behaviour again and can read values once again.
% All these steps are intermediate steps within one clock cycle. So at the "end" of the dependency graph, the step looks similar to the others, but it is registered as the clock and the next clock cycle begins. In principal, all these steps could be the clock, since the step is the same, but a step is simply chosen to be the clock, based on the dependency graph.
% By treating the communication like this within a clock cycle, the values can propagate through the network and the internal state of the processes are also kept. The original TAPS version could only verify all input for a system, but if the system was internally affected by values from a previous clock cycle, then the system could not verify it. It is not a problem in the seven segment example, since no values are dependent on previous values. But the Addone network do depend on what happened in the last clock cycle.
% With this solution it is possible to verify a specific number of clock cycles.
%
% It is important to note that SME processes usually do not end, but continue running forever. However, when we simulate with SME, the simulation end after some specific number of clock cycles, and this is what we wish to verify.
% It might be interesting to use the "running forever" thing to verify stuff on it. But at the moment I cant see what we would verify. Deadlocks would never happen in SME, so it would be possible to use it to check that the translation have been successful, but this is not so relevant for other people.
% It might be that the stuff we wish to verify is not possible in a limited clock cycle range and that it is necessary to run it "forever". I am not sure.



% (From design)
% \section{Clock cycle problem}
% % When trying to model a clocked hardware system in \cspm{}, we want to create a synchronous network where the system has a controlled termination, and where it can be verified that all processes end, by behaving like the \texttt{SKIP} process.

In this example, we wish to create a network called 'Addone'. The network consists of two processes where one process, the 'Add' process, increments a variable by one and sents it along to the other process, the 'Id' process, which then sents the value back to the 'Add' process.
% TODO: Add a picture of the network, like Truls example.
The network is a two process loop and it is therefore essential that there is a way to initialise the loop as well as terminating it properly.\\

We wish to model a \cspm{} network which reflects the SME model and therefore we have to adhere to the SME model structure. There is three different states for each clock cycle; the read state, the calculation statem, and the write state. A process must read before writing in a clock cycle.\\
For the 'Add' and 'Id' processes to comply with these states, they would have to read first, then the calculate phase, in which the 'Id' process does nothing, and then they would write the result onto a channel. A problem occurs, since they both have to read first, no one can read because no processes have written anything yet. To solve this, we implement two buffers which for each clock cycle reads the output that the process writes and then writes the value to a channel. Thus the buffer structure is the reverse of a 'normal' process since it will write and then read in a clock cycle. \\
If we give the buffers an initial value, they can begin the clock cycle writing the value which the 'Add' and 'Id' processes can read and thereby they will comply with the SME model structure.
The buffers will be instantiated will a 'dummy' value which is also how it is typically done in hardware. The dummy value is simply to indicate that the system should ignore the first clock cycle and then continue with the systems actual values. \\
Each process is also instantiated with a value, which is then used instead of the dummy value from the buffer process. After this initial cycle, the process loop will continue and the communication will hold according to the description of the network explained above.\\\\


We have to limit the number of verified clock cycles, otherwise, FDR4 would never stop. Even though the standard SME process also never stops, when simulating the system in SMEIL, it is only simulating a finite number of clock cycles. Therefore we need to implement a way to have all processes of the \cspm{} network terminate in a timely manner. This is done with the \texttt{Clock} process. The clock process is initialised with a value and syncronise on the \texttt{clock} channel, which all other processes does as well. When the specified number of clock cycles has passed, the \texttt{Clock} process stops clocking and instead behaves as \texttt{SKIP}. This means that all other processes won't be able to syncronise on the \texttt{clock} channel anymore and therefore they will then instead behave as \texttt{SKIP} and that way the system terminates as planned.\\
Since the processes all must read, calculate and then write, the buffer processes, as mentioned before, must behave the opposite way. This means that the last write the processes make before they \texttt{SKIP} will be left in the buffers since there are no processes to read the value from the channels. This means that the buffers must be able to either write a value or \texttt{SKIP}.\\
The \texttt{clock} channel is used as a two-way clock synchronisation, where the same channel is used for syncronising up as well as down. Thus all processes syncronise before they read and before they write. The \texttt{Clock} process then needs to syncronise twice before incrementing its counter. \\

When trying to verify that this system terminates as expected in FDR4, we came across an error while FDR4 was compiling the program.
FDR4 is complaining that a value, the system is trying to send, is not a part of the set of values defined for the channel.
The channels, used for communicating the value between the processes and the buffers, are defined for a specific range and since the 'Add' process has to read, calculate and write before it can terminate it will always write a value out that have been incremented with one.\\
For instance, if the channels were defined with the range \{0..5\} and the \texttt{Clock} process would stop after 10 clock cycles. This would mean that on the last clock cycle, before terminating, the 'Add' process would write a 6 onto the channel, which of course is not possible since the channel is defined only for the range \{0..5\}, so FDR4 complains about this.\\
However, what we experienced was that when the ranges of the channels were set to a larger number than the system would ever reach, within the defined number of clock cycles, FDR4 would still fail with the same reason.
This caused some problems since we were not able to verify the system and we were not interested in FDR4 trying to verify a communication that would never occur within the network.
Since we are syncronising the processes on the events in the channels, it seems odd that FDR4 still considers events which should not be possible to reach.  \\\\
After some time working with the problem and trying to understand the reason for FDR4s error message, we found that if we simply add a guard or an if-then-else statement that tests the value to be written, FDR4 will gladly verify the system and when using Probe on the network, it is clear that FDR4 does not consider the trace with the wrong values. \\

So the suggested solution, or fix, to this problem, is to add an if-then-else before all writes in a program. The statement then tests the value to be written against the max value of the range of the channel and if the value is not within the range, then the process behaves as the \texttt{SKIP} process, otherwise, it continues with write.
\begin{minted}[linenos=false, escapeinside=||, mathescape=true]{cspm_lexer.py:CSPmLexer -x}
channel c : {0..20}

|$\vdots$|

    if (i+1) > 20  -- Check the upper limit of the channel
        then SKIP  -- SKIP if the value is above
        else (c_r ! (i+1) -> Add(i))) -- Otherwise write and continue
\end{minted}
In this case, it is only necessary to add an upper limit test, since the network only increments, but as a general rule, it would be necessary to test for both upper and lower bounds.


% The problem ocurred several times with different versions of the solution. Also, Ohm had the same problem with the Commstime problem. It makes sense why FDR4 wants to check, and maybe it is a way to save verification time: if it checks things in parallel or something.
% However it is not a good solution that we have to put in a upper/lower bound check that is actually never relevant.
% Of course it might be good, in any case, since the programmer then do not need to worry. On the other hand it might make the program look like it is terminating properly when it is actually failing because it tries to write a value that is not allowed. This case might happen, and if all other processes do not notice it and then also SKIP according to their specifications, then the verification passes even though it might be wrong.
% A solution might be to ensure that all processes synchronise before skipping, because then the problem (i think) would not occur, since the failing process would simply skip, then the other processes cannot skip because they have not syncrhonised yet. however, I am not sure this is possible since it might be that they either syncronise or they skip.

% Kenneths version, which I believe is how the SME model works, is having a process or bus in the middle og all steps. By using a dependency graph (Explain more?) it is possible to see which processes communicate to witch processes and, more importantly, in which order. For each communication step (or maybe for each communication) a process/bus will receive all writes. In SME a process can write several times to the same channel but only the last one before the clock signal will be written, the others are just overwritten. Since we have the dependency graph, we also know which processes we need communication from, and when the process have written all it has to write, then it sends a ready signal to the "bus" process, which then waits for all the ready signals (because it knows how many it should get. And if it is one process/bus pr. communication then it only needs one of course.). When all ready signals are in, the bus-process change behaviour and it is not writing instead of reading. It writes all possible values out and the processes that are supposed to receive the values (which we know from the dependency graph) will receive the values. And the processes then need to send a ready signal back to the bus process to let it know that it have read all it needed. When all ready signals are received, the bus process when change behaviour again and can read values once again.
% All these steps are intermediate steps within one clock cycle. So at the "end" of the dependency graph, the step looks similar to the others, but it is registered as the clock and the next clock cycle begins. In principal, all these steps could be the clock, since the step is the same, but a step is simply chosen to be the clock, based on the dependency graph.
% By treating the communication like this within a clock cycle, the values can propagate through the network and the internal state of the processes are also kept. The original TAPS version could only verify all input for a system, but if the system was internally affected by values from a previous clock cycle, then the system could not verify it. It is not a problem in the seven segment example, since no values are dependent on previous values. But the Addone network do depend on what happened in the last clock cycle.
% With this solution it is possible to verify a specific number of clock cycles.
%
% It is important to note that SME processes usually do not end, but continue running forever. However, when we simulate with SME, the simulation end after some specific number of clock cycles, and this is what we wish to verify.
% It might be interesting to use the "running forever" thing to verify stuff on it. But at the moment I cant see what we would verify. Deadlocks would never happen in SME, so it would be possible to use it to check that the translation have been successful, but this is not so relevant for other people.
% It might be that the stuff we wish to verify is not possible in a limited clock cycle range and that it is necessary to run it "forever". I am not sure.



% (From design)
% \section{Clock cycle problem}
% % When trying to model a clocked hardware system in \cspm{}, we want to create a synchronous network where the system has a controlled termination, and where it can be verified that all processes end, by behaving like the \texttt{SKIP} process.

In this example, we wish to create a network called 'Addone'. The network consists of two processes where one process, the 'Add' process, increments a variable by one and sents it along to the other process, the 'Id' process, which then sents the value back to the 'Add' process.
% TODO: Add a picture of the network, like Truls example.
The network is a two process loop and it is therefore essential that there is a way to initialise the loop as well as terminating it properly.\\

We wish to model a \cspm{} network which reflects the SME model and therefore we have to adhere to the SME model structure. There is three different states for each clock cycle; the read state, the calculation statem, and the write state. A process must read before writing in a clock cycle.\\
For the 'Add' and 'Id' processes to comply with these states, they would have to read first, then the calculate phase, in which the 'Id' process does nothing, and then they would write the result onto a channel. A problem occurs, since they both have to read first, no one can read because no processes have written anything yet. To solve this, we implement two buffers which for each clock cycle reads the output that the process writes and then writes the value to a channel. Thus the buffer structure is the reverse of a 'normal' process since it will write and then read in a clock cycle. \\
If we give the buffers an initial value, they can begin the clock cycle writing the value which the 'Add' and 'Id' processes can read and thereby they will comply with the SME model structure.
The buffers will be instantiated will a 'dummy' value which is also how it is typically done in hardware. The dummy value is simply to indicate that the system should ignore the first clock cycle and then continue with the systems actual values. \\
Each process is also instantiated with a value, which is then used instead of the dummy value from the buffer process. After this initial cycle, the process loop will continue and the communication will hold according to the description of the network explained above.\\\\


We have to limit the number of verified clock cycles, otherwise, FDR4 would never stop. Even though the standard SME process also never stops, when simulating the system in SMEIL, it is only simulating a finite number of clock cycles. Therefore we need to implement a way to have all processes of the \cspm{} network terminate in a timely manner. This is done with the \texttt{Clock} process. The clock process is initialised with a value and syncronise on the \texttt{clock} channel, which all other processes does as well. When the specified number of clock cycles has passed, the \texttt{Clock} process stops clocking and instead behaves as \texttt{SKIP}. This means that all other processes won't be able to syncronise on the \texttt{clock} channel anymore and therefore they will then instead behave as \texttt{SKIP} and that way the system terminates as planned.\\
Since the processes all must read, calculate and then write, the buffer processes, as mentioned before, must behave the opposite way. This means that the last write the processes make before they \texttt{SKIP} will be left in the buffers since there are no processes to read the value from the channels. This means that the buffers must be able to either write a value or \texttt{SKIP}.\\
The \texttt{clock} channel is used as a two-way clock synchronisation, where the same channel is used for syncronising up as well as down. Thus all processes syncronise before they read and before they write. The \texttt{Clock} process then needs to syncronise twice before incrementing its counter. \\

When trying to verify that this system terminates as expected in FDR4, we came across an error while FDR4 was compiling the program.
FDR4 is complaining that a value, the system is trying to send, is not a part of the set of values defined for the channel.
The channels, used for communicating the value between the processes and the buffers, are defined for a specific range and since the 'Add' process has to read, calculate and write before it can terminate it will always write a value out that have been incremented with one.\\
For instance, if the channels were defined with the range \{0..5\} and the \texttt{Clock} process would stop after 10 clock cycles. This would mean that on the last clock cycle, before terminating, the 'Add' process would write a 6 onto the channel, which of course is not possible since the channel is defined only for the range \{0..5\}, so FDR4 complains about this.\\
However, what we experienced was that when the ranges of the channels were set to a larger number than the system would ever reach, within the defined number of clock cycles, FDR4 would still fail with the same reason.
This caused some problems since we were not able to verify the system and we were not interested in FDR4 trying to verify a communication that would never occur within the network.
Since we are syncronising the processes on the events in the channels, it seems odd that FDR4 still considers events which should not be possible to reach.  \\\\
After some time working with the problem and trying to understand the reason for FDR4s error message, we found that if we simply add a guard or an if-then-else statement that tests the value to be written, FDR4 will gladly verify the system and when using Probe on the network, it is clear that FDR4 does not consider the trace with the wrong values. \\

So the suggested solution, or fix, to this problem, is to add an if-then-else before all writes in a program. The statement then tests the value to be written against the max value of the range of the channel and if the value is not within the range, then the process behaves as the \texttt{SKIP} process, otherwise, it continues with write.
\begin{minted}[linenos=false, escapeinside=||, mathescape=true]{cspm_lexer.py:CSPmLexer -x}
channel c : {0..20}

|$\vdots$|

    if (i+1) > 20  -- Check the upper limit of the channel
        then SKIP  -- SKIP if the value is above
        else (c_r ! (i+1) -> Add(i))) -- Otherwise write and continue
\end{minted}
In this case, it is only necessary to add an upper limit test, since the network only increments, but as a general rule, it would be necessary to test for both upper and lower bounds.


% The problem ocurred several times with different versions of the solution. Also, Ohm had the same problem with the Commstime problem. It makes sense why FDR4 wants to check, and maybe it is a way to save verification time: if it checks things in parallel or something.
% However it is not a good solution that we have to put in a upper/lower bound check that is actually never relevant.
% Of course it might be good, in any case, since the programmer then do not need to worry. On the other hand it might make the program look like it is terminating properly when it is actually failing because it tries to write a value that is not allowed. This case might happen, and if all other processes do not notice it and then also SKIP according to their specifications, then the verification passes even though it might be wrong.
% A solution might be to ensure that all processes synchronise before skipping, because then the problem (i think) would not occur, since the failing process would simply skip, then the other processes cannot skip because they have not syncrhonised yet. however, I am not sure this is possible since it might be that they either syncronise or they skip.

% Kenneths version, which I believe is how the SME model works, is having a process or bus in the middle og all steps. By using a dependency graph (Explain more?) it is possible to see which processes communicate to witch processes and, more importantly, in which order. For each communication step (or maybe for each communication) a process/bus will receive all writes. In SME a process can write several times to the same channel but only the last one before the clock signal will be written, the others are just overwritten. Since we have the dependency graph, we also know which processes we need communication from, and when the process have written all it has to write, then it sends a ready signal to the "bus" process, which then waits for all the ready signals (because it knows how many it should get. And if it is one process/bus pr. communication then it only needs one of course.). When all ready signals are in, the bus-process change behaviour and it is not writing instead of reading. It writes all possible values out and the processes that are supposed to receive the values (which we know from the dependency graph) will receive the values. And the processes then need to send a ready signal back to the bus process to let it know that it have read all it needed. When all ready signals are received, the bus process when change behaviour again and can read values once again.
% All these steps are intermediate steps within one clock cycle. So at the "end" of the dependency graph, the step looks similar to the others, but it is registered as the clock and the next clock cycle begins. In principal, all these steps could be the clock, since the step is the same, but a step is simply chosen to be the clock, based on the dependency graph.
% By treating the communication like this within a clock cycle, the values can propagate through the network and the internal state of the processes are also kept. The original TAPS version could only verify all input for a system, but if the system was internally affected by values from a previous clock cycle, then the system could not verify it. It is not a problem in the seven segment example, since no values are dependent on previous values. But the Addone network do depend on what happened in the last clock cycle.
% With this solution it is possible to verify a specific number of clock cycles.
%
% It is important to note that SME processes usually do not end, but continue running forever. However, when we simulate with SME, the simulation end after some specific number of clock cycles, and this is what we wish to verify.
% It might be interesting to use the "running forever" thing to verify stuff on it. But at the moment I cant see what we would verify. Deadlocks would never happen in SME, so it would be possible to use it to check that the translation have been successful, but this is not so relevant for other people.
% It might be that the stuff we wish to verify is not possible in a limited clock cycle range and that it is necessary to run it "forever". I am not sure.



% (From design)
% \section{Clock cycle problem}
% % \input{chapters/clock_cycle_problem}
%
% CSP was not initially developed for hardware modeling, and therefore it is not evident how to handle the clock cycle, which is an essential part of hardware modeling. When we transpile the SME network into \cspm{}, the SMEIL simulation have provided the ranges of all values from the simulation and therefore all clock cycles. This means that when FDR4 asserts a property it asserts on all possible communication combinations for all the simulated clock cycles. Therefore, even though we are transpiling from an SME model, where the clock is crucial, we can simply translate ``one-to-one" from the SMEIL program and still get an accurate assertion on the properties.
%
%
%
%
% % It is important to mention that the FDR version of the SMEIL program are represented as one clock cycle and therefore we do not have to handle implicit clock cycle issues. we can just translate one-to-one, because FDR models one clock cycle and the input represents all possible input in one clock cycle.
%
%
%
% % TODO: Async vs. sync processes
%
% %%% Generator processes - clock cycles
% %TODO: Write something here?
%
%
% % TODO: Write that the read of a process does not make sense in the network and that we move it to be internal inside the process.


% (from verification in design)
% TODO: Maybe I should do it with the FD model, but in this case it does not really make sense, since all processes end after one iterations (they all SKIP). But it is worth mentioning it in the new system, because there it becomes relevant
%
% CSP was not initially developed for hardware modeling, and therefore it is not evident how to handle the clock cycle, which is an essential part of hardware modeling. When we transpile the SME network into \cspm{}, the SMEIL simulation have provided the ranges of all values from the simulation and therefore all clock cycles. This means that when FDR4 asserts a property it asserts on all possible communication combinations for all the simulated clock cycles. Therefore, even though we are transpiling from an SME model, where the clock is crucial, we can simply translate ``one-to-one" from the SMEIL program and still get an accurate assertion on the properties.
%
%
%
%
% % It is important to mention that the FDR version of the SMEIL program are represented as one clock cycle and therefore we do not have to handle implicit clock cycle issues. we can just translate one-to-one, because FDR models one clock cycle and the input represents all possible input in one clock cycle.
%
%
%
% % TODO: Async vs. sync processes
%
% %%% Generator processes - clock cycles
% %TODO: Write something here?
%
%
% % TODO: Write that the read of a process does not make sense in the network and that we move it to be internal inside the process.


% (from verification in design)
% TODO: Maybe I should do it with the FD model, but in this case it does not really make sense, since all processes end after one iterations (they all SKIP). But it is worth mentioning it in the new system, because there it becomes relevant
%
% CSP was not initially developed for hardware modeling, and therefore it is not evident how to handle the clock cycle, which is an essential part of hardware modeling. When we transpile the SME network into \cspm{}, the SMEIL simulation have provided the ranges of all values from the simulation and therefore all clock cycles. This means that when FDR4 asserts a property it asserts on all possible communication combinations for all the simulated clock cycles. Therefore, even though we are transpiling from an SME model, where the clock is crucial, we can simply translate ``one-to-one" from the SMEIL program and still get an accurate assertion on the properties.
%
%
%
%
% % It is important to mention that the FDR version of the SMEIL program are represented as one clock cycle and therefore we do not have to handle implicit clock cycle issues. we can just translate one-to-one, because FDR models one clock cycle and the input represents all possible input in one clock cycle.
%
%
%
% % TODO: Async vs. sync processes
%
% %%% Generator processes - clock cycles
% %TODO: Write something here?
%
%
% % TODO: Write that the read of a process does not make sense in the network and that we move it to be internal inside the process.


% (from verification in design)
% TODO: Maybe I should do it with the FD model, but in this case it does not really make sense, since all processes end after one iterations (they all SKIP). But it is worth mentioning it in the new system, because there it becomes relevant

